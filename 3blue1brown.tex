\documentclass[12pt]{article}
\usepackage{enumerate}
\usepackage{mathematics}
\usepackage{oxford}

\begin{document}

\begin{mdframed}
  Choose 3 points uniformly over the circumference of a circle. What's the
  probability that the inscribed triangle contains the center of the circle?
\end{mdframed}

The first point is arbitrary. The second point is separated from the first by
an arc of $X \sim \Unif(0, \pi)$ radians. Given this first arc of $X$ radians,
consider the arc of $X$ radians that lies on the opposite side of the
circle. If the third point falls within this arc then the 3 points define a
triangle containing the centre. Let $q$ be the desired probability and $\phi$
be the probability density of $X$. Then
\begin{align*}
  q &= \int_0^{\pi} \frac{x}{2\pi}\phi_X(x) \dx\\
    &= \int_0^{\pi} \frac{x}{2\pi}\frac{1}{\pi} \dx\\
    &= \frac{1}{2\pi^2} \frac{\pi^2}{2}\\
    &= \frac{1}{4}.
\end{align*}


\begin{mdframed}
  Choose 4 points uniformly over the surface of a sphere. What's the
  probability that the inscribed tetrahedron contains the center of the sphere?
\end{mdframed}

The first point is arbitrary. The second point lies on a great circle passing
through the first point, and is separated from the first point by an arc of
$X_1 \sim \Unif(0, \pi)$ radians. The third point is specified by a latitude
$X_2 \sim \Unif(0, \pi)$ and a longitude $X_3 \sim \Unif(0, \pi)$.  $X_1$,
$X_2$ and $X_3$ define a subset of the surface area of the sphere (a hyperbolic
triangle lying in the surface of the sphere).

There is a corresponding subset on the opposite side of the sphere with the
same shape and surface area (For each of the 3 points, draw a line through the
centre of the sphere and find its intersection with the surface beyond. The
second subset is the hyperbolic triangle defined by these 3 points of
intersection).

If the fourth point lies in this second subset then the tetrahedron contains
the center of the sphere. Therefore the desired probability is the area of the
hyperbolic triangle, as a proportion of the total surface area, averaged over
possible values of $X_1, X_2, X_3$.

Also, note that $\Big(\frac{1}{4}\Big)^{3/2} = \frac{1}{8}$, i.e. this is the
value expected from a straightforward scaling argument when passing from 2D to
3D.

\end{document}
