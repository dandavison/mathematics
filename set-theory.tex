\section{Combinatorics}
The number of $k$-tuples that can be chosen from $n$ items without
replacement is
\begin{align*}
  (n)_k = n \cdot (n-1) \cdots (n - k + 1) = \frac{n!}{(n-k)!}
\end{align*}

The number of sets of size $k$ that can be chosen from $n$ items is
\begin{align*}
  {n \choose k} = \frac{(n)_k}{k!} = \frac{n!}{k!(n-k)!}.
\end{align*}

The number of ways that $n$ items can be put into $k$ buckets, with no empty
buckets, recording only the counts in each bucket (not the identities of the
items), is
\begin{align*}
  {n-1 \choose k -1}.
\end{align*}
\textit{Proof}: Represent this as $n$ unlabeled stars, and $k-1$ bars
representing the partition of the stars into different buckets. There are $n-1$
gaps where the bars can be placed, hence ${n-1 \choose k -1}$ ways of dividing
up the items.

The number of ways that $n$ items can be put into $k$ buckets, with no empty
buckets, recording the identities of the items in each bucket, is the number of
\textit{partitions} of size $k$ of a set of size $n$. It is equal to the
Stirling number of the second kind:
\begin{align*}
  S(n, k) = \frac{1}{k!} \sum_{i=0}^k(-1)^i{k \choose i}(k-i)^n. ~~ \blue{\text{(check this)}}
\end{align*}
\textit{Proof}:

\section{Relations and partitions}
A relation on a set $A$ is a subset of $A^2$. Thus for a pair
$(a_1, a_2) \in A^2$ the relation says whether $a_1$ is related to $a_2$.

An equivalence relation is a relation that is reflexive, symmetric, and
transitive, and thus makes sense as defining a partitioning of the set into
groups of equivalent elements.

The equivalence relation doesn't tell you explicitly which group a pair belongs
to (it just tells you that they are in the same group). But the information is
there: the groups are the connected components in the graph in which two
vertices are connected if they are related. There are fewer equivalence
relations than assignments to labeled buckets, since the equivalence relation
does not identify the buckets. \blue{How many equivalence relations are there,
  compared to Stirling II number and stars-and-bars count configurations?}
