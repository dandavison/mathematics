
\section{Logic}
\subsection{Propositional logic}

\begin{definition*}
  An atom is a logical proposition which cannot be decomposed. A formula is a tree in which each
  node is either
  \begin{enumerate}
  \item an atom, or
  \item a logical operator with one child node for each argument.
  \end{enumerate}
  The logical operators are:
  \begin{itemize}
  \item $\lnot p$ negation
  \item $p \lor q$ disjunction
  \item $p \land q$ conjunction
  \item $p \limplies q$ implication
  \item $p \lequiv q$ equivalence
  \end{itemize}
  An interpretation is an assignment of true/false values to the atoms in a formula (well, except
  for logical constants, those are atoms but are constant true or constant false).

  A truth table is a list of all interpretations together with the resulting values of the formula.

  A model for a formula is an interpretation (row of truth table) for which the formula evaluates
  to true.

  A formula is satisfiable if there exists a model for it.

  A formula is valid (aka a tautology) if every interpretation is a model.

  The negation of a valid formula is not satisfiable.
\end{definition*}


\begin{theorem*}
  All other logical
\end{theorem*}

\newpage
\section{Combinatorics}

\begin{mdframed}
\includegraphics[width=300pt]{img/discrete-mathematics-tucker-combinatorics-summary.png}
\end{mdframed}

\begin{theorem*}[Subtuples]~\\
  The number of $k$-tuples that can be formed from a set of size $n$ without replacement is
  \begin{align*}
    (n)_k := n \cdot (n-1) \cdots (n - k + 1) = \frac{n!}{(n-k)!}.
  \end{align*}
\end{theorem*}

\begin{remark*}
  As a special case, the number of $n$-tuples (i.e. permutations/arrangements) is $n!$. (This is also
  the number of $n-1$ tuples.)
\end{remark*}

\begin{theorem*}[Subsets]~\\
  The number of subsets of size $k$ that can be formed from a set of size $n$ is
  \begin{align*}
    C(n, k) = {n \choose k} := \frac{(n)_k}{k!} = \frac{n!}{(n-k)!~k!}.
  \end{align*}
\end{theorem*}

\begin{proof}
  Each distinct $k$-subset gives rise to $k!$ $k$-tuples by assigning position labels. Therefore
  $(n)_k = {n \choose k}k!$.
\end{proof}

\begin{theorem*}[Multiset arrangements]~\\
  Consider a multiset comprising $n$ distinct elements, with $r_i \geq 1$ repeats of the $i$-th
  element. The number of $n$-tuples that can be formed from such a multiset is
  \begin{align*}
    P(n; r_1, \ldots, r_k)
    :=& {n \choose r_1}{n - r_1 \choose r_2}\cdots{n - r_1 \cdots - r_{n-1} \choose r_n}\\
    % &=\frac{n!}{(n - r_1)!r_1!}
    %   \frac{(n - r_1)!}{(n - r_1 - r_2)!r_2!}
    %   \cdots
    %   \frac{(n - r_1 \cdots - r_{n-1})!}{r_n!}
     =& \frac{n!}{r_1!r_2!\cdots r_n!}.
  \end{align*}
\end{theorem*}

\begin{proof}
  The $r_1$ copies of the first element must all go somewhere. ${n \choose r_1}$ counts the number
  of distinct positions they can occupy. Then there are $n - r_1$ empty positions left. Etc.
\end{proof}

\begin{remark*}
  The number $n!$ of permutations of a set is a special case of this with $r_i = 1$ for all $i$.
\end{remark*}

\begin{example*}~\\
  \begin{enumerate}
  \item {\bf How many ways are there to assign 100 different diplomats to five different
      continents?}\\
    $5^{100}$
  \item {\bf How many ways if 20 diplomats must be assigned to each continent?}\\
    $P(100; 20, 20, 20, 20, 20)$. Arrange the 100 diplomats in an arbitrary order. Now we have a
    multiset of country labels with 20 repeats of each label. Given the fixed ordering of the
    diplomats, there's a one-to-one correspondence between distinct permutations of the multiset
    and assignments of diplomats to countries.
  \item {\bf How many ways are there to distribute 20 (identical) sticks of red licorice and 15
    (identical) sticks of black licorice among five children?}\\
  ${20 + 5 -1 \choose 5 - 1}$${15 + 5 -1 \choose 5 - 1}$.
  \end{enumerate}
\end{example*}

\begin{theorem*}
  How many $k$-tuples for $k \leq n$ can be formed from such a multiset?
\end{theorem*}
\red{TODO}

\begin{theorem*}[Stars and bars]~\\
  Consider the number of ways that $n$ identical objects can be put into $k$ buckets, recording
  only the counts in each bucket (not the identities of the objects).

  With no empty buckets, the answer is
  \begin{align*}
    {n - 1 \choose k - 1} ~~~~~~~\text{($k-1$ bars to be placed in $n-1$ gaps between $n$ stars)}.
  \end{align*}

  With empty buckets allowed, the answer is
  \begin{align*}
    {n + k - 1 \choose k - 1} = P(n + k - 1; n, k - 1)~~~~~~~\text{(number of arrangements of $n$ stars and $k-1$ bars)}.
  \end{align*}
\end{theorem*}

\begin{proof}
  Represent this as $n$ unlabeled stars, and $k-1$ bars representing the partition of the stars
  into different buckets.

  With no empty buckets allowed, there are $n-1$ gaps where the bars can be placed, hence
  ${n - 1 \choose k - 1}$ ways of dividing up the items.

  With empty buckets allowed, there could be multiple bars in the same position. The number of
  $(n + k - 1)$-tuples that can be formed from the star and bar symbols is
  \begin{align*}
    P(n + k - 1; n, k - 1) &= C(n + k - 1, k - 1)C(n, n)\\
                           &= C(n + k - 1, k - 1)\\
                           &= C(n + k - 1, n)C(k - 1, k - 1)\\
                           &= C(n + k - 1, n).
  \end{align*}

  Note that ${n - 1 \choose k - 1}$ for the no-empty-buckets version can also be derived as
  follows:
  \begin{enumerate}
  \item Place one item into each bucket.
  \item Now there are $n - k$ items into $k$ buckets and empty buckets are allowed for the
    subsequent allocations. So the answer is
    ${(n - k) + k - 1 \choose k - 1} = {n - 1 \choose k - 1}$ by the empty-buckets-allowed theorem.
  \end{enumerate}
\end{proof}

\begin{theorem*}[Stars and bars]~\\
  The number of ways that $n$ items can be put into $k$ buckets, with empty buckets allowed,
  recording only the counts in each bucket (not the identities of the items), is

\end{theorem*}

\begin{theorem*}[Partitions]~\\
  The number of ways that $n$ items can be put into $k$ buckets, with no empty
  buckets, recording the identities of the items in each bucket, is the number of
  \textit{partitions} of size $k$ of a set of size $n$. It is equal to the
  Stirling number of the second kind:
  \begin{align*}
    S(n, k) = \frac{1}{k!} \sum_{i=0}^k(-1)^i{k \choose i}(k-i)^n. ~~ \blue{\text{(check this)}}
  \end{align*}
\end{theorem*}

\begin{proof}
  \red{TODO}
\end{proof}

\begin{claim}
  Consider the assignment of $n$ items $x_1, x_2, \ldots, x_n$ to $k$ buckets. Define $S_i$ to be the sum of items assigned to bucket $i$. The assignments for which $\max_i S_i$ is minimized is the assignment for which $\var S_i$ is minimized.
\end{claim}

Not true?

\begin{theorem*}[Identities]~\\
  \begin{align*}
    {m + n \choose r} = \sum_{i=0}^r {m \choose i}{n \choose r - i}
  \end{align*}
\end{theorem*}

\subsection{Tucker - Applied Combinatorics - Exercises}

\begin{enumerate}
\item[(5.1)] {\bf General Counting Method for Arrangements and Selections}
  \begin{enumerate}
  \item[(37)] {\bf If three distinct dice are rolled, what is the probability that the highest
      value is
      twice the smallest value?}\\~\\
    $\frac{(3 \times 2 \times 3) + (3 \times 3!)}{6^3}$\\~\\
    An outcome is a 3-tuple such as $(1,1,1)$. Outcomes that match the criterion belong to two
    disjoint subsets:
    \begin{enumerate}
    \item Outcomes with two distinct values, such as $(1,1,2)$. There are $3 \times 2 \times 3$
      such outcomes ($3$ choices of unordered pairs of numbers, each with two alternative labelings
      and $3$ distinct permutations).
    \item Outcomes with three distinct values, such as $(2,3,4)$. There are $3 \times 3!$ such
      outcomes ($1 + 2$ unordered triples of numbers, each with $3!$ distinct permutations)
    \end{enumerate}
  \end{enumerate}
  \newpage
\item[(5.2)] {\bf Simple arrangements and selections}
  \begin{enumerate}
  \item[(Example 2)] {\bf How many ways are there to arrange the 7 letters of the word SYSTEMS...}
    \begin{enumerate}
    \item {\bf ...?}\\
      \begin{align*}
        7_{(7 - 3)} = 7\cdot 6\cdot 5\cdot 4 ~~~~~~~\text{(Choose positions of the other 4 letters, then Ss determined.)}
      \end{align*}
    \item {\bf ...with the 3 Ss consecutive?}
      \begin{align*}
        5_{(5)} = 5! ~~~~~~~\text{(Consider as 5-letter word S$^3$YTEM.)}
      \end{align*}
    \item {\bf ...with E before M?}
      \begin{align*}
        {7 \choose 2}5_{(5 - 3)} = {7 \choose 2}5\cdot 4 ~~~~~~~\text{(Choose position of E,M, then choose position of non-Ss.)}
      \end{align*}
    \item {\bf ...with E before M and 3 Ss consecutive?}
      \begin{align*}
        {5 \choose 2} 3! ~~~~~~~\text{(Consider as 5-letter word S$^3$YTEM, choose position of E,M, then choose positions for remaining letters.)}
      \end{align*}
    \end{enumerate}
  \item[(Example 6)] How
  \end{enumerate}
\end{enumerate}


\subsection{Generating functions}

\begin{definition*}[Generating function]
  Let $a_r$ be the number of ways to select $r$ objects in some counting procedure. Then $g(x)$ is
  a generating function for $a_r$ if $g(x)$ has the polynomial expansion
  \begin{align*}
    a_0 + a_1x + \ldots + a_nx^n.
  \end{align*}
\end{definition*}


\begin{example*}
  Find the generating function for $a_r$, the number of ways to select $r$ balls from $3$ green,
  $3$ white, $3$ blue, and $3$ gold balls.


\end{example*}

\newpage
\section{Pythagorean triples}

\subsection*{Project Euler question 9}

\begin{mdframed}
  A Pythagorean triplet is a set of three natural numbers, $a < b < c$, for which
  $a^2 + b^2 = c^2$.  For example, $3^2 + 4^2 = 9 + 16 = 25 = 5^2$.

  There exists exactly one Pythagorean triplet for which $a + b + c = 1000$.  Find the product
  $abc$.
\end{mdframed}

\begin{proof}~\\
  Let $m, n \in \N$.

  Recall that $|m + ni| := \sqrt{m^2 + n^2}$ and that $|wz| = |w| |z|$ for $w, z \in \C$.

  Note that $|(m + ni)^2| = |(m^2 - n^2) + 2mni| = m^2 + n^2 \in \Z$.

  Therefore $(m^2 - n^2, 2mn, m^2 + n^2)$ is a pythagorean triple for all $m, n \in
  \N$. (Claim: all pythagorean triples are of this form.)

  Therefore we seek $m, n \in \Z$ such that $m > n$ and
  \begin{align*}
    m^2 - n^2 + 2mn + m^2 + n^2             &= 1000\\
    m^2 + mn                                &= 500\\
    \(m + \frac{n}{2}\)^2 - \frac{n}{4} - 500 &= 0\\
    m                                       &= \sqrt{\frac{n}{4} + 500} - \frac{n}{2}
  \end{align*}
  Therefore (?) $\sqrt{\frac{n}{4} + 500} \in \Z$. So $\frac{n}{4} + 500 = a^2$
  for some $a \in \Z$.


\end{proof}
