\section{aio}
\begin{itemize}
\item A promise is (result, callbacks)
\item To {\it subscribe} to a promise means to add your callback (aio-listen).
\item Callbacks take one argument: the {\it value function} that will be supplied when the callback is resolved.
\item To {\it resolve} a promise means to call all the callbacks, passing them a {\it value function} (aio-resolve)
\item The simplest example of the promise lifecyle is an non-async function creating and resolving a promise:

\begin{minted}{emacs-lisp}
(setq lexical-binding t)

(defun make-call-function-promise (seconds function)
  (let ((promise (aio-promise))
        (value-function (lambda () (funcall function))))
    (aio-listen promise (lambda (value-function) (funcall value-function)))
    (prog1 promise
      (run-at-time seconds nil #'aio-resolve promise value-function))))

(make-call-function-promise 1 (lambda () (message "Hello world!")))
\end{minted}

\subsection{Chaining promises}

To chain two promises means to create an awaitable object which:
\begin{enumerate}
\item Can be kicked off
\item When resolved, will kick off the second
\end{enumerate}


\begin{minted}{emacs-lisp}
  (aio-listen promise-1 (lambda (value-function) (schedule promise-2)))
\end{minted}
