\section{Bayes' rule}

The probability of hypothesis $h$ after observing evidence $e$ is
\begin{align*}
  \p(h | e)
  = \frac{\p(h, e)}{\p(h, e) + \p(\neg h, e)}.
\end{align*}
Interpretations:
\begin{enumerate}
\item Suppose we know the joint probability distribution of $H$ and $E$. Then Bayes' rule says that the posterior
  probability of $h$ is equal to the proportion of probability mass associated with $(h, e)$ out of all the
  probability mass associated with $(*, e)$.

\item Alternatively, suppose we do not know the joint distribution directly. If we know the likelihoods
  $\p(e|h)$ and $p(e|\neg h)$ and the prior $p(h)$ then we can do the same calculation, since $\p(h, e) = \p(h)\p(e | h)$.
\end{enumerate}

\subsection{Disease testing}

Let the hypotheses be
\begin{align*}
  h      &= \text{have disease} \\
  \neg h &= \text{do not have disease}.
\end{align*}

What is the probability that you have the disease given a positive test ($e$)? We could
\begin{enumerate}
\item Estimate the true-positive $\p(e|h)$ and false-positive $\p(e|\neg h)$ rates by running the test on control samples.
\item Estimate the prior probability $\p(h)$ from the population frequency of the disease.
\end{enumerate}


\section{The Book of Why}

\subsection{Conditional probabilities and confounding}


We use an upper case letter, such as $X$, to represent a random variable, and a lower case symbol, such
as $x_1$ or $x^*$ to represent a concrete observed value.

For example, $X$ could be ``smoker? yes/no'', and $x^*$ could be ``yes''. And $U$ could be ``age'', and $u_1$
could be ``1''.

A tuple $(\ldots, U, V, W, X, Y, Z, \ldots)$ represents a single sample taken from reality of some things that
we can measure. Each slot in the tuple, such as $W$, is a random variable representing the value of some thing
we can measure.

Let $X$ be a variable we're interested in as a possible cause (e.g. smoker? yes/no), and let $Y$ be
an outcome variable of interest (e.g. symptomatic Covid-19 diagnosis? yes/no).

The traditional conditional probability $\p(Y|X=x^*)$ is the distribution of $y$ values in all tuples in
which $X=x^*$. I.e. the distribution of $y$ values in tuples of the
form $(\ldots, U, V, W, x^*, Y, Z, \ldots)$. You can estimate this conditional probability distribution by
taking a large sample of random tuples, {\it filtering} down to those in which $X=x^*$, and looking at the $Y$ values in
the remaining tuples.

But {\it filtering} possible tuples down to just those where $X=x^*$ may alter the distribution of another variable, such
as $U$, in the remaining tuples (relative to unfiltered tuples). And the value of $U$ may influence the value
of $Y$ via mechanisms that don't involve $X$. For example, $U$ could be age.

In contrast, an experimental intervention is {\it not} filtering random tuples: it is equivalent to the following procedure:
\begin{enumerate}
\item Sample a single tuple randomly from reality. Say the values you observe are $(u_3, v_{17}, w_4, x_2, y_3, z_{10})$.
\item Mutate the value in the $X$ slot so that it has the treatment value of interest. For this one sample that
  would yield $(u_3, v_{17}, w_4, x^*, y_3, z_{10})$.
- Allow the mutation to have any causal effects that it may have.
\item Repeat.
\end{enumerate}
The distribution of $Y$ values in a sample obtained in this way is what Pearl calls $\p(Y|\pdo(X = x^*))$.

In general, $\p(Y|\pdo(X = x^*)) \neq \p(Y|X=x^*)$. Variables such as $U$ that give rise to these two
distributions not being the same are known as {\it confounding variables}.

A problem is that, in this example, the experimental intervention would involve demanding that certain study
participants start smoking.



In ch 5/6 of Book of Why. why does it seem that magnitudes / strengths of causal connections didn’t come into play?






\section{Lindley - Causality review}

\subsection{2. Multivariate Distributions}

Consider 3 random variables with a joint distribution $\p(u, x, y)$. This can be factorized:

$\p(u, x, y) = \p(u)\p(x|u)\p(y|u, x)$.

A factorization requires choosing an order and corresponds to a network diagram:
\begin{tabular}{|c|c|}
 \textbf{Factorization} & \textbf{Network} \\
 \hline
 $\p(u)\p(x|u)\p(y|u, x)$ &
    \begin{tikzpicture}
      \graph {u -> {x, y}, x -> y};
    \end{tikzpicture} \\
  \hline
 $\p(x)\p(u|x)\p(y|x, u)$ &
    \begin{tikzpicture}
      \graph {u <- x, y <- {x, u}};
    \end{tikzpicture}
\end{tabular}

When a causal mechanism is adopted, some of these factorizations become inapplicable. For example, let $x$ be
foot size and $y$ be hand size, and let $u$ be a common genetic factor. Then the second factorization above is
inapplicable. The appropriate diagram is a variant of the first in which $y$ is independent of $x$,
i.e. $\p(y|u,x) = \p(y|u)$:
\begin{tikzpicture}
\graph {u -> {x, y}};
\end{tikzpicture}
\subsection{3. Causal Mechanisms}
Consider again the ordering/factorization: $\p(u, x, y) = \p(u)\p(x|u)\p(y|u, x)$
\begin{tikzpicture}
  \graph {u -> {x, y}, x -> y};
\end{tikzpicture}
What happens when $x$ is replaced by $\pdo(x)$? The factorization becomes
\begin{align*}
  \p(u, \pdo(x), y) = \p(u)\p(y|u, x).
\end{align*}
I.e.
\begin{enumerate}
\item $x$ now has no uncertainty, so $\p(x|u) = 1$
\item This doesn't affect $\p(u)$ (assumption)
\item $\p(y|u, \pdo(x)) = \p(y|u, x)$ (assumption)
\end{enumerate}
Consider $\p(y|x)$. When $x$ is random this is
\begin{align*}
  \p(y|x)
  &= \int \p(u, y|x) \du \\
  &= \frac{1}{\p(x)}\int \p(u)\p(x|u)\p(y|u, x) \du
\end{align*}
And when $x$ is selected this is
\begin{align*}
  \p(y|\pdo(x))
  &= \int \p(u, y|\pdo(x)) \du \\
  &= \int \p(u)\p(y|u, x) \du.
\end{align*}
Relatedly, if a different ordering were chosen: $\p(u, x, y) = \p(x)\p(u|x)\p(y|u, x)$ then we would have
\begin{align*}
  \p(u, \pdo(x), y) = \p(u|x)\p(y|u, x),
\end{align*}
leading to a third value for the regression of $y$ on $x$:
\begin{align*}
  \p(y|\pdo(x))
  &= \int \p(u|x)\p(y|u, x) \du.
\end{align*}
