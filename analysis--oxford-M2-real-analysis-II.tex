\documentclass[12pt]{article}
\usepackage{mathematics}

\begin{document}

\title{Oxford M2 - Real Analysis II - Continuity and Differentiability
  \footnotetext{\url{https://courses.maths.ox.ac.uk/node/37497}}} \author{Dan Davison}
\author{}
\date{}
\maketitle


\section{Misc}
\newpage
\begin{definition*}[floor and ceil]
  Let $x \in \R$. Then floor of $x$ is $\floor{x} = \max\{n \in \Z ~|~ n \leq x\}$ and ceil of $x$
  is $\ceil{x} = \min\{n \in \Z ~|~ n \geq x\}$.
\end{definition*}

\begin{theorem*}[$\floor{x}$ and $\ceil{x}$ exist]~\\
  Let $x \in \R$. Define $S = \{n \in \Z ~|~ n \geq x\} \subset \R$. Note that $S$ is bounded below
  by $x$. Also $S$ is non-empty by the Archimedean Property of $\N$, since otherwise $x$ would be
  an upper bound for $\N$. Therefore $\ceil{x} = \min S$ exists by Well-Ordering.

  Similarly, $\floor{x}$ exists.
\end{theorem*}

\begin{mdframed}
  (Question from Alex Coward)\\

  Define $f:\R\to\R$ by
  \begin{align*}
    f(x) =
    \begin{cases}
      0 ~~~~~~~~~~  & x \in (\R\setminus\Q)\cup\{0\}\\
     \frac{1}{q}    & x = \frac{p}{q} \in \Q, \gcd(p, q) = 1, q > 0.
    \end{cases}
  \end{align*}
  Where is $f$ continuous?
\end{mdframed}

\begin{definition*}[Stride]
  Let $S = \big\{\frac{1}{q} ~|~ q \in \N\big\}$. For $x = \frac{p}{q} \in \Q$, where
  $\gcd(p, q) = 1$, $q > 0$, define the stride function $s:\Q\to S$ by $x \mapsto \frac{1}{q}$. We
  say that the stride of $x$ is $s(x)$.
\end{definition*}

% \begin{intuition*}
%   At rational $x$, $f(x)$ returns the largest stride that hits $x$ when starting from zero. At
%   irrational $x$ (and zero), $f(x) = 0$.
% \end{intuition*}

\begin{claim*}
  $f$ is continuous at $0$, i.e. $\lim_{x \to 0}f(x) = 0$.
\end{claim*}

\begin{proof}
  Fix $\epsilon > 0$ and take $\delta = \epsilon$. We claim that
  $0 < |x| < \delta \implies |f(x)| < \epsilon$.

  At irrational values $0 < |x| < \delta$, we have $|f(x)| = 0 < \epsilon$.

  At rational values $0 < |x| < \delta$, we have $x = \frac{p}{q} \in \Q, \gcd(p, q) = 1$ for some
  $p \in \Z, q \in \N$ with $p, q \neq 0$. Therefore
  \begin{align*}
    |f(x)| = \Big|\frac{1}{q}\Big| \leq \Big|\frac{p}{q}\Big| = |x| < \delta = \epsilon.
  \end{align*}
\end{proof}

\begin{claim*}
  $f$ is continuous on $\R\setminus\Q$.
\end{claim*}

\begin{proof}
  Let $a \in \R\setminus\Q$ and fix $0 < \epsilon < 1$. Note that $f(a) = 0$. We now construct an
  interval centered at $a$ that excludes rational numbers with stride greater than or equal to
  $\epsilon$.

  Formally, we claim that there exists $\delta > 0$ such that
  $0 < |x - a| < \delta \implies |f(x)| < \epsilon$.

  Define $T = \{\frac{1}{n} ~|~ n \in \N, \frac{1}{n} > \epsilon\}$. Note that $T$ is non-empty
  since $\epsilon < 1$ therefore $1 \in T$. Since $T$ is non-empty, finite, and bounded below,
  $\min T$ exists. Let $\frac{1}{m} = \min T$.

  Let $K = \{k \in \N ~|~ k < m\}$, so that $\frac{1}{k} \geq \epsilon$ for all $k \in K$.

  For each $k \in K$ define $U_k$ by $U_k = \{\frac{n}{k} ~|~ \frac{n}{k} < a, n \in \N\}$. Note
  that $\max U_k$ exists and is equal to $\frac{\floor{ak}}{k}$ (\red{Proof needed?}). Let
  $u^* = \max ~ \{\max U_K ~|~ k \in K\}$.

  Now perform the analogous procedure for values greater than $a$. For each $k \in K$ define $V_k$
  by $V_k = \{\frac{n}{k} ~|~ \frac{n}{k} > a, n \in \N\}$. Note that $\min V_k$ exists and is
  equal to $\frac{\ceil{ak}}{k}$. Let $v^* = \min ~ \{\min V_K ~|~ k \in K\}$.

  Set $\delta = \min(|a - u^*|, |a - v^*|)$. Then
  $0 < |x - a| < \delta \implies |f(x)| < \epsilon$.
\end{proof}

\begin{claim*}
  $f$ is not continuous on $\Q$.
\end{claim*}

\begin{proof}
  Let $a \in \Q\setminus\{0\}$. Then $f(a) = s(a)$, the stride of $a$. Suppose for a contradiction
  that $f$ is continuous at $a$. Let $\epsilon = s(a)/2$. Then there exists $\delta > 0$ such that
  $|x - a| < \delta \implies |f(x) - s(a)| < \epsilon$. Note that there exists
  $b \in (a, a + \delta) \cap \R\setminus\Q$, since there is an irrational number between any two
  real numbers. But $|f(b) - s(a)| = s(a) > \epsilon$, a contradiction. Therefore $f$ is not
  continuous at $a$.
\end{proof}

(I thought I would need the following as a lemma to prove that minima of $U_k$ and $V_k$ above
exist, but I have not used it.)
\begin{theorem*}[well-order isomorphism]
  Let $S$ be a set with a well-ordering. I.e. every non-empty subset of $S$ has a least
  element. Let $T$ be a set and let $f:S \to T$ be a bijection with the property that
  \begin{align*}
    s_1 \leq s_2 \iff f(s_1) \leq f(s_2).
  \end{align*}
  Then $<$ is a well-ordering on $T$.
\end{theorem*}
\begin{proof}
  Let $\emptyset \neq U \subseteq T$. We claim that $\min T$ exists and is given by
  $\min T = f(\min S)$. Indeed suppose that there exists $t \in T$ such that $t < f(\min S)$. But
  then $f^\1(t) < \min S$, a contradiction. Therefore $\min T = f(\min S)$ as claimed.
\end{proof}


\end{document}