\documentclass[12pt]{article}
\usepackage{mathematics}

\begin{document}

Statistics and Machine Learning have two objectives:

\begin{enumerate}
\item Make a statement about an unobserved quantity.
\item Make a statement about the process by which an observed quantity came into existence.
\end{enumerate}


\subsection*{The view from Machine Learning}

\subsubsection*{One-dimensional data}

\subsubsection*{Regression}
We make $N$ observations. Each observation consists of an input $x$ and an output $y$. Thus the
data set is $(x_1, y_1), \ldots, (x_N, y_N)$.

We specify a model. A model is a function $M(x, \theta) \mapsto \hat y$. It takes in an observed data value $x$
and a parameter vector $\theta$, and outputs a predicted value $\hat y$.

We specify a loss function $L(\hat y, y) \mapsto \text{loss}$. If the predicted value $\hat y$ is close to the true
value $y$, then $L(\hat y, y)$ outputs a small ``loss'' value, otherwise the loss value is large.

The goal is to choose $\theta$ so that the model makes good predictions. We do this by finding


\newpage
\section*{Bias - Variance Decomposition}

Let $X$ and $y$ be data draw from some random data-generating process.

Let $\hat\theta$ be the value of the parameter vector estimated from data $X, y$.

Let $f_{\hat\theta}(z)$ be the output value predicted by the model for input value $z$ (where the
model uses the estimated parameter vector $\hat\theta$).

Notice that $X$ and $y$ are random variables, so $\hat\theta$ is also a random variable, and so,
for fixed $z$, the model prediction $f_{\hat\theta}(z)$ is also a random variable.

The loss function for linear regression is $\(f_{\hat\theta}(z) - \gamma\)^2$.

In general the loss function is $\ell(z, \gamma, X, y)$.

\begin{definition*}[True Risk]~\\
  The true risk of the model is the expected value of the loss for a new (input, output)
  observation: $\E_{X, y}$
\end{definition*}
\end{document}
