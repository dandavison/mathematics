\section{Choimet \& Queffélec: Twelve Landmarks of Twentieth-Century Analysis}

\begin{claim*}
  For real $0 \leq x < 1$ and non-negative integer $n$, the following inequality holds:
  \begin{align*}
    1 - x^n \leq n(1 - x)
  \end{align*}
\end{claim*}

\begin{proof}[Proof 1 (induction)]~\\
  It is true for $n=0$. Suppose it is true for $n=k$. Then, after multiplying
  both sides by $x$, we have
  \begin{align*}
    x(1 - x^k)  &\leq kx(1 - x).
  \end{align*}
  Multiplying out and rearranging gives
  \begin{align*}
    x - x^{k+1}         &\leq kx - kx^2\\
    1 - 1 + x - x^{k+1} &\leq kx - kx^2\\
    1 - x^{k+1}         &\leq 1 + (k - 1)x - kx^2.
  \end{align*}
  The RHS factorises as $(1 + kx)(1 - x)$ which is less than
  $(1+k)(1-x)$. Therefore
   \begin{align*}
     1 - x^{k+1} &\leq (k + 1)(1 - x),
  \end{align*}
  which proves the claim by induction.
\end{proof}

\begin{proof}[Proof 2]~\\
  The claim is true for $n=0$, so restrict attention to $n > 0$. We want to show that
  \begin{align*}
    \frac{1 - x^n}{n(1 - x)} \leq 1.
  \end{align*}
  Note that
  \begin{align*}
    (1 - x)\sum_{i=0}^{n-1}x^i &= (1 + x + x^2 + \ldots + x^{n-1}) - (x + x^2 + x^3 + \ldots + x^n)\\
                              &= 1 - x^n.
  \end{align*}
  (This can be arrived at by performing long division of $1-x$ into $1 - x^n$.)

  Therefore the ratio is
  \begin{align*}
    \frac{1 - x^n}{n(1 - x)} &= \frac{(1 - x)\sum_{i=0}^{n-1}x^i}{n(1 - x)}\\
                             &= \frac{1}{n}\sum_{i=0}^{n-1}x^i,
  \end{align*}
  which is less than 1, as required.

\end{proof}
