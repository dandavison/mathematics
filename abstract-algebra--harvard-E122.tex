\section{Exercises - Harvard E122}
Exercises from Artin *Algebra* 1st edition.

\begin{itemize}
\item [Harvard E122](http://wayback.archive-it.org/3671/20150528171650/https://www.extension.harvard.edu/open-learning-initiative/abstract-algebra)
\item [Harvard 122](http://www.math.harvard.edu/~ctm/home/text/class/harvard/122/02/html/hw.html)
\end{itemize}

\hrule

\subsection{E122 Homework 1}

** 1. Read 1.1, pp. 38-42 **

** 1.1.7 Find a formula for $\smatt{1}{1}{1}
                                   { }{1}{1}
                                   { }{ }{1}^n$ and prove it by induction. **

** 1.1.16 **

** 1.1.17 **

\hrule
\subsection{E122 Homework 2}
**Read 2.1, 2.2**

** 2.1.5 **

** 2.1.7 **

** 2.2.1 **

** 2.2.14 (122) Let $G$ be a cyclic group of order $n$, and let $r$ be an integer
dividing $n$. Prove that $G$ contains exactly one subgroup of order $r$. **

If $n$ is prime then the only subgroups of $G$ are $\{e\}$ and $G$. The
integers which divide $n$ are 1 and $n$ and so it is true that for every such
integer $r$ there is one subgroup of order $r$.

If $n$ is not prime then $G$ has non-trivial subgroups for every integer that
divides it.

For example, if $G$ has order 8, the claim is that $G$ has only one subgroup of
order 2. Well, all subgroups of order 2 are isomorphic to $\{e, \tau\}$. So the
claim implies that groups of permutations of order 4, 6, 8 etc (i.e. $S_4, S_6,
S_8$) contain only one transposition.


A solution is given in the
[Harvard 122 materials](http://www.math.harvard.edu/~ctm/home/text/class/harvard/122/02/html/home/solns/hw2.pdf).

The starting point is to note that $<g^{n/r}>$ is a subgroup of order $r$
($n/r$ fits $r$ times into each chunk of $n$ items). So in the example with $G$
of order 8, $<g^{4}>$ has order 2. The question is whether there is any other
subgroup of order 2. Intuitively it seems obvious that the answer is no, since
only 4 divides up chunks of 8 in that way.

Formally, the proof proceeds by supposing that $H' = <g^m>$ also has order $r$
(using a lemma that every subgroup of a cyclic group is cyclic). It then shows
that $m$ must divide $n$ (otherwise leads to a contradiction). Therefore the
order of $H'$ is $n/m$. Therefore $m = n/r$ which shows that $H'$ is $H$.

** 2.2.15 **

** 2.2.20(a) **


\hrule

\subsection{E122 Homework 3}

** 2.3.1 **

** 2.3.11 **

** 2.3.12 **

** 2.4.3 **

** 2.4.5 (122) **

** 2.4.6 **

** 2.4.8 (122) **

** 2.4.11 **

** 2.4.16 (122) **

** 2.4.23 (122) **


\hrule
\subsection{E122 Homework 4}
** 1. Read Artin 1.4.**

** 2. Let $V$ denote the Klein 4-group. Show that Aut($V$) is isomorphic to $S_3$.**

Every automorphism sends the identity to itself. This is because an
automorphism must preserve structure, therefore we require $\rho(e) =
\rho(ee^\1) = \rho(e)\rho(e^\1) = \rho(e)\rho(e)^\1$ which implies $\rho(e) =
e$.

Therefore, elements of Aut($V$) are distinguished by their effect on the 3
non-identity elements and there is a 1-1 correspondence $f$ between elements of
Aut($V$) and $S_3$.

We need to show that $f$ is a homomorphism, i.e. that $f(\rho_1 \rho_2) =
f(\rho_1) f(\rho_2)$. The operation is composition of permutations in both
groups...the identity seems obvious.

**3.**

Define $f : \GL_n(\R) \rightarrow \GL_n(\R)$ by $f(A) =~ ^tA^\1$ (where $^tA$
is the transpose of $A$). Show that $f$ is an automorphism, but not an inner
automorphism for $n \geq 1$.

To show that $f$ is an automorphism we need to show that it preserves structure
and is a bijection.

*Preservation of structure:* We require that $^t(AB)^\1 =~ ^tA^\1~ ^tB^\1$.
The RHS is $^tA^\1~ ^tB^\1 = (^tB^tA)^\1 = ~^t(AB)^\1$ as required.

*Bijection:* $f^\1(A) =~ ^tA$. Since an inverse mapping exists, the original
mapping must be bijective.

Finally, we need to show that $f$ is not an inner automorphism for $n \geq
1$. An inner automorphism is an automorphism defined by $\rho(A) = BAB^\1$ for
some fixed $B$. So we need to show that there is no $B$ for which $^tA^\1 =
BAB^\1$ for all $A$. $BAB^\1$ corresponds to the action of $A$ performed in a
basis defined by $B$. The determinant of $A$ is invariant under change of
basis. However, $\det ~^tA^\1 = (\det A)^\1$. Therefore $^tA^\1 = BAB^\1$ is
not true in general since it is not true when $\det A \neq \pm 1$.

http://math.stackexchange.com/questions/98378/fx-tx-1-is-an-automorphism-of-gl-n-mathbbr

I'm also watching these lectures and trying to do the homework, so by no means
an expert. In the final part of the proof, a variant would be to use
determinants:

Assume that $f$ is an inner automorphism. Therefore for some fixed $B$, $^tA^\1
= BAB^\1$ for all $A$. But this implies $(\det A)^\1 = \det A$ which is true
only for $\det A = \pm 1$. Therefore $f$ is not an inner automorphism.


**4. 1.4.5 Prove that the transpose of a permutation matrix P is its inverse.**

\subsection{E122 Homework 5}
\hrule

**2.5.1 Prove that the nonempty fibres of a map form a partition of the domain.**

We need to show

1. That every element of the domain is in some fibre, and
2. That no element is in more than one fibre

Let $f$ be a map from set $S$ to set $T$. A fibre $\phi_t$ of $f$ is the set
$\{s: f(s) = t\}$.

For any element $s$ in the domain, $f(s)$ exists and $s$ is in fibre
$\phi_{f(s)}$. This proves (1).

To prove (2), suppose that $s$ belongs to fibres $\phi_{t_1}$ and
$\phi_{t_2}$. Then $f(s) = t_1$ and $f(s) = t_2$ which is a contradiction
showing that no element belongs to two fibres.

\hrule

**2.5.6**

**(a) Prove that the relation $x$ conjugate to $y$ in a group $G$ is an
  equivalence relation on $G$.**

$x$ conjugate to $y$ means $y = gxg^{-1}$ for some $g \in G$. We need to show
that the relation is reflexive, transitive and symmetric.

*Reflexivity*: $x$ is equal to $x$ conjugated with the identity element: $x = exe^{-1}$.

*Symmetry*: Let $y = gxg^{-1}$. Then multiplying on the right by $g$ and on the
            left by $g^{-1}$ gives $g^{-1}yg = x$.

*Transitivity*: Let $y = g_1xg_1^{-1}$ and $z = g_2yg_2^{-1}$. Then $z =
                g_2(g_1xg_1^{-1})g_2^{-1} = (g_2g_1)x(g_1^{-1}g_2^{-1})$


**(b) Describe the elements $a$ whose conjugacy class (= equivalence class)
consists of the element $a$ alone.**

\hrule

**2.6.2 Prove directly that distinct cosets do not overlap.**

Let $H$ be a subgroup of $G$ and let $g_1H = \{g_1h:h\in H\}$ be a coset of $H$
in $G$. Consider an element $g_3 \in g_1H$. This means that $g_3 = g_1h$ for a
unique $h \in H$.

Suppose that $g_3$ is also in another coset $g_2H$. Then $g_3 = g_1h = g_2h'$
for $h, h' \in H$ and therefore $g_1hh'^{-1} = g_2$. But $hh'^{-1} \in H$ so
$g_2$ is in coset $g_1H$, which shows that cosets $g_1H$ and $g_2H$ are the
same.

\hrule

**2.6.4 Give an example showing that left cosets and right cosets of $\GL_2(\R)$
  in $\GL_2(\C)$ are not always equal.**

We don't even need to consider matrices outside $\GL_2(\R)$. Let A =
$\smat{1}{0}
      {0}{0}$ and let $H = \GL_2(\R)$.

Geometrically, this matrix projects points in 2D onto the x-axis.

The left coset of $\GL_2(\R)$ containing $A$ is the set of matrices of the form

$$
\smat{1}{0}
     {0}{0} \smat{a}{b}
                 {c}{d} = \smat{a}{b}
                               {0}{0}
$$

where $a,b,c,d \in \R$. Geometrically, this composition first sends the basis
vectors to new locations with x-coordinates $a$ and $b$, and then projects onto
the x axis.

The right coset of $\GL_2(\R)$ containing $A$ is the set of matrices of the form

$$
\smat{a}{b}
     {c}{d} \smat{1}{0}
                 {0}{0} = \smat{a}{0}
                               {c}{0}
$$

Geometrically, this composition first sends the $\scvec{0}{1}$ basis vector to
the origin, and then performs an arbitrary linear transformation. But once sent
to the origin, the $\scvec{0}{1}$ basis vector stays there, regardless of the
nature of the second transformation.

\hrule

**2.6.5 Let $H$, $K$ be subgroups of a group $G$ of orders 3, 5
  respectively. Prove that $H \cap K = \{e\}$.**

Since $H$ and $K$ are of prime order, they are isomorphic to the cyclic groups
of orders 3 and 5 respectively. Therefore the non-identity elements of $H$ are
of order 3, and the non-identity elements of $K$ are of order 5 (because the
order of an element in a group must divide the order of the group). Therefore,
while $e$ is in $H \cap K$, no other element is.

\hrule


** 2.6.10 (122) **

** 2.6.11 (122) **

** 2.8.2 (122) **

** 2.8.10 (122) **


\subsection{E122 Homework 6}

\hrule
** 2.9.2 2. **

** (a) Prove that the square $a^2$ of an integer $a$ is congruent to 0 or 1 modulo 4. **

Suppose $a$ is even, so $a = 2n$ for some $n \in \Z$. Then $a^2 = 4n^2 \equiv
0$ (mod 4). Alternatively, suppose $a$ is odd, i.e. $a = 2n + 1$ for some $n
\in \Z$. Then $a^2 = 4n^2 + 4n +1 \equiv 1$ (mod 4).

** (b) What are the possible values of $a^2$ modulo 8? **

If $a$ is even, then $a^2 = 4n^2$. If $n^2$ is odd then $a^2 \equiv 4$ (mod 8)
and if even then $a^2 \equiv 0$ (mod 8). If $a$ is odd, then $a^2 = 4n(n+1) + 1
\equiv 1$ (mod 8). So the possible values are 0, 1, 4.


\hrule

** 2.9.4 Prove that every integer a is congruent to the sum of its decimal
digits modulo 9. **

Let $a = d_0 + d_1 10 + d_2 10^2 + \ldots$.

[*Good solution*](https://github.com/AMouri/artin-algebra/blob/3083860baf553b472495fd01ef62489db9a261ee/Chapter%202%20-%20Groups/Section%209%20-%20Modular%20Arithmetic/solutions.tex#L47):
Note that $10 \equiv 1$ (mod 9). So $a \equiv d_0 + d_1 + d_2 + \ldots$ (mod 9).

*My solution*: we require that the difference is a multiple of 9. The difference
is $\sum_{i=0}^\infty d_i 10^i - \sum_{i=0}^\infty d_i = \sum_{i=0}^\infty
d_i(10^i - 1)$...which is a multiple of 9.


\hrule

** 2.9.5 Solve the congruence $2x \equiv 5$ **

** (a) modulo 9 **

[*Good solution*](https://github.com/AMouri/artin-algebra/blob/3083860baf553b472495fd01ef62489db9a261ee/Chapter%202%20-%20Groups/Section%209%20-%20Modular%20Arithmetic/solutions.tex#L52):
$x = 2^\1 5 \equiv 5 \cdot 5 \equiv 7$ (mod 9)

The possible values of $2x$ (mod 9) are 0, 2, 4, 6, 8, 1, 3, 5, 7, 9. So 2 (mod
9) does have a multiplicative inverse (5).


*My solution*: $14 = 2 \cdot 7 \equiv 5$ (mod 9). So the solution is $x \equiv 7$ (mod 9).

** (b) modulo 6 **
[*Good solution*](https://github.com/AMouri/artin-algebra/blob/3083860baf553b472495fd01ef62489db9a261ee/Chapter%202%20-%20Groups/Section%209%20-%20Modular%20Arithmetic/solutions.tex#L54)
The possible values of $2x$ (mod 6) are 0, 2, 4. Therefore $2x \equiv 5$ has no
solution.

I.e. 2 (mod 6) has no multiplicative inverse?

*My solution*: $\bar 5$ (mod 6) $= 5 + 6\Z = 1 + 2\cdot2 + 2\cdot3\Z = 1 + 2(2 + 3\Z)$ is an
equivalence class of odd numbers. Therefore $2x \equiv 5$ (mod 6) has no
solutions.



\hrule

**2.9.8 Use Proposition (2.6) to prove the Chinese Remainder Theorem: Let $m$,
$n$, $a$, $b$ be integers, and assume that the greatest common divisor of $m$
and $n$ is 1. Then there is an integer $x$ such that $x \equiv a$ (mod $m$)
and $x \equiv b$ (mod $n)$.**

Proposition 2.6 is the theorem describing the fact that, for two integers $i$
and $j$, $\gcd(i, j)$ generates the subgroup $i\Z + j\Z$.

Here's the proof from [A Mouri's solutions](https://github.com/AMouri/artin-algebra/blob/3083860baf553b472495fd01ef62489db9a261ee/Chapter%202%20-%20Groups/Section%209%20-%20Modular%20Arithmetic/solutions.tex#L66):

\hrule

From Proposition 2.6, $1 = an + bm$. Then

$1 \equiv an + bm$ (mod $m$) $\rightarrow 1 \equiv an$ (mod $m$).

Since $\gcd(n, m) = 1$, then $n^{-1} \equiv a \mod m$. Similarly, $m^{-1}
\equiv b \mod n$.

\hrule
My attempt:

We need to show that for any values $a,b$, we can always find an integer $x$
which exceeds the previous multiple of $m$ by $a$ and exceeds the previous
multiple of $n$ by $b$.

So we want to show that the intersection of $a + m\Z$ and $b + n\Z$ is
non-empty. I.e. we need to show that we can always find integers $r, s$
satisfying $a + rm = b +sn$. That's equivalent to $rm - sn = b - a$. We know
that $m\Z + n\Z = \gcd(m, n)\Z$, and thus since $\gcd(m, n) = 1$ we know that
$m\Z + n\Z$ is all of $\Z$. Therefore the integer $b - a$ must be reachable by
taking some number $b$ of steps of length $m$ and some number $-a$ of steps of
length $n$.

\subsection{E122 Homework 7}

\hrule


** 2.10.1 (E122 \& 122) Let $G$ be the group of invertible real upper triangular
2 x 2 matrices. Determine whether or not the following conditions describe
normal subgroups $H$ of $G$. If they do, use the First Isomorphism Theorem to
identify the quotient group $G/H$. **

$G = \left\{ \mat{a}{b}
                 {0}{d} \right\}$

** (a) $a_{11} = 1$ **

$H = \left\{ \mat{1}{b}
                 {0}{d} \right\}$

This is closed, contains the identity and contains inverses, so it is a
subgroup. And closed under conjugation, so normal.

Geometrically, the elements of the subgroup $H$ are matrices representing a
stretch in the direction of the second basis vector ("Y axis") and shearing
parallel to the first basis vector ("X axis). Conjugation yields matrices
$ghg^\1$ which correspond to doing the same transformation in a different
basis. However, this class of change-of-basis operations preserves the
direction of the first basis vector. Therefore the transformation, in the new
basis, also corresponds to stretching in the direction of the second basis
vector and shearing parallel to the first basis vector, and thus the conjugated
matrix is still in the subgroup.

\begin{align*}
&\mat{e}{f}
     {0}{h} \mat{1}{b}
                {0}{d} \mat{h}{-f}
                           {0}{e} \frac{1}{eh} \\
= &\mat{e}{f}
       {0}{h} \mat{h}{-f + be}
                  {0}{de     } \frac{1}{eh} \\
= &\mat{eh}{e(-f + be) + fde}
       {0}{hde} \frac{1}{eh}
\end{align*}


The theory of quotient groups involves identifying establishing a group
structure on a set of cosets of a normal subgroup: this set of cosets is then a
quotient group $G/H$. There's a homomorphism $f:G \rightarrow G/H$ defined by
$f(g) =$ (coset containing $g$), and the multiplication law between cosets is

(coset containing $g$) x (coset containing $g'$) = (coset containing $gg'$)

The normal subgroup is the kernel of this homomorphism, since $f(1)$ = (coset
containing $1$) = $1H$.

So we have a subgroup $H$ of the group $G$ of real upper-triangular 2x2
matrices.  We've determined that the normal subgroup is $H$, a particular
subset of real upper-triangular matrices. Each coset is therefore, for some
fixed $g \in G$, the set of matrices that can be obtained by multiplying $g$ by
some $h \in H$. So there are infinitely many cosets and each has infinitely
many elements. Each coset contains matrices of the form

\begin{align*}
\mat{1}{b}
    {0}{d} \mat{e}{f}
               {0}{h} = \mat{e}{b + df}
                            {0}{dh    }.
\end{align*}

Since $b$ and $d$ can be any real numbers, we could write this as

\begin{align*}
\mat{1}{*}
    {0}{*} \mat{a}{b}
               {0}{d} = \mat{a}{*}
                            {0}{*},
\end{align*}

thus the cosets are distinguished by the real number in their top-left
entry. So the quotient group $G/H$ is $\left\{\smat{a}{*} {0}{*}: a \in
\R\right\}$; it is isomorphic to the positive real numbers (elements of $G$
would not be invertible if $a$ were $0$). Viewed as transformations of the
plane, the elements of $G/H$ are distinguished by the magnitude of their
stretch in the x-direction.

\hrule
** (b) $a_{12} = 0$ **

$H = \left\{ \mat{a}{0}
                 {0}{d} \right\}$

This is closed, contains identity and inverses, hence is a subgroup. However
again, it is not closed under conjugation, so not normal.

Geometrically, the matrix represents a stretch in orthogonal x- and
y-directions. The non-normality is because the matrix that performs the
transformation in the changed basis does not stretch in the directions of the
basis vectors in the original basis.

\begin{align*}
&\mat{e}{f}
     {0}{h} \mat{a}{0}
                {0}{d} \mat{h}{-f}
                           {0}{e } \frac{1}{eh} \\
= &\mat{e}{f}
       {0}{h} \mat{ah}{-af}
                  {0 }{de } \frac{1}{eh} \\
= &\mat{eah}{-eaf + fde}
       {0  }{hde       } \frac{1}{eh}
\end{align*}

\hrule
** (c) $a_{11} = a_{22}$ **

$H = \left\{ \mat{a}{b}
                 {0}{a} \right\}$

Closed and contains identity and inverses, so a subgroup, but again not closed
under conjugation.

Geometrically, the matrix represents a stretch by the same amount in orthogonal
x- and y-directions, plus a shear. The non-normality is because the matrix that
performs the transformation in the changed basis does not stretch equally in
the directions of the basis vectors in the original basis.

\begin{align*}
&\mat{e}{f}
     {0}{h} \mat{a}{b}
                {0}{d} \mat{h}{-f}
                           {0}{e } \frac{1}{eh} \\
= &\mat{e}{f}
       {0}{h} \mat{ah}{-af + be}
                  {0 }{de      } \frac{1}{eh} \\
= &\mat{eah}{e(-af + be) + fde}
       {0  }{hde              } \frac{1}{eh}
\end{align*}


\hrule
** (d) $a_{11} = a_{22} = 1$ **

$H = \left\{ \mat{1}{b}
                 {0}{1} \right\}$

Closed, contains identity and inverses. It performs a shear, with no
stretching. Closed under conjugation and therefore normal.

Geometrically, the matrix represents a shear, with no stretching. The normality
is because the matrix that performs that transformation in the changed basis
also performs a shear without stretching in the original basis.


\begin{align*}
&\mat{e}{f}
     {0}{h} \mat{1}{b}
                {0}{1} \mat{h}{-f}
                           {0}{e} \frac{1}{eh} \\
= &\mat{e}{f}
       {0}{h} \mat{h}{-f + be}
                  {0}{e      } \frac{1}{eh} \\
= &\mat{eh}{e(-f + be) + fe}
       {0  }{eh             } \frac{1}{eh}
\end{align*}

Each coset contains matrices of the form

\begin{align*}
\mat{1}{b}
    {0}{1} \mat{e}{f}
               {0}{h} = \mat{e}{f + bh}
                            {0}{h     }.
\end{align*}

which could be written as

\begin{align*}
\mat{1}{*}
    {0}{1} \mat{e}{f}
               {0}{h} = \mat{e}{*}
                            {0}{h}.
\end{align*}

The elements of the quotient group $G/H$ are thus distinguished by their
diagonal entries. I.e., $G/H$ is isomorphic to $\R^2$.


\hrule

** 2.10.3 Let $P$ be a partition of a group $G$ with the property that for any
pair of elements $A, B$ of the partition, the product set $AB$ is contained
entirely within another element $C$ of the partition. Let $N$ be the element of
$P$ which contains 1. Prove that $N$ is a normal subgroup of $G$ and that $P$
is the set of its cosets. **

** Prove that $N$ is a normal subgroup of $G$: **

We know that $1 = 1^\1$ is in $N$. Therefore we know that the product set $NN$
must lie entirely within $N$, since one example of a product
(i.e. $1\cdot1^\1$) is in $N$. I.e. $N$ is closed. Does it contain inverses?
Consider an arbitrary element $n$ of $N$. Let $n^\1$ be in some element $A \in
P$ of the partition. Then the product set $NA$ must lie entirely within $N$,
since $nn^\1$ does. So either $A = N$, or $N$ and $A$ are distinct elements of
the partition but with the property that $NN = NA$. I'm going to say that $NN =
NA$ implies $N = A$ by the definition of the problem, and thus that $N$
contains inverses, but I may be missing something here.

So, $N$ contains the identity, is closed and contains inverses. It inherits
associativity from $G$. So $N$ is a subgroup. Is it normal? Consider $gNg^\1$
for some fixed $g$. Since $1 \in N$, we conclude that $gN$ is a subset of the
element $A$ of the partition that contains $g$. And since $g \in gN$,
$(gN)g^\1$ must be a subset of $N$. Therefore $N$ is normal.

** Prove that $P$ is the set of its cosets: **

The set of cosets of $N$ are $\{gN: g \in G\}$ by definition.

We need to show that exactly one partition satisfying the definition of $P$
exists, and that it is the set of cosets of $N$.

First, we show that the set of cosets satisfy the definition of $P$. Consider
two cosets $gN$ and $hN$. Their product set is $(gN)(hN)$. We know that $g \in
gN$ and $h \in hN$ so we conclude that $(gN)(hN)$ is a subset of the element of
the partition that contains $gh$.

So the set of cosets satisfy the definition of $P$, showing that at least one
such partition exists. Now we show that any such partition must be the set of
cosets. Consider subsets $A, B$ of $G$ that satisfy the definition of $P$. We
need to show that $A$ and $B$ are cosets of $N$. Fix an arbitrary element $a$
of $A$. Since $1 \in N$ and $a1 \in A$ we conclude that the product set $AN$ is
equal to $A$ and therefore that $A$ is equal to the coset $aN$. Similarly, $B$
is a coset of $N$.

\hrule

** 2.10.5 Identify the quotient group $\R^\times/P$, where $P$ denotes the subgroup of
positive real numbers. **

The cosets of $P$ are the positive and negative reals, $\{P, -1P\}$. We define
composition of cosets to be $(xP)(yP) = (xy)P$. The group operation is
commutative, therefore $P$ is normal and the operation on cosets is
well-defined. Therefore the quotient group is isomorphic to the group of size
2: $P$ is the identity and $(-1P)(-1P) = P$.

\hrule

** 2.10.6 Let $H = \{\pm 1, \pm i\}$ be the subgroup of $G = \Cx$ of fourth roots
of unity. Describe the cosets of $H$ in $G$ explicitly, and prove that $G/H$ is
isomorphic to $G$. **

The cosets of $H$ in $G$ are sets of the form $zH = \{\pm z, \pm zi\}$, where $z \in
\Cx$. Geometrically, they are the four points of a cross in the complex
plane; a rotated and scaled version of the unit cross corresponding to $\{\pm 1,
\pm i\}$.

If $z \neq z'$ then $zH \neq z'H$. So there is a bijection between $G/H$ and
$G$. I.e. $G/H$ is isomorphic to $G$.

\hrule

** 2.10.10 (122) Describe the quotient groups $\Cx / P$ and $\Cx /
U$, where $U$ is the subgroup of complex numbers of absolute value $1$ and $P$
denotes the positive reals. **

** $\Cx / P$ **

The cosets of $P$ are the set of radial lines emanating from the origin in the
complex plane. So a single coset is $e^{i\theta}P = \{pe^{i\theta} | p \in P\}$
and there is one coset for every value of $\theta \in [0, 2\pi)$. We define the
following operation on the set of cosets:
$(e^{i\theta_1}P)(e^{i\theta_2}P) = e^{i(\theta_1 + \theta_2)}P$. The group operation is commutative, therefore $P$ is normal and the operation is well-defined.

So the quotient group is isomorphic to the group of angles $[0, 2\pi)$ under addition, and to $U$. This is the "circle group" $T$.


** $\Cx / U$ **

The cosets of $U$ are concentric circles in the complex plane. So a single coset is $re^{i\phi}U = rU = \{re^{i\theta}|0\leq\theta<2\pi\}$, and there is one coset for every $r \in P$. We define the
following operation on the set of cosets: $(r_1U)(r_2U) = r_1r_2U$. The group operation is commutative, therefore $U$ is normal and the operation is well-defined.

So the quotient group is isomorphic to $P$, the positive reals under multiplication.

\subsection{E122 Homework 8}


\hrule

** 3.1.1 Which of the following subsets of the vector space of real $n \times
n$ matrices is a subspace?**

** (a) symmetric matrices ($A = A^t$) **

This is a subspace: it's closed under addition, contains the identity (zero
matrix), and contains additive inverses, and is closed under scalar
multiplication.

** (b) invertible matrices **

This is not a subspace, since the additive identity (zero matrix) is not an
invertible matrix. (It's closed under scalar multiplication but I'm not sure if
it contains additive inverses.)

** (c) upper-triangular matrices **

This is a subspace, since the additive identity (zero matrix) is upper
triangular, and it is closed under scalar multiplication and contains additive
inverses.

\hrule

** 3.1.5 Prove that the classification of subspaces of $\R^3$ stated after
(1.2) is complete **

> The subspaces of $\R^3$ are of four types
>
> (i) The zero vector
>
> (ii) Vectors lying in a line through the origin
>
> (iii) Vectors lying in a plane through the origin
>
> (iv) The whole space $\R^3$

Subspaces must contain the identity. So $\{0\}$ is one subspace. Now consider a
subspace that contains $x \neq 0$ in addition to $0$. In order to be closed
under scalar multiplication the subspace must also contain all vectors $x' =
cx$ for every $c \in \R$. So that's a line through the origin and $x$. One such
subspace exists for every distinct direction that a line through the origin can
take. Next planes...


** 3.2.1 Prove that the set of numbers of the form $a + b \sqrt 2$, where $a,
b$ are rational numbers, is a field. **

A field is by definition a set with an addition and a multiplication
operation. For each operation there must be associativity, closure,
commutativity, identity, inverses. A distributive law must hold: $(a + b)c =
ac + bc$.

More briefly, for a set $\F$ to be a field, there must be an addition law that
turns $\F$ into an Abelian group (identity $0$), and there must be a
multiplication law that turns $\F - \{0\}$ into an Abelian group (identity
$1$). Distributivity must hold: $(a + b)c = ac + bc$.

Take the **addition** operation first. Closure, commutativity and associativity
obviously hold. The identity is $0 = 0 + 0\sqrt 2$. The inverse is $(a + b
\sqrt 2)^\1 = (-a + -b \sqrt 2)$ (this is unique since the product of a
rational and irrational is irrational).

Now **multiplication**.

Closure and commutativity: $(a + b\sqrt 2)(c + d\sqrt 2) = (c + d\sqrt 2)(a +
b\sqrt 2) = (ac + 2bd) + (cb + ad)\sqrt 2$.

Associativity:
\begin{align*}
\((a + b\sqrt 2)(c + d\sqrt 2)\)(e + f\sqrt 2) = \\
\((ac + 2bd) + (cb + ad)\sqrt 2\)(e + f\sqrt 2) = \\
\((ac + 2bd)e + 2(cb + ad)f\) + \((cb + ad)e + (ac + 2bd)f\)\sqrt 2 = \\
\(ace + 2adf + 2bcf + 2bde\) + \(acf + ade + bce + 2bdf\)\sqrt 2 = \\
\end{align*}

\begin{align*}
(a + b\sqrt 2)\((c + d\sqrt 2)(e + f\sqrt 2)\) = \\
(a + b\sqrt 2)\((ce + 2df) + (cf + de)\sqrt 2\) = \\
\(a(ce + 2df) + 2b(cf + de)\) + \(a(cf + de) + b(ce + 2df)\)\sqrt 2 = \\
\(ace + 2adf + 2bcf + 2bde\) + \(acf + ade + bce + 2bdf\)\sqrt 2
\end{align*}

Identity is $1 + 0\sqrt 2 = 1$.

Inverses: $(a + b\sqrt 2)(a - b\sqrt 2) = (a^2 - 2b^2) \implies (a + b\sqrt
2)^\1 = \frac{1}{(a^2 - 2b^2)}(a - b\sqrt 2)$

Distributivity: ...bored now.

** 3.2.7 Define homomorphism of fields, and prove that every homomorphism of
fields is injective. **

A homomorphism between fields $F$ and $G$ is a function $f: F \rightarrow G$
such that $f(a+b) = f(a) + f(b)$ and $f(ab) = f(a)f(b)$.

Suppose that $f$ is not injective. Then $f(a) = f(b)$ for some $a \neq b$. So
$f(a-b) = f(a) + -f(b) = 0$. Then $f((a-b)(a-b)^\1) = f(a-b)f((a-b)^\1) = 0$
since the product of $0$ with anything is $0$; and yet $f((a-b)(a-b)^\1) = f(1)
= 1$. This is a contradiction which proves that homomorphisms must be
injective.

(So what about linear maps with non-empty nullspace / non full rank? Presumably
they are not homomorphisms?)

** 3.2.15 **

** (a) By pairing elements with their inverses, prove that the product of all
nonzero elements of $\F_p$ is $-1$. **

\begin{align*}
\prod aa^\1 bb^\1 ...
\end{align*}

** (b) Let $p$ be a prime integer. Prove Wilson's Theorem: $(p- 1)! \equiv -1$ (mod $p$). **
