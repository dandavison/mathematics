\documentclass[12pt]{article}
\usepackage{mathematics}
\newcommand{\op}{\otimes}
\newcommand{\id}{\text{id}}

\title{Structure-preserving maps}
\date{}

\begin{document}

\maketitle

\begin{definition*}
  Let $S$ be a set equipped with a binary operation $\op$. Then $f:S \to S$ is a
  structure-preserving map if $f(a \op b) = f(a) \op f(b)$.
\end{definition*}

\begin{examples*}\hspace{0pt}
  \begin{enumerate}

  \item The identity map is always structure-preserving since
    $\id(a \op b) = a \op b = \id(a) \op \id(b)$.

  \item Consider the reals under addition and let $f(x) = 2x$. Then
    $f(a + b) = 2(a + b) = f(a) + f(b)$.

  \item Let $V$ be a vector space and let $f:V \to V$ be a linear map. Then $f$ preserves additive
    structure since $f(u + v) = f(u) + f(v)$ for all $u, v \in V$.

  \item {\bf Non-example} (not a linear map): Consider the reals under addition and let $f(x) = x + 1$. Then $f(a + b) = a + b + 1 \neq (a + 1) + (b + 1)$.

  \end{enumerate}
\end{examples*}

\begin{claim*}
  Let $V$ be a vector space and let $f: V \to V$. Then $f$ preserves additive structure if and only
  if $f$ is a linear map.
\end{claim*}

\begin{proof}
  Let $V$ be a vector space and let $f:V \to V$.

  Clearly if $f$ is a linear map then it preserves additive structure.


  Suppose $f$ preserves the additive structure, i.e.  $f(u + v) = f(u) + f(v)$ for all
  $u, v \in V$.

  Then for $a \in \Z$ we have $f(au) = f(u + u + \ldots + u) = af(u)$.

  Also $f(0) = 0$ since $f(0) = f(0 + 0) = f(0) + f(0) = 2f(0)$.

  But what about $a \in \R \setminus \Z$?
\end{proof}


\end{document}