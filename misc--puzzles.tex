\documentclass[12pt]{article}
\usepackage{mathematics}

\begin{document}

\section*{Tower struts}

\begin{mdframed}
  A tower of base width $w$ and height $h$ is composed of two sides inclined towards each other at
  the same angle $\beta$. A series of diagonal struts at angle $\alpha < \beta$ join the two
  sides. The two sides of the tower do not meet at the top.

  \includegraphics[width=200pt]{img/puzzles-tower-struts.png}

  There are $n$ diagonal struts, with the first strut starting at height 0 and the last strut
  ending at height $h$.

  $w, h, \beta$ and $n$ are given. What must $\alpha$ be?
\end{mdframed}
Note that the width of the base of the second storey is $w_2 = w - 2pw = (1 - 2p)w$.

Similarly, $w_3 = (1 - 2p)w_2 = (1 - 2p)^2w$.

So the final width is $w_{n+1} = (1 - 2p)^nw$.


Define the $n$-th storey to be the section of the tower containing the $n$-th strut.

Let $p = \frac{\tan\beta}{\tan\alpha + \tan\beta}$ be the ratio of the two angles.

The first strut intersects the right side at a horizontal distance $x = pw$ from the bottom right
hand corner (this can be proved by solving the equation $x\tan\beta = (w - x)\tan\alpha$, which
describes the height of intersection of the first strut with the right hand side).

Therefore the height of the first storey is $pw\tan\beta$.

Note that the width of the base of the second storey is $w - 2pw = (1 - 2p)w$.

Therefore the height of the second storey is $p(1 - 2p)w\tan\beta$, and the height of the $i$-th
storey is $p(1 - 2p)^{i-1}w\tan\beta$.

\newpage
% Therefore we need to find the $\alpha$ that solves $h = pw\tan\beta\sum_{i=0}^{n-1}(1 - 2p)^i$.

\begin{tabular}{l|l}
  $n$&$\frac{h}{w\tan\beta}$\\
  \hline\\
  $1$& $p$\\
  $2$& $p - 2p^2$\\
  $3$& $2p - 6p^2 + 4p^3$\\
  $4$& $3p - 12p^2 + 16p^3 - 8p^4$\\ \\
  $n$& $?$
\end{tabular}


$n = 1$
\begin{align*}
  \frac{h}{w\tan\beta}
  &= p
\end{align*}

$n = 2$
\begin{align*}
  \frac{h}{w\tan\beta}
  &= p(1 - 2p)\\
        &= p - 2p^2
\end{align*}

$n = 3$
\begin{align*}
  \frac{h}{w\tan\beta}
  &= p\Big((1 - 2p) + (1 - 2p)^2\Big)\\
  &= p(1 - 2p)(2 - 2p)\\
  &= (p - 2p^2)(2 - 2p)\\
  &= 2p - 6p^2 + 4p^3
\end{align*}

$n = 4$
\begin{align*}
  \frac{h}{w\tan\beta}
  &= p\Big((1 - 2p) + (1 - 2p)^2 + (1 - 2p)^3\Big)\\
  &= p\Big((1 - 2p)(2 - 2p) + (1 - 2p)^3\Big)\\
  &= p(1 - 2p)((2 - 2p) + (1 - 2p)^2)\\
  &= p(1 - 2p + 1 - 4p + 4p^2 + (1 - 2p)(1 - 4p + 4p^2))\\
  &= p(1 - 2p + 1 - 4p + 4p^2 + 1 - 4p + 4p^2 - 2p + 8p^2 - 8p^3)\\
  &= 3p - 12p^2 + 16p^3 - 8p^4\\
\end{align*}



\section{n-omino}
\begin{mdframed}
  You have a $n$-omino. (Like a tetris piece but with arbitrarily many pieces.)
  Now pick two points with uniform distribution on that piece. Prove that the
  probability that the line between them is contained on the piece is of the
  form $p - q\ln(r)$ where $p,q$ and $r$ are rational.
\end{mdframed}
~\\

Assumptions:
\begin{enumerate}
\item A random $n$-omino is generated as follows: place a square tile at an
  arbitrary location in a plane; while the number of tiles is less than $n$,
  choose an edge uniformly from among the available edges and place a tile
  adjoining that edge.
\end{enumerate}

Define an ``internal'' line to be a line connecting two points on an $n$-omino
that is contained in the $n$-omino.

Let $\omega$ denote the desired probability that a line between two points
chosen uniformly on an $n$-omino is internal.

For $n=1$ and $n=2$ we have $\omega = 1$, since the only possible
configurations are rectangles, and these are convex polygons.

For $n=3$ the configuration is a rectangle with probability $\frac{1}{3}$ and
an L-shape with probability $\frac{2}{3}$. In the former case the line is
always internal. In the latter case, if the first point is in the corner
square, then the line is always internal; otherwise, the probability that the
line is internal is equal to the
\begin{align*}
  \omega = \frac{1}{3}\cdot 1 + \frac{2}{3} \Big(\text{random area enclosed by line through center}\Big)
\end{align*}

\section{IMO}

\newpage
\begin{mdframed}
\includegraphics[width=400pt]{img/puzzles-imo-2017-5.png}
\end{mdframed}

\begin{enumerate}
\item Reverse direction: starting from a row of $2N$ satisfying the conditions, show that it is
  possible to generate an arbitrary superrow of $N(N+1)$ by inserting players?
\item Characterize the starting row as a permutation of $1, \ldots, N(N+1)$?

  Let $S_n$ be the set of permutations of $n$ objects.

  Let the starting row be $r \in S_{N(N+1)}$.

  Let $R_n \subset S_n$ be the subset of permutations of $n$ objects that satisfy the conditions.

  \begin{claim*}
    For all $\sigma \in S_{N(N+1)}$ there exists $\rho \in R_{2N}$ such that $\rho$ can be
    transformed into $\sigma$ via insertions.
  \end{claim*}

\item Formalize using a tuple of binary indicator variables $(e_1, e_2, \ldots, e_{N(N+1)})$, where
  $e_i \in \{0, 1\}$. Let $R_{N(N+1)}$ be the set of all possible starting rows and let $E_{N(N+1)}$
    be the set of all indicator tuples.

  Define the removal function
  $$f:R_{N(N+1)} \times E_{N(N+1)} \to R_{2N}$$
  and show for all $r \in R_{N(N+1)}$ there exists $e \in E_{N(N+1)}$ such that $f(r, e)$ satisfies
  the conditions?
\item Clearly the final number of players must be even, but what is the significance of the number
  $N(N+1)$?

\begin{verbatim}
|    |    <1 |     1 |     2 |     3 |     4 |
|----+-------+-------+-------+-------+-------|
| <1 | 12345 | 21345 | 23145 | 23415 | 23451 |
|  1 | 21345 | 12345 | 13245 | 13425 | 13452 |
|  2 | 31245 | 13245 | 12345 | 12435 | 12453 |
|  3 | 41235 | 14235 | 12435 | 12345 | 12354 |
|  4 | 51234 | 15234 | 12534 | 12354 | 12345 |
\end{verbatim}
\end{enumerate}

\newpage
\begin{mdframed}
   I pick ten points in the plane. I have ten coins.

   Show that I can cover all the points with the ten coins, with none of the coins overlapping.
\end{mdframed}

\begin{proof}
  Show (by identifying tiling region and using trigometry) that maximal packing of circles in plane
  covers just over $90\%$ of the plane.

  Therefore the expected number of points covered by exceeds 9.

  Therefore it is sometimes 10.
\end{proof}

\newpage
\section{Prisoners}
\begin{mdframed}
  There are 100 prisoners numbered 1 to 100. In a room there are 100 closed boxes. In each box
  there is a number between 1 and 100. Every box contains a different number. The prisoners enter
  the room one at a time. Once inside the room, the prisoner opens 50 boxes. If all prisoners
  encounter their own number then they all go free; otherwise they remain prisoners.
\end{mdframed}

Let $b_{ij}$ be the number of the $j$-th box chosen by prisoner $i$.

Represent the 50 boxes as 50 bits.

Thus $b_{ij} \in \{2^0, 2^1, 2^2, \ldots, 2^{99}\}$.

A selection of 50 boxes corresponds to an integer in $\{0, 1, 2, 3, \ldots, 2^{100} - 1\}$ which
has 50 bits set.

A selection $s$ contains box number $i$ if the binary representation of $s$ has the $i$-th bit set.

The prisoners are freed if and only if $s_1$

Does an optimal strategy have the property that every box is looked at by exactly 50 prisoners?

Divide prisoners into an odd and an even group. Then probability of being freed is $(1/2)^{50}$,
the same as each prisoner choosing a subset of boxes independently from all others.

\newpage
\section{Gas stations}

\begin{mdframed}
  There are $n$ gas stations at points $x_1, \ldots, x_n \in [0, 2\pi)$ of a circle.

  $2\pi$ units of gas are required to drive around the circle.

  The $i$-th gas station has $g_i$ units of gas, where $\sumin g_i = 2\pi$.

  Prove that there exists a gas station at which you can start, and make it around the circle
  without running out of gas.
\end{mdframed}

We assume, without loss of generality, that the $x_i$ are sorted in ascending order.

Any valid $\vec x$, $\vec g$ can be constructed as described below.

Make a cut in the circle at points $0, g_1, g_1 + g_2, \ldots, \sum_i^{n-1}g_i$.

The arcs thus defined represent the $g_i$, and their left endpoints represent the $x_i$.

Note that in this initial configuration the claim is true, since the arcs lie end-to-end with no
overlaps and cover the entire circle.

Take the left endpoint of the $g_1$ arc and drag it around the circle anticlockwise until it lies
at $x_1$. Shift the other arcs around to remove the overlap created.

Take the left endpoint of the $g_2$ arc and drag it around the circle anticlockwise until it lies
at $x_2$.

If this overlaps with $g_1$ then set $g_1 \leftarrow x_2, g_2 \leftarrow g_2 + (g_1 - x_2)$.

If this does not overlap with $g_1$... no not sure this leads to a proof.

\end{document}
