\documentclass[12pt]{article}
\usepackage{notes}
\usepackage{mdframed}

\begin{document}

Differential calculus is a way to compute quantities related to functions by
treating the smooth curve or surface of function output values as being
comprised of many local linear functions. Each linear approximation applies
over a tiny (arbitrarily small) local interval; the linear approximation in the
next interval will in general have a slightly different gradient.

A central concept in differential calculus is the \textit{differential}: the
change in output value caused by a small change in the input value, at some
starting input value. This describes the way in which the function output
changes in response to changes in input. Differentials are often used to
compute a \textit{derivative}: the ratio of change in output value to the
change in some input value. Derivatives define a local \textit{linear
  approximation} to the function: over a small local region we consider the
real function to be approximated by a line with gradient equal to the
derivative at that point.

The above is differential calculus. Integral calculus is concerned with
``summing'' the output values of a function associated with some region in the
input space. In the familiar case, the input space is a section of the real
number line, and the output values are also real numbers. So ``summing'' the
output values corresponds to calculating the area under a curve (i.e. under the
graph of the function).

Now allow the input space to be a higher dimensional Euclidean space, e.g. some
region of the plane $\R^2$, but keep the output values as being simply real
numbers. One question is what is the value of the integral along some
1-dimensional \textit{path} through the input space. We imagine dividing the
input space up into many small sections (vectors) $\Delta x_i$, as
usual. However, when computing the contribution from one such infinitisimal
section, it is not sufficient to say simply that this is $f(x_i)|\Delta
x_i|$. The reason is that the appropriate contribution might depend not only on
the position $x_i$ but on the direction of the infinitisimal displacement
vector $\Delta x_i$. Therefore, we define $\omega_{x_i}$ to be the linear
mapping that takes as input $\Delta x_i$ and outputs the ``height'' $f(x_i)$.

What does this look like in the simple case where the answer is insensitive to
the direction of the infinitisimal displacement vector $\Delta x_i$? I think
$\omega$ would depend on $|\Delta x_i|$ only and not otherwise on $\Delta x_i$?

Another question is what is the value of the integral over some higher
dimensional region of input space (e.g. a subset of the plane).

\section*{Objects}

A function is a rule associating input values from one set with output values
from another; a function is a set of (input, output) pairs in which each input
value occurs at most once.

A curve in $d$ dimensions is a set of $d$-dimensional points that form a
``connected'' 1-dimensional object. A curve can be specified as the set of
points satisfying some condition (e.g. $x^2 + y^2 = R^2$) or by specifying that
one dimension records the value of a function whose inputs are the other
dimensions.

A surface is a similar concept to a curve, but is 2-dimensional.

The dimensionality of an object is equal to the dimensionality of the ambient
space, minus the number of independent equations.

Let $f$ be $\R \to \R$. The graph of $f$ is the set of points $(x,y)$
satisfying $y = f(x)$. This defines a curve in 2D (which never ``turns back on
itself''; the tangent line to the curve is never vertical.)

\subsection*{Parametric form}

Specifying a curve as the graph of a function (i.e. specifying one coordinate
as a function of the others) is one way to specify a curve, but it is limited
to functions, and even then it may not be convenient.

An alternative is parametric form: suppose the x-coordinate is given by $f(t)$
and the y-coordinate by $g(t)$. Then the curve is the set of points
$\big(f(t), g(t)\big)$ for some range of the parameter $t$.

\subsubsection*{Area under a curve}

What is the area $A$ under the curve from $t=a$ to $t=b$? It's just
$\int_\alpha^\beta y \dx$ as usual\footnote{$(\alpha, \beta) = (f(a), f(b))$},
but how do we express this as an integral with respect to $t$?

Well, $y = g(t)$; what about $\dx$? $x = f(t)$ (displacement), therefore
$\dx = \dt f'(t)$ (velocity $\times$ time; local linear approximation). So, the
area under the curve bounded by start and end $t$-values is
$A = \int_a^b g(t) f'(t) \dt$.

Thus, if the x-coordinate is increasing rapidly with $t$, then the area is
larger.

\subsubsection*{Length of a curve}

The length of a curve is $L = \int \sqrt{\dx^2 + \dy^2}$, over some interval.

This can be expressed as an integral with respect to $x$ (non-parametric form):
$L = \int_\alpha^\beta \sqrt{1 + (\frac{\dy}{\dx})^2} \dx$.

Or it can be expressed as an integral over an interval of $t$ values (parametric form):
$L = \int_a^b \sqrt{ (\frac{\dx}{\dt})^2 + (\frac{\dy}{\dt})^2} \dt$

\subsubsection*{Area of a surface formed by revolution of a curve}

Suppose a curve is revolved around the $x$-axis. To compute the area, we divide
the surface into vertical strips\footnote{I'm a bit confused. Why exactly do we
  construct these strips using the hypotenuse, whereas when approximating the
  area under a graph we construct rectangles $y\dx$?}:
$A = \int_\alpha^\beta 2\pi y \sqrt{\dx^2 + \dy^2}$.


\subsection*{Polar coordinates}


\subsubsection*{Area of a sector bounded by a curve}

What's the area of the sector bounded by the two rays and a curve, between $\theta=a$ and $\theta=b$?

Note that the area of a sector of $\phi$ radians of a circle is $\pi r^2 \times \frac{\phi}{2\pi} = \frac{1}{2}\phi r^2$.

We're considering a curve defined by $r = f(\theta)$. We divide it up into many
sectors each with angle $\dtheta$. The area is
$\int_a^b \frac{1}{2}f(\theta)^2\dtheta$.


\newpage
\subsection*{Math 53 Midterm I Februrary 2011 Frenkel}

Consider the curve in $\R^2$ defined by the equation

$$
r = \cos(2\theta)
$$.

(a) Sketch this curve.
\begin{mdframed}
\end{mdframed}

(b) Find the area of the region enclosed by one loop of this curve.

\begin{mdframed}
The area of a sector of $\dtheta$ radians is $\frac{\dtheta}{2\pi}\pi r^2 = \frac{\dtheta}{2}\cos^2(2\theta)$. Recall the identity $\cos^2 t = \frac{1}{2} + \frac{1}{2}\cos(2t)$. Therefore the area of one loop is

\begin{align*}
  \int_{-\pi/4}^{\pi/4} \frac{\dtheta}{2}\cos^2(2\theta)
  =&\frac{1}{4}\int_{-\pi/4}^{\pi/4} (1 + \cos 4\theta) \dtheta \\
  =& \Big|\frac{1}{4} + \frac{1}{16}\sin(4\theta)\Big|_{-\pi/4}^{\pi/4} \\
  =& \frac{\pi}{8}.
\end{align*}

\end{mdframed}


\end{document}