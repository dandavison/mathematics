\documentclass[12pt]{article}
\usepackage{notes}
\usepackage{mdframed}
\newcommand{\dydx}{\frac{\dy}{\dx}}

\newtheorem*{theorem}{Theorem}

\begin{document}

Differential calculus is a way to compute quantities related to functions by
treating the smooth curve or surface of function output values as being
comprised of many local linear functions. Each linear approximation applies
over a tiny (arbitrarily small) local interval; the linear approximation in the
next interval will in general have a slightly different gradient.

A central concept in differential calculus is the \textit{differential}: the
change in output value caused by a small change in the input value, at some
starting input value. This describes the way in which the function output
changes in response to changes in input. Differentials are often used to
compute a \textit{derivative}: the ratio of change in output value to the
change in some input value. Derivatives define a local \textit{linear
  approximation} to the function: over a small local region we consider the
real function to be approximated by a line with gradient equal to the
derivative at that point.

The above is differential calculus. Integral calculus is concerned with
``summing'' the output values of a function associated with some region in the
input space. In the familiar case, the input space is a section of the real
number line, and the output values are also real numbers. So ``summing'' the
output values corresponds to calculating the area under a curve (i.e. under the
graph of the function).

Now allow the input space to be a higher dimensional Euclidean space, e.g. some
region of the plane $\R^2$, but keep the output values as being simply real
numbers. One question is what is the value of the integral along some
1-dimensional \textit{path} through the input space. We imagine dividing the
input space up into many small sections (vectors) $\Delta x_i$, as
usual. However, when computing the contribution from one such infinitisimal
section, it is not sufficient to say simply that this is $f(x_i)|\Delta
x_i|$. The reason is that the appropriate contribution might depend not only on
the position $x_i$ but on the direction of the infinitisimal displacement
vector $\Delta x_i$. Therefore, we define $\omega_{x_i}$ to be the linear
mapping that takes as input $\Delta x_i$ and outputs the ``height'' $f(x_i)$.

What does this look like in the simple case where the answer is insensitive to
the direction of the infinitisimal displacement vector $\Delta x_i$? I think
$\omega$ would depend on $|\Delta x_i|$ only and not otherwise on $\Delta x_i$?

Another question is what is the value of the integral over some higher
dimensional region of input space (e.g. a subset of the plane).

\section*{Curves and surfaces}

A function is a rule associating input values from one set with output values
from another; a function is a set of (input, output) pairs in which each input
value occurs at most once.

A curve in $d$ dimensions is a set of $d$-dimensional points that form a
``connected'' 1-dimensional object.

A surface is a similar concept to a curve, but is 2-dimensional.

The dimensionality of an object is equal to the dimensionality of the ambient
space, minus the number of independent equations.

\section*{Specifying a curve or surface}

\textbf{Cartesian equation:} A curve can be specified as the set of points
satisfying some condition (e.g. $x^2 + y^2 = R^2$) or by specifying that one
dimension records the value of a function whose inputs are the other
dimensions ($z = 3 + 1.5(x-1) - 2.7(y-2)$).

\textbf{Graph:} Let $f$ be $\R \to \R$. The graph of $f$ is the set of points
$(x,y)$ satisfying $y = f(x)$. This defines a curve in 2D (which never ``turns
back on itself''; the tangent line to the curve is never vertical.)

A curve in 3D would require two equations (to reduce the dimensionality of the
ambient space to that of the object being specified; i.e. the intersection of
two surfaces). In practice, curves in 3D are usually specified in parametric
form.

\textbf{Parametric form:} For a curve in 2D, suppose the x-coordinate is given
by $f(t)$ and the y-coordinate by $g(t)$. Then the curve is the set of points
$\big(f(t), g(t)\big)$ for some range of the parameter $t$. E.g. a line
represented in parametric form using vector notation:
$\vec r = \vec r_0 + \vec v t$. (A surface would require 2 parameters, so they
are often specified using Cartesian equations.)


\section*{Area under a curve}

What is the area $A$ under the curve from $t=a$ to $t=b$? It's just
$\int_\alpha^\beta y \dx$ as usual\footnote{$(\alpha, \beta) = (f(a), f(b))$},
but how do we express this as an integral with respect to $t$?

Well, $y = g(t)$; what about $\dx$? $x = f(t)$ (displacement), therefore
$\dx = \dt f'(t)$ (velocity $\times$ time; local linear approximation). So, the
area under the curve bounded by start and end $t$-values is
$A = \int_a^b g(t) f'(t) \dt$.

Thus, if the x-coordinate is increasing rapidly with $t$, then the area is
larger.

\section*{Length of a curve}

The length of a curve is $L = \int \sqrt{\dx^2 + \dy^2}$, over some interval.

This can be expressed as an integral with respect to $x$ (non-parametric form):
$L = \int_\alpha^\beta \sqrt{1 + (\frac{\dy}{\dx})^2} \dx$.

Or it can be expressed as an integral over an interval of $t$ values (parametric form):
$L = \int_a^b \sqrt{ (\frac{\dx}{\dt})^2 + (\frac{\dy}{\dt})^2} \dt$

\section*{Area of a surface formed by revolution of a curve}

Suppose a curve is revolved around the $x$-axis. To compute the area, we divide
the surface into vertical strips\footnote{I'm a bit confused. Why exactly do we
  construct these strips using the hypotenuse, whereas when approximating the
  area under a graph we construct rectangles $y\dx$?}:
$A = \int_\alpha^\beta 2\pi y \sqrt{\dx^2 + \dy^2}$.


\section*{Polar coordinates}

E.g. the curve $r = \cos(\theta)$ is a circle of radius 1 centered at $(x, y) = (\frac{1}{2}, 0)$. (?)

\subsection*{Area of a sector bounded by a curve}

What's the area of the sector bounded by the two rays and a curve, between $\theta=a$ and $\theta=b$?

Note that the area of a sector of $\phi$ radians of a circle is $\pi r^2 \times \frac{\phi}{2\pi} = \frac{1}{2}\phi r^2$.

We're considering a curve defined by $r = f(\theta)$. We divide it up into many
sectors each with angle $\dtheta$. The area is
$\int_a^b \frac{1}{2}f(\theta)^2\dtheta$.

\section*{Surfaces}

\subsection*{Planes}
Given a normal vector $\vec n = \cveccc{d}{e}{f}$, and a point in the plane
$P = \cveccc{x_0}{y_0}{z_0}$, an equation specifying the plane is

\begin{align*}
  d(x - x_0) + e(y - y_0) + f(z - z_0) = 0 \\
  dx + ey + fz = C.
\end{align*}

So the normal vector can be read off from the equation.

Similarly the general equation of a line in 2D is

\begin{align*}
  d(x - x_0) + e(y - y_0) = 0,
\end{align*}

(TODO: explain this and other content towards end of L11)

so $\cvecc{d}{e}$ is a normal vector to the line.


\subsection*{Quadric surfaces}
Ellipsoids, hyperboloids, paraboloids. Also cylinders (one variable not
specified, e.g. $x^2 + y^2 = 1$), and cones (e.g. $z^2 = x^2 + y^2$).

\section*{Tangent spaces}

\subsection*{Tangent lines}
E.g. a tangent vector is given by differentiating the parametric equation for a
curve, giving an equation for the tangent line:

\begin{align*}
  \vec r = \vec r_0 + \cveccc{x'(t_0)}{y'(t_0)}{z'(t_0)}s = \vec r_0 + \vec v' s.
\end{align*}

\subsection*{Tangent planes}

\begin{align*}
  (z - z_o) = (x - x_0)f_x(x_0, y_0) + (y - y_0)f_y(x_0, y_0)
\end{align*}

And what's the normal vector to that tangent plane? It's
$\cveccc{f_x(x_0, y_0)}{f_y(x_0, y_0)}{-1}$.


\section*{Limits (L8)}

$\frac{x^2}{x^2 + y^2}$ has no limit at $(0, 0)$.
Easy to prove by exhibiting paths with different limits: e.g. along x-axis vs. y-axis.
Lack of limit related to degree of numerator and denominator being same.

But $\frac{2x^3}{x^2 + y^2}$ does have a limit at $(0, 0)$.

Proof: consider a disk of radius $r$. For points in this disk, $x^2 + y^2 \leq r^2$ and so $x \leq r$.
Now

\begin{align*}
  \left|\frac{2x^3}{x^2 + y^2}\right| = 2|x|\left|\frac{x^2}{x^2 + y^2}\right| \leq 2r,
\end{align*}

so for any desired closeness to the limiting value 0, we can find an $r$ that will do it.

\section*{Partial derivatives (L8)}

Clairot's theorem: equality of mixed partials under certain continuity
conditions.

\begin{quote}
``Same commutative structure as multiplication''; all that matters
is how many times you have differentiated w.r.t. $x$, and to $y$;
``differentiation is in a sense opposite to multiplication''.
\end{quote}

\section*{Differentials (L8)}

\begin{quote}
  ``The differential is the function whose graph the tangent line (plane) is,
  but with the coordinate axes shifted to the point at which it is being
  evaluated.''
\end{quote}

A differential, defined at a particular point in the input space, is the
function describing the linear approximation at that point: it maps a
displacement in the input space to a displacement in the output space.

It's the function whose graph is the tangent space at that point, in a
coordinate space shifted to have its origin at that point. So in 1D, if
$z = f(x)$, then the differential at $x_0$ is

\begin{align*}
  \dz(x) = (x - x_0)f'(x_0).
\end{align*}


Not to be confused with $\Delta f$ --- the increment in the \textit{actual
  function} value --- whereas the differential refers to the increment in the
linear approximation.


\section*{Directional derivatives (L11)}

\theorem{
  The directional derivative of $f(x, y)$ in the direction of a unit
  vector $u = \cvecc{a}{b}$ is
  \begin{align*}
    D_u f = a\partiald{f}{x} + b\partiald{f}{y} = \nabla f \cdot \vec u.
  \end{align*}
}

\proof{ Since $u$ is unit length, $\cvecc{ha}{hb}$ is a displacement of length
  $h$ in the direction of $u$. Then\footnote{The proof in the lecture and in
    Stewart is slightly different, involving defining these quantities as
    functions of $h$ and considering the derivative w.r.t. $h$.}
  \begin{align*}
    D_u f(x_0, y_0)
    :=& \lim_{h \to 0} \frac{f(x_0 + ha, y_0 + hb) - f(x_0, y_0)}{h} \\
    =& \lim_{h \to 0} \frac{f(x_0, y_0) + ha\partiald{f}{x}(x_0, y_0) + hb\partiald{f}{y}(x_0, y_0) - f(x_0, y_0)}{h} \\
    =& a\partiald{f}{x}(x_0, y_0) + b\partiald{f}{y}(x_0, y_0) \qed
  \end{align*}

}

\section*{Gradient}
$\nabla f(x_0, y_0)$ is normal to the level curve that cuts $f$ at $z = z_0$.

Recall that $\cveccc{f_x(x_0,y_0)}{f_y(x_0,y_0)}{-1}$ is a normal vector to the
tangent plane at $(x_0,y_0)$.

\newpage
\section*{Math 53 2017 Frenkel - Homework}

\subsection*{10.2.30}
Find equations of the tangents to the curve $x = 3t^2 + 1$, $y = 2t^3 + 1$,
that pass through the point $(4, 3)$.

~\\
\begin{mdframed}
  We seek points $(x_0, y_0)$ on the curve at which the tangent intersects
  $(4, 3)$. Such points satisfy both the linear tangent equation, and the
  Cartesian equation for the curve:
  \begin{align*}
    \begin{cases}
      y_0 = 4 + (x_0 - 3)\frac{\dy}{\dx}(x_0)\\
      y_0 = 2\(\frac{x_0-1}{3}\)^{3/2} + 1.
    \end{cases}
  \end{align*}

  The derivative is $\dydx = (x - 1)^{1/2}$, so $x_0$ satisfies
  \begin{align*}
    \frac{2}{3^{3/2}}(x_0 - 1)^{3/2} - (x_0 - 3)(x_0 - 1)^{1/2} - 3 = 0.
  \end{align*}

  Letting $A = x_0 - 1$,
  \begin{align*}
    \frac{2}{3^{3/2}}A^{3/2} - (A - 2)A^{1/2} - 3 = 0.
  \end{align*}
  Letting $B = A^{1/2}$,
  \begin{align*}
    \frac{2}{3^{3/2}}B^3 - (B^2 - 2)B - 3 = 0 \\
    \(\frac{2}{3^{3/2}} - 1\)B^3 + 2B - 3 = 0 \\
  \end{align*}

  But this doesn't seem to have a simple solution.

\begin{verbatim}
In [80]: sp.solve(Eq(c * B**3 + 2*B - 3, 0), B)
Out[80]:
[-(-1/2 - sqrt(3)*I/2)*(sqrt(6561/c**2 + 864/c**3)/2 - 81/(2*c))**(1/3)/3 + 2/(c*(-1/2 - sqrt(3)*I/2)*(sqrt(6561/c**2 + 864/c**3)/2 - 81/(2*c))**(1/3)),
 -(-1/2 + sqrt(3)*I/2)*(sqrt(6561/c**2 + 864/c**3)/2 - 81/(2*c))**(1/3)/3 + 2/(c*(-1/2 + sqrt(3)*I/2)*(sqrt(6561/c**2 + 864/c**3)/2 - 81/(2*c))**(1/3)),
 -(sqrt(6561/c**2 + 864/c**3)/2 - 81/(2*c))**(1/3)/3 + 2/(c*(sqrt(6561/c**2 + 864/c**3)/2 - 81/(2*c))**(1/3))]
\end{verbatim}

Alternatively, we have $\dy\dx = \frac{6t^2}{6t} = t$, and so

\begin{align*}
  3 - (2t^3 + 1) = t\(4 - (3t^2 + 1)\) \\
  t^3(-2 + 3) + t(-4 + 1) + 3 - 1= 0 \\
  t^3 -3t + 2 = 0 \\
  (t - 1)^2(t + 2)
\end{align*}

\end{mdframed}

\subsection*{10.2.32}
Find the area enclosed by the curve $x = t^2 - 2t$, $y = \sqrt{t}$, and the y-axis.

\begin{mdframed}
  The curve starts at the origin, goes up and left to a turning point then goes
  up and right to $(0, \sqrt{2})$ and continues up and right.

  The desired area can be expressed as a sum of horizontal strips:

  \begin{align*}
    \int_{t=0}^{t=2} x \dy
    &= \int_{t=0}^{t=2} \(t^2 - 2t\) \frac{1}{2}t^{-1/2} \dt \\
    &= \frac{1}{2} \int_{t=0}^{t=2} \(t^{3/2} - 2t^{1/2}\) \dt \\
    &= \frac{1}{2} \left[ \frac{2}{5}t^{5/2} - \frac{4}{3} t^{3/2} \right]_{t=0}^{t=2} \\
    &= \frac{1}{2} \(\frac{2}{5} \sqrt{32} - \frac{4}{3} \cdot 2\sqrt{2}\) \\
    &= \(\frac{12}{15} \sqrt{2} - \frac{20}{15} \sqrt{2}\) \\
    &= \frac{-8}{15} \sqrt{2}
  \end{align*}

  (Sign is wrong)

\end{mdframed}

\newpage
\section*{Math 53 Midterm I Februrary 2011 Frenkel}

1. Consider the curve in $\R^2$ defined by the equation

$$
r = \cos(2\theta)
$$.

(a) Sketch this curve.

(b) Find the area of the region enclosed by one loop of this curve.\\

\begin{mdframed}
The area of a sector of $\dtheta$ radians is $\frac{\dtheta}{2\pi}\pi r^2 = \frac{\dtheta}{2}\cos^2(2\theta)$. Recall the identity $\cos^2 t = \frac{1}{2} + \frac{1}{2}\cos(2t)$. Therefore the area of one loop is

\begin{align*}
  \int_{-\pi/4}^{\pi/4} \frac{\dtheta}{2}\cos^2(2\theta)
  =&\frac{1}{4}\int_{-\pi/4}^{\pi/4} (1 + \cos 4\theta) \dtheta \\
  =& \Big|\frac{1}{4} + \frac{1}{16}\sin(4\theta)\Big|_{-\pi/4}^{\pi/4} \\
  =& \frac{\pi}{8}.
\end{align*}

\end{mdframed}~\\

2. Find an equation of the surface consisting of all points in $\R^3$ that are equidistant
from the point $(0, 0, 1)$ and the plane $z = 2$.\\

\begin{mdframed}
Such points satisfy $\sqrt{x^2 + y^2 + (z - 1)^2} = z - 2$, which simplifies as

\begin{align*}
x^2 + y^2 + z^2 - 2z + 1 &= z^2 - 4z + 4 \\
z &= -\frac{x^2}{2} -\frac{y^2}{2} + \frac{3}{2} \\
\end{align*}

\end{mdframed}~\\

(b) Sketch this surface. What is it called?

\begin{mdframed}
Concave-down paraboloid
\end{mdframed}

3. Show that the function $\frac{x^{50}y^{50}}{x^{100} + y^{200}}$ does not
have a limit at $(x, y) = (0, 0)$.

\begin{mdframed}
  First consider approaching $(0, 0)$ along the line $x=0$: this gives a limiting value of 0.
  Now consider approaching along the line $x=y$: this gives a limiting value of $\frac{x^{100}}{x^{100} + x^{200}} \to 1$.
  Since the two approaches gives different results, the limit does not exist.
\end{mdframed}~\\

4. Consider the function $f(x,y) = x\cos(y) + y^2e^x + x$.

(a) Find the differential of this function

\begin{mdframed}
  \begin{align*}
  \d f(x, y) &= \partiald{f}{x}\d x + \partiald{f}{y}\d y\\
             &= (\cos(y) + y^2e^x + 1)\d x + (-x\sin(y) + 2ye^x) \d y
  \end{align*}
\end{mdframed}

(b) Find an equation of the tangent plane to the graph of this function at the
point $(0, \pi, \pi^2)$.

\begin{mdframed}
  Points in the plane passing through $(x_0, y_0, z_0)$ satisfy
  \begin{align*}
    z - z_0 = (x - x_0)\partiald{f}{x} + (y - y_0)\partiald{f}{y},
  \end{align*}
  so in this case
  \begin{align*}
    z - \pi^2 &= (x - 0)\(\cos(\pi) + \pi^2e^0 + 1\) + (y - \pi)\(-0\sin(\pi) + 2\pi e^0\)\\
    z &= \pi^2x + 2\pi y - \pi^2
  \end{align*}
\end{mdframed}~\\

5. Suppose we need to know an equation of the tangent plane to a surface $S$ at
the point $P = (1, 3, 2)$. We don't have an equation for $S$, but we know that the
curves

\begin{align*}
r_1(t) &= \langle 1 + 5t,    3 - t^2,       2 + t - t^3    \rangle, \\
r_2(s) &= \langle 3s - 2s^2, s + s^3 + s^4, s - s^2 + 2s^3 \rangle
\end{align*}
both lie in $S$. Find an equation of the tangent plane to $S$ at the point $P$.


~\\\begin{mdframed}
First note that $P$ lies in both curves, since $r_1(0) = r_2(1) = P$.

Now we use the two curves to obtain two vectors that lie in the tangent
plane. Their cross product is a vector normal to the tangent plane and provides
the coefficients for the equation of the plane.

The derivative of a curve $r(t)$ with respect to the parameter $t$ is a
function giving a tangent vector. The derivatives are:

\begin{align*}
  \dot r_1(t) = \cveccc{5}{-2t}{1 - 3t^2}, ~~~
  \dot r_2(s) = \cveccc{3 - 4s}{1 + 3s^2 + 4s^3}{1 - 2s + 6s^2}.
\end{align*}

Evaluated at $P = \cveccc{1}{3}{2}$ these are

\begin{align*}
  \dot r_1(t) = \cveccc{5}{0}{1}, ~~~
  \dot r_2(s) = \cveccc{-1}{8}{5}.
\end{align*}

Their cross product is

\begin{align*}
  \cveccc{5}{0}{1} \times \cveccc{-1}{8}{5} =
  \cveccc{(0 \times 5) - (1 \times 8)}
         {(1 \times -1) - (5 \times 5)}
         {(5 \times 8) - (0 \times -1)} =
  \cveccc{-8}{-26}{40}
\end{align*}


Therefore points in the plane satisfy

\begin{align*}
  -8(x - x_0) - 26(y - y_0) + 40(z - z_0) = 0 \\
  -8(x - 1)   - 26(y - 3)   + 40(z - 2) = 0.
\end{align*}

Would another way of doing this, without using the cross product, be to start
by saying that the plane is the set of points $(x, y, z)$ that satisfy

\begin{align*}
  \cveccc{x}{y}{z} = \cveccc{1}{3}{2} + u\cveccc{5}{0}{1} + v \cveccc{-1}{8}{5},
\end{align*}

and then to somehow change basis/variables from $(u, v)$ to $(x, y)$?

\end{mdframed}

\end{document}
