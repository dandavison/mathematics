\documentclass[12pt]{article}
\usepackage{enumerate}
\usepackage{mathematics}

\DeclareMathOperator{\diam}{\mathrm{diam}}

\begin{document}

\section{Math 1A Final (Adiredja)}

\subsection*{1.c}

\begin{mdframed}
In order to find the maximum, we want the derivative of
\begin{align*}
  f(x) = \int_0^x t^2 - 1 \dt.
\end{align*}
We could integrate it explicitly:
\begin{align*}
  f(x) = \left[\frac{t^3}{3} - t \right]_0^x = \frac{x^3}{3} - x.
\end{align*}
So this is a function telling us the area under the curve for a given $x$. Now
we want the maximum of this function, so we have to differentiate:
\begin{align*}
  \ddx \( \frac{x^3}{3} - x \) = x^2 - 1 = (x+1)(x-1)
\end{align*}
But, ``of course'', this was obvious from FTC.

And that is zero at -1, and 1. But at 1, it's a minimum, not a maximum.
\end{mdframed}

\newpage
\subsection*{3.b}
$\lim_{x\to 2} x^2$ is obviously 4. Prove it.\\

\begin{mdframed}
  We want a procedure that, given a distance $\epsilon$ in the output space,
  shows how to pick a distance $\delta$ in the input space. The $\delta$ must satisfy the following:

  For any $x$ within the radius of $\delta$ in the input space, $x^2$ will be
  within the radius of $\epsilon$ in the output space.

  \textbf{Guess}\\
  First we try to ``guess'' a rule giving a $\delta$ in terms of the $\epsilon$.

  If it is true that $x^2$ is within the radius of $\epsilon$, then that's the
  same as saying
  \begin{align*}
    |x^2 - 4| < \epsilon\\
    x^2 < \epsilon + 4\\
    x < \sqrt{\epsilon + 4}\\
  \end{align*}
  Now, we're trying to make a statement about the size of the window in the
  input space. The distance from the point of interest is $x - 2$, and so we
  know that
  \begin{align*}
    x - 2 < \sqrt{\epsilon + 4} - 2.
  \end{align*}
  So that suggests that the following rule will give a $\delta$ satisfying the
  requirement:

  Choose $\delta = \sqrt{\epsilon + 4} - 2$.

  \textbf{Prove}\\
  We know that $2 \pm \delta$ gets mapped onto the boundary of the output
  window (because we chose $\delta$ to have that property).

  Now, we need to prove that it is true that any $x$ within that input window
  will be mapped into the output window.

  Consider some $x$ in the input window. Where does it get mapped to? Answer:
  $x^2$. Also, we know that $x^2 < (2 + \delta)^2$ (because $2+\delta$ is at
  the outer edge of our input window). And, we've \textit{chosen} a value for
  $\delta$: it's $\sqrt{\epsilon + 4} - 2$. So we can write the following
  inequality:
  \begin{align*}
    x^2 < (2 + \sqrt{\epsilon + 4} - 2)^2
  \end{align*}
  Recall that what we're trying to show is that this $x$ (which was chosen to
  be in the \textit{interior} of our input window) gets mapped into the
  \textit{interior} of the output window.

  Simplifying our inequality:
  \begin{align*}
    x^2 < (2 + \sqrt{\epsilon + 4} - 2)^2\\
    x^2 < (\sqrt{\epsilon + 4})^2\\
    x^2 < 4 + \epsilon \qed
  \end{align*}
  And that proved it: $f(x) = x^2$ is less than $\epsilon$ from the hypothesized limit, 4.

  Well, that isn't valid. See\\
  \url{https://www.youtube.com/watch?v=gLpQgWWXgMM}\\
  for the correct proof, and also\\
  \url{https://math.stackexchange.com/questions/330297/prove-that-lim-x-to-2x2-4-using-epsilon-delta-definition}
  \url{https://math.stackexchange.com/questions/1344493/epsilon-delta-proof-of-lim-x-to-2-x2-4}
  \begin{align*}
  \end{align*}
\end{mdframed}

\newpage
\subsubsection*{b}
Using definition of definite integral (as limit of Riemann sums).

This example illustrates aspects of the Fundamental Theorem of Calculus: that
using antiderivatives to evaluate a definite integral gives the same result as
computing the limit of the Riemann sums directly.

\begin{mdframed}
  \begin{align*}
  \int_0^2 (2 - x^2) \dx
    &= \lim_{N \to \infty}\sum_{i=1}^N \frac{2}{N}\(2 - \(\frac{2i}{N}\)^2\) \\
    &= \lim_{N \to \infty}\sum_{i=1}^N \frac{4}{N} - \frac{8i^2}{N^3} \\
    &= \lim_{N \to \infty}\(4  - \frac{8}{N^3}\sum_{i=1}^Ni^2 \)\\
    &= \lim_{N \to \infty}\(4  - \frac{8}{N^3}\frac{N(N+1)(2N+1)}{6} \)\\
    &= \lim_{N \to \infty}\(4  - 8\frac{(N+1)(2N+1)}{6N^2} \)\\
    &= \lim_{N \to \infty}\(4  - 8\frac{2 + 3N^{-1} + N^{-2}}{6} \)\\
    &= 4  - \frac{8}{3} = \frac{4}{3}\\
  \end{align*}
  Alternatively,
  \begin{align*}
  \int_0^2 (2 - x^2) \dx
    &= \left[2x - \frac{x^3}{3}\right]_0^2 \\
    &= 4 - \frac{8}{3} = \frac{4}{3} \qed
  \end{align*}

\end{mdframed}




\end{document}