\documentclass[12pt]{article}
\usepackage{mathematics}

\begin{document}

\section{Fibonacci - matrix multiplication}

\begin{definition*}
  The Fibonacci sequence $0, 1, 1, 2, 3, 5, 8, ...$ is defined by
  \begin{align*}
    x_0 &= 0\\
    x_1 &= 1\\
    x_{n} &= x_{n-1} + x_{n-2}, ~~~~~~~ n \geq 2.
  \end{align*}
\end{definition*}

\begin{remark*}
  The sequence can be generated by taking $\vecMM{x_1}{x_0} = \vecMM{1}{0}$ as the initial state and multiplying
  repeatedly by $A = \matMMxNN{1}{1}
                              {1}{0}$, yielding the sequence
  \begin{align*}
    \vecMM{1}{0}, \vecMM{1}{1}, \vecMM{2}{1}, \vecMM{3}{2}, \vecMM{5}{3}, \vecMM{8}{5}, ....
  \end{align*}
  Thus $\vecMM{x_{n+1}}{x_n} = A^n\vecMM{1}{0}$.\footnote{The book describes a trick for
    efficiently raising a matrix $A$ to an integer power $n$ involving using the binary expansion
    of $n$ to determine the computations to perform. So $\log_2(n)$ matrix multiplications rather
    than $n$.}
\end{remark*}

\newpage
\section{Fibonacci - formula}

Let $V$ be the vector space containing all sequences of real numbers
$u_0, u_1, \ldots$.

\begin{comment}
  \begin{proof}
    This is a vector space since:
    \begin{enumerate}
    \item It's an Abelian group under addition (the zero sequence is the additive identity, inverse
      is obtained by negating each element, addition is associative and commutative)
    \item Closed under scalar multiplication since $\lambda u_0, \lambda u_1, ... \in V$.
    \end{enumerate}
  \end{proof}
\end{comment}

Let $W$ be the subspace of $V$ containing sequences such that $u_{n+2} = u_{n+1} + u_n$ for all
$n \geq 0$.

\begin{comment}
  \begin{proof}
    $W$ is a subspace because:
    \begin{enumerate}
    \item $W$ contains the zero sequence.
    \item Let $(u)_{n\geq 0}, (v)_{n\geq 0} \in W$. Then
      $$(u + v)_{n+2} = u_{n+1} + u_n + v_{n + 1} + v_n = (u + v)_{n+1} + (u + v)_n,$$ and
      $$(\lambda u)_{n+2} = \lambda u_{n+1} + \lambda u_n = (\lambda u)_{n+1} + (\lambda u)_n.$$
    \end{enumerate}
  \end{proof}
\end{comment}

% Note that every sequence $u \in W$ is determined by the first two values $(u_0, u_1)$.

\begin{claim*}
  A basis for $W$ is
  \begin{align*}
    e_1 &= 0, 1, 1, 2, 3, 5, 8, \ldots\\
    e_2 &= 1, 0, 1, 1, 2, 3, 5, \ldots.
  \end{align*}
\end{claim*}

\begin{proof} We need to show that $e_1$ and $e_2$ are linearly independent and spanning.

  Note that every sequence $u \in W$ is determined by the first two values $(u_0, u_1)$.

  Define the projection $P: W \to \R^2$ by $p(u) := (u_0, u_1)$. Note that $P$ is linear \footnote{
  \begin{align*}
    P(\lambda u) &= (\lambda u_0, \lambda u_1) = \lambda P(u)\\
    P(u + v)     &= (u_0 + v_0, u_1 + v_1) = (u_0, u_1) + (v_0, v_1) = P(u) + P(v).
  \end{align*}
  }.
  Furthermore, $P$ is injective (TODO). Therefore $P$ defines an isomorphism between $W$ and $\R^2$.

  {\bf Linear independence}: suppose $\lambda_1 e_1 + \lambda_2 e_2 = 0$, i.e. the zero
  sequence.


\end{proof}

\begin{proof}
  We need to show that $e_1$ and $e_2$ are linearly independent and spanning.

  {\bf Linear independence}: suppose $\lambda_1 e_1 + \lambda_2 e_2 = 0$, i.e. the zero
  sequence. Then, considering just the first two elements, we have $
  \begin{cases}
    0 + \lambda_2 = 0\\
    \lambda_1 + 0 = 0
  \end{cases}
$ so $\lambda_1 = \lambda_2 = 0$ as required.

  {\bf Spanning}: Let $u \in W$. Then $u_{n+2} = u_{n+1} + u_n$ for all $n \geq 0$.
\end{proof}

% Consider the two elements:

% What's the dimension of this subspace?




\end{document}