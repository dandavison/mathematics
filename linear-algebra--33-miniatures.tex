\section{Matousek -- 33 Miniatures}

\subsection{Fibonacci - matrix multiplication}\label{fibonacci-matrix-multiplication}

\begin{definition*}
  The Fibonacci sequence $0, 1, 1, 2, 3, 5, 8, ...$ is defined by
  \begin{align*}
    x_0 &= 0\\
    x_1 &= 1\\
    x_{n} &= x_{n-1} + x_{n-2}, ~~~~~~~ n \geq 2.
  \end{align*}
\end{definition*}

\begin{remark*}
  The sequence can be generated by taking $\vecMM{x_1}{x_0} = \vecMM{1}{0}$ as the initial state and multiplying
  repeatedly by $A = \matMMxNN{1}{1}
  {1}{0}$, yielding the sequence
  \begin{align*}
    \vecMM{1}{0}, \vecMM{1}{1}, \vecMM{2}{1}, \vecMM{3}{2}, \vecMM{5}{3}, \vecMM{8}{5}, ....
  \end{align*}
  Thus $\vecMM{x_{n+1}}{x_n} = A^n\vecMM{1}{0}$.\footnote{The book describes a trick for
    efficiently raising a matrix $A$ to an integer power $n$ involving using the binary expansion
    of $n$ to determine the computations to perform. So $\log_2(n)$ matrix multiplications rather
    than $n$.}
\end{remark*}

\newpage
\subsection{Fibonacci - formula}

Let $V$ be the vector space containing all sequences of real numbers
$u_0, u_1, \ldots$.

\begin{comment}
  \begin{proof}
    This is a vector space since:
    \begin{enumerate}
    \item It's an Abelian group under addition (the zero sequence is the additive identity, inverse
      is obtained by negating each element, addition is associative and commutative)
    \item Closed under scalar multiplication since $\lambda u_0, \lambda u_1, ... \in V$.
    \end{enumerate}
  \end{proof}
\end{comment}

Let $W$ be the subspace of $V$ containing sequences such that $u_{n+2} = u_{n+1} + u_n$ for all
$n \geq 0$.

\begin{comment}
  \begin{proof}
    $W$ is a subspace because:
    \begin{enumerate}
    \item $W$ contains the zero sequence.
    \item Let $(u)_{n\geq 0}, (v)_{n\geq 0} \in W$. Then
      $$(u + v)_{n+2} = u_{n+1} + u_n + v_{n + 1} + v_n = (u + v)_{n+1} + (u + v)_n,$$ and
      $$(\lambda u)_{n+2} = \lambda u_{n+1} + \lambda u_n = (\lambda u)_{n+1} + (\lambda u)_n.$$
    \end{enumerate}
  \end{proof}
\end{comment}

\begin{claim*}
  A basis for $W$ is
  \begin{align*}
    e_1 &= 0, 1, 1, 2, 3, 5, 8, \ldots\\
    e_2 &= 1, 0, 1, 1, 2, 3, 5, \ldots.
  \end{align*}
\end{claim*}
\begin{proof} We need to show that $e_1$ and $e_2$ are linearly independent and spanning.
  Note that every sequence $u \in W$ is determined by the first two values $(u_0, u_1)$.
  Define the projection $P: W \to \R^2$ by $p(u) := (u_0, u_1)$. Note that $P$ is linear \footnote{
    \begin{align*}
      P(\lambda u) &= (\lambda u_0, \lambda u_1) = \lambda P(u)\\
      P(u + v)     &= (u_0 + v_0, u_1 + v_1) = (u_0, u_1) + (v_0, v_1) = P(u) + P(v).
    \end{align*}
  }, injective and invertible.
  Let $i = (0, 1) \in \R^2$ and $j = (1, 0) \in \R^2$.
  Therefore $P^\1(i), P^\1(j) = e_1, e_2$ is a basis for $W$, by theorem (\ref{transformed-basis-is-a-basis}).
\end{proof}

Now we look for a different basis of $W$. Specifically, we have an inspiration: we seek sequences
$u \in W$ of the form $u_n = \tau^n$ for some $\tau$. Thus $\tau$ must satisfy
$\tau^{n+2} = \tau^{n+1} + \tau^n$ for all $n \geq 0$. We solve this for $n=0$.  We have
$\tau^2 = \tau + 1$, therefore $\tau_1 = \frac{1 + \sqrt{5}}{2}$ and
$\tau_2 = \frac{1 - \sqrt{5}}{2}$.

Define two new sequences $e'_1 = \tau_1^0, \tau_1^1, \ldots$ and
$e'_2 = \tau_2^0, \tau_2^1, \ldots$.

\begin{claim*}
  $e'_1, e'_2$ is another basis for $W$.
\end{claim*}

\begin{proof}
  We need only they are linearly independent. If they are, then they span $W$ since $W$ is
  2-dimensional.
  So suppose $\lambda_1e'_1 + \lambda_2e'_2 = 0$. Then, from considering the first two elements, we
  have
  $\begin{cases}
    \lambda_1 + \lambda_2 = 0\\
    \lambda_1\tau_1 + \lambda_2\tau_2 = 0.
  \end{cases}$
  Therefore $\lambda_1(\tau_1 - \tau_2) = 0$, so $\lambda_1 = \lambda_2 = 0$, as required.
\end{proof}

Therefore for all $u \in W$ there exist $\lambda_1, \lambda_2$ such that
$u = \lambda_1e'_1 + \lambda_2e'_2$.

In particular, there exist $\lambda_1, \lambda_2$ such that
$e_1 = 0, 1, 1, 2, 3, 5, 8, \ldots = \lambda_1e'_1 + \lambda_2e'_2$.

We can use the first two elements of the sequence to solve for $\lambda_1$ and $\lambda_2$. We have
$\begin{cases}
  0 = \lambda_1 + \lambda_2\\
  1 = \lambda_1\tau_1 + \lambda_2\tau_2
\end{cases}$, therefore $\lambda_1 = \frac{1}{\tau_1 - \tau_2} = \frac{1}{\sqrt{5}}$ and
$\lambda_2 = \frac{-1}{\sqrt{5}}$.

The $n$-th element of the Fibonacci sequence is therefore
\begin{align*}
  \lambda_1\tau_1^n + \lambda_2\tau_2^n =
  \frac{1}{\sqrt{5}}
  \(
  \(\frac{1 + \sqrt{5}}{2}\)^n -
  \(\frac{1 - \sqrt{5}}{2}\)^n
  \).
\end{align*}

\newpage
\subsection*{$n$-th Fibonacci number via eigenvector change of basis}

(Not in book.)

Recall from (\ref{fibonacci-matrix-multiplication}) that the Fibonacci sequence can be generated by
taking $\vecMM{1}{0}$ as the initial state and multiplying repeatedly by
$A = \matMMxNN{1}{1} {1}{0}$. Thus $\vecMM{x_{n+1}}{x_n} = A^n\vecMM{1}{0}$.

The characteristic polynomial of $A$ is $\tau^2 - \tau - 1 = 0$, therefore the eigenvalues are
$\tau_1 = \frac{1 + \sqrt{5}}{2}$ and $\tau_2 = \frac{1 - \sqrt{5}}{2}$.

Note that $\tau^2 = \frac{3 \pm \sqrt{5}}{2}$.

\begin{align*}
  \(A - \tau I\)v &= 0\\
  \matMMxNN{1 - \tau}{1}
           {1}       {-\tau} \vecMM{v_1}{v_2} &= 0\\
  &\begin{cases}
    v_1(1 - \tau) + v_2 = 0~~~~~&(1)\\
    v_1 - \tau v_2 = 0~~~~~&(2)\\
    \sqrt{v_1^2 + v_2^2} = 1 ~~~~~&(3)~~~\text{(choose arbitrary length for eigenvectors)}
  \end{cases}\\
  (2, 3):~~~~~~~ \sqrt{1 - v_2^2} &= \tau v_2\\
  1 &= v_2^2(\tau^2 + 1)\\
  v_2 &= \sqrt{\frac{1}{\tau^2 + 1}}\\
  v_2 &= \sqrt{\frac{2}{5 \pm \sqrt 5}}\\
  (2): ~~~~~~~v_1 = \tau v_2 &= \frac{1 \pm \sqrt 5}{2}\sqrt{\frac{2}{5 \pm \sqrt 5}}\\
  v_1 &= \frac{1 \pm \sqrt 5}{\sqrt 2 \sqrt{5 \pm \sqrt 5}}\\
  (1): ~~~~~~~v_1\(\frac{-1 \mp \sqrt{5}}{2}\) &= -\sqrt{\frac{2}{5 \pm \sqrt 5}}\\
              v_1 &= \frac{-2}{-1 \mp \sqrt{5}}\sqrt{\frac{2}{5 \pm \sqrt 5}}\\
\end{align*}

%   \tau v_2(1 - \tau) + v_2 &= 0\\
%   v_2(\tau - \tau^2 + 1) &= 0




\newpage
\subsection{The Clubs of Oddtown}

Oddtown has two rules about clubs:
\begin{itemize}
\item Every club must have an {\it odd} number of members.
\item The number of members in the intersection of any two distinct clubs must be {\it even}.
\end{itemize}

\begin{theorem}
  Under these rules, the number of clubs is less than the number of members.
\end{theorem}

\begin{proof}
  Let $m$ be the number of clubs and $n$ be the number of members.

  Let $A = (a_{ij})$ be a matrix over the finite field $F_2$, where $a_{ij} =
  \begin{cases}
    1 ~~~~ \text{if person $j$ is a member of club $i$},\\
    0 ~~~~ \text{otherwise}.
  \end{cases}
$

Thus rows correspond to clubs, and the $(i,j)$-th entry of $AA^T$ is 1 if the intersection of club
$i$ and club $j$ is odd, and 0 if even.

Therefore the rules imply that $AA^T = I_m$. I.e. the rank of $AA^T$ is $m$.

But the rank of a product of matrices cannot be larger than the minimum rank of any one factor, so
$m \leq n$.
\end{proof}
