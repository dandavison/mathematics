\section*{Math 202a - HW4 - Dan Davison - \texttt{ddavison@berkeley.edu}}
\begin{mdframed}
  \includegraphics[width=400pt]{img/analysis--berkeley-202a-hw04-f604.png}
\end{mdframed}
\begin{claim*}
  $\mu$ is a measure.
\end{claim*}
\begin{proof}
  It is given that $\mu$ is non-negative and that $\mu(\emptyset) = 0$. We must prove that $\mu$ is countably
  additive.

  So let $B_1, B_2, \dots$ be a pairwise disjoint countable collection of subsets of $X$. We want to show
  that $\mu(\bigcup_{i=1}^\infty B_i) = \sum_{i=1}^\infty \mu(B_i)$.

  Let's construct an increasing sequence of sets. Define $A_j = \bigcup_{i \leq j} B_i$ for $j=1, 2, \dots$, so
  that $A_1, A_2, \dots$ is an increasing sequence of sets. Note
  that $\bigcup_{i=1}^\infty B_i = \bigcup_{j=1}^\infty A_j$ therefore, by
  hypothesis,
  $\mu\big(\bigcup_{i=1}^\infty B_i\big) = \mu\big(\bigcup_{j=1}^\infty A_j\big) = \lim_{j\to\infty} \mu(A_j)$.

  Now, from finite additivity we have
  \begin{align*}
    \mu(A_j) = \mu(\bigcup_{i \leq j} B_i) = \sum_{i\leq j} \mu(B_i),
  \end{align*}
  therefore
  \begin{align*}
    \mu\big(\bigcup_{i=1}^\infty B_i\big) = \lim_{j\to\infty} \sum_{i\leq j} \mu(B_i) = \sum_{i=1}^\infty \mu(B_i),
  \end{align*}
  as required.
\end{proof}

\newpage
\begin{mdframed}
\includegraphics[width=400pt]{img/analysis--berkeley-202a-hw04-c187.png}
\end{mdframed}

\begin{proof}
  It is given that $\mu$ is non-negative and that $\mu(\emptyset) = 0$. We must prove that $\mu$ is countably
  additive.

  Let $A_1, A_2, \dots$ be a sequence of sets in $\mc A$ that decrease to $\emptyset$.

  Define $B_i = A_i \setminus \bigcup_{j > i} A_j = A_i \setminus A_{i+1}$. Then $B_1, B_2, \dots$ are a
  pairwise disjoint and countable collection of subsets of $X$. We want to show
  that $\mu(\bigcup_{i=1}^\infty B_i) = \sum_{i=1}^\infty \mu(B_i)$.

  We must use:
  \begin{enumerate}
  \item Finite additivity of $\mu$
  \item The fact that $\lim_{i\to\infty} \mu(A_i) = 0$.
  \end{enumerate}

  Morally, the result is true because some of the $B_i$ have non-zero measure, and yet their measure decreases
  to zero, and their sum is bounded above by $\mu(A_1)$.


  Note that $\bigcup_{i=1}^\infty B_i = \bigcup_{i=1}^\infty A_i = A_1$.

  From finite additivity we have that for all $n \in \N$
  \begin{align*}
    \mu(\bigcup_{i=1}^n B_i) = \sum_{i=1}^n \mu(B_i).
  \end{align*}
  Also



\end{proof}


\newpage
\begin{mdframed}
\includegraphics[width=400pt]{img/analysis--berkeley-202a-hw04-b168.png}
\end{mdframed}
\begin{proof}
  $\emptyset$ is countable, therefore we have $\mu(\emptyset) = 0$ as required, and it remains to show
  that $\mu$ is countably additive.

  So let $B_1, B_2, \dots$ be a disjoint countable collection of subsets of $X$. We want to show
  that $\mu(\bigcup_{i=1}^\infty B_i) = \sum_{i=1}^\infty \mu(B_i)$.

  % Consider $\bigcup_{i=1}^\infty B_i$. It could be uncountable, since the $B_i$ could be a countable partition
  % of the entire set $X$. Could it also be countable? Yes, since the $B_i$ could be singletons. So we must
  % handle both cases.

  First suppose $\bigcup_{i=1}^\infty B_i$ is countable. Then no $B_i$ is uncountable. Therefore $\mu(B_i) = 0$
  for all $i$ and we have
  \begin{align*}
    \sum_{i=1}^\infty \mu(B_i) = \sum_{i=1}^\infty 0 = 0 = \mu\big(\bigcup_{i=1}^\infty B_i\big),
  \end{align*}
  as required.

  Next, suppose $\bigcup_{i=1}^\infty B_i$ is uncountable. We want to show
  that $\sum_{i=1}^\infty \mu(B_i) = 1$. Equivalently, we want to show that exactly one of the $B_i$ is
  uncountable. Note that $\ms A = \sigma(\ms A)$, and therefore we have by hypothesis that either $B_i$ is
  countable or $B_i^c$ is countable, for all $i$. Clearly some $B_i$ is uncountable or else we would
  have $\sum_{i=1}^\infty \mu(B_i) = \sum_{i=1}^\infty 0 = 0 \neq \mu\big(\bigcup_{i=1}^\infty B_i\big)$.
  Suppose for a contradiction that there exists $j \neq k$ such that $B_j$ and $B_k$ are uncountable. Note
  that $B_j$ and $B_k$ are disjoint, therefore $B_k \subseteq B_j^c$. But $B_j^c$ is countable, therefore $B_k$
  is countable; a contradiction. Therefore no such pair $j, k$ exists and we conclude that exactly one of
  the $B_i$ is uncountable, as required.
\end{proof}

\newpage
\begin{mdframed}
\includegraphics[width=400pt]{img/analysis--berkeley-202a-hw04-c88b.png}
\end{mdframed}

\begin{remark*}
  The text states that this exercise guarantees that the completion of a $\sigma$-algebra exists.

  A subset $A \subset X$ (not necessarily in $\mc A$) is a \defn{null set} if $A$ is a subset of some element of
  $\mc A$ which has zero measure. $(X, \mc A, \mu)$ is a \defn{complete} measure space if all null sets are in
  $\mc A$. The \defn{completion} of $\mc A$ is the smallest complete $\sigma$-algebra $\bar{\mc A}$ containing
  $\mc A$ such that $(X, \bar{\mc A}, \bar \mu)$ is a complete measure space, where $\bar \mu$ is an extension
  of $\mu$ from $\mc A$ to $\bar{\mc A}$.
\end{remark*}

\begin{claim*}
  $B \in \mc B$ if and only if there exists $A \in \mc A$ and $N \in \mc N$ such that $B = A \cup N$.
\end{claim*}

\begin{proof}
  Let $\mc B = \sigma(\mc A \cup \mc N)$.

  Suppose that there exists $A \in \mc A$ and $N \in \mc N$ such that $B = A \cup N$.
  Then $A \in \mc A \cup \mc N$, therefore $A \in \mc B$. And $N \in \mc A \cup \mc N$,
  therefore $N \in \mc B$. Therefore $B = A \cup N \in \mc B$.

  For the other direction, suppose that $B \in \mc B$. We want to show that there exists $A \in \mc A$
  and $N \in \mc N$ such that $B = A \cup N$.
\end{proof}

\begin{claim*}
  $\bar{\mu}(B)$ is uniquely defined for each $B \in \mc B$.
\end{claim*}

\begin{proof}
  The way one shows uniqueness is: let $\bar{\mu'}$ be another...show that $\bar{\mu'} = \bar{\mu}$.
\end{proof}

\begin{claim*}
  $\bar{\mu}(B)$ is a measure on $\mc B$.
\end{claim*}

\begin{proof}

\end{proof}

\begin{claim*}
  $(X, \mc B, \bar{\mu})$ is complete.
\end{claim*}

\begin{proof}

\end{proof}

\begin{claim*}
  $(X, \mc B, \bar{\mu})$ is the completion of $(X, \mc A, \mu)$.
\end{claim*}

\begin{proof}

\end{proof}
\newpage
\begin{mdframed}
  \includegraphics[width=400pt]{img/analysis--berkeley-202a-hw-0d98.png}
\end{mdframed}


\newpage
\begin{mdframed}
\includegraphics[width=400pt]{img/analysis--berkeley-202a-hw04-c0b6.png}
\end{mdframed}

\newpage


\begin{comment}
  EXERCISE 3.3 FROM OLD VERSION
  \begin{mdframed}
    \includegraphics[width=400pt]{img/analysis--berkeley-202a-hw-2389.png}
  \end{mdframed}

  \begin{proof}
    From finite additivity we have
    \begin{align*}
      \mu(A \cup B) &= \mu(A) + \mu(A \setminus B). \\
      \mu(A \cup B) &= \mu(B) + \mu(B \setminus A).
    \end{align*}
    Summing these gives
    \begin{align*}
      2\mu(A \cup B)
      = \mu(A) + \mu(B) + \mu(A \triangle B)
    \end{align*}
    But $\{A \triangle B, A \cap B\}$ is a partition of $A \cup B$ hence from finite additivity
    again $\mu(A \triangle B) + \mu(A \cap B) = \mu(A \cup B)$. Therefore
    \begin{align*}
      2\mu(A \cup B) = \mu(A) + \mu(B) + \mu(A \cup B) - \mu(A \cap B),
    \end{align*}
    or equivalently
    \begin{align*}
      \mu(A \cup B) = \mu(A) + \mu(B) - \mu(A \cap B).
    \end{align*}
  \end{proof}

  \begin{mdframed}
    \includegraphics[width=400pt]{img/analysis--berkeley-202a-hw-3a79.png}
  \end{mdframed}

  \begin{claim*}
    If $m$ and $n$ have the same value on any open interval in $B$ then they have the same value on any set
    in $\mc B$.
  \end{claim*}

  \begin{proof}
    Let $\mc O$ be the collection of open subsets of $\R$.

    Let $O \in \mc O$. Then $O$ can be written as a countable union of disjoint open intervals, $O = \bigcup_i I_i$. Therefore
    \begin{align*}
      m(O) = m(\bigcup_i I_i) = \sum_i m(I_i) = \sum_i n(I_i) = n(O).
    \end{align*}
    By definition, $\mc B = \sigma(\mc O)$. We want to show that measures $m$ and $n$ agree on every set $A \in \mc B$.

    Let $A \in \mc B$ and suppose $m(A) = n(A)$. Then
    \begin{align*}
      m(A^c) = m(X) - m(A) = n(X) - n(A) = n(A^c).
    \end{align*}
  \end{proof}
\end{comment}
