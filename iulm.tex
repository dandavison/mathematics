\documentclass[12pt]{article}
\usepackage{notes}
\usepackage{enumerate,mdframed}

\newcommand{\LHS}{\text{LHS}}
\newcommand{\RHS}{\text{RHS}}

\begin{document}
\title{Introduction to University Mathematics}
\maketitle
\tableofcontents

\section{Sheet 1}
\newpage
\subsection{Let $A, B, C$ be sets. Write out a proof that\\
                    $A \cap (B \cup C) = (A \cap B) \cup (A \cap C)$.}
\begin{mdframed}

The following things are defined already: intersection, union, subset.

Define $\LHS = A \cap (B \cup C)$ and $\RHS = (A \cap B) \cup (A \cap C)$.

We have to prove
\begin{enumerate}
\item that $\LHS \subset \RHS$, and
\item that $\RHS \subset \LHS$.
\end{enumerate}

\subsubsection*{Proof of (1)}
If $\LHS = \emptyset$, then (1) is true.

Alternatively, suppose $\LHS \neq \emptyset$ and let $x \in \LHS$. Then by the
definition of intersection, $x \in A$ and $x \in B \cup C$, and therefore by
the definition of union either $x \in B$ or $x \in C$ or it is in both.

Suppose $x \in B$. Then we have $x \in A$ and $x \in B$, so by the definition
of intersection $x \in A \cap B$. The RHS is the union of two sets, of which
$A \cap B$ is one. Therefore by the definition of union $x \in \RHS$.

Alternatively we can suppose $x \in C$. Since union and intersection commute,
$B$ and $C$ play equivalent roles and $x \in C$ leads to the same conclusion
that $x \in \RHS$ by an identical argument.

Therefore $\LHS \subset \RHS$.

\subsubsection*{Proof of (2)}
If $\RHS = \emptyset$, then (2) is true.

Alternatively, suppose $\RHS \neq \emptyset$ and let $x \in \RHS$. Then by the
definition of union, $x \in A \cap B$ or $x \in A \cap C$. Suppose
$x \in A \cap B$ (and label this ``supposition 1''). Then $x \in A$ and
$x \in B$. $x \in B \implies x \in B \cup Z$ for any set $Z$, so
$x \in B \cup C$, and therefore $x \in \LHS$.

Alternatively, at supposition 1, we could suppose $x \in A \cap C$. Due to
the symmetry resulting from commutativity of union and interesection, this also
leads to the conclusion $x \in \LHS$ by an identical argument.

Therefore $\RHS \subset \LHS$. \qed
\end{mdframed}

\section{Sheet 2}
\newpage
\subsection{}
\begin{mdframed}
  The attempted proof is flawed because it has the following form:
  \begin{enumerate}
  \item Suppose [the thing we are trying to prove]
  \item Deduce from this a tautology
  \item Conclude that [the thing we are trying to prove] is true.
  \end{enumerate}
  The problem is that it is possible to deduce a tautology from a false
  statement. For example $-1 = 1 \implies 1 = 1$ by squaring both sides.

  A correct proof follows.

  We are given that $a \geq 0, b \geq 0$. We start by noting that
  $0 \leq (a - b)^2$ is a tautology (true for all values of $a$ and
  $b$). Therefore
  \begin{align*}
              0 &\leq a^2 -2ab + b^2\\
    \implies  4ab &\leq a^2 + 2ab + b^2 = (a + b)^2\\
    \implies  2\sqrt{ab} &\leq a+b. \qed
  \end{align*}
\end{mdframed}

\subsection{}
\begin{mdframed}
  \begin{enumerate}[(a)]
  \item 2 is prime or 2 is odd.\\
    True. 2 is prime.
  \item 2 is prime or 2 is even.\\
    True. 2 is prime.
  \item If 2 is odd then 2 is prime.\\
    True. Premise is false so implication is true for all values of conclusion.
  \item If 2 is even then 2 is prime.\\
    True. Premise is true and conclusion is true, so implication is true.
  \item Forall $n \in \N$, if $n$ is a square number then $n$ is not prime.\\
    False, 1 is a square number yet not prime.
  \item Forall $n \in \N$, $n$ is not prime if and only if $n$ is a square number.\\
    False. 6 is not prime and yet not a square number.
  \item For all even primes $p > 2$, $p^2 = 2016$.\\
    True. There are no even primes greater than 2;
    $\forall ~ x \in \emptyset, ~p(x)$ is true for any predicate $p$.
  \end{enumerate}
\end{mdframed}


\subsection{}
\begin{mdframed}
Poorly-worded question. ``A card that has an even number on one side has a
vowel on the other'' might mean either of the following:
\begin{enumerate}
\item $\exists c: \text{even}(c) \text{~and~} \text{vowel}(c)$
\item $\text{even}(c) \implies \text{vowel}(c)$.
\end{enumerate}
If we assume the intended meaning is (2), then we could disprove the hypothesis
by finding an even card with a consonant. So we could try cards 2 and B. That
might lead to a counter-example. If it does not, then the experiment is
inconclusive, because a counter-example may exist in the unavailable cards in
the rest of the pack.

If we assume the intended meaning is (1), then we could try turning over 2 and
A. That might lead to an example confirming the truth of the hypothesis. If it
does not, then the experiment is inconclusive, because a confirmation may exist
in the unavailable cards in the rest of the pack.
\end{mdframed}
\subsection{}
\begin{mdframed}
\subsubsection*{Mathematical induction}
Consider a predicate function $p: \N \to \{\text{true},\text{false}\}$. We want to show
that $p(i)$ is true for all $i \in N$. In other words,

\textbf{Proposition}: $\forall ~ i \in N ~ p(i)$.\\

\textbf{Proof by induction}: First demonstrate that $p(0)$ is true. Next,
demonstrate that $p(i) \implies p(i+1)$.

\end{mdframed}

\subsection{}
\begin{mdframed}
  For all real numbers $a$ and positive reals $\epsilon$, there exists a
  positive real $\delta$ for which the following is true: for all real numbers
  $x$, $f(x)$ lies within a distance $\epsilon$ of $f(a)$ whenever $x$ lies
  within a distance $\delta$ of $a$.
\end{mdframed}

\subsection{}
\begin{mdframed}
  \begin{align*}
    \forall a \in \R: \forall x \in \R: \forall \epsilon \in \R^{>0}: \exists \delta \in \R^{>0}: |x - a| < \delta \implies |f(x) - f(a)| < \epsilon.
  \end{align*}
Yes it says much the same thing.
\end{mdframed}

\subsection{}

\newpage
\subsection{}
Let’s write $n = d_md_{m-1} \cdots d_2d_1d_0$ where $0 \leq d_i \leq 9$ for
$0 \leq i \leq m$ and $d_m \neq 0$, to mean that $n$ is an $(m+1)$-digit natural
number and the given string of digits is its decimal representation. Prove that
$n$ and $d_0 + d_1 + ··· + d_m−1 + d_m$ leave the same remainder
when divided by 9.\\
\begin{mdframed}
  We have $n = \sum_{i=0}^m d_i 10^m$, and let $s(n) = \sum_{i=0}^m d_i$. The
  claim is that $$n \mod 9 = s(n) \mod 9$$.

  We can sort of see this informally, in that every time we count through a
  block of 10 numbers, the remainder $\mod 9$ ``slips'' by 1. The coefficient
  in the 10s column records the ``number of slips''. When we get to 100, we've
  almost ``slipped through'' a whole block, and the 1 in the 100s column
  accounts for that.

  \subsubsection*{Proof}
  Note that $10 = 1 \mod 9$. Also note that $ab \mod j = (a \mod j)(b \mod j)$
  and $(a+b) \mod j = (a \mod j) + (b \mod j)$ (proof?). Then
  \begin{align*}
    n \mod 9 &= \(\sum_{i=0}^m d_i 10^m\) \mod 9\\
             &= \(\sum_{i=0}^m \(d_i 10^m \mod 9\)\) \mod 9\\
             &= \(\sum_{i=0}^m ((d_i \mod 9) (10^m \mod 9) \mod 9) \)\mod 9\\
             &= \(\sum_{i=0}^m ((d_i \mod 9) (\prod_{j=1}^m 10 \mod 9) \mod 9) \)\mod 9\\
             &= \(\sum_{i=0}^m d_i \mod 9 \)\mod 9\\
             &= \(\sum_{i=0}^m d_i \mod 9 \)\mod 9\\
  \end{align*}

  \begin{align*}
    n \mod 9 = s(n) \mod 9
    &\iff n - s(n) = 0 \mod 9\\
    &\iff \sum_{i=0}^m d_i 10^m - d_i = 0 \mod 9\\
    &\iff (10^m - 1)\sum_{i=0}^m d_i = 0 \mod 9\\
  \end{align*}
  which is true for all valid values of $m$ and $d_i$.

\end{mdframed}


\subsection{}
\end{document}
