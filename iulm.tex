\documentclass[12pt]{article}
\usepackage{notes}
\usepackage{mdframed}

\newcommand{\LHS}{\text{LHS}}
\newcommand{\RHS}{\text{RHS}}

\begin{document}
\title{Introduction to University Mathematics}
\maketitle
\tableofcontents

\section{Sheet 1}
\newpage
\subsection{Let $A, B, C$ be sets. Write out a proof that\\
                    $A \cap (B \cup C) = (A \cap B) \cup (A \cap C)$.}
\begin{mdframed}

The following things are defined already: intersection, union, subset.

Define $\LHS = A \cap (B \cup C)$ and $\RHS = (A \cap B) \cup (A \cap C)$.

We have to prove
\begin{enumerate}
\item that $\LHS \subset \RHS$, and
\item that $\RHS \subset \LHS$.
\end{enumerate}

\subsubsection*{Proof of (1)}
If $\LHS = \emptyset$, then (1) is true.

Alternatively, suppose $\LHS \neq \emptyset$ and let $x \in \LHS$. Then by the
definition of intersection, $x \in A$ and $x \in B \cup C$, and therefore by
the definition of union either $x \in B$ or $x \in C$ or it is in both.

Suppose $x \in B$. Then we have $x \in A$ and $x \in B$, so by the definition
of intersection $x \in A \cap B$. The RHS is the union of two sets, of which
$A \cap B$ is one. Therefore by the definition of union $x \in \RHS$.

Alternatively we can suppose $x \in C$. Since union and intersection commute,
$B$ and $C$ play equivalent roles and $x \in C$ leads to the same conclusion
that $x \in \RHS$ by an identical argument.

Therefore $\LHS \subset \RHS$.

\subsubsection*{Proof of (2)}
If $\RHS = \emptyset$, then (2) is true.

Alternatively, suppose $\RHS \neq \emptyset$ and let $x \in \RHS$. Then by the
definition of union, $x \in A \cap B$ or $x \in A \cap C$. Suppose
$x \in A \cap B$ (and label this ``supposition 1''). Then $x \in A$ and
$x \in B$. $x \in B \implies x \in B \cup Z$ for any set $Z$, so
$x \in B \cup C$, and therefore $x \in \LHS$.

Alternatively, at supposition 1, we could suppose $x \in A \cap C$. Due to
the symmetry resulting from commutativity of union and interesection, this also
leads to the conclusion $x \in \LHS$ by an identical argument.

Therefore $\RHS \subset \LHS$. \qed
\end{mdframed}

\end{document}
