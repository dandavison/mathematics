\begin{comment}
  \section*{Math 202a - HW1 - Dan Davison - \texttt{ddavison@berkeley.edu}}

  \begin{mdframed}
    \includegraphics[width=400pt]{img/analysis--berkeley-202a--homework-1-a75a.png}
  \end{mdframed}


  \begin{proof}
    $d$ is a metric if it satisfies (I) $d(f,f) = 0$, (II) $d(f,g) = d(g, f)$, and (III) $d(f,g) + d(g, h) \le d(f, h)$.

    (I) is satisfied: $d(f, f) = \int_{[0,1]}|f(x) - f(x)| \dx = 0$.

    (II) is satisfied:
    \begin{align*}
      d(f, g)
      &= \int_{[0,1]}|f(x) - g(x)| \dx \\
      &= \int_{[0,1]}|g(x) - f(x)| \dx \\
      &= d(g, f),
    \end{align*}

    (III) is satisfied:
    \begin{align*}
      d(f, g) + d(g, h)
      &= \int_{[0,1]} |f(x) - g(x)| \dx + \int_{[0,1]} |g(x) - h(x)| \dx \\
      &= \int_{[0,1]} |f(x) - g(x)| + |g(x) - h(x)| \dx \\
      &\le \int_{[0,1]} |f(x) - g(x) + g(x) - h(x)| \dx \\
      &= \int_{[0,1]} |f(x) - h(x)| \dx \\
      &= d(f, h).
    \end{align*}
  \end{proof}

  \newpage
  \begin{mdframed}
    \includegraphics[width=400pt]{img/analysis--berkeley-202a--homework-1-d1d3.png}
  \end{mdframed}

  \begin{enumerate}
  \item
    \begin{proof}
      $d$ is a metric on the function space $\mathcal C\([0, 1], \R\)$ if it satisfies (I) $d(f,f) = 0$,
      (II) $d(f,g) = d(g, f)$, and (III) $d(f,g) + d(g, h) \le d(f, h)$.

      (I) is satisfied: $\sup_{x\in [0,1]} |f(x) - f(x)| = \sup_{x\in [0,1]} 0 = 0$.

      (II) is satisfied:
      \begin{align*}
        d(f, g)
        &= \sup_{x \in [0,1]}|f(x) - g(x)| \\
        &= \sup_{x \in [0,1]}|g(x) - f(x)| \\
        &= d(g, f),
      \end{align*}
      (III) is satisfied:
      \begin{align*}
        d(f, g) + d(g, h)
        &=   \sup_{x \in [0,1]} |f(x) - g(x)| + \sup_{x \in [0,1]} |g(x) - h(x)| \\
        &=   \sup_{x \in [0,1]} \Big(|f(x) - g(x)| + |g(x) - h(x)|\Big) \\
        &\le \sup_{x \in [0,1]} |f(x) - h(x)| \\
        &=   d(f, h).
      \end{align*}
    \end{proof}
  \item
    \begin{claim*}
      $\mc C\([0, 1], \R\)$ is separable.
    \end{claim*}

    \begin{proof}
      Let $\mc C$ be the set of continuous functions $[0, 1] \to \R$.

      Fix an arbitrary function $f \in \mc C$ and fix some $\epsilon > 0$.

      Define $g^*_n: [0, 1] \to \R$ as follows:
      \begin{enumerate}
      \item For all $i \in 0, 1, 2, \ldots, n$ set $x_i = i/n$.
      \item For all $i \in \{0, 1, 2, \ldots, n\}$, set $y^*_i = f(x_i)$. (Note that $y^*_i$ is in general not a rational
        number; we will account for this later.)
      \item For all $i \in \{1, 2, \ldots, n\}$ draw a straight line segment connecting $(x_{i-1}, y^*_{i-1})$
        and $(x_i, y^*_i)$.
      \item Define $g^*_n: [0, 1] \to \R$ to be the function whose graph was just drawn. (It is possible to give an
        explicit procedure for computing $g^*_n(x)$ by finding the interval in which $x$ lies and then using linear
        interpolation.)
      \end{enumerate}

      We now modify the definition of the family of approximating functions so that the $y$-coordinates of the
      endpoints are rational. Define $g_n: [0, 1] \to \R$ as follows:
      \begin{enumerate}
      \item Construct the sets of points $\{(x_i, y^*_i) ~|~ i \in \{0, 1, 2, \ldots, n\}\}$ as above.
      \item For $i \in \{0, 1, 2, \ldots, n\}$ set $y_i$ equal to a rational number in the
        interval $(y^*_i - \epsilon/4, y^*_i)$. (Such a rational number exists: for example, set $k$ equal to the
        smallest natural number such that $1/k < \epsilon/4$, and then set $j$ equal to the smallest natural number
        such that $j/k > y^*_i - \epsilon/4$. Then $j/k$ is a rational number in $(y^*_i - \epsilon/4, y^*_i)$.)
      \item For all $i \in \{1, 2, \ldots, n\}$ draw a straight line segment connecting $(x_{i-1}, y_{i-1})$
        and $(x_i, y_i)$.
      \item Define $g_n: [0, 1] \to \R$ to be the function whose graph was just drawn.
      \end{enumerate}
      Note that, since $f$ is continuous on a compact domain, $f$ is uniformly continuous. Fix some $\delta > 0$ such
      that $|x - x'| < \delta \implies |f(x) - f(x')| < \epsilon$ for all $(x, x') \in [0, 1]^2$.

      Set $m$ equal to the smallest natural number such that $1/m < \delta/2$ and note
      that $|f(x) - g_n(x)| < \epsilon$ for all $x \in [0, 1]$ due to the uniform continuity of $f$. (Informally,
      this is true because we can view uniform continuity as stating that a rectangle of base $\delta$ and
      height $\epsilon$ can be positioned over the graph at any point such that the graph intersects the left and
      right edges of the rectange but does not otherwise leave the rectangle. Our piecewise affine function
      consists of straight line segments that fit within such rectangles.) Therefore $d(g_n, f) < \epsilon$ for
      all $n \geq m$ and so $\{g_n | n \in \N\}$ is dense in $\mc C$.

      Finally we must show that $\{g_n ~|~ n \in \N\}$ is countable. Note that $g_n$ is piecewise affine for a
      given $n$, and that the $x$-coordinates of the endpoints are fixed. Thus for a given $n$, the cardinality
      of $\{g_n\}$ is equal to the cardinality of the set of possible $y$-coordinates. The latter set is $\Q^n$.
      Thus the cardinality of $\{g_n ~|~ n \in \N\}$ is equal to the cardinality of the
      set $\bigcup_{n\in \N} \Q^n$. This is a countable union of countable sets and is therefore countable.
    \end{proof}

  \item
    \begin{proof}
      Let $f_s: (0, 1) \to \R$ be given by $f_s(x) = \frac{1}{r(s)x}$, where $s \in 2^\N$ and $r(s) \in [0, 1]$ is
      the real number corresponding to $s$, i.e. the number $r(s) = 0.d_1d_2d_3\ldots$ where
      \begin{align*}
        d_i =
        \begin{cases}
          1, ~~~~ \text{if} ~~~~ i \in s,\\
          0, ~~~~ \text{otherwise}.
        \end{cases}
      \end{align*}
      Note that for real $a, b$ we have
      \begin{align*}
        \frac{1}{ax} - \frac{1}{bx} = \frac{b - a}{abx}
      \end{align*}
      and therefore the supremum distance between any two elements $f_{s_1}$ and $f_{s_2}$ is unbounded
      as $x \to 0$.
    \end{proof}
  \item
    \begin{proof}
      Let $\mc C$ be the set of continuous functions $f: (0, 1) \to \R$.

      Assume for a contradiction that $\mc C$ is separable. Let $\mc G \subset \mc C$ be a countable dense set of
      functions.

      Recall that in part (3) we found an uncountable set $\mc H \subset \mc C$ with the property that every pair
      of elements in $\mc H$ is at supremum distance at least one.

      But this is a contradiction, since we can establish a bijection between $\mc H$ and $\mc G$ as follows:

      Pick an element $h_1 \in \mc H$. Since $\mc G$ is dense in $\mc C$, there exists $g_1 \in \mc G$ such
      that $d(h_1, g_1) < 1/2$. Now pick $h_2 \in \mc H$ such that $h_1 \neq h_2$. Again, there
      exists $g_2 \in \mc G$ such that $d(h_2, g_2) < 1/2$. Furthermore, by the triangle
      inequality, $g_2 \neq g_1$. Continuing in this fashion, on the $i$-th iteration we pick $h_i \in \mc H$ and
      find a nearby $g_i \in \mc G$ such that $d(h_i, g_i) < 1/2$, and by the triangle inequality conclude
      that $g_i \neq g_j$ for all $j < i$.

      Thus we can associate each element of $\mc H$ with a unique element of $\mc G$ and conclude that the
      cardinality of $\mc G$ equals that of $\mc H$, which is that of the power set of the natural numbers.
      But $\mc G$ is countable; a contradiction. Therefore no such countable dense set $\mc G$ exists and $\mc C$
      is not separable.
    \end{proof}


    But can find bounded fns with image [0, 1]

    \begin{mdframed}
      \includegraphics[width=400pt]{img/analysis--berkeley-202a-hw-7459.png}
    \end{mdframed}
  \end{enumerate}


  \newpage
  \begin{mdframed}
    \includegraphics[width=400pt]{img/analysis--berkeley-202a--homework-1-8349.png}
  \end{mdframed}


  \begin{proof}
    Let $E_m^a = \{x \in [0, 1] : |f_m(x) < a|\}$ and let $T = \bigcap_{k=1}^\infty \bigcup_{l=1}^\infty \bigcap_{m > l} E_m^{1/k}$.

    Informally, $E_m^a$ is the set of points for which $f_m$ is within $a$ of zero.

    Let $f_n: [0, 1] \to \R$ be a sequence of functions and let $S \subseteq [0, 1]$ be the set of points $x$
    such that $f_n(x) \to 0$ as $n \to \infty$.

    First we prove that $x \in S \implies x \in T$.

    So let $x \in S$. Then from the definition of limit we have
    \begin{align*}
      &\forall \epsilon>0 ~~~~ \exists l \in \N ~~~~ \forall m \geq l ~~~~  x \in E_m^\epsilon \\
      \iff &\forall \epsilon>0 ~~~~ \exists l \in \N ~~~~                        x \in \bigcap_{m \geq l} E_m^\epsilon \\
      \iff &\forall \epsilon>0 ~~~~                                              x \in \bigcup_{l=1}^\infty \bigcap_{m \geq l} E_m^\epsilon \\
      \iff &                                                              x \in \bigcap_{k=1}^\infty \bigcup_{l=1}^\infty \bigcap_{m \geq l} E_m^{1/k} = T,
    \end{align*}
    as required.

    Secondly we prove that $x \in T \implies x \in S$.

    So let $x \in T$. We have
    \begin{align*}
      x \in \bigcap_{k=1}^\infty \bigcup_{l=1}^\infty \bigcap_{m \geq l} E_m^{1/k},
    \end{align*}
    which is equivalent to the statement
    \begin{align*}
      \forall k>0 ~~~~ \exists l \in \N ~~~~ \forall m \geq l ~~~~  |f_m(x)| < \frac{1}{k}.
    \end{align*}
    Let $\epsilon > 0$ be a real number. Then there exists $k \in \N$ such that $\frac{1}{k} < \epsilon$. Therefore we have
    \begin{align*}
      \forall \epsilon>0 ~~~~ \exists l \in \N ~~~~ \forall m \geq l ~~~~  |f_m(x)| < \epsilon
    \end{align*}
    which is equivalent to $x \in S$, as required.
  \end{proof}

  \begin{proof}
    Let $S \subseteq [0, 1]$ be the set of points $x$ for which $f_n(x)$ converges as $n \to \infty$. Since every
    convergent sequence in the reals is Cauchy, we have that $x \in S$ is equivalent to
    \begin{align*}
      \forall \epsilon > 0 ~~~~ \exists l \in \N ~~~~ \forall m \geq l ~~~~ \forall n \geq l ~~~~ |f_m(x) - f_n(x)| < \epsilon,
    \end{align*}
    which is equivalent to
    \begin{align*}
      \forall \epsilon > 0 ~~~~ x \in \bigcup_{l=1}^\infty \bigcap_{m \geq l} \bigcap_{n \geq l} E_{m,n}^\epsilon.
    \end{align*}
    Therefore, we have that $x \in S$ implies
    \begin{align*}
      x \in \bigcap_{k=1}^\infty \bigcup_{l=1}^\infty \bigcap_{m \geq l} \bigcap_{n \geq l} E_{m,n}^{1/k}.
    \end{align*}
    As before, the reverse implication also holds since, for any given $\epsilon > 0$, we can find a $k \in \N$
    such that $\frac{1}{k} < \epsilon$.
  \end{proof}

  \newpage
  \begin{mdframed}
    \includegraphics[width=400pt]{img/analysis--berkeley-202a--homework-1-f175.png}
  \end{mdframed}


  \begin{proof}
    Let $X \subset [0, 1]$ be the subset of real numbers without any 6 in their decimal expansion. Let $x \in X$ and let
    \begin{align*}
      d_n(x) =
      \begin{cases}
        0, ~~~~ n\text{-th decimal place of }x \text{ is }0\\
        1, ~~~~ \text{otherwise}.
      \end{cases}
    \end{align*}
    Define $f: X \to [0, 1]$ by setting $f(x)$ equal to the real number whose binary expansion
    is $0.d_1(x)d_2(x)\cdots$.

    Note that for any real number $\om \in [0, 1]$, there exists $x \in X$ such that $f(x) = \om$. To find such
    an $x$, we could for example choose the number whose decimal expansion is equal to the binary expansion
    of $\om$.

    Therefore $f$ is a non-injective surjection from $X$ to the reals in $[0, 1]$, and so the cardinality of $X$
    is at least that of the reals. Since $X \subset \R$ we conclude that the cardinality of $X$ is equal to that
    of the reals.
  \end{proof}

  \newpage
  \begin{mdframed}
    \includegraphics[width=400pt]{img/analysis--berkeley-202a--homework-1-5192.png}
  \end{mdframed}



  \begin{proof}
    Let $X \subseteq [0, 1]$ be the set of points at which $f$ is continuous. We want to show that $f$ is
    continuous at $x$ iff $x$ is not rational.

    Suppose for a contradiction that $f$ is continuous at a rational point $x = p/q$, where $p, q$ are
    non-negative integers. Then $f(x) = 1/q$. But there will always be irrational points within a given
    distance $\delta$ of $x$, now matter how small $\delta$ is, and at such an irrational point $x'$ we
    have $|f(x) - f(x')| = |1/q - 0| = 1/q$. Therefore $f$ is not continuous at $x$ since the definition of
    continuity does not hold for $\epsilon < 1/q$.

    Now let $x$ be irrational, so that $f(x) = 0$. Fix an arbitrary $\epsilon > 0$. We want to show that there
    exists a $\delta$ such that $1/q < \epsilon$ for any rational point $p/q$ lying within $\delta$ of $x$,
    where $p/q$ is in reduced terms. If $\epsilon > 1/2$ then any $\delta$ will work, so
    assume $\epsilon \leq 1/2$. Let $k$ be the largest natural number such that $1/k \geq \epsilon$, let $i$ be
    the largest natural number such that $i/k < x$ and let $j$ be the smallest natural number such
    that $j/k > x$. Then a choice of $\delta = \frac{1}{2}\min\{x - i/k, j/k - x\}$ will work to prove continuity
    of $f$ at irrational $x$.
  \end{proof}


  \begin{mdframed}
    \includegraphics[width=400pt]{img/analysis--berkeley-202a-hw-af4c.png}
  \end{mdframed}


  \begin{mdframed}
    \includegraphics[width=400pt]{img/analysis--berkeley-202a-hw-5fee.png}
  \end{mdframed}




  \newpage
  \begin{mdframed}
    \includegraphics[width=400pt]{img/analysis--berkeley-202a--homework-1-f5e8.png}
  \end{mdframed}





  \begin{definition*}
    $g: [0, 1] \to \R$ is Riemann integrable if
    \begin{align*}
      \sup_{\phi^-} I(\phi^-) = \inf_{\phi^+} I(\phi^+).
    \end{align*}
    Here $\phi^-$ and $\phi^+$ are step functions adapted​ to some
    partition $0 \leq x_1 \leq x_2 \leq \ldots \leq x_{n-1} \leq 1$, such that $\phi(x) = c_i$
    for $x \in (x_{i-1}, x_i)$. $I(\phi)$ is (informally) the area under the step function $\phi$:
    \begin{align*}
      I(\phi) = \sum_{i=1}^n c_i(x_i - x_{i-1}).
    \end{align*}
    And the supremum is over all minorants $\phi^- \leq g$ and the infimum is over all
    majorants $\phi^+ \geq g$, where the length $n$ of the partition is allowed to vary as well as the constant
    values $\{c_1, c_2, \ldots, c_n\}$ of the step function within each segment.
  \end{definition*}

  \begin{enumerate}
  \item
    \begin{claim*}
      The specified function $f$ is not Riemann integrable.
    \end{claim*}

    \begin{proof}
      Consider the first segment of any partition: $(0, x_1)$. No matter how small $x_1$ is, there
      exists $n \in \N$ such that $1/n < x_1$. Therefore for all majorants we have $c_1 \geq 1$ and yet for all
      minorants we have $c_1 \leq 0$. So, when restricted to this first segment, we have $I(\phi^-) > I(\phi^+)$
      for all $\phi^-, \phi^+$ and, since every majorant is elsewhere less than every minorant, it is not
      possible that $\sup_{\phi^-} I(\phi^-) = \inf_{\phi^+} I(\phi^+)$ and hence the Riemann integral is
      undefined.
    \end{proof}

  \item
    \begin{claim*}
      $\int_0^1 f > 0$
    \end{claim*}
    \begin{proof}
      Suppose for a contradiction that $\int_0^1 f = 0$. Fix an arbitrary minorant $\phi^-$, adapted to a
      partition of length $n$. Then we have that $\sum_{i=1}^n c_i(x_i - x_{i-1}) \leq 0$.
      Since $x_i \geq x_{i-1}$ for all $i$, and since $x_0 = 0 < x_n = 1$, it must be the case
      that $x_i - x_{i-1} > 0$ for some $i$, and therefore that $c_i > 0$ for some $i$. Therefore $f$ vanishes at
      at least one point. This contradiction proves that $\int_0^1 f \neq 0$.

      To see that it's not negative, note that for every majorant $\phi^+$ we have $x_i - x_{i-1} \geq 0$
      and $c_i > 0$ for all $i$ and therefore $I(\phi^+) = \sum_{i=1}^n c_i(x_i - x_{i-1}) \geq 0$. Therefore $\int_0^1 f > 0$.
    \end{proof}

  \end{enumerate}

  \newpage
  \begin{mdframed}
    \includegraphics[width=400pt]{img/analysis--berkeley-202a--homework-1-a577.png}
  \end{mdframed}

  \begin{proof}
    Suppose for a contradiction that $[0, 1] \subset \R$ is countable. Fix an enumeration $\{u_n ~|~ n \in \N\}$
    of the elements of $[0, 1]$.

    If $u_1 > u_2$ then relabel them so that $u_1 < u_2$. Set $U_1 = (u_1, u_2)$.

    Continue examining the numbers in the enumeration (starting at $u_3$) until two numbers have been encountered
    that are both in $(u_1, u_2)$. Form an interval from this pair and label it $U_2$. Continue examining the
    numbers in the enumeration until two numbers are encountered that are both in $U_2$; label this
    interval $U_3$. Continue in this fashion indefinitely.

    We will write $(U_{i1}, U_{i2})$ to refer to the endpoints of interval $i$.

    There are two cases:

    \begin{enumerate}
    \item {\bf The process terminates.}\\
      Then there is a last interval $U_L = (U_{L1}, U_{L2})$. It is possible that there is one (but not more than
      one) element $u^*$ of the original enumeration that is present in the interval $U_L$. If that is so, then
      every element of $U_L \setminus \{u^*\}$ is a real number not in the original enumeration; otherwise every
      element of $U_L$ is a real number not in the original enumeration.

    \item {\bf The process does not terminate.}\\
      Note that the sequence of interval lower bounds $(U_{i1})_{i\in\N}$ forms a strictly increasing sequence
      bounded above by $u_2$ and that the sequence of interval upper bounds $(U_{i2})_{i\in\N}$ forms a strictly
      decreasing sequence bounded below by $u_1$. By the Monotone Convergence theorem, both sequences converge:
      let these limits be $\alpha$ and $\beta$ respectively. There are two cases:
      \begin{enumerate}
      \item {\bf $\alpha < \beta$}\\
        Then every element of $(\alpha, \beta)$ is a real number not in the original enumeration.
      \item {\bf $\alpha = \beta$}\\
        Then $\alpha$ is a real number not in the original enumeration.
      \end{enumerate}
    \end{enumerate}

    In all cases, we found a real number that was not present in the original enumeration. But this is a
    contradiction, since the original enumeration contains all real numbers. Therefore no such enumeration exists
    and the real numbers are not countable.
  \end{proof}



\section*{Math 202a - HW2 - Dan Davison - \texttt{ddavison@berkeley.edu}}

\begin{mdframed}
  \includegraphics[width=400pt]{img/analysis--berkeley-202a-ebe4.png}
\end{mdframed}

\begin{intuition*}
  $O \subset \R$ is a countable union of disjoint open intervals.

  $x \sim y$ iff $x$ and $y$ are in the same interval.

  The length of $O$ should be the sum of the lengths of the intervals.
\end{intuition*}


\begin{enumerate}[label=(1.\arabic*)]
\item
  \begin{claim*}
    $\sim$ is an equivalence relation on $O$.
  \end{claim*}
  \begin{proof}
    \begin{enumerate}
    \item {\bf Reflexivity}\\
      $x \sim x$ since $[x, x] = \{x\} \subseteq O$.

    \item {\bf Symmetry}\\
      Let $x, y \in \R$ such that $x \sim y$. Then $[\min\{x, y\}, \max\{x, y\}] \subseteq O$.
      Therefore $[\min\{y, x\}, \max\{y, x\}] \subseteq O$. Therefore $y \sim x$.

    \item {\bf Transitivity}\\
      Let $x, y, z \in \R$ such that $x \sim y$ and $y \sim z$. Then $[\min\{x, y\}, \max\{x, y\}] \subseteq O$
      and $[\min\{y, z\}, \max\{y, z\}] \subseteq O$. Therefore $[\min\{x, y\}, \max\{y, z\}] \subseteq O$.
      Therefore $x \sim z$.
    \end{enumerate}
  \end{proof}

\item
  \begin{claim*}
    $O$ may be written as a countable union of disjoint open intervals.
  \end{claim*}

  \begin{proof}
    Let $\mc I = I_1, I_2, \ldots$ be the set of equivalence classes of $O$ under $\sim$.

    Since $\sim$ is an equivalence relation, the elements of $\mc I$ are disjoint and their union is equal
    to $O$.

    We now show that the elements of $\mc I$ are open sets. Let $I \in \mc I$ and suppose for a contradiction
    that $I$ is not open. Then there exists $x \in I$ such that no neighborhood of $x$ is contained within $I$.
    Let $x$ be such a point. Since $O$ is open, we may choose $\eps > 0$ such
    that $(x-\eps, x+\eps) \subseteq O$. Since $I$ is not open, for all $\eps' < \eps$, we have
    that $(x-\eps', x+\eps')$ contains a point outside $I$ in some other interval $J \in \mc I$
    where $J \neq I$. But this contradicts the disjointness of the partition $\mc I$. Therefore $I$ is open.

    Note that $I$ has at least two elements since $I$ is an equivalence class. It follows from transitivity of
    the equivalence relation that the elements of $\mc I$ are intervals.

    % Finally we show that the elements of $\mc I$ are intervals. Let $I \in \mc I$. Note that $I$ has at least
    % two elements since $I$ is an equivalence class. Suppose $I$ is bounded below and above. Then for
    % any $\epsilon > 0$ there exists $a \in I$ such that $a - \inf O < \epsilon$. Note that $[a, x] \subseteq I$
    % for all $x \in I$ where $a < x$. By a similar argument, $[x, b] \subseteq I$ for all $x \in I$
    % where $b = \sup I$ and $x < b$.  of the form $(I_a, I_b)$

    Finally we show that this is a countable union.

    Note that every rational number is in zero or one interval, but not more than one. Furthermore, every open
    interval contains at least one rational.

    Therefore there is a non-injective surjection from a subset of the rationals to the set of intervals.

    Therefore the cardinality of the set of intervals is not greater than the cardinality of the rationals.

    Therefore the set of intervals is countable.
  \end{proof}

\item
  \begin{proof}
    We may assign a length $\mu(O)$ to $O$ as follows:

    If $O = \emptyset$ then $\mu(O) := 0$.

    Otherwise, if $O$ is not bounded below, or if $O$ is not bounded above, then $\mu(O) := \infty$.

    Otherwise, if the series $\sum_i |I_i|$ diverges, then $\mu(O) := \infty$.

    Otherwise, $\mu(O) := \sum_i |I_i|$.

    Note that every term of the series is positive. In order for this definition to be unambiguous, the
    value $\mu(O)$ must not depend on the ordering of the series. This is true by the lemma below.
  \end{proof}

  \begin{lemma*}
    Let $\sum_i a_i$ be a series with $a_i > 0$ for all $i$. Then
    \begin{enumerate}
    \item If the series diverges for any ordering of the series, it diverges for all orderings.
    \item If the series converges for any ordering of the series, it converges to the same value for all
      orderings.
    \end{enumerate}
  \end{lemma*}
  \begin{proof}
    A sketch proof of the second statement is as follows: given any $\epsilon > 0$ we can identify a tail of
    the sequence whose sum is less than $\epsilon$. Thus the sum of the series is determined by the finite
    head. The sum of this finite head does not depend on its ordering, by commutativity of addition.
  \end{proof}
\end{enumerate}


\newpage
\begin{mdframed}
  \includegraphics[width=400pt]{img/analysis--berkeley-202a-8d78.png}
\end{mdframed}

\begin{enumerate}[label=(2.\arabic*)]
\item
  \begin{claim*}
    $C$ is compact.
  \end{claim*}
  \begin{proof}
    Since $C \subset \R$ it suffices to show that $C$ is closed and bounded. Then it follows from the Heine-Borel
    theorem that $C$ is compact.

    $C$ is bounded below by $0$ and above by $1$, since it is constructed by removing points from $[0, 1]$.

    To show that $C$ is closed we may show that $C^c$ is open. Since $C = \bigcap_{n=0}^\infty C_n$, we
    have $C^c = \bigcup_{n=0}^\infty C_n^c$. Note that $C_n$ is a union of closed intervals; therefore $C_n^c$
    is a union of open intervals and therefore open (if an interval contains a neighborhood of each one of its
    points then the union of intervals also contains neighborhoods of those points); therefore $C^c$ is a union
    of open intervals and therefore open. Therefore $C$ is closed.
  \end{proof}

\item
  \begin{claim*}
    $C$ is uncountable.
  \end{claim*}

  \begin{proof}
    Note that $\om \in C$ if and only if the base 3 (ternary) expansion of $\om$ contains no $1$s.

    Consider the map $f:C \to [0, 1]$ defined by the following rule: $f(\om)$ is equal to the real number whose
    binary expansion is formed by substituting every $2$ with a $1$ in the ternary expansion of $\om$.

    This map is a bijection, therefore the cardinality of $C$ is equal to that of $[0, 1]$, therefore $C$ is
    uncountable.
  \end{proof}

\item \begin{claim*}
    $C$ is negligible.
  \end{claim*}

  \begin{proof}
    Fix an arbitrary $\epsilon > 0$. We will show that there exists a countable union of
    intervals $I_1, I_2, \ldots$ that cover $C$ and for which $\sum_k |I_k| < \epsilon$.

    Note that $C_n$ comprises $2^n$ disjoint intervals each of length $3^{-n}$. Therefore the total length
    of $C_n$ is $|C_n| = \big(\frac{2}{3}\big)^n$ and we see that $|C_n| < \epsilon$ for
    all $n > \Big\lceil\frac{\log\eps}{\log 2/3} \Big\rceil$. Since $C \subset C_n$ for all $n$, we see
    that $|C| < \eps$. We can write $C$ as $C = \bigcup_{k=1}^\infty I_k$, and so we can construct an efficient
    cover for $C$ as follows: for $k \in \{1, 2, \ldots\}$ place an interval of length $2^{-k}\eps$ over $I_k$.
    The total length of the cover is $\eps\sum_{k=1}^\infty 2^{-k} = \eps$.

%    \red{TODO} Tie this up by giving explicit coordinates for the intervals in the covering.
  \end{proof}

\end{enumerate}

\newpage
\begin{mdframed}
  \includegraphics[width=400pt]{img/analysis--berkeley-202a-6b7a.png}
\end{mdframed}

\begin{enumerate}[label=(3.\arabic*)]

\item
  \begin{definition*}
    Let $X$ be a set formed by iteratively removing open intervals from $[0, 1]$. Let $I_1, I_2, \ldots$ be the
    open intervals that were removed in the formation of $X$. Note that these are disjoint, since a point can
    not be removed more than once. Define the measure of $X$ to be $1 - \sum_k |I_k|$.
  \end{definition*}

\item
  \begin{definition*}[Cantor set of measure $a$]
    Let $a \in (0, 1)$. The Cantor set of measure $a$ is formed as follows:

    Instead of removing $1/3$ at each iteration, we will remove a smaller fraction.

    Note that $\sum_{n=1}^\infty \frac{1 - a}{2^n} = 1 - a$. So we will design an algorithm that
    removes $\frac{1-a}{2^n}$ at each iteration, for $n=1, 2, \ldots$. Note that at the start of iteration $n$
    we have $n$ intervals. Therefore, in order to remove a length of $\frac{1-a}{2^n}$ we will remove the
    middle $\frac{1-a}{n2^n}$ from each interval.
  \end{definition*}
\end{enumerate}



\newpage
\begin{mdframed}
  \includegraphics[width=400pt]{img/analysis--berkeley-202a-8ce8.png}
\end{mdframed}

\begin{lemma}\label{lemma-4-1}
  $F_n(1) = 1$ for all $n$.
\end{lemma}

\begin{proof}
  For all $n$ we have that $C_n$ is a union of $2^n$ intervals, each of length $1/3^n$, therefore the total
  length is decreasing: $|C_n| = (2/3)^n$. The function $f_n$ has the value $(3/2)^n$ on each interval in $C_n$
  and zero elsewhere. Therefore $F_n(1) = \(\frac{3}{2}\)^n\(\frac{2}{3}\)^n = 1$ for all $n$.
\end{proof}

\begin{lemma}\label{lemma-4-2}
  Let $x \in C^c$. There exists $m$ such that $F_{n+1}(x) = F_n(x)$ for all $n > m$.
\end{lemma}
\begin{proof}
  Let $x \in C^c$. Then there exists $m$ such that for all $n > m$ we have $x \in C_n^c$. Let $m$ be such a
  value and fix an arbitrary $n > m$. We write $C_n$ as a union of $2^n$ closed
  intervals, $C_n = \bigcup_{i=1}^{2^n} [a_i, b_i]$, and let $k = |\{i \in \{1, \ldots, 2^n\} ~:~ b_i < x\}|$
  be the number of these intervals whose right endpoints are less than $x$. We have
  \begin{align*}
    F_n(x) = k\Big(\frac{3}{2}\Big)^n\Big(\frac{1}{3}\Big)^n = \frac{k}{2^n}.
  \end{align*}
  At the next generation, there are $2k$ of these intervals whose right endpoints are less than $x$, and we
  have
  \begin{align*}
    F_{n+1}(x) = 2k\Big(\frac{3}{2}\Big)^{n+1}\Big(\frac{1}{3}\Big)^{n+1} = \frac{k}{2^n}.
  \end{align*}
  Therefore $F_{n+1}(x) = F_n(x)$ for all $n > m$.
\end{proof}


\begin{lemma}\label{lemma-4-3}
  Let $[a, b] \subset C$. Then $\int_a^b \big|f_{n+1}(x) - f_n(x)\big| \dx = \frac{1}{3}\frac{1}{2^{n-1}}$.
\end{lemma}

\begin{proof}
  Let $[a, b] \subset C$. For $x$ in the left or right thirds (closed) of this interval we have
  \begin{align*}
    f_{n+1}(x) - f_n(x)
    = \Big(\frac{3}{2}\Big)^{n+1} - \Big(\frac{3}{2}\Big)^{n}
    = \frac{1}{2}\Big(\frac{3}{2}\Big)^n,
  \end{align*}
  and for $x$ in the middle third (open) of this interval we have
  \begin{align*}
    f_{n+1}(x) - f_n(x)
    = -\Big(\frac{3}{2}\Big)^{n}.
  \end{align*}
  Since the interval has length $(1/3)^n$ we have
  \begin{align*}
    \int_a^b \big|f_{n+1}(x) - f_n(x)\big| \dx
    &=  \frac{2}{3}\Big(\frac{1}{3}\Big)^n\frac{1}{2}\Big(\frac{3}{2}\Big)^n
      + \frac{1}{3}\Big(\frac{1}{3}\Big)^n\Big(\frac{3}{2}\Big)^n \\
      % &= \Big(\frac{1}{3}\Big)^n\Big(\frac{3}{2}\Big)^n\Big(\frac{2}{3}\frac{1}{2} + \frac{1}{3}\Big) \\
    &= 2\Big(\frac{1}{3}\Big)^{n+1}\Big(\frac{3}{2}\Big)^n\\
    &= \frac{1}{3}\frac{1}{2^{n-1}}.
  \end{align*}
\end{proof}


\begin{enumerate}[label=(4.\arabic*)]

\item
  \begin{claim*}
    For each $x \in [0, 1]$ the limit $F(x) = \lim_{n\to\infty} F_n(x)$ exists.
  \end{claim*}

  \begin{proof}
    We will study the difference $|F_{n+1}(x) - F_n(x)|$ and show that this decreases with $n$ in a way that
    implies that the sequence $F_0(x), F_1(x), \ldots$ is Cauchy for all $x \in [0, 1]$.

    First consider $x \in C^c$. Then from lemma \ref{lemma-4-2} we have that $F_{n+1}(x) - F_{n}(x) = 0$ for
    sufficiently large $n$ and so the sequence is obviously Cauchy at a point $x \in C^c$.

    Next let $x \in C$ and let $[a, b] \subset C$ be the closed interval containing $x$. Then
    \begin{align*}
      F_{n+1}(x) - F_n(x)
      &= \Big(\int_0^a f_{n+1}(t) \dt + \int_a^x f_{n+1}(t) \dt\Big)
        - \Big(\int_0^a f_{n}(t) \dt   + \int_a^x f_{n}(t) \dt\Big) \\
      &= F_{n+1}(a) - F_{n}(a)
        + \Big(\int_a^x f_{n+1}(t) \dt - \int_a^x f_{n}(t) \dt\Big).
    \end{align*}
    Now, from lemma \ref{lemma-4-2} we have that $F_{n+1}(x) - F_{n}(x) = 0$ for sufficiently large $n$,
    where $x \in C^c$. But this result also holds for $x$ an endpoint of a closed interval in $C$, since such
    an endpoint is arbitrarily close to a point of $C^c$. Thus we have $F_{n+1}(a) - F_{n}(a) = 0$ and, using
    lemma \ref{lemma-4-3},
    \begin{align*}
      \Big|F_{n+1}(x) - F_n(x)\Big|
      &=    \Big|\int_a^x f_{n+1}(t) \dt - f_{n}(t) \dt\Big| \\
      &\leq \int_a^x \big|f_{n+1}(t) \dt - f_{n}(t)\big| \dt \\
      &\leq \int_a^b \big|f_{n+1}(t) \dt - f_{n}(t)\big| \dt \\
      &=    \frac{1}{3}\frac{1}{2^{n-1}} \\
      &<    \frac{1}{2^n}.
    \end{align*}
    In order to show that $F_0(x), F_1(x), \ldots$ is Cauchy, fix $0 < \eps < 1$, let $m \in \N$ be such
    that $\sum_{k=1}^m \frac{1}{2^k} \geq 1 - \eps$, and let $i, j > m$. Then
    \begin{align*}
      \Big|F_i(x) - F_j(x)\Big| \leq \sum_{k=m+1}^\infty \frac{1}{2^k} < \eps.
    \end{align*}
    Therefore the sequence $F_0(x), F_1(x), \ldots$ is Cauchy at a point $x \in C$.

    Since the sequence $F_0(x), F_1(x), \ldots$ is Cauchy for all $x \in [0, 1]$, the
    limit $F(x) = \lim_{n\to\infty} F_n(x)$ exists for all $x \in [0, 1]$.

    Furthermore, the same $m$ works for all $x$, i.e. the sequence is uniformly Cauchy.
  \end{proof}

\item
  \begin{claim*}
    $F$ is continuous on $[0, 1]$ with $F(0) = 0$ and $F(1) = 1$.
  \end{claim*}

  \begin{proof}
    For continuity of $F$ it suffices to prove that the $F_n$ are continuous, since in part (4.1) we proved
    that they are uniformly Cauchy and hence converge uniformly to $F$. Continuity of $F$ then follows from the
    uniform limit theorem.

    Informally, the $F_n$ are piecewise affine and thus obviously continuous. To prove this, note that from
    the fundamental theorem of calculus we have that $F'_n = f_n$. Therefore $|F'_n|$ is bounded above
    by $(3/2)^n$, hence $F_n$ is Lipschitz continuous and therefore continuous.

    Finally, we have
    \begin{align*}
      F(0) &= \lim_{n\to\infty} \int_0^0 f_n(0) \dt = 0,
    \end{align*}
    and, from lemma \ref{lemma-4-1} we have
    \begin{align*}
      F(1)
      &= \lim_{n\to\infty} \int_0^1 f_n(t) \dt \\
      &= \lim_{n\to\infty} 1 \\
      &= 1.
    \end{align*}
  \end{proof}


\item
  \begin{proof}
    For every point $x \in C^c$, $F$ is constant within a neighborhood of $x$. Since we showed in question 2
    that $C$ is a null set (i.e. negligible), this implies that for $x$ outside a null set (``almost
    every $x$​'') $F$ is differentiable with $F'(x) = 0$.
  \end{proof}
\end{enumerate}


\newpage
\begin{mdframed}
  \includegraphics[width=400pt]{img/analysis--berkeley-202a-hw-8c2b.png}
\end{mdframed}

\begin{enumerate}[label=(5.\arabic*)]

\item
  \begin{claim*}
    The orbit $S_{\lambda}(x)$ is dense for all $x \in (0, 1]$ iff there exists an $x$ for which it is dense.
  \end{claim*}
  \begin{proof}
    Let $x$ be such that $S_\lambda(x)$ is dense. Let $y \neq x$. Since the orbit of $x$ is dense, the sequence
    starting at $x$ will visit a point $y'$ arbitrarily close to $y$. The set of points in the tail of the
    sequence, after visiting $y'$, is the orbit of $y'$. Since the orbit of $x$ is dense, the set of points in
    any tail is also dense, hence $\orb(y')$ is dense.

    But $y'$ differs from $y$ by an arbitrarily small epsilon. Since the transformation is additive, the $i$-th
    element in the sequence starting at $y$ differs from the corresponding element in the sequence starting
    at $y'$ by this same $\eps$. It follows that $\orb(y)$ is dense also.
  \end{proof}

\item
% \begin{lemma}\label{lemma-5-1}
%   \begin{align*}
%     S^n_\lambda(x)
%     &= x + S^n_\lambda(1) \mod 1 \\
%     &= x + n\lambda \mod 1
%   \end{align*}
% \end{lemma}
  \begin{claim*}
    The orbits are dense iff $\lambda \in (0, 1]$ is irrational.
  \end{claim*}
  \begin{proof}
    Let $j, k \in \N$ and let $\lambda = \frac{j}{k}$ be a rational number in reduced form.

    Note that $S^n_\lambda(x) = x + n\lambda \mod 1$. Therefore
    \begin{align*}
      S^k_\lambda(x)
      &= x + k\frac{j}{k} \mod 1 \\
      &= x + 1 \mod 1 \\
      &= x.
    \end{align*}
    Therefore if $\lambda$ is rational, the sequence returns to its starting point after finitely many
    iterations. Therefore $\orb(x)$ under $S_\lambda$ is a finite set, therefore it is not dense in $(0, 1]$.

    Now suppose $\lambda$ is irrational. Then
    \begin{align*}
      S^k_\lambda(x) = x + k\lambda \mod 1.
    \end{align*}
    Since $k\lambda = i$ has no solutions for irrational $\lambda$ and integers $i, k$, we see that the
    sequence never returns to its starting point.
  \end{proof}


\item
  \begin{definition*}
    The $n$-th occupation fraction of $I$ by $S_\lambda$ is
    \begin{align*}
      O_n(I, \lambda) = \frac{1}{n}\Big|\Big\{i \in \{1, \ldots, n\} ~:~ a_n \in I\Big\}\Big|.
    \end{align*}
  \end{definition*}
  \begin{claim*}
    The occupation fraction has a limiting value $\lim_{n\to\infty}O_n(I, \lambda) = b - a$.
  \end{claim*}
\item

  \begin{proof}[Proof sketch]
    Focus on the $j$-th digit in the binary expansion of $a_n$, and consider the orbit of that digit alone.
    That's a sequence of $0$s and $1$s that we may interpret as a real number in $[0, 1]$. The claimed result
    would be proved if we can prove the following:

    \begin{enumerate}
    \item The real number corresponding to the sequence visited by the $j$-th digit is normal, for all $j$.
    \item The sequence for digit $j$ becomes (in an appropriate sense) uncorrelated with the sequence for
      digit $k \neq j$.
    \end{enumerate}

    Those two results together would imply that each digit is visiting $0$ and $1$ with equal frequency,
    independently of other digits, and therefore that $a_n$ itself has no tendency to occupy any particular
    dyadic interval more than any other dyadic interval, from which the claimed result follows.

    In order to prove those results, I would investigate the following direction:
    \begin{enumerate}
    \item Note that $\lambda$ is irrational, therefore has a non-repeating binary expansion.

    \item Study the behavior of the binary expansion of $a_n$ under repeated addition of $\lambda$, i.e. with carrying and the mod 1 operation.
    \end{enumerate}
  \end{proof}
\end{enumerate}
\end{comment}


\section*{Math 202a - HW3 - Dan Davison - \texttt{ddavison@berkeley.edu}}

\begin{enumerate}
\item~\\
  \begin{mdframed}
    \includegraphics[width=400pt]{img/analysis--berkeley-202a-hw-e32d.png}
  \end{mdframed}
  \green{DONE}
  \begin{enumerate}[label=(\alph*)]
  \item
    \begin{claim*}[countable union of algebras is an algebra]
      Let $\ms F_1, \ms F_2, \ldots$ be algebras on $\Omega$ (collections of events).
      Then $\bigcup_{n=1}^\infty \ms{F}_n$ is an algebra.
    \end{claim*}

    \begin{proof}
      Let $\ms F = \bigcup_{n=1}^\infty \ms F_n$. We must show that
      \begin{enumerate}
      \item $\Omega \in \ms F$ \\
        {\bf Proof:} $\ms F_1$ is an algebra, therefore $\Omega \in \ms F_1$, therefore $\Omega \in \ms F$.
      \item If $A \in \ms F$ then $A^c \in \ms F$ \\
        {\bf Proof:} If $A \in \ms F$ then $A \in F_n$ for some $n$. Therefore $A^c \in \ms F_n$. Therefore $A^c \in \ms F$.
      \item If $A, B \in \ms F$ then $A \cup B \in \ms F$ \\
        {\bf Proof:} If $A, B \in \ms F$ then for some $m$ and $n$ we have $A \in \ms F_m$ and $B \in \ms F_n$.
        Suppose $m = n$. Then $A \cup B \in \ms F_m$, therefore $A \cup B \in \ms F$. Alternatively
        suppose $m \neq n$. Then either $\ms F_m \subset \ms F_n$ or $\ms F_n \subset \ms F_m$. Suppose without
        loss of generality that $\ms F_m \subset \ms F_n$. Then $A, B \in \ms F_n$,
        therefore $A \cup B \in \ms F_n$, therefore $A \cup B \in \ms F$.
      \end{enumerate}
    \end{proof}
  \item
    \begin{claim*}[countable union of nested $\sigma$-algebras may not be a $\sigma$-algebra]
      Let $\ms F_1, \ms F_2, \ldots$ be $\sigma$-algebras satisfying $\ms F_n \subset \ms F_{n+1}$.
      Then $\bigcup_{n=1}^\infty \ms{F}_n$ may not be a $\sigma$-algebra.
    \end{claim*}
    % \begin{proof}
    %   It suffices to exhibit an example.

    %   Let $\Omega = \{1, 2, \ldots\}$.

    %   Define
    %   \begin{align*}
    %     \ms F_n = &\{A ~:~ A \subset \{1, \ldots, n\} \} ~ \bigcup ~ \{A ~:~ A^c \subset \{1, \ldots, n\} \}.
    %   \end{align*}


    % \end{proof}

    \begin{proof}
      Let $\Omega = (0, 1]$, let $\ms A_n$ be the set of rank-$n$ dyadic intervals in $[0, 1]$ and
      define $\ms F_n = \sigma(\ms A_n)$, the $\sigma$-algebra generated by $\ms F_n$. Then for example we have
      \begin{align*}
        \ms F_1 &= \Big\{\emptyset,
                         (0, .5], (.5, 1],
                         (0, 1]
                   \Big\} \\
        % \ms F_2 &= \Big\{\emptyset,
        %                  (0, .25], (.25, .5], (.5, .75], (.75, 1],
        %                  (0, .5], (.5, 1],
        %                  (0, .75], (.25, .75],
        %                  (0, 1],
        %                  (.25, 1]
        %             \Big\}
      \end{align*}
      Note however that $(1 - 2^{-n}, 1] \in \ms F_n$ and that $\bigcap_{n=1}^\infty (1 - 2^{-n}, 1] = \{1\}$.
      Therefore if $\bigcup_{n=1}^\infty \ms F_n$ is a $\sigma$-algebra
      then $\{1\} \in \bigcup_{n=1}^\infty \ms F_n$.

      But $\{1\}$ is not a dyadic interval, therefore there is no $n$ for which $\{1\} \in \ms A_n$.
      Furthermore there is no $n$ for which $\{1\} \in \sigma(\ms A_n)$ (justification below).

      Therefore $\{1\} \notin \bigcup_{n=1}^\infty \ms F_n$ and so $\bigcup_{n=1}^\infty \ms F_n$ is not a $\sigma$-algebra.

      ~\\
      \textbf{Justification that there is no $n$ for which $\{1\} \in \sigma(\ms A_n)$:}

      By definition, $\sigma(\ms A_n)$ is the intersection of all $\sigma$-algebras that include $\ms A_n$.
      Suppose $\{1\} \in \sigma(\ms A_n)$. Now form a new class of sets $\sigma^*(\ms A_n)$ by removing
      from $\sigma(\ms A_n)$ every set that contains $1$ as an isolated point, and its complement. We claim
      that $\sigma^*(\ms A_n)$ is a $\sigma$-algebra. Note that none of the removed sets were in $\ms A_n$
      (since they are not dyadic intervals). But
      then $\ms A_n \subseteq \sigma^*(\ms A_n) \subset \sigma(\ms A_n)$ which contradicts the definition
      of $\sigma(\ms A_n)$. Therefore $\{1\} \notin \sigma(\ms A_n)$.
    \end{proof}
  \end{enumerate}

\newpage
\item~\\
 \green{DONE}
  \begin{mdframed}
    \includegraphics[width=400pt]{img/analysis--berkeley-202a-hw-ab18.png}
  \end{mdframed}
  Let $\ms F_1, \ms F_2, \ldots$ be the collection of all algebras in $\Omega$ for which $\ms A \subset \ms F_n$,
  so that $f(\ms A) = \bigcap_{n}\ms F_n$.

  \begin{enumerate}[label=(\alph*)]

  \item
    \begin{claim*}
      $f(\ms A)$is an algebra
    \end{claim*}
    \begin{proof}
      We must show that
      \begin{enumerate}
      \item $\Omega \in f(\ms A)$\\
        {\bf Proof:} $\Omega \in \ms F_n$ for all $n$, therefore $\Omega \in \bigcap_{n}\ms F_n = f(\ms A)$.

      \item If $X \in f(\ms A)$ then $X^c \in f(\ms A)$\\
        {\bf Proof:} If $X \in f(\ms A)$ then $X \in \ms F_n$ for all $n$, therefore $X^c \in \ms F_n$ for all $n$, therefore $X^c \in \bigcap_{n}\ms F_n = f(\ms A)$.

      \item If $X, Y \in f(\ms A)$ then $X \cup Y \in f(\ms A)$\\
        {\bf Proof:} If $X, Y \in f(\ms A)$ then $X, Y \in \ms F_n$ for all $n$, therefore $X \cup Y \in \ms F_n$ for all $n$, therefore $X \cup Y \in f(\ms A)$.
      \end{enumerate}
    \end{proof}

    \begin{claim*}
      $\ms A \subset f(\ms A)$
    \end{claim*}
    \begin{proof}
      Let $X \in \ms A$. Then $X \in \ms F_n$ for all $n$. Therefore $X \in \bigcap_{n}\ms F_n = f(\ms A)$. Therefore $\ms A \subset f(\ms A)$.
    \end{proof}

    \begin{claim*}
      $f(\ms A)$ is minimal in the sense that if $\ms G$ is an algebra and $\ms A \subset \ms G$, then $f(\ms A) \subset \ms G$.
    \end{claim*}
    \begin{proof}
      If $\ms G$ is an algebra with $\ms A \subset \ms G$ then $\ms G \in \{\ms F_1, \ms F_2, \ldots\}$, therefore $\ms G \supset \bigcap_{n}\ms F_n = f(\ms A)$.
    \end{proof}

  \item
    \begin{proof}
      Let $\ms B$ be the class of sets of the form $\bigcup_{i=1}^m \bigcap_{j=1}^{n_i} A_{ij}$, where
      either $A_{ij} \in \ms A$ or $A^c_{ij} \in \ms A$,
      with $\big\{\bigcap_{j=1}^{n_i} A_{ij} ~ : ~ i \in \{1, \ldots m\}\big\}$ disjoint.

      For inclusion in one direction, let $B \in \ms B$. Then $B$ is formed from the elements of $\ms A$ by
      taking complements, finite unions and finite intersections. Therefore $B \in f(\ms A)$.
      Therefore $\ms B \subseteq f(\ms A)$.

      To prove equality, we will show that taking finite unions and complements never leads to an element
      outside $f(\ms A)$.

      First consider finite intersection. If $B = \bigcup_{i=1}^m \bigcap_{j=1}^{n_i} A_{ij}$
      and $B' = \bigcup_{i=1}^{m'} \bigcap_{j=1}^{n'_i} A_{ij}$, with the $A_{ij}$ disjoint as prescribed in
      both cases, then $B \cup B'$ is also a union of products of the form prescribed.

      Next consider complements. We see that
      \begin{align*}
        \Big(\bigcup_{i=1}^m \bigcap_{j=1}^{n_i} A_{ij}\Big)^c
        = \bigcap_{i=1}^m \bigcup_{j=1}^{n_i} A^c_{ij}
        \in f(\ms A),
      \end{align*}
      i.e. the complement of a set in $\ms B$ is a set in $f(\ms A)$.
    \end{proof}

  \end{enumerate}
\newpage
\item~\\
  \begin{mdframed}
    \includegraphics[width=400pt]{img/analysis--berkeley-202a-hw-fc85.png}
  \end{mdframed}
    \begin{definition*}
      Let $\Omega = (0, 1]$ and let $\ms O$ be the collection of open subsets of $\Omega$. Each element of the
      Borel $\sigma$-algebra $\ms B := \sigma(\ms O)$ is a Borel set.
    \end{definition*}

    \begin{lemma}\label{hw3-2-11-lemma-1}
      Let $\ms I_1 = \{(a, b) ~:~ a, b \in \R\}$. Then $\sigma(\ms I_1) = \ms B$.
    \end{lemma}
    \begin{proof}
      Let $\Omega = (0, 1]$ and let $\ms O$ be the collection of open subsets of $\Omega$.

      Let $\ms I_1 = \{(a, b) ~:~ a, b \in \R\}$. We want to show that $\sigma(\ms I_1) = \ms B := \sigma(\ms O)$.

      In one direction, every element of $\ms I_1$ is open, so clearly $\sigma(\ms I_1) \subseteq \sigma(\ms O)$.

      For the other direction, let $X \in \ms O$ be an open subset of $\R$. Then $X$ is a countable union of
      open intervals (i.e. finite intervals and open rays). Every finite interval is in $\ms I_1$. But an open
      ray is also a countable union of finite intervals: $(-\infty, a) = \bigcup_n^\infty (a-n, a)$
      and $(a, \infty) = \bigcup_n^\infty (a, a + n)$. Therefore $X \in \sigma(\ms I_1)$, i.e. every open set
      is in the $\sigma$-algebra generated by open intervals. This is equivalent to the
      statement $\ms O \subseteq \sigma(\ms I_1)$, i.e. the collection of all open sets is a subset of
      that $\sigma$-algebra. Therefore $\sigma(\ms O) \subseteq \sigma(\sigma(\ms I_1)) = \sigma(\ms I_1)$.
    \end{proof}

  \begin{enumerate}[label=(\alph*)]

  \item
    \begin{claim*}
      The $\sigma$-algebra $\ms B$ of Borel sets is countably generated.
    \end{claim*}
    \begin{proof}
      let $\ms O$ be the collection of open sets of $\Omega = (0, 1]$, so that $\ms B = \sigma(\ms O)$.

      Let $\ms I_1 = \{(a, b) ~:~ a, b \in \R\}$ and $\ms I_2 = \{(p, q) ~:~ p, q \in \Q\}$

      Note that $\ms I_2$ is a countable set (the rationals are countable, and any finite Cartesian product of
      countable sets is countable.)

      We claim that $\sigma(\ms I_2) = \sigma(\ms O)$.

      Inclusion in one direction is immediate, since $\ms I_2 \subset \ms O$ and
      therefore $\sigma(\ms I_2) \subseteq\sigma(\ms O)$.

      For inclusion in the other direction, note that for all $a, b \in \R$ with $a < b$
      \begin{align*}
        (a, b) = \bigcup_{\substack{p, q \in \Q\\a < p < q < b}} (p, q).
      \end{align*}
      Therefore $\sigma(\ms I_1) \subseteq \sigma(\ms I_2)$. But $\sigma(\ms I_1) = \sigma(\ms O)$ from lemma
      (\ref{hw3-2-11-lemma-1}) hence $\sigma(\ms O) \subseteq \sigma(\ms I_2)$.
    \end{proof}
  \item
    \begin{claim*}
      Let $\ms F$ be a $\sigma$-algebra containing the countable and cocountable subsets of $\Omega$ ($A$ being
      cocountable if $A^c$ is countable). Then $\ms F$ is countably generated if and only if $\Omega$ is
      countable.
    \end{claim*}
    \begin{proof}
      First let $\Omega$ be countable. We want to show that there exists a countable class of sets that
      generates $\ms F$. Indeed, the class of all singletons generates $\ms F$ and is countable.

      Next let $\Omega$ be uncountable. We want to show that every class of sets that generates $\ms F$ is
      uncountable. Suppose for a contradiction that $\ms F$ is countably generated and let $\ms A$ be a
      countable class of sets that generates $\ms F$. Note that there exists a countable class of singletons
      from which every element of $\ms A$ can be formed as a countable union. Therefore $\ms F$ can be
      generated by this countable class of singletons. $\Omega_0$. Now consider $\Omega^c_0$. Let $\ms S$ be
      the class of singleton subsets of $\Omega^c_0$. We want to derive a contradiction, and presumably that
      contradiction is going to be concluding that $\Omega^c_0$ is countable when in fact we know it is
      uncountable, because $\Omega$ is. \red{TODO}
    \end{proof}
  \end{enumerate}

\newpage
\item~\\
  \begin{mdframed}
    \includegraphics[width=400pt]{img/analysis--berkeley-202a-hw-af2a.png}
  \end{mdframed}
  \begin{claim*}
    A $\sigma$-algebra cannot be countably infinite.
  \end{claim*}
  \begin{proof}
    Let $\ms A$ be a non-finite $\sigma$-algebra on $\Omega$. If (\red{TODO}) we can show that $\ms A$ contains a
    countable set of singletons then we are done, because then $\ms A$ contains all subsets that can be formed
    from those singletons by countable unions, in which case its cardinality is at least $2^{\aleph_0}$.
  \end{proof}


  \begin{claim*}
    An algebra can be countably infinite.
  \end{claim*}

  \begin{proof}

    Let $\Omega$ be $\{1, 2, \ldots\}$.

    An algebra is a collection of subsets of $\Omega$.

    \begin{align*}
      \{\emptyset, \{1, 2, \ldots\}, \{1\}, \{2, 3, \ldots\}, \{1, 2\}, \{3, 4, \ldots\}, \ldots\}
    \end{align*}

    \begin{align*}
      \{\emptyset, \{1, 2, \ldots\}, \{1\}, \{3, \ldots\}, \{1, 3\}, \{5, 7, \ldots\}, \ldots\}
    \end{align*}


    They must be closed under
    \begin{itemize}
    \item complements
    \item unions
    \item intersections
    \end{itemize}

  \end{proof}
\newpage
\item~\\
  \begin{mdframed}
    \includegraphics[width=400pt]{img/analysis--berkeley-202a-hw-b476.png}
  \end{mdframed}

  \begin{enumerate}[label=(\alph*)]

  \item (a)
    \begin{claim*}
      $D$ is finitely but not countably additive on $\ms D$.
    \end{claim*}
    \begin{proof}
      Let $A_1$ and $A_2$ be disjoint subsets of $\Omega = \{1, 2, \ldots\}$. Then
      \begin{align*}
        D(A_1 \cup A_2)
        &= \lim_{n\to\infty} \big|\big\{m ~:~ 1 \leq m \leq n, m \in A_1 \cup A_2 \big\}\big| \\
        &= \lim_{n\to\infty} \big|\big\{m ~:~ 1 \leq m \leq n, m \in A_1 \big\} \bigcup \big\{m ~:~ 1 \leq m \leq n, m \in A_2 \big\}\big|\\
        &= \lim_{n\to\infty} \big|\big\{m ~:~ 1 \leq m \leq n, m \in A_1 \big\}\big| + \big|\big\{m ~:~ 1 \leq m \leq n, m \in A_2 \big\}\big| \\
        &= \lim_{n\to\infty} \big|\big\{m ~:~ 1 \leq m \leq n, m \in A_1 \big\}\big| + \lim_{n\to\infty} \big|\big\{m ~:~ 1 \leq m \leq n, m \in A_2 \big\}\big| \\
        &= D(A_1) + D(A_2).
      \end{align*}
      Finite additivity then follows by induction. \red{TODO}

      To show that $\ms D$ is not countably additive, it is sufficient to provide a counter-example.

      Let $A_i = \{i\}$ for all $i \in \{1, 2, \ldots\}$ and let $\ms A = \bigcup_{i=1}^\infty A_i$ be the collection of all the singleton sets.

      Note that $D(A_i) = 0$ for all $i$. However $\bigcup_{i=1}^\infty A_i = \N$ and
      therefore $D\big(\bigcup_{i=1}^\infty A_i\big) = 1$. Therefore $D$ is not countably additive, since
      \begin{align*}
        \sum_{i=1}^\infty D(A_i) = \sum_{i=1}^\infty 0 = 0 \neq 1 = D\big(\bigcup_{i=1}^\infty A_i\big).
      \end{align*}
    \end{proof}

  \item (b)
    \begin{enumerate}
    \item
      \begin{claim*}
         $\ms D$ contains $\emptyset$ and $\Omega$.
      \end{claim*}
      \begin{proof}
        $P_n(\emptyset) = 0$ for all $n$, therefore the limit exists and
        is $D(\emptyset) := \lim_{n\to\infty} 0 = 0$, therefore $\emptyset \in \ms D$.

        $P_n(\Omega) = 1$ for all $n$, therefore the limit exists and
        is $D(\Omega) := \lim_{n\to\infty} 1 = 1$, therefore $\Omega \in \ms D$.
      \end{proof}
    \item
      \begin{claim*}
         $\ms D$ is closed under complementation, proper differences, and finite disjoint unions.
      \end{claim*}
      \begin{proof}
        \red{TODO}
      \end{proof}
    \item
      \begin{claim*}
         $\ms D$ is not closed under countable disjoint unions.
      \end{claim*}
      \begin{proof}
        An example of a subset of $\{1, 2, \ldots\}$ that has no density is the set of positive integers whose
        binary representation has an odd number of digits. This can be formed as a countable union of disjoint
        sets: (one-digit) $\bigcup$ (three-digits) $\bigcup \ldots$.

        Informally, this set consists of a stretch of consecutive integers that are included, followed by a
        longer stretch that are excluded, followed by a longer still stretch that are included, and so on. The
        reason there is no limiting density is that the density fluctuates, and the way in which the stretches
        increase in length means that the amplitude of the density fluctuations does not decrease to zero.

        \red{TODO} prove that $D(A) := \lim_{n\to\infty} P_n(A)$ does not exist.
      \end{proof}
    \item
      \begin{claim*}
         $\ms D$ is not closed under finite unions that are not disjoint.
      \end{claim*}
    \end{enumerate}

  \item (c)
    \begin{claim*}
      \begin{enumerate}
      \item $f(\ms M)$ is contained in $\ms D$.
      \item
      \end{enumerate}
    \end{claim*}
    \begin{proof}
      We have
      \begin{align*}
        M_1 &= \{1, 2, 3, \ldots\} \\
        M_2 &= \{2, 4, 6, \ldots\} \\
        M_3 &= \{3, 6, 9, \ldots\} \\
        \vdots
      \end{align*}

    \end{proof}
    \begin{claim*}
      $D$ is completely determined on $f(\ms M)$ by the value it gives for each $a$ to the event that $m$ is
      divisible by $a$.

      In other words:

      Let $A \in f(\ms M)$. Then $D(A)$ can be computed knowing only the values $D(M_1), D(M_2), D(M_3), \ldots$.
    \end{claim*}

  \item (d)
    \begin{claim*}
      $D$ is finitely additive but not countably additive on $f(\ms M)$.
    \end{claim*}
    \begin{proof}
      For $n \geq 2$ define
      \begin{align*}
        L_n = M_n \setminus \bigcup_{2 \leq i < n} M_i.
      \end{align*}
      Note that
      \begin{align*}
        D(L_2) &= 1/2 \\
        D(L_3) &< 1/3 \\
        D(L_4) &= 0 \\
        D(L_5) &< 1/5 \\
        D(L_6) &= 0 \\
        D(L_7) &< 1/7 \\
        D(L_8) &= 0 \\
        \vdots
      \end{align*}
      The $L_n$ are disjoint, and we have $D(\bigcup_{n \geq 2} L_n) = D(\{2, \ldots\}) = 1$.

      We want to show that $\sum_n D(L_n) \neq 1$ but that is not clear: it is given that the sum of the
      reciprocals of the primes diverges, but our sum is smaller than that.

      \red{TODO}
    \end{proof}

  \item (e)
    \begin{claim*}
      Let $\varphi(n)$ be Euler's function. Then
      \begin{align*}
        \frac{\varphi(n)}{n} = \prod_{p|n}\big(1 - \frac{1}{p}\big).
      \end{align*}
    \end{claim*}
    \begin{proof}
      Fix $n \in \{1, 2, \ldots\}$ and let $p_1, \ldots, p_r$ be the distinct prime factors of $n$, so that $n = \prod_{i=1}^rp_i^{k_i}$.
      Then
      \begin{align*}
        \frac{\varphi(n)}{n} &= \frac{\#\big\{i ~:~ 1 \leq i < n, i \text{~coprime with~} n\big\}}{n} \\
                             &= \frac{(n - 1) - \#\big\{i ~:~ 1 \leq i < n, i \text{~is a multiple of~} p_j \text{~for some~} 1 \leq j \leq r\big\}}{n} \\
                             &= \frac{(n - 1) - \#\big\{i ~:~ 1 \leq i < n, i \in \bigcup_{j=1}^r M_{p_j}\big\}}{n} \\
      \end{align*}



      Fix $n \in \{1, 2, \ldots\}$ and let $p_1, \ldots, p_r$ be the distinct prime factors of $n$.
      Then
      \begin{align*}
        \varphi(n)
        &= n P_n\Big(\big(\bigcup_{i=1}^r M_p\big)^c\Big) \\
        &= n \Big(1 - P_n\big(\bigcup_{i=1}^r M_p\big)\Big) \\
        &= n \Big(1 - \Big(\frac{1}{p_1} + \frac{1}{p_2} + \frac{1}{p_3} + \ldots - \frac{1}{p_1p_2} - \frac{1}{p_1p_3} - \ldots + (-1)^r\frac{1}{p_1p_2p_3\cdots p_r}\Big)\Big) \\
        &= n \prod_{i=1}^r(1 - p_i).
      \end{align*}
    \end{proof}

  \item (f)
    \begin{claim*}
	    $D(A) = x$ has a solution for all $0 \leq x \leq 1$.
    \end{claim*}
    \begin{proof}
      A set $A$ with density $ 0 \leq x \leq 1$ can be constructed by finding a series
      \begin{align*}
        \sum_{i=1}^\infty D(A_i)
      \end{align*}
      that converges to $x$ where the $A_i$ are disjoint elements of an algebra on which $D$ is countably
      additive.

      But we haven't yet identified any algebra on which $D$ is countably additive.
    \end{proof}

  \item (g)
    \begin{claim*}
      $D$ is translation invariant.
    \end{claim*}
  \end{enumerate}
  [Note that a in part (c) is a positive integer.]

\end{enumerate}


\newpage
\section*{Math 202a - HW4 - Dan Davison - \texttt{ddavison@berkeley.edu}}

\begin{mdframed}
\includegraphics[width=400pt]{img/analysis--berkeley-202a-hw-1e56.png}
\end{mdframed}
\begin{claim*}
  $\mu$ is a measure.
\end{claim*}
\begin{proof}
  It is given that $\mu$ is non-negative and that $\mu(\emptyset) = 0$. We must prove that $\mu$ is countably
  additive.

  So let $B_1, B_2, \dots$ be a disjoint collection of subsets of $X$. We want to show
  that $\mu(\bigcup_{i=1}^\infty B_i) = \sum_{i=1}^\infty \mu(B_i)$.

  Let's construct an increasing sequence of sets. Define $A_j = \bigcup_{i \leq j} B_i$ for $j=1, 2, \dots$, so
  that $A_1, A_2, \dots$ is an increasing sequence of sets. Note
  that $\bigcup_{i=1}^\infty B_i = \bigcup_{j=1}^\infty A_j$ therefore, by
  hypothesis,
  $\mu\big(\bigcup_{i=1}^\infty B_i\big) = \mu\big(\bigcup_{j=1}^\infty A_j\big) = \lim_{j\to\infty} \mu(A_j)$.

  Now, from finite additivity we have
  \begin{align*}
    \mu(A_j) = \mu(\bigcup_{i \leq j} B_i) = \sum_{i\leq j} \mu(B_i),
  \end{align*}
  therefore
  \begin{align*}
    \mu\big(\bigcup_{i=1}^\infty B_i\big) = \lim_{j\to\infty} \sum_{i\leq j} \mu(B_i) = \sum_{i=1}^\infty \mu(B_i),
  \end{align*}
  as required.
\end{proof}

\newpage
\begin{mdframed}
\includegraphics[width=400pt]{img/analysis--berkeley-202a-hw-8824.png}
\end{mdframed}

\begin{proof}
  $\emptyset$ is countable, therefore we have $\mu(\emptyset) = 0$ as required, and it remains to show
  that $\mu$ is countably additive.

  So let $B_1, B_2, \dots$ be a disjoint countable collection of subsets of $X$. We want to show
  that $\mu(\bigcup_{i=1}^\infty B_i) = \sum_{i=1}^\infty \mu(B_i)$.

  % Consider $\bigcup_{i=1}^\infty B_i$. It could be uncountable, since the $B_i$ could be a countable partition
  % of the entire set $X$. Could it also be countable? Yes, since the $B_i$ could be singletons. So we must
  % handle both cases.

  First suppose $\bigcup_{i=1}^\infty B_i$ is countable. Then no $B_i$ is uncountable. Therefore $\mu(B_i) = 0$
  for all $i$ and we have
  \begin{align*}
    \sum_{i=1}^\infty \mu(B_i) = \sum_{i=1}^\infty 0 = 0 = \mu\big(\bigcup_{i=1}^\infty B_i\big),
  \end{align*}
  as required.

  Next, suppose $\bigcup_{i=1}^\infty B_i$ is uncountable. We want to show
  that $\sum_{i=1}^\infty \mu(B_i) = 1$. Equivalently, we want to show that exactly one of the $B_i$ is
  uncountable. Note that $\ms A = \sigma(\ms A)$, and therefore we have by hypothesis that either $B_i$ is
  countable or $B_i^c$ is countable, for all $i$. Clearly some $B_i$ is uncountable or else we would
  have $\sum_{i=1}^\infty \mu(B_i) = \sum_{i=1}^\infty 0 = 0 \neq \mu\big(\bigcup_{i=1}^\infty B_i\big)$.
  Suppose for a contradiction that there exists $j \neq k$ such that $B_j$ and $B_k$ are uncountable. Note
  that $B_j$ and $B_k$ are disjoint, therefore $B_k \subseteq B_j^c$. But $B_j^c$ is countable, therefore $B_k$
  is countable; a contradiction. Therefore no such pair $j, k$ exists and we conclude that exactly one of
  the $B_i$ is uncountable, as required.
\end{proof}
\newpage
\begin{mdframed}
\includegraphics[width=400pt]{img/analysis--berkeley-202a-hw-2389.png}
\end{mdframed}

\begin{proof}

\end{proof}
\newpage
\begin{mdframed}
\includegraphics[width=400pt]{img/analysis--berkeley-202a-hw-3a79.png}
\end{mdframed}

\begin{claim*}
  If $m$ and $n$ have the same value on any open interval in $B$ then they have the same value on any set
  in $\mc B$.
\end{claim*}

\begin{proof}
  Let $\mc O$ be the collection of open subsets of $\R$.

  Let $O \in \mc O$. Then $O$ can be written as a countable union of disjoint open intervals, $O = \bigcup_i I_i$. Therefore
  \begin{align*}
    m(O) = m(\bigcup_i I_i) = \sum_i m(I_i) = \sum_i n(I_i) = n(O).
  \end{align*}
   By definition, $\mc B = \sigma(\mc O)$. We want to show that measures $m$ and $n$ agree on every set $A \in \mc B$.

   Let $A \in \mc B$ and suppose $m(A) = n(A)$. Then
   \begin{align*}
     m(A^c) = m(X) - m(A) = n(X) - n(A) = n(A^c).
   \end{align*}




\end{proof}
Read ``complete measure space​'' at end of Bass ch.3

\newpage
\begin{mdframed}
\includegraphics[width=400pt]{img/analysis--berkeley-202a-hw-0d98.png}
\end{mdframed}


\newpage
\begin{mdframed}
\includegraphics[width=400pt]{img/analysis--berkeley-202a-hw-e9e2.png}
\end{mdframed}
