\begin{comment}
\section*{Math 202a - HW1 - Dan Davison - \texttt{ddavison@berkeley.edu}}

\begin{mdframed}
  \includegraphics[width=400pt]{img/analysis--berkeley-202a--homework-1-a75a.png}
\end{mdframed}


\begin{proof}
  $d$ is a metric if it satisfies (I) $d(f,f) = 0$, (II) $d(f,g) = d(g, f)$, and (III) $d(f,g) + d(g, h) \le d(f, h)$.

  (I) is satisfied: $d(f, f) = \int_{[0,1]}|f(x) - f(x)| \dx = 0$.

  (II) is satisfied:
  \begin{align*}
    d(f, g)
    &= \int_{[0,1]}|f(x) - g(x)| \dx \\
    &= \int_{[0,1]}|g(x) - f(x)| \dx \\
    &= d(g, f),
  \end{align*}

  (III) is satisfied:
  \begin{align*}
    d(f, g) + d(g, h)
    &= \int_{[0,1]} |f(x) - g(x)| \dx + \int_{[0,1]} |g(x) - h(x)| \dx \\
    &= \int_{[0,1]} |f(x) - g(x)| + |g(x) - h(x)| \dx \\
    &\le \int_{[0,1]} |f(x) - g(x) + g(x) - h(x)| \dx \\
    &= \int_{[0,1]} |f(x) - h(x)| \dx \\
    &= d(f, h).
  \end{align*}
\end{proof}

\newpage
\begin{mdframed}
  \includegraphics[width=400pt]{img/analysis--berkeley-202a--homework-1-d1d3.png}
\end{mdframed}

\begin{enumerate}[label=(2.\arabic*)]
\item
  \begin{proof}
    $d$ is a metric on the function space $\mathcal C\([0, 1], \R\)$ if it satisfies (I) $d(f,f) = 0$,
    (II) $d(f,g) = d(g, f)$, and (III) $d(f,g) + d(g, h) \le d(f, h)$.

    (I) is satisfied: $\sup_{x\in [0,1]} |f(x) - f(x)| = \sup_{x\in [0,1]} 0 = 0$.

    (II) is satisfied:
    \begin{align*}
      d(f, g)
      &= \sup_{x \in [0,1]}|f(x) - g(x)| \\
      &= \sup_{x \in [0,1]}|g(x) - f(x)| \\
      &= d(g, f),
    \end{align*}
    (III) is satisfied:
    \begin{align*}
      d(f, g) + d(g, h)
      &=   \sup_{x \in [0,1]} |f(x) - g(x)| + \sup_{x \in [0,1]} |g(x) - h(x)| \\
      &=   \sup_{x \in [0,1]} \Big(|f(x) - g(x)| + |g(x) - h(x)|\Big) \\
      &\le \sup_{x \in [0,1]} |f(x) - h(x)| \\
      &=   d(f, h).
    \end{align*}
  \end{proof}
\item
  \begin{claim*}
    $\mc C\([0, 1], \R\)$ is separable.
  \end{claim*}

  \begin{proof}
    Let $\mc C$ be the set of continuous functions $[0, 1] \to \R$.

    Fix an arbitrary function $f \in \mc C$ and fix some $\epsilon > 0$.

    Define $g^*_n: [0, 1] \to \R$ as follows:
    \begin{enumerate}
    \item For all $i \in 0, 1, 2, \ldots, n$ set $x_i = i/n$.
    \item For all $i \in \{0, 1, 2, \ldots, n\}$, set $y^*_i = f(x_i)$. (Note that $y^*_i$ is in general not a rational
      number; we will account for this later.)
    \item For all $i \in \{1, 2, \ldots, n\}$ draw a straight line segment connecting $(x_{i-1}, y^*_{i-1})$
      and $(x_i, y^*_i)$.
    \item Define $g^*_n: [0, 1] \to \R$ to be the function whose graph was just drawn. (It is possible to give an
      explicit procedure for computing $g^*_n(x)$ by finding the interval in which $x$ lies and then using linear
      interpolation.)
    \end{enumerate}

    We now modify the definition of the family of approximating functions so that the $y$-coordinates of the
    endpoints are rational. Define $g_n: [0, 1] \to \R$ as follows:
    \begin{enumerate}
    \item Construct the sets of points $\{(x_i, y^*_i) ~|~ i \in \{0, 1, 2, \ldots, n\}\}$ as above.
    \item For $i \in \{0, 1, 2, \ldots, n\}$ set $y_i$ equal to a rational number in the
      interval $(y^*_i - \epsilon/4, y^*_i)$. (Such a rational number exists: for example, set $k$ equal to the
      smallest natural number such that $1/k < \epsilon/4$, and then set $j$ equal to the smallest natural number
      such that $j/k > y^*_i - \epsilon/4$. Then $j/k$ is a rational number in $(y^*_i - \epsilon/4, y^*_i)$.)
    \item For all $i \in \{1, 2, \ldots, n\}$ draw a straight line segment connecting $(x_{i-1}, y_{i-1})$
      and $(x_i, y_i)$.
    \item Define $g_n: [0, 1] \to \R$ to be the function whose graph was just drawn.
    \end{enumerate}
    Note that, since $f$ is continuous on a compact domain, $f$ is uniformly continuous. Fix some $\delta > 0$ such
    that $|x - x'| < \delta \implies |f(x) - f(x')| < \epsilon$ for all $(x, x') \in [0, 1]^2$.

    Set $m$ equal to the smallest natural number such that $1/m < \delta/2$ and note
    that $|f(x) - g_n(x)| < \epsilon$ for all $x \in [0, 1]$ due to the uniform continuity of $f$. (Informally,
    this is true because we can view uniform continuity as stating that a rectangle of base $\delta$ and
    height $\epsilon$ can be positioned over the graph at any point such that the graph intersects the left and
    right edges of the rectange but does not otherwise leave the rectangle. Our piecewise affine function
    consists of straight line segments that fit within such rectangles.) Therefore $d(g_n, f) < \epsilon$ for
    all $n \geq m$ and so $\{g_n | n \in \N\}$ is dense in $\mc C$.

    Finally we must show that $\{g_n ~|~ n \in \N\}$ is countable. Note that $g_n$ is piecewise affine for a
    given $n$, and that the $x$-coordinates of the endpoints are fixed. Thus for a given $n$, the cardinality
    of $\{g_n\}$ is equal to the cardinality of the set of possible $y$-coordinates. The latter set is $\Q^n$.
    Thus the cardinality of $\{g_n ~|~ n \in \N\}$ is equal to the cardinality of the
    set $\bigcup_{n\in \N} \Q^n$. This is a countable union of countable sets and is therefore countable.
  \end{proof}

\item
  \begin{proof}
    Let $f_s: (0, 1) \to \R$ be given by $f_s(x) = \frac{1}{r(s)x}$, where $s \in 2^\N$ and $r(s) \in [0, 1]$ is
    the real number corresponding to $s$, i.e. the number $r(s) = 0.d_1d_2d_3\ldots$ where
    \begin{align*}
      d_i =
      \begin{cases}
        1, ~~~~ \text{if} ~~~~ i \in s,\\
        0, ~~~~ \text{otherwise}.
      \end{cases}
    \end{align*}
    Note that for real $a, b$ we have
    \begin{align*}
      \frac{1}{ax} - \frac{1}{bx} = \frac{b - a}{abx}
    \end{align*}
    and therefore the supremum distance between any two elements $f_{s_1}$ and $f_{s_2}$ is unbounded
    as $x \to 0$.
  \end{proof}
\item
  \begin{proof}
    Let $\mc C$ be the set of continuous functions $f: (0, 1) \to \R$.

    Assume for a contradiction that $\mc C$ is separable. Let $\mc G \subset \mc C$ be a countable dense set of
    functions.

    Recall that in part (3) we found an uncountable set $\mc H \subset \mc C$ with the property that every pair
    of elements in $\mc H$ is at supremum distance at least one.

    But this is a contradiction, since we can establish a bijection between $\mc H$ and $\mc G$ as follows:

    Pick an element $h_1 \in \mc H$. Since $\mc G$ is dense in $\mc C$, there exists $g_1 \in \mc G$ such
    that $d(h_1, g_1) < 1/2$. Now pick $h_2 \in \mc H$ such that $h_1 \neq h_2$. Again, there
    exists $g_2 \in \mc G$ such that $d(h_2, g_2) < 1/2$. Furthermore, by the triangle
    inequality, $g_2 \neq g_1$. Continuing in this fashion, on the $i$-th iteration we pick $h_i \in \mc H$ and
    find a nearby $g_i \in \mc G$ such that $d(h_i, g_i) < 1/2$, and by the triangle inequality conclude
    that $g_i \neq g_j$ for all $j < i$.

    Thus we can associate each element of $\mc H$ with a unique element of $\mc G$ and conclude that the
    cardinality of $\mc G$ equals that of $\mc H$, which is that of the power set of the natural numbers.
    But $\mc G$ is countable; a contradiction. Therefore no such countable dense set $\mc G$ exists and $\mc C$
    is not separable.
  \end{proof}
\end{enumerate}


\newpage
\begin{mdframed}
  \includegraphics[width=400pt]{img/analysis--berkeley-202a--homework-1-8349.png}
\end{mdframed}


\begin{proof}
  Let $E_m^a = \{x \in [0, 1] : |f_m(x) < a|\}$ and let $T = \bigcap_{k=1}^\infty \bigcup_{l=1}^\infty \bigcap_{m > l} E_m^{1/k}$.

  Informally, $E_m^a$ is the set of points for which $f_m$ is within $a$ of zero.

  Let $f_n: [0, 1] \to \R$ be a sequence of functions and let $S \subseteq [0, 1]$ be the set of points $x$
  such that $f_n(x) \to 0$ as $n \to \infty$.

  First we prove that $x \in S \implies x \in T$.

  So let $x \in S$. Then from the definition of limit we have
  \begin{align*}
    &\forall \epsilon>0 ~~~~ \exists l \in \N ~~~~ \forall m \geq l ~~~~  x \in E_m^\epsilon \\
    \iff &\forall \epsilon>0 ~~~~ \exists l \in \N ~~~~                        x \in \bigcap_{m \geq l} E_m^\epsilon \\
    \iff &\forall \epsilon>0 ~~~~                                              x \in \bigcup_{l=1}^\infty \bigcap_{m \geq l} E_m^\epsilon \\
    \iff &                                                              x \in \bigcap_{k=1}^\infty \bigcup_{l=1}^\infty \bigcap_{m \geq l} E_m^{1/k} = T,
  \end{align*}
  as required.

  Secondly we prove that $x \in T \implies x \in S$.

  So let $x \in T$. We have
  \begin{align*}
    x \in \bigcap_{k=1}^\infty \bigcup_{l=1}^\infty \bigcap_{m \geq l} E_m^{1/k},
  \end{align*}
  which is equivalent to the statement
  \begin{align*}
    \forall k>0 ~~~~ \exists l \in \N ~~~~ \forall m \geq l ~~~~  |f_m(x)| < \frac{1}{k}.
  \end{align*}
  Let $\epsilon > 0$ be a real number. Then there exists $k \in \N$ such that $\frac{1}{k} < \epsilon$. Therefore we have
  \begin{align*}
    \forall \epsilon>0 ~~~~ \exists l \in \N ~~~~ \forall m \geq l ~~~~  |f_m(x)| < \epsilon
  \end{align*}
  which is equivalent to $x \in S$, as required.
\end{proof}

\begin{proof}
  Let $S \subseteq [0, 1]$ be the set of points $x$ for which $f_n(x)$ converges as $n \to \infty$. Since every
  convergent sequence in the reals is Cauchy, we have that $x \in S$ is equivalent to
  \begin{align*}
    \forall \epsilon > 0 ~~~~ \exists l \in \N ~~~~ \forall m \geq l ~~~~ \forall n \geq l ~~~~ |f_m(x) - f_n(x)| < \epsilon,
  \end{align*}
  which is equivalent to
  \begin{align*}
    \forall \epsilon > 0 ~~~~ x \in \bigcup_{l=1}^\infty \bigcap_{m \geq l} \bigcap_{n \geq l} E_{m,n}^\epsilon.
  \end{align*}
  Therefore, we have that $x \in S$ implies
  \begin{align*}
    x \in \bigcap_{k=1}^\infty \bigcup_{l=1}^\infty \bigcap_{m \geq l} \bigcap_{n \geq l} E_{m,n}^{1/k}.
  \end{align*}
  As before, the reverse implication also holds since, for any given $\epsilon > 0$, we can find a $k \in \N$
  such that $\frac{1}{k} < \epsilon$.
\end{proof}

\newpage
\begin{mdframed}
  \includegraphics[width=400pt]{img/analysis--berkeley-202a--homework-1-f175.png}
\end{mdframed}


\begin{proof}
  Let $X \subset [0, 1]$ be the subset of real numbers without any 6 in their decimal expansion. Let $x \in X$ and let
  \begin{align*}
    d_n(x) =
    \begin{cases}
      0, ~~~~ n\text{-th decimal place of }x \text{ is }0\\
      1, ~~~~ \text{otherwise}.
    \end{cases}
  \end{align*}
  Define $f: X \to [0, 1]$ by setting $f(x)$ equal to the real number whose binary expansion
  is $0.d_1(x)d_2(x)\cdots$.

  Note that for any real number $\om \in [0, 1]$, there exists $x \in X$ such that $f(x) = \om$. To find such
  an $x$, we could for example choose the number whose decimal expansion is equal to the binary expansion
  of $\om$.

  Therefore $f$ is a non-injective surjection from $X$ to the reals in $[0, 1]$, and so the cardinality of $X$
  is at least that of the reals. Since $X \subset \R$ we conclude that the cardinality of $X$ is equal to that
  of the reals.
\end{proof}

\newpage
\begin{mdframed}
  \includegraphics[width=400pt]{img/analysis--berkeley-202a--homework-1-5192.png}
\end{mdframed}



\begin{proof}
  Let $X \subseteq [0, 1]$ be the set of points at which $f$ is continuous. We want to show that $f$ is
  continuous at $x$ iff $x$ is not rational.

  Suppose for a contradiction that $f$ is continuous at a rational point $x = p/q$, where $p, q$ are
  non-negative integers. Then $f(x) = 1/q$. But there will always be irrational points within a given
  distance $\delta$ of $x$, now matter how small $\delta$ is, and at such an irrational point $x'$ we
  have $|f(x) - f(x')| = |1/q - 0| = 1/q$. Therefore $f$ is not continuous at $x$ since the definition of
  continuity does not hold for $\epsilon < 1/q$.

  Now let $x$ be irrational, so that $f(x) = 0$. Fix an arbitrary $\epsilon > 0$. We want to show that there
  exists a $\delta$ such that $1/q < \epsilon$ for any rational point $p/q$ lying within $\delta$ of $x$,
  where $p/q$ is in reduced terms. If $\epsilon > 1/2$ then any $\delta$ will work, so
  assume $\epsilon \leq 1/2$. Let $k$ be the largest natural number such that $1/k \geq \epsilon$, let $i$ be
  the largest natural number such that $i/k < x$ and let $j$ be the smallest natural number such
  that $j/k > x$. Then a choice of $\delta = \frac{1}{2}\min\{x - i/k, j/k - x\}$ will work to prove continuity
  of $f$ at irrational $x$.
\end{proof}

\newpage
\begin{mdframed}
  \includegraphics[width=400pt]{img/analysis--berkeley-202a--homework-1-f5e8.png}
\end{mdframed}


\begin{definition*}
  $g: [0, 1] \to \R$ is Riemann integrable if
  \begin{align*}
    \sup_{\phi^-} I(\phi^-) = \inf_{\phi^+} I(\phi^+).
  \end{align*}
  Here $\phi^-$ and $\phi^+$ are step functions adapted​ to some
  partition $0 \leq x_1 \leq x_2 \leq \ldots \leq x_{n-1} \leq 1$, such that $\phi(x) = c_i$
  for $x \in (x_{i-1}, x_i)$. $I(\phi)$ is (informally) the area under the step function $\phi$:
  \begin{align*}
    I(\phi) = \sum_{i=1}^n c_i(x_i - x_{i-1}).
  \end{align*}
  And the supremum is over all minorants $\phi^- \leq g$ and the infimum is over all
  majorants $\phi^+ \geq g$, where the length $n$ of the partition is allowed to vary as well as the constant
  values $\{c_1, c_2, \ldots, c_n\}$ of the step function within each segment.
\end{definition*}

\begin{enumerate}[label=(6.\arabic*)]
\item
  \begin{claim*}
    The specified function $f$ is not Riemann integrable.
  \end{claim*}

  \begin{proof}
    Consider the first segment of any partition: $(0, x_1)$. No matter how small $x_1$ is, there
    exists $n \in \N$ such that $1/n < x_1$. Therefore for all majorants we have $c_1 \geq 1$ and yet for all
    minorants we have $c_1 \leq 0$. So, when restricted to this first segment, we have $I(\phi^-) > I(\phi^+)$
    for all $\phi^-, \phi^+$ and, since every majorant is elsewhere less than every minorant, it is not
    possible that $\sup_{\phi^-} I(\phi^-) = \inf_{\phi^+} I(\phi^+)$ and hence the Riemann integral is
    undefined.
  \end{proof}

\item
  \begin{claim*}
    $\int_0^1 f > 0$
  \end{claim*}
  \begin{proof}
    Suppose for a contradiction that $\int_0^1 f = 0$. Fix an arbitrary minorant $\phi^-$, adapted to a
    partition of length $n$. Then we have that $\sum_{i=1}^n c_i(x_i - x_{i-1}) \leq 0$.
    Since $x_i \geq x_{i-1}$ for all $i$, and since $x_0 = 0 < x_n = 1$, it must be the case
    that $x_i - x_{i-1} > 0$ for some $i$, and therefore that $c_i > 0$ for some $i$. Therefore $f$ vanishes at
    at least one point. This contradiction proves that $\int_0^1 f \neq 0$.

    To see that it's not negative, note that for every majorant $\phi^+$ we have $x_i - x_{i-1} \geq 0$
    and $c_i > 0$ for all $i$ and therefore $I(\phi^+) = \sum_{i=1}^n c_i(x_i - x_{i-1}) \geq 0$. Therefore $\int_0^1 f > 0$.
  \end{proof}

\end{enumerate}

\newpage
\begin{mdframed}
  \includegraphics[width=400pt]{img/analysis--berkeley-202a--homework-1-a577.png}
\end{mdframed}

\begin{proof}
  Suppose for a contradiction that $[0, 1] \subset \R$ is countable. Fix an enumeration $\{u_n ~|~ n \in \N\}$
  of the elements of $[0, 1]$.

  If $u_1 > u_2$ then relabel them so that $u_1 < u_2$. Set $U_1 = (u_1, u_2)$.

  Continue examining the numbers in the enumeration (starting at $u_3$) until two numbers have been encountered
  that are both in $(u_1, u_2)$. Form an interval from this pair and label it $U_2$. Continue examining the
  numbers in the enumeration until two numbers are encountered that are both in $U_2$; label this
  interval $U_3$. Continue in this fashion indefinitely.

  We will write $(U_{i1}, U_{i2})$ to refer to the endpoints of interval $i$.

  There are two cases:

  \begin{enumerate}
  \item {\bf The process terminates.}\\
    Then there is a last interval $U_L = (U_{L1}, U_{L2})$. It is possible that there is one (but not more than
    one) element $u^*$ of the original enumeration that is present in the interval $U_L$. If that is so, then
    every element of $U_L \setminus \{u^*\}$ is a real number not in the original enumeration; otherwise every
    element of $U_L$ is a real number not in the original enumeration.

  \item {\bf The process does not terminate.}\\
    Note that the sequence of interval lower bounds $(U_{i1})_{i\in\N}$ forms a strictly increasing sequence
    bounded above by $u_2$ and that the sequence of interval upper bounds $(U_{i2})_{i\in\N}$ forms a strictly
    decreasing sequence bounded below by $u_1$. By the Monotone Convergence theorem, both sequences converge:
    let these limits be $\alpha$ and $\beta$ respectively. There are two cases:
    \begin{enumerate}
    \item {\bf $\alpha < \beta$}\\
      Then every element of $(\alpha, \beta)$ is a real number not in the original enumeration.
    \item {\bf $\alpha = \beta$}\\
      Then $\alpha$ is a real number not in the original enumeration.
    \end{enumerate}
  \end{enumerate}

  In all cases, we found a real number that was not present in the original enumeration. But this is a
  contradiction, since the original enumeration contains all real numbers. Therefore no such enumeration exists
  and the real numbers are not countable.
\end{proof}

\end{comment}

\section*{Math 202a - HW2 - Dan Davison - \texttt{ddavison@berkeley.edu}}


\begin{mdframed}
  \includegraphics[width=400pt]{img/analysis--berkeley-202a-ebe4.png}
\end{mdframed}

\begin{intuition*}
  $O \subset \R$ is a countable union of disjoint open intervals.

  $x \sim y$ iff $x$ and $y$ are in the same interval.

  The length of $O$ should be the sum of the lengths of the intervals.
\end{intuition*}


\begin{itemize}
\item (1)
  \begin{claim*}
    $\sim$ is an equivalence relation on $O$.
  \end{claim*}
  \begin{proof}
    \begin{enumerate}
    \item {\bf Reflexivity}\\
      $x \sim x$ since $[x, x] = \{x\} \subseteq O$.

    \item {\bf Symmetry}\\
      Let $x, y \in \R$ such that $x \sim y$. Then $[\min\{x, y\}, \max\{x, y\}] \subseteq O$.
      Therefore $[\min\{y, x\}, \max\{y, x\}] \subseteq O$. Therefore $y \sim x$.

    \item {\bf Transitivity}\\
      Let $x, y, z \in \R$ such that $x \sim y$ and $y \sim z$. Then $[\min\{x, y\}, \max\{x, y\}] \subseteq O$
      and $[\min\{y, z\}, \max\{y, z\}] \subseteq O$. Therefore $[\min\{x, y\}, \max\{y, z\}] \subseteq O$.
      Therefore $x \sim z$.
    \end{enumerate}
  \end{proof}

\item (2)
  \begin{claim*}
    $O$ may be written as a countable union of disjoint open intervals.
  \end{claim*}

  \begin{proof}
    Let $\mc I = I_1, I_2, \ldots$ be the set of equivalence classes of $O$ under $\sim$.

    Since $\sim$ is an equivalence relation, the elements of $\mc I$ are disjoint and their union is equal
    to $O$.

    We now show that the elements of $I$ are open intervals:

    Let $I \in \mc I$. Note that $I$ has at least two elements since $I$ is an equivalence class. Suppose $I$
    is bounded below and above. Then for any $\epsilon > 0$ there exists $a \in I$ such
    that $a - \inf O < \epsilon$. Note that $[a, x] \subseteq I$ for all $x \in I$ where $a < x$. By a similar
    argument, $[x, b] \subseteq I$ for all $x \in I$ where $b = \sup I$ and $x < b$.

    Note that either one of the elements of $I$ has no lower bound, or none do.

    \red{TODO}

    \red{TODO} We want to show that the elements of $\mc I$ are intervals of the form $(I_a, I_b)$

    Let $I \in \mc I$.

    We now show that the elements of $\mc I$ are open sets. Let $I \in \mc I$ and suppose for a contradiction
    that $I$ is not open. Then there exists $x \in I$ such that no neighborhood of $x$ is contained within $I$.
    Let $x$ be such a point and let $N(x)$ be a neighborhood of $x$. We claim
    that $N(x) \cap (\R \setminus O) \neq \emptyset$, i.e. that $N(x)$ contains a point outside $O$.

    \red{TODO} We need to show that this is a countable union.

    We can enumerate the intervals as follows:

    Let $q_1, q_2, \ldots$ be an enumeration of the rationals.

    Note that every rational is in zero or one interval, but not more than one. Furthermore, every interval
    contains at least one rational.

    Therefore there is a non-injective surjection from a subset of the rationals to the set of intervals.

    (We can make this a bijection by choosing one rational from each interval?)

    Therefore the cardinality of the set of intervals is not greater than the cardinality of the rationals.

    Therefore the set of intervals is countable.
  \end{proof}

\item (3) \green{DONE}
  \begin{proof}
    We may assign a length $\mu(O)$ to $O$ as follows:

    If $O = \emptyset$ then $\mu(O) := 0$.

    Otherwise, if $O$ is not bounded below, or if $O$ is not bounded above, then $\mu(O) := \infty$.

    Otherwise, if the series $\sum_i |I_i|$ diverges, then $\mu(O) := \infty$.

    Otherwise, $\mu(O) := \sum_i |I_i|$.

    Note that every term of the series is positive. In order for this definition to be unambiguous, the
    value $\mu(O)$ must not depend on the ordering of the series. This is true by the lemma below.
  \end{proof}

  \begin{lemma}
      Let $\sum_i a_i$ be a series with $a_i > 0$ for all $i$. Then
      \begin{enumerate}
      \item If the series diverges for any ordering of the series, it diverges for all orderings.
      \item If the series converges for any ordering of the series, it converges to the same value for all
        orderings.
      \end{enumerate}
  \end{lemma}
  \begin{proof}
    A sketch proof of the second statement is as follows: given any $\epsilon > 0$ we can identify a tail of
    the sequence whose sum is less than $\epsilon$. Thus the sum of the series is determined by the finite
    head. The sum of this finite head does not depend on its ordering, by commutativity of addition.
  \end{proof}
\end{itemize}



\begin{mdframed}
  \includegraphics[width=400pt]{img/analysis--berkeley-202a-8d78.png}
\end{mdframed}

\begin{itemize}
\item
  \begin{claim*}
    $C$ is compact.
  \end{claim*}
  \begin{proof} \green{DONE}
    Since $C \subset \R$ it suffices to show that $C$ is closed and bounded. Then it follows from the Heine-Borel
    theorem that $C$ is compact.

    $C$ is bounded below by $0$ and above by $1$, since it is constructed by removing points from $[0, 1]$.

    To show that $C$ is closed we may show that $C^c$ is open. Since $C = \bigcap_{n=0}^\infty C_n$, we
    have $C^c = \bigcup_{n=0}^\infty C_n^c$. Note that $C_n$ is a union of closed intervals; therefore $C_n^c$
    is a union of open intervals and therefore open (if an interval contains a neighborhood of each one of its
    points then the union of intervals also contains neighborhoods of those points); therefore $C^c$ is a union
    of open intervals and therefore open. Therefore $C$ is closed.
  \end{proof}

\item
  \begin{claim*}
    $C$ is uncountable.
  \end{claim*}

  \begin{proof} \red{TODO}
    Note that $\om \in C$ if and only if the base 3 (ternary) expansion of $\om$ contains no $1$s.

    Consider the map $f:C \to (0, 1]$ defined by the following rule: $f(\om)$ is equal to the real number whose
    binary expansion is formed by substituting every $2$ with a $1$ in the ternary expansion of $\om$.

    This map is a bijection, therefore $C$ is uncountable.


    We want to show that the countably infinite intersection of closed intervals does not result in a single
    point, or in an empty set.

    For any $n$ we can exhibit an $\eps > 0$ such that $[0, \eps) \subset C$. Since this is an uncountable set, $C_n$ is uncountable.

    So $C_n$ is uncountable for all $n$. How do we show that $C$ is uncountable?

    elements of Cantor set are ternary digits bijection <-> R

  \end{proof}

\item \begin{claim*}
    $C$ is negligible.
  \end{claim*}

  \begin{proof}
    Fix an arbitrary $\epsilon > 0$. We will show that there exists a countable union of
    intervals $I_1, I_2, \ldots$ that cover $C$ and for which $\sum_k |I_k| < \epsilon$.

    Note that $C_n$ comprises $2^n$ disjoint intervals each of length $3^{-n}$. Therefore the total length
    of $C_n$ is $|C_n| = \big(\frac{2}{3}\big)^n$. Set $m = \Big\lceil\frac{\log\eps}{\log 2/3} \Big\rceil$. We see that for all $n > m$ we have $|C_n| < \epsilon$.




    \red{TODO} Tie this up by showing explicitly that the limiting set $C$ can be covered.
  \end{proof}

\end{itemize}


\begin{mdframed}
  \includegraphics[width=400pt]{img/analysis--berkeley-202a-6b7a.png}
\end{mdframed}

\begin{enumerate}[label=(3.\arabic*)]

\item (1)
  \begin{definition*}
    Let $X$ be a set formed by iteratively removing open intervals from $[0, 1]$. Let $I_1, I_2, \ldots$ be the
    open intervals that were removed in the formation of $X$. Note that these are disjoint, since a point can
    not be removed more than once. Define the measure of $X$ to be $1 - \sum_k |I_k|$.

    $1r + 2r^2 + 2^2r^3 + \ldots$ geometric series
  \end{definition*}

\item (2)
  \begin{definition}[Cantor set of measure $a$]
    Let $a \in (0, 1)$. The Cantor set of measure $a$ is formed as follows:

    \red{TODO}
  \end{definition}

\end{enumerate}




\begin{mdframed}
  \includegraphics[width=400pt]{img/analysis--berkeley-202a-8ce8.png}
\end{mdframed}

Devil's staircase
\begin{enumerate}[label=(4.\arabic*)]

\item (1) It seems that this limit will have the form

  (measure on $C$ restricted to $(0, x)$) $\times$ ($(3/2)^n$)

  which is

  $0 \times \infty$.

show $F_n$ is a Cauchy sequence
\item (2) $F$ is piecewise affine, hence continuous.

\item (3) $F$ is differentiable on the constant and non-constant sections, but not on the endpoints.

\end{enumerate}



\begin{mdframed}
  \includegraphics[width=400pt]{img/analysis--berkeley-202a-ccfa.png}
\end{mdframed}

\begin{enumerate}[label=(5.\arabic*)]

\item
  \begin{claim*}
    The orbit $S_{\lambda}(x)$ is dense for all $x \in (0, 1]$ iff there exists an $x$ for which it is dense.
  \end{claim*}
  \begin{proof}
    \red{TODO}
  \end{proof}

\item
  \begin{claim*}
    The orbits are dense iff $\lambda \in (0, 1]$ is irrational.
  \end{claim*}
  \begin{proof}
    \red{TODO}
  \end{proof}

\item
  \begin{definition*}
    The $n$-th occupation fraction of $I$ by $S_\lambda$ is
    \begin{align*}
      O_n(I, \lambda) = \frac{1}{n}\Big|\Big\{i \in \{1, \ldots, n\} ~:~ a_n \in I\Big\}\Big|.
    \end{align*}
  \end{definition*}
  \begin{claim*}
   The occupation fraction has a limiting value: $\lim_{n\to\infty}O_n(I, \lambda) = b - a$.
  \end{claim*}
\item

Hi Nathan, Scott, Ali: I had a thought about Q5 and wanted to ask you whether you think it's a direction worth exploring.

Focus on the j-th digit in the binary expansion of a_n, and consider the orbit of that digit alone. That's a sequence of 0s and 1s that we may interpret as a real number in [0, 1]. My thought was that if we could prove the following then that would solve the question:

The real number corresponding to the sequence visited by the j-th digit is normal, for all j.
The sequence for digit j is, or becomes, (in an appropriate sense) uncorrelated with the sequence for digit k ≠ j.

I think those two results would imply that a_n itself has no tendency to accumulate in any dyadic interval, which is what we want to show.

I'm not sure how to show them though. Possibly the argument would involve observing that lambda is irrational and hence has a non-repeating binary expansion, and then making an argument involving the mechanics of addition of binary sequences (with overflow and the mod 1 operation).


  \begin{proof}[Proof sketch]
    Suppose the process runs for many iterations. We want to show that, if we pick a generation $n$ at random,
    then $a_n$ (now viewed as a random variable) is uniformly distributed over the interval $I$.

    We want to show that picking a generation $n$ at random yields an $a_n$ which is like a uniform pick
    from $(0, 1]$.

    \begin{itemize}
    \item Consider the sequence of values taken by the $j$-th digit in the binary expansion of $a_n$.
    \item We want to show that this sequence, when interpreted as the binary expansion of a real number in
      $[0, 1]$ is normal, and uncorrelated from the sequence of values taken by the $k$-th digit for $j \neq k$.
    \item These results would imply that each digit is visiting $0$ and $1$ with equal frequency, independently of
      other digits, and therefore that $a_n$ has no tendency to occupy any dyadic interval more than any other
      dyadic interval.
    \end{itemize}

    Let $\lambda \in (0, 1) \setminus \Q$.

    \red{TODO}

  \end{proof}
\end{enumerate}