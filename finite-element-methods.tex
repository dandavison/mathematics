\section{DSM / FEM (ch 1 \& 2)}
\begin{enumerate}
\item A plane truss is a system of bars (elements), meeting at joints (nodes) .
\item We model it, basically, as a system of springs, each element behaving a bit like a spring under Hooke's law.
\item The truss has some equilibrium configuration in which forces are balanced.
\item Each node has a resultant force and a displacement away from its equilibrium position. It's a 2D model so that's 4 numbers in total.
\item In a $n$-node system there are $2n$ forces and $2n$ displacements.
\item In a DSM analysis, the basic question is: Suppose some of the nodes are displaced away from that equilibrium configuration. What are the resulting
  forces?
\item The DSM model says that there is a linear transformation $K$ which computes the vector of forces $f$, given the vector of
  displacements $u$, i.e. $f = K u$
\item In some sense (?) this is like Hooke's law, which says that the force on the end of a single spring is equal to some constant times the spring
  displacement.
\item The entries in the matrix $K$involve three considerations:
  \begin{enumerate}
  \item Certain displacements are irrelevant to certain forces.
  \item Certain displacements are positively related to certain forces while others are negatively related.
  \item The actual value of the non-zero entries is determined by physical considerations (elasticity, cross-sectional area, length), similar to
    how the Hooke's law coefficient reflects physical characteristics of the spring.
  \end{enumerate}
\item In the first step of a FEM analysis, one considers each element separately, so the force and displacement vectors have length 4 (two nodes),
  i.e. the equation is a linear tranformation $\R^4 \to \R^4$. You write the matrix equations for each element separately, and then there is some
  procedure for coming up with a combined equation for the overall system (i.e.$\R^{2n} \to \R^{2n}$). (I'm vague on this bit, but it didn't seem to be an
  iterative algorithm, it seemed like linear algebra gave you a way of just computing a consistent equation for the combined system in one step)
\item Some nodal displacements / forces are known a priori, the others can then be solved for using standard linear algebra techniques (I'm vague on this
  too).
\end{enumerate}

\section{Variational approach (ch. 11)}
\begin{enumerate}
\item Consider a system comprising a single bar. $x$ measures location along the bar.
\item The question has now changed. Whereas previously we asked what the nodal forces are given nodal displacements (i.e. the bars always remained
  straight lines?), now we say: Suppose the nodal displacements cause the bars to be deformed. What function describes the shape that the bars adopts
  when subjected to the nodal displacaments?
\item This function is the axial displacement function $u(x)$.
\item So we're now seeking a continuous function, rather than a finite-dimensional vector.
\item We can write down a functional that, given a candidate axial displacement function, computes the associated total potential energy TPE. Then we can
  use calculus of variations arguments to solve for $u(x)$on the basis that the unknown function $u(x)$ will be whatever minimizes the TPE functional.
\item The TPE functional has the form (potential energy) = (internal energy) - (external energy)
\item The internal energy is modeled as an integral over the length of the bar. The quantity being integrated models stress and strain at location
  $x$.
\item The internal energy at a location $x$ depends not only on the location $x$, but also on the rate-of-change of axial displacement $u'$. I.e. "strain"
  is greater in a region in which the displacement is changing more rapidly.
\item This involvement of $u'$ is similar to the situation in classical mechanics, where the motion of a particle is determined not only by its location
  (which determined its potential energy) but also by its momentum (velocity; which determines its kinetic energy). So whereas in FEM the independent
  variable is $x$ and we want to know the axial displacement function $u(x)$, in classical mechanics the independent variable is time $t$ and we want
  to know the particle's trajectory $y(t)$. In classical mechanics the Principle of Least Action says that the trajectory $y(t)$ of the particle will
  be whatever function $y(t)$ minimizes the functional
  \begin{align*}
    \int \text{KE}\(y'(t)\) - \text{PE}\(y(t)\) \dt.
  \end{align*}
  In FEM we are trying to find a function $u(x)$ which minimizes the functional
  \begin{align*}
    \int \text{IE}\(u(x), u'(x)\) - \text{EE}\(x, u(x)\) \dx.
  \end{align*}
\item In principle one could use the Euler-Lagrange equations, which would give a differential equation which the unknown function $u(x)$ must
  satisfy. However in ch. 11 the author assumes that $u$ has a piecewise-linear form. I think this is in order to demonstrate that the same linear
  equation from DSM connecting nodal displacements and forces via a stiffness matrix can be derived under that assumption?
\end{enumerate}
