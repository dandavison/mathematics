\documentclass[12pt]{article}
\usepackage{notes}

\begin{document}

\section{n-omino}
\begin{mdframed}
  You have a $n$-omino. (Like a tetris piece but with arbitrarily many pieces.)
  Now pick two points with uniform distribution on that piece. Prove that the
  probability that the line between them is contained on the piece is of the
  form $p - q\ln(r)$ where $p,q$ and $r$ are rational.
\end{mdframed}
~\\

Assumptions:
\begin{enumerate}
\item A random $n$-omino is generated as follows: place a square tile at an
  arbitrary location in a plane; while the number of tiles is less than $n$,
  choose an edge uniformly from among the available edges and place a tile
  adjoining that edge.
\end{enumerate}

Define an ``internal'' line to be a line connecting two points on an $n$-omino
that is contained in the $n$-omino.

Let $\omega$ denote the desired probability that a line between two points
chosen uniformly on an $n$-omino is internal.

For $n=1$ and $n=2$ we have $\omega = 1$, since the only possible
configurations are rectangles, and these are convex polygons.

For $n=3$ the configuration is a rectangle with probability $\frac{1}{3}$ and
an L-shape with probability $\frac{2}{3}$. In the former case the line is
always internal. In the latter case, if the first point is in the corner
square, then the line is always internal; otherwise, the probability that the
line is internal is equal to the
\begin{align*}
  \omega = \frac{1}{3}\cdot 1 + \frac{2}{3} \Big(\text{random area enclosed by line through center}\Big)
\end{align*}

\end{document}