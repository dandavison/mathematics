\documentclass[12pt]{article}
\usepackage{mathematics}

\begin{document}

\section{Languages}
\begin{definition*}[dsll]
  A dsll file declares
  \begin{itemize}
  \item Type constructors
  \item Operators
  \item Predicates
  \end{itemize}
\end{definition*}

\newpage
\section{Linear algebra domain}

Suppose we have the following {\tt .sub} file
\begin{verbatim}
VectorSpace U,V

LinearMap f
From(f, U, V)

Vector p, q, x, y, z1, z2
In(p, U)
In(q, U)
In(x, V)
In(y, V)
In(z1, V)
In(z2, V)
x := Apply(f, p)
y := Apply(f, q)
z1 := x + y
z2 := Apply(f, u1 + u2)
\end{verbatim}

In the visualization it must be the case that $z_1 = z_2$, because $f$ is linear. How do we
implement the {\tt .sty} so that this is true?

We could declare that one of the outputs of $f$, say $x$, is a parameter to be optimized, and then
solve for $f$ such that $f(p) = x$.

Note that in this example, $U = V = \R^2$.

\begin{problem*}
  Let $p, x \in \R^2$. Find, under the standard basis, the set of matrices $A$ such that $Ap = x$.
\end{problem*}

\begin{proof}
  Suppose that $p$ and $x$ are linearly independent. Let $p = \bvecc{p_1}{p_2}$,
  $x = \bvecc{x_1}{x_2}$ under the standard basis.

  Note that under the basis $(p, x)$ the matrix $A = \bmatt{0}{a}
                                                           {1}{b}$ takes $p \mapsto x$ for
  all $a, b \in \R$.

  Let's further constrain the solution by demanding that the determinant is 1, which implies that $a = -1$.

  And let's take $b = 0$ because it seems like a choice that will result in simpler calculations/results.

  Thus we have $A_{(p,x)} = \bmatt{0}{-1}
                                 {1}{0}$ under the basis $(p, x)$.

  Under the standard basis this becomes
  \begin{align*}
    A_{\text{standard}}
    &=
      \bmatt{p_1}{x_1}
            {p_2}{x_2}
      \bmatt{0}{-1}
            {1}{0}
      \bmatt{p_1}{x_1}
            {p_2}{x_2}^{-1}\\
    &=
      \bmatt{x_1}{-p_1}
            {x_2}{-p_2}
      \bmatt{x_2}{-x_1}
            {-p_2}{p_1}
      \frac{1}{p_1x_2 - p_2x_1}\\
    &=
      \bmatt{x_2x_1 + p_2p_1}{-x_1^2 -p_1^2}
            {x_2^2 + p_2^2}{-x_1x_2 - p_1p_2}
      \frac{1}{p_1x_2 - p_2x_1}\\
  \end{align*}




\begin{verbatim}
#+begin_src mathematica
getMatrix[p_, x_] := Module[
    {p1, p2, x1, x2},
    p1 = p[[1]]; p2 = p[[2]]; x1 = x[[1]]; x2 = x[[2]];
    (1 / (p1 x2 - p2 x1)) {{x2 x1 + p2 p1, -x1^2  - p1^2},
                           {x2^2  + p2^2,  -x1 x2 - p1 p2}}
];
getMatrix[{i1, i2}, {j1, j2}]. {i1, i2} // Simplify
#+end_src

#+RESULTS:
| j1 | j2 |
\end{verbatim}


  % Under the standard basis, this is
  % \begin{align*}
  %   A
  %   &=
  %     \bmatt{p_1}{x_1}
  %           {p_2}{x_2}^{-1}
  %     \bmatt{0}{a}
  %           {1}{b}
  %     \bmatt{p_1}{x_1}
  %           {p_2}{x_2}\\
  %   &=
  %     \frac{1}{p_1x_2 - p_2x_1}
  %     \bmatt{x_2}{-x_1}
  %           {-p_2}{p_1}
  %     \bmatt{p_2a}{x_2a}
  %           {p_1 + p_2b}{x_1 + x_2b}
  % \end{align*}
\end{proof}

\end{document}