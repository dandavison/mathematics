\documentclass[12pt]{article}
\usepackage{mathematics}
\begin{document}

\title{Haskell}
\date{}
\author{Dan Davison}
\maketitle
\tableofcontents

\section{Function}
\begin{notation*}
  Function application is written without parentheses; parentheses are used for grouping
  only. Functions of two arguments can be written as infix binary operators. Here
  are some examples of traditional mathematical notation and Haskell equivalents:\\~\\
  \begin{tabular}{|l|l|l|l|}
    $f(x, y)$         &                     & \mih{f x y}     &\\
    $f(g(x))$         &                     & \mih{f (g x)}       & \mih{f $ g x} \\  % $
    $f(x + y)$        & \mih{f (+ x y)}     & \mih{f (x + y)} & \mih{f $ x + y}\\ % $
    $f(x) + f(y)$     & \mih{+ (f x) (f y)} & \mih{f x + f y} &\\
    $(f \circ g)(x)$  &                     & \mih{(f . g) x} &\\
  \end{tabular}

  The following syntax has no standard equivalent in traditional mathematical notation:\\~\\
  \begin{tabular}{|l|l|l|}
    list concatenation           & \mih{list1 ++ list2} & E.g. \mih{[1,2] ++ [3,4] == [1, 2, 3, 4]}\\
    prepend (``cons'') to a list &\mih{x:list1}         & E.g. \mih{1:[2,3,4] == [1, 2, 3, 4]}\\
  \end{tabular}

\end{notation*}

\section{Type}
\begin{definition*}
  A {\bf type} is a set.
\end{definition*}

\begin{comment}
  \begin{remark*}
    The type is defined at compile time. The elements of the set are data values (including functions) that exist at run time.
  \end{remark*}
\end{comment}

\begin{notation*}\hspace{0pt}\\
  The type definition syntax \mih{x :: A} means $x \in A$, i.e. value {\tt x} has type {\tt A}.

  The type definition syntax \mih{f :: A -> B} means $f:A \to B$, i.e. $f$ is an element of the set
  of functions with domain $A$ and codomain $B$.
\end{notation*}

\begin{remark*}
  It follows that the type definition syntax \mih{f :: A -> (B -> C)} means $f$ is a function with
  domain $A$ and codomain the set of functions $B \to C$. Note that:
  \begin{enumerate}
  \item $f$ can also be thought of as a function with domain $A \times B$ and codomain
    $C$.\footnote{\mih{A -> B -> C} is known as the ``curried'' version and \mih{(A X B) -> C} is
      known as the ``uncurried'' version.}
  \item By convention the parentheses are ommitted and \mih{->} is right-associative. Thus in real
    Haskell code this would always be written \mih{f :: A -> B -> C}.
  \end{enumerate}

\end{remark*}

\begin{example*}
  An example of a type is the set of lists containing items of type $a$.

  Here we declare that variable {\tt x} is of type list-of-integers and assign a value to
  {\tt x}:
  \begin{minted}{haskell}
    x :: [Int]
    x = [1, 1]
  \end{minted}

  And here we define a new type which contains a list of integers, and create a variable of
  that type:
  \begin{minted}{haskell}
    data IntList = MakeIntList [Int]

    y :: IntList
    y = MakeIntList [2, 2]
  \end{minted}

  We can extract the native list and {\tt Int} values from the {\tt IntList} value by pattern-matching:
  \begin{minted}{haskell}
    IntList myList = y
    IntList [i, j] = y
  \end{minted}
  Now {\tt myList} is {\tt  [2, 2]} and {\tt i} and {\tt j} are both {\tt 2}.

\end{example*}

\section{Type class}
\begin{definition*}
  A {\bf typeclass} is a set of types.
\end{definition*}

\begin{remark*}\hspace{0pt}
  \begin{enumerate}
  \item A type class is defined using the \mih{class} keyword followed by some methods that all
    instances of the type class must have. For example
    \begin{normalfont}
      \begin{minted}[escapeinside=\\`\\`]{haskell}
        class Monoid a where
            mempty :: a
            mappend :: a `$\to$` a `$\to$` a
      \end{minted}
    \end{normalfont}
  \item We can declare that type \mih{A} is an instance of type class \mih{X} in two ways:
    \begin{enumerate}
    \item At type-definition time. The class method implementations will be derived
      automatically. E.g. \mih{data A = MakeA deriving X}
    \item After type-definition time. Here we can provide implementations of class methods. E.g.
      \begin{normalfont}
        \begin{minted}[escapeinside=\\`\\`]{haskell}
          instance A Monoid
              where mappend x y = x ++ y
        \end{minted}
      \end{normalfont}
    \end{enumerate}
  \item A type signature can have a type class constraint: \mih{f :: X a => a -> B} means: let
    \mih{a} be any type that is\footnote{In fact, in Haskell one says ``has an instance of'', not
      ``is an instance of''.} an instance of type class \mih{X}; then \mih{f :: a -> B}. This is an
    example of a polymorphic function definition.
  \end{enumerate}
\end{remark*}


\section{Functor}
\begin{definition*}\hspace{0pt}
  \begin{normalfont}
    \begin{minted}[escapeinside=\\`\\`]{haskell}
      class Functor f where
          (<>) :: (a `$\to$` b) `$\to$` f a `$\to$` f b
    \end{minted}
  \end{normalfont}
\end{definition*}

%           (<$) :: a `$\to$` [b] `$\to$` [a]
%           (<$) = fmap . const

\begin{remark*}\hspace{0pt}
  \begin{enumerate}
  \item Note that the functor \mih{f} is an operator on types. It maps a type to a new type.
  \item The new type can be viewed as a ``container'' or a ``context'' for objects of type \mih{a}.
  \item \mih{(<>)}, also called \mih{fmap}, operates on a function \mih{g :: a -> b}, ``lifting''
    it to operate on objects of type \mih{f a}.
  \end{enumerate}
\end{remark*}


\begin{example*}\hspace{0pt}
  \begin{enumerate}

  \item An example of a functor is the list constructor \mih{[]}. In this case \mih{fmap} is just
    \mih{map}:
    \begin{minted}[escapeinside=\\`\\`]{haskell}
      instance Functor [] where
          (<>) g (x:xs) = g x : (g <> xs)
    \end{minted}

  \item A second example of a functor is \mih{((->) e)}. It takes a type \mih{a} and maps it to the
    function type \mih{e -> a}. An example is discussed below under Applicative.

  \end{enumerate}
\end{example*}


\section{Applicative}

\begin{definition*}\hspace{0pt}
  \begin{normalfont}
    \begin{minted}[escapeinside=\\`\\`]{haskell}
      class Applicative f where
          <*> :: f (a `$\to$` b) `$\to$` (f a `$\to$` f b)
    \end{minted}
    % $
  \end{normalfont}
\end{definition*}

\begin{example*}\hspace{0pt}
  \begin{enumerate}

  \item An example of a functor is \mih{((->) e)}. It takes a type \mih{a} and maps it to the
    function type \mih{e -> a}. Let's look into this in more detail.

    Suppose we have the predicate functions
    \begin{minted}[escapeinside=\\`\\`]{haskell}
    isAlpha :: Char `$\to$` Bool
    isDigit :: Char `$\to$` Bool
    \end{minted}
    and what we want is a nicer version of
    \begin{minted}[escapeinside=\\`\\`]{haskell}
    eitherPredicate :: (Char `$\to$` Bool) -> (Char `$\to$` Bool) -> (Char -> Bool)
    eitherPredicate p1 p2 c = p1 c || p2 c
    \end{minted}

    \begin{mdframed}
    From {\tt ghc/libraries/base/GHC/Base.hs}:
    \begin{minted}[escapeinside=\\`\\`]{haskell}
    instance Functor ((->) r) where
        fmap = (.)

    instance Applicative ((->) a) where
        pure = const
        (<*>) f g x = f x (g x)
        liftA2 q f g x = q (f x) (g x)
    \end{minted}
    \end{mdframed}

    Consider the type of \mih{liftA2}:
    \begin{minted}[escapeinside=\\`\\`]{haskell}
    liftA2 :: Applicative f => (a `$\to$` b `$\to$` c) `$\to$` (f a `$\to$` f b `$\to$` f c)
    \end{minted}
    In the instance where \mih{f} is \mih{((->) e)} this is
    \begin{minted}[escapeinside=\\`\\`]{haskell}
    liftA2 :: (a `$\to$` b `$\to$` c) `$\to$` ((e `$\to$` a) `$\to$` (e `$\to$` b) `$\to$` (e `$\to$` c))
    \end{minted}
    So in this Applicative instance, what \mih{liftA2} does is lift a binary operator so that it
    operates on the {\it outputs} of two functions.

    \begin{minted}[escapeinside=\\`\\`]{haskell}
    liftA2 (||) isAlpha isDigit :: Char `$\to$` Bool
    \end{minted}

    So \mih{liftA2} has taken \mih{||} and lifted it to operate on \mih{Char -> Bool} functions.

  \end{enumerate}
\end{example*}


\section{Monoid}
\newcommand{\map}{\text{{\tt map}}}
\newcommand{\foldr}{\text{{\tt foldr}}}
\newcommand{\op}{\oplus}
\newcommand{\id}{\epsilon}

\begin{definition*}
  A monoid is a set $S$ with an associative binary operation $\op$ and an identity element $\id$.
\end{definition*}

\begin{remark*}
  A group is a monoid in which each element has an inverse.
\end{remark*}

\begin{definition*}
  A {\bf type} is a set.
\end{definition*}

\begin{definition*}
  A {\bf list} containing items of type $a$ is a sequence $a_1, a_2, \ldots$ where $a_i \in a$ for
  all $i \in \N$.
\end{definition*}

\begin{example*}\hspace{0pt}
  The set of lists of type \mintinline{haskell}{[a]} is a monoid (the binary operation is
  concatenation and the identity is the empty list).
\end{example*}

\begin{definition*}
  $f:S \to T$ is a homomorphism if $f(x \op y) = f(x) \op f(y)$ for all $x, y \in S$.\footnote{Note that $\op$ is used here for two potentially distinct operators: one in $S$ and one
    in $T$.}
\end{definition*}

\begin{remark*}\hspace{0pt}
  \begin{enumerate}
  \item Equivalent notation: $(f \circ \op)(x, y) = \op(f(x), f(y))$.
  \item In some sense this is saying that for a homomorphism $f$, it doesn't matter in what order
    we perform $f$ and $\op$.
  \end{enumerate}

\end{remark*}

\begin{definition*}[$\map$, $\foldr$]\hspace{0pt}\\
  $\map$ takes a function of one argument and a list of input values and returns a list of output
  values:
  \begin{align*}
    \map(g, [x, y, z, ...]) = [g(x), g(y), g(z), ...]
  \end{align*}

  $\foldr$ takes a function of two arguments, an initial value, and a sequence of input values, and
  performs a right-associative ``reduce'' operation:
  \begin{align*}
    \foldr(g, \alpha, [x, y, z]) = g(x, f(y, f(z, \alpha))).
  \end{align*}
\end{definition*}

\begin{remark*}\hspace{0pt}
  \begin{enumerate}
  \item When $\alpha$ is an identity element for $g$ and the input is a length-2 list, {\tt foldr}
    is just function application:
  \begin{align*}
    \foldr(g, \id, [x, y]) = g(x, g(y, \id)) = g(x, y).
  \end{align*}
\item   Here are type definitions and implementations of {\tt map} and {\tt foldr} in Haskell:
  \begin{normalfont}
    \begin{minted}[escapeinside=\\`\\`]{haskell}
      map :: (a `$\to$` b) `$\to$` [a] `$\to$` [b]
      map g (x:xs) = g x : map g xs
      map g [] = []

      foldr :: (a `$\to$` b `$\to$` b) `$\to$` b `$\to$` [a] `$\to$` b
      foldr g `$\alpha$` (x:xs) = g x (foldr f `$\alpha$` xs)
      foldr g `$\alpha$` [] = `$\alpha$`
    \end{minted}
  \end{normalfont}
  \end{enumerate}
\end{remark*}

Instances of the {\tt Monoid} typeclass are defined to have an identity element and a closed binary
operation:
\begin{minted}[escapeinside=\\`\\`]{haskell}
  class Monoid a where
    `$\id$` :: a
    `$\op$` :: a `$\to$` a `$\to$` a
\end{minted}

In addition a function {\tt mconcat} is defined for Monoid instances. This ``reduces'' a list of elements:
\begin{minted}[escapeinside=\\`\\`]{haskell}
  mconcat :: Monoid a => [a] `$\to$` a
  mconcat = foldr `$\op$` `$\id$`
\end{minted}

\subsection{Free Monoid Homomorphism theorem}


\newpage
\begin{theorem*}
  Let $S$ be a monoid with identity $\id$. Let $x, y \in S$ and let $f:S \to T$ be a
  homomorphism. Then
  \begin{align*}
    f(x \op y) = \foldr(\op, f(\id), \map(f, [x, y])).
  \end{align*}
  I.e.
  \begin{normalfont}
    \begin{minted}[escapeinside=\\`\\`]{haskell}
               f (x `$\op$` y) = foldr `$\op$` f(`$\id$`) (map f [x, y])
      `$\iff$`  (f . `$\op$`) x y = (foldr `$\op$` f(`$\id$`) . (map f)) [x, y]
      `$\iff$`  (f . `$\op$`) x y = (mconcat . (map f)) [x, y]
      `$\iff$`  (f . `$\op$`) = mconcat . map f                      -- well, not exactly, but similar
    \end{minted}
  \end{normalfont}
\end{theorem*}

\begin{proof}
  From the definition of $\map$ and $\foldr$ we have:
  \begin{align*}
    \foldr(\op, f(\id), \map(f, [x, y]))
    &= \foldr(\op, f(\id), [f(x), f(y)])\\
    &= f(x) \op (f(y) \op f(\id))\\
    &= f(x) \op f(y),
  \end{align*}
  which is equal to $f(x \op y)$ since $f$ is a homomorphism.
\end{proof}

\begin{remark*}
  What is this saying?


\end{remark*}

Yorgey (2012) gives the following:

\begin{definition*}\hspace{0pt}
  \begin{normalfont}
    \begin{minted}{haskell}
      hom :: Monoid m => (a -> m) -> ([a] -> m)
    \end{minted}
  \end{normalfont}
\end{definition*}

\begin{theorem*}\hspace{0pt}
  \begin{normalfont}
    \begin{minted}{haskell}
      hom f = mconcat . map f
    \end{minted}
  \end{normalfont}
\end{theorem*}

{\bf Are these saying the same thing?}\\
Let's look at the types of the different components:
\begin{normalfont}
  \begin{minted}[escapeinside=\\`\\`]{haskell}
    map f :: [a] `$\to$` [f a]
    mconcat . map f :: Monoid f a => [a] `$\to$` f a
  \end{minted}
  So \mih{f a} is a monoid, i.e.
  \begin{normalfont}
    \begin{minted}[escapeinside=\\`\\`]{haskell}
      f :: Monoid m => a `$\to$` m
    \end{minted}
  \end{normalfont}
\end{normalfont}


\end{document}