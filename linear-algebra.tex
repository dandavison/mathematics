\documentclass[12pt]{article}
\usepackage{notes}
\usepackage{fullpage,amsfonts,amsmath,amssymb,mathpazo,graphicx,verbatim,parskip,footnote}
\usepackage[left=2cm,top=2cm,right=2cm,bottom=2cm,head=2cm,foot=1cm]{geometry}
\usepackage[colorlinks=true,urlcolor=blue]{hyperref}
% \newcommand{i}{\mathbf{i}}
% \newcommand{j}{\mathbf{j}}
\renewcommand{\vec}[1]{\mathbf{#1}}


\begin{document}

\begin{center}
  \section*{
    Linear Algebra
    \footnote{
      \href{https://www.youtube.com/playlist?list=PLZHQObOWTQDPD3MizzM2xVFitgF8hE_ab}{Essence of Linear Algebra} video series by \href{http://www.3blue1brown.com/}{Grant Sanderson / 3blue1brown}
    }
  }
\end{center}

\section{ Linear transformations and matrices}

A linear transformation is completely specified by

\begin{enumerate}
\item Some basis vectors $i$ and $j$
\item Where those basis vectors are taken to by the transformation.
\end{enumerate}

How the transformation affects any other point follows from those two pieces of
information.

So $i$ might be taken to $ai + bj$, and $j$ might be taken to $ci + dj$.
In this case we would use the following matrix to describe the
transformation:

$$
\mat{a}{c}
    {b}{d}
$$

Some examples are

$$
\begin{array}{ll}
\text{stretch by a in the i-direction} & \mat{a}{0}
                                             {0}{1}
\\\\
\text{stretch by a in the i-direction and shear right} & \mat{a}{b}
                                                             {0}{1}
\\\\
\text{rotate anticlockwise 90°} & \mat{0}{-1}
                                      {1}{ 0}
\end{array}
$$

Note that we haven't said what $i$ and $j$ are yet; they \textit{define} the
2-dimensional space that we're considering. But, we can think of them for now
as the usual orthogonal unit vectors in 2D space.

So the matrix tells us where the basis vectors have been taken to. Any other
vector $fi + gj$ is taken to wherever that is using the transformed basis
vectors:

$$
fi + gj \longrightarrow f\cvec{a}{b} + g\cvec{c}{d} = \cvec{fa + gc}{fb + gd}
$$


And that's how matrix multiplication is defined:

$$
\mat{a}{c}
    {b}{d} \cvec{f}{g} = \cvec{fa + gc}{fb + gd}
$$


A matrix represents a linear transformation by showing where the basis vector
are taken to.


\section{ Change of basis}

Suppose person B uses some other basis vectors to describe locations in
space. Specifically, in our coordinates, their basis vectors are
$\scvec{2}{1}$ and $\scvec{-1}{1}$.


\textbf{When they state a vector, what is it in our coordinates?}

If they say $\scvec{-1}{2}$, what is that in our coordinates?

Well, if they say $\scvec{1}{0}$, that's $\scvec{2}{1}$ in our coordinates. And
if they say $\scvec{0}{1}$, that's $\scvec{-1}{1}$ in our coordinates. So the
matrix containing \textit{their basis vectors expressed using our coordinate system}
transforms a point expressed in their coordinate system into one expressed in
ours. That last sentence is critical, so hopefully it makes sense! So, the answer is

$$
\mat{2}{-1}
    {1}{ 1} \cvec{-1}{2} = \cvec{-4}{1}.
$$


\textbf{When we state a vector, what is it in their coordinates?}

We give the vector $\scvec{3}{2}$. What is that in their coordinate system? By
definition, the answer is the weights that scales their basis vectors to hit
$\scvec{3}{2}$. So, the solution to

$$
\mat{2}{-1}
    {1}{1} \cvec{a}{b} = \cvec{3}{2}.
$$


Computationally, we can see that we can get the solution by multiplying both
sides by the inverse:

$$
\cvec{a}{b} = \mat{2}{-1}
                  {1}{1}^{-1} \cvec{3}{2}.
$$

Conceptually, we have

$$
\mat{2}{-1}
    {1}{1} =
\begin{bmatrix}\text{matrix converting their}\\\text{representation to ours} \\ \end{bmatrix}
$$

where "their representation" means the vector expressed using their coordinate
system. So the role played by the inverse is

$$
\cvec{a}{b} =
\begin{bmatrix}\text{matrix converting our}\\\text{representation to theirs} \\ \end{bmatrix}
\cvec{3}{2}.
$$

\textbf{When we state a transformation, what is it in their coordinates?}

We state a 90° anticlockwise rotation of 2D space:

$$
\mat{0}{-1}
    {1}{0}
$$

what is that transformation in their coordinates? The answer is

$$
\begin{bmatrix}\text{matrix converting our}\\\text{representation to theirs} \\ \end{bmatrix}
\mat{0}{-1}
    {1}{0}
\begin{bmatrix}\text{matrix converting their}\\\text{representation to ours} \\ \end{bmatrix}
$$

since the composition of those three transformations defines a single
transformation that takes in a vector expressed in their coordinate system,
converts it to our coordinate system, transforms it as requested, and then
converts back to theirs.


\section*{Linear and quadratic approximations to a function
  \footnote{
    \href{https://www.khanacademy.org/math/multivariable-calculus/applications-of-multivariable-derivatives/optimizing-multivariable-functions/a/reasoning-behind-the-second-partial-derivative-test}{khanacademy - Grant Sanderson - second partial derivative test}
  }
}



We construct first- and second-order approximations to a differentiable
function $f: \R^2 \rightarrow \R$. The approximation is made at some point
$(x_0, y_0) = \vec x_0 \in \R^2$; we demand that the value of the approximation, and the
first and second derivatives, match those of $f$ exactly at that point.

\subsection*{Linear approximation to a function $f(x, y)$ near $(x_0, y_0)$:}

\begin{align*}
L(x, y) &=
~
f(x_0, y_0) ~+
(x - x_0)f_x(x_0,y_0) +
(y - y_0)f_y(x_0,y_0)
\\\\
&= f(\vec x) + (\vec x - \vec x_0) \cdot \nabla_f(\vec x_0)
\end{align*}

Note that, at $(x_0, y_0)$, the first partial derivatives of $L$ are equal to
those of $f$, as they must be. (In fact, we could say that the coefficients are
determined by this requirement; see the quadratic case below. But the linear
case is obvious without ``deriving'' the coefficients.)


\subsection*{Quadratic approximation to a function $f(x, y)$ near $(x_0, y_0)$:}

First note that the ``quadratic form'' $ax^2 + 2bxy + cy^2$ can be written as
\begin{align*}
\cvec{x}{y}^\T \mat{a}{b}
                   {b}{c} \cvec{x}{y}.
\end{align*}

\begin{align*}
Q(x, y) &=
f(\vec x_0) + (x - x_0)f_x(\vec x_0) +
(y - y_0)f_y(\vec x_0) ~+ \\
&~~~~~~~\frac{1}{2} f_{xx}(\vec x_0)(x - x_0)^2 +
f_{xy}(\vec x_0)(x - x_0)(y - y_0) +
\frac{1}{2} f_{yy}(\vec x_0)(y - y_0)^2 \\\\
&= f(\vec x_0) +
(\vec x - \vec x_0) \cdot \nabla f(\vec x_0) +
\frac{1}{2}(\vec x - \vec x_0)^\T \nabla^2 f(\vec x_0)(\vec x - \vec x_0),
\end{align*}
where $\nabla^2 f(\vec x_0)$ is the Hessian matrix $\mat{f_{xx}}{f_{xy}}
                                                        {f_{yx}}{f_{yy}}$ evaluated at $\vec x_0$.

\subsection*{Second partial derivative test and positive definiteness of Hessian}

The second partial derivative test for a function of two variables states that
we examine the determinant of the Hessian evaluated at the critical point:
$$
D = \det \nabla^2 f(\vec x_0) = f_{xx}(\vec x_0)f_{yy}(\vec x_0) - f_{xy}(\vec x_0)^2.
$$

Notice that $D \geq 0$ implies that the sign of $f_{xx}$ and $f_{yy}$ agree
(because we're subtracting the square of the mixed partial $f_{xy}$, i.e. a
positive number).

\begin{tabular}{ l l l l l }
  $D$    & roots          & $f_{xx}$ &  & Hessian \\
  \hline
  $+$    & no real roots  & $+$     & minimum        & positive definite \\
  $+$    & no real roots  & $-$     & maximum        & negative definite \\
  $0$    & one real root  & $+$     & minimum        & positive semidefinite \\
  $0$    & one real root  & $-$     & maximum        & negative semidefinite \\
  $-$    & two real roots & n/a     & saddle point   & - \\
\end{tabular}

\subsubsection*{Explanation}
At a critical point $\vec x_0$, the gradient is zero and the quadratic approximation is therefore
$$
Q(x, y) = f(\vec x_0) + \frac{1}{2}(\vec x - \vec x_0)^\T \nabla^2 f(\vec x_0)(\vec x - \vec x_0).
$$
So if this is a minimum (concave-up paraboloid) then this quadratic form is
positive for all $\vec x \neq \vec x_0$ (and if it's a maximum then it's
negative for all $\vec x \neq \vec x_0$).

Basically the argument is that, instead of analyzing the function $f$ itself,
we analyze its quadratic approximation at the critical point. So the question
comes down to: how do we determine whether a quadratic form is always positive,
always negative, or takes positive and negative values?

To answer that, consider a generic quadratic form $ax^2 + 2bxy + cy^2$. Let $y$
be constant at $y_0$; then we have a quadratic in $x$, the roots of which are
\begin{align*}
  x
  = \frac{-2by_0 \pm \sqrt{4b^2y_0^2 - 4acy_0^2}}{2a}
  = y_0\frac{-b \pm \sqrt{b^2 - ac}}{a}.
\end{align*}
So, whether this is a saddle point or a minimum/maximum depends on whether the
quadratic form has real roots. If there are no real roots, then whether it's a
minimum or a maximum depends on the sign of $f_{xx}$ (this sign will be the
same as that of $f_{yy}$ in the no real roots case).

\subsection*{Derivation of quadratic approximation coefficients}
\begin{align*}
Q(x, y) =
&f(\vec x_0) + (x - x_0)f_x(\vec x_0) +
(y - y_0)f_y(\vec x_0) ~+ \\
&a(x - x_0)^2 +
b(x - x_0)(y - y_0) +
c(y - y_0)^2
\end{align*}

What are the coefficients $a,b,c$? They are determined by the requirement that
the second partial derivatives are identical at the point of approximation
$\vec x_0$.

First look at the first partial derivatives:

\begin{align*}
  Q_x &= f_x(\vec x_0) + 2a(x - x_0) + b(y - y_0)\\
  Q_y &= f_y(\vec x_0) + b(x - x_0) + 2c(y - y_0)\\
\end{align*}
so the quadratic approximation is an exact first-order approximation at $\vec x_0$, as required:
\begin{align*}
  Q_x(\vec x_0) &= f_x(\vec x_0) \\
  Q_y(\vec x_0) &= f_y(\vec x_0),
\end{align*}

Now look at the second derivatives:
\begin{align*}
  Q_{xx} &= 0 + 2a + 0 \\
  Q_{xy} &= 0 + 0 + b \\
  Q_{yx} &= 0 + b + 0 \\
  Q_{yy} &= 0 + 0 + 2c \\
\end{align*}
Since we require that these match those of $f$ exactly at $\vec x_0$, we have
\begin{align*}
  a &= \frac{1}{2} f_{xx}(\vec x_0) \\
  b &= f_{xy}(\vec x_0) = f_{yx}(\vec x_0) \\
  c &= \frac{1}{2} f_{yy}(\vec x_0),
\end{align*}
so the quadratic approximation is
\begin{align*}
Q(x, y) =
&f(\vec x_0) + (x - x_0)f_x(\vec x_0) +
(y - y_0)f_y(\vec x_0) ~+ \\
&\frac{1}{2} f_{xx}(\vec x_0)(x - x_0)^2 +
f_{xy}(\vec x_0)(x - x_0)(y - y_0) +
\frac{1}{2} f_{yy}(\vec x_0)(y - y_0)^2
\end{align*}

\end{document}