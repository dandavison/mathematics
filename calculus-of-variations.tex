\section{}
\footnotetext{Notes from Classical Mechanics by John R. Taylor, ch. 6}

The calculus of variations can be used to find a function $y(x)$ that minimizes a scalar quantity
that is expressed as an integral $\int_{x_0}^{x_1} f[y(x), y'(x), x] \dx$. Here are two such
problems:

\begin{question*}
  What is the shortest path between two points in a plane?
\end{question*}

\begin{proof}
  Let the points be $(x_0, y_0)$ and $(x_1, y_1)$ and let them be joined by some path $y(x)$ of
  length $L$. Consider a short section of the path of length $\Delta l$ above a section of the
  $x$-axis of length $\Delta x$, and make a linear approximation to the path in this region. The
  length of the hypotenuse is
  \begin{align*}
    \Delta l = \sqrt{(\Delta x)^2 + (y'(x)\Delta x)^2} = \sqrt{1 + y'(x)^2} \Delta x.
  \end{align*}
  Therefore a shortest path is a function $y(x)$ that minimizes
  \begin{align*}
    L = \int_{x_0}^{x_1} \d l = \int_{x_0}^{x_1} \sqrt{1 + y'(x)^2} \dx,
  \end{align*}
  with the constraint that the endpoints are fixed at $y(x_0) = y_0$ and $y(x_1) = y_1$.

  \todo{Find the function $y$ that minimizes this integral.}
\end{proof}


\begin{question*}
  In 1662 Fermat proposed that light, when passing from one point to another through a material with
  varying refractive index, takes the path which takes least time\footnote{\todo{In fact, the path
      taken is a stationary point with respect to the action? time? ...not necessarily least}}. What
  is this path?
\end{question*}

\begin{proof}
  Again consider a short section of the path of length $\Delta l$ above a section of the $x$-axis of
  length $\Delta x$. Let $c$ be the speed of light and $n$ be the refractive index in this
  region. This means that the light travels at speed $c/n$, and therefore takes time $(n/c)\Delta l$
  to pass along the hypotenuse. The refractive index $n$ can vary with both $x$ and $y$, therefore a
  least-time path is a function that minimizes
  \begin{align*}
    T = \int_{x_0}^{x_1} n(x, y(x)) \d l = \int_{x_0}^{x_1} n(x, y(x)) \sqrt{1 + y'(x)^2} \dx,
  \end{align*}
  with the constraint that the endpoints are fixed at $y(x_0) = y_0$ and $y(x_1) = y_1$.

  \todo{Find the function $y$ that minimizes this integral.}
\end{proof}

Note that in both problems, the integral can be viewed as a scalar-valued \emph{functional} that
depends on the \emph{functions} $y(x)$ and $y'(x)$.
