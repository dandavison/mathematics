\section{1. The Nature of  Classical Mechanics}
\begin{enumerate}
\item Discrete state machines: simplest example of dynamical system (directed graph)
\item To be valid in classical mechanics, dynamical law must be:
  \begin{enumerate}
  \item Deterministic (There's only one next state)
  \item Reversible (Conservation of information: there's only one previous state)
  \end{enumerate}
\item Even in a cycle, there's still one in-edge and one out-edge
\item There can be separate cycles: these correspond to conservation laws (The system stays in the cycle it started in.)
\end{enumerate}

\section{2. Motion}

\section{3. Dynamics}
$F = ma$

\section{4.    Systems of  More Than One Particle}
\begin{enumerate}
\item The force vector acting on a particle is determined by the locations of all the particles:
  \begin{align*}
    m_i\ddot{\x}_i = \vec{F_i} = \vec{F_i}(\x_1, \x_2, \ldots, \x_N).
  \end{align*}
\item Note that
  \begin{enumerate}
  \item There are $N$ such equations.
  \item Each equation is really 3 equations: one for each spatial dimension.
  \end{enumerate}
\item So, the {\it configuration space} is $3N$-dimensional. Given a point in configuration space, we can compute the acceleration
  vector of each particle in the system.
\item But that is not enough to specify the evolution of the system: we need to know the current velocities (momenta)
  also.
\item So our {\it state space} is $6N$-dimensional \footnote{In physics, this full state space is called ``phase space'': ``configuration space plus momentum space
    equals phase space''.}.
\item Note that the dynamical law can be written
  \begin{align*}
    \vec{\dot{p}}_i = \vec{F_i} = \vec{F_i}(\x_1, \x_2, \ldots, \x_N).
  \end{align*}
\item A consequence of Newton's 3rd Law is that the net force on a closed system of  $N$ particles is zero.
\item Therefore the rate of change of momentum is zero: the law of \emph{conservation of momentum}.
\item Recall from chapter 1 the notion that cycles in a dynamical system correspond to conservation laws. Here we have
  an example:
  \begin{enumerate}
  \item We can label points in $6N$-dimensional state space according to the total momentum of the system.
  \item Conservation of momentum means that these partitions of state space are cycles / unconnected components of the
    graph.
  \end{enumerate}
\item So the system corresponds to a point $(\x, \vec{p})$ in a $6N$-dimensional state space. Our system evolves according
  to a trajectory $(\x(t), \vec{p}(t))$ in this state space. This trajectory never jumps between points
  with different values of total system momentum.

\end{enumerate}

\todo{Explain why the second-order DE is equivalent to two first-order
  DEs, i.e. why the $3N$ N2L equations are really $6N$ equations.}

\section{5. Energy}
\begin{enumerate}
\item Configuration $\x$ is a point in a $3N$-dimensional space.
\item Potential energy function $V(\x)$.
\item Force as negative derivative of PE $V(\x)$.
\item Time-derivative of PE + KE is zero under N2L.
\end{enumerate}
\section{6. The Principle of Least Action}

\subsection*{Definition of Action and Lagrangian}
The \textit{action} of a trajectory $x(t)$ for a single particle is
\begin{align*}
    \mathcal{A}[x]
    &=  \int_{t_0}^{t_1} \(\frac{1}{2}m\dot{x}(t)^2 - V\(x(t)\)\) \dt \\
    &= \int_{t_0}^{t_1} \(T - V\) \dt \\
    &= \int_{t_0}^{t_1} \Lag\(x(t), \dot{x}(t)\) \dt.
  \end{align*}

  \red{Why does he seem to deny that you need to know velocity for the Lagrangian? p 108, 109}

  \subsection*{The Principle of Least Action}
  The {\it principle of least action} states that the trajectory $x(t)$ that is taken is that for which the
  action is minimal (actually, a stationary point).

  \subsection*{N particles}
  For $N$ particles we have $N$ trajectory functions $x_1, \ldots, x_N$, and the action is a functional depending
  on all of them:

  \begin{align*}
    \mathcal{A}[x_1, \ldots, x_N]
    &=  \int_{t_0}^{t_1} \(\frac{1}{2}\sum_im_i\dot{x_i}(t)^2 - V(x_1(t), \ldots, x_n(t)\) \dt \\
    &= \int_{t_0}^{t_1} \(T - V\) \dt \\
    &= \int_{t_0}^{t_1} \Lag(x_1(t), \ldots, x_2(t), \dot{x_1}(t), \ldots, \dot{x_2}(t)) \dt.
  \end{align*}


  \subsection*{The Euler-Lagrange equations}
  The least action set of trajectories (trajectory of the system through configuration space) satisfies the
  following {\it at every point $t$ in time}.  :
  \begin{align*}
    \begin{cases}
      \ddt \pdLdxdOne - \pdLdxOne = 0 \\
      \vdots \\
      \ddt \pdLdxdN - \pdLdxN = 0
    \end{cases}
  \end{align*}

  \subsection*{Informal derivation of Euler-Lagrange equations}

  \subsubsection*{One-dimensional system}

  Consider a particle moving along a line. The (unknown) trajectory it follows is written as $x(t)$.

  The ``action'' for the system is the functional
  \begin{align*}
      \mathcal{A}[x]
      &= \int_{t_a}^{t_b} \Lag\( x(t), \dot{x}(t) \) \dt.
  \end{align*}
  \begin{theorem*}[Euler-Lagrange equations]
    The trajectory $x(t)$ that minimize $\mathcal{A}$ satisfies the following for all $t \in (t_a, t_b)$:
    \begin{align*}
      \ddt \pdLdxd = \pdLdx.
    \end{align*}
  \end{theorem*}
  \begin{proof} (sketch)

    We pretend that time is discrete and that the system evolves via $n$ instantaneous ``jumps'' each
    $\Delta t$ seconds apart, so that the particle visits locations $x_1, x_2, \ldots, x_n$ at clock
    ticks $1, 2, \ldots, n$.

    The discretized version of the functional we wish to minimize is
    \begin{align*}
      \mathcal{A}[\x]
      &= \sum_i \Lag\( x_i, \dot{x}_i \) \Delta t.
    \end{align*}
    Now, instead of seeeking a minimizing function in an infinite-dimensional function space, we are working in a
    finite-dimensional space: we need to find the values $x_1, \ldots, x_n$ that minimize the action. So we
    differentiate with respect to each $x_i$, set these derivatives equal to zero, and solve the resulting system
    of equations.

    In our discrete-time model, we make the following replacements: at the $i$-th clock tick, the position
    is $\frac{x_{i+1} + x_i}{2}$, and velocity is $\frac{x_{i+1} - x_{i}} {\Delta t}$, so we have
    \begin{align*}
      \mathcal{A}[\x]
      &= \sum_i \Lag\( \frac{x_{i+1} + x_i}{2}, \frac{x_{i+1} - x_i} {\Delta t} \) \Delta t.
    \end{align*}
    The partial derivative with respect to a particular $x_i$ involves only the $(i-1)$-th and $i$-th terms of the sum:
  \begin{align*}
    \frac{1}{\Delta t}\frac{\partial \mathcal{A}}{\partial x_i}
    &= \frac{\partial}{\partial x_i} \(
    \Lag\( \frac{x_i + x_{i-1}}{2}, \frac{x_i - x_{i-1}}{\Delta t} \) +
    \Lag\( \frac{x_{i+1} + x_i}{2}, \frac{x_{i+1} - x_{i}}{\Delta t} \) \) \\
    &=
    \frac{1}{2}\pdLdx\Big|_{x=x_{i-1}} + \frac{1}{\Delta t}\pdLdxd\Big|_{x=x_{i-1}} +
    \frac{1}{2}\pdLdx\Big|_{x=x_i}     - \frac{1}{\Delta t}\pdLdxd\Big|_{x=x_i} \\
    &= \(\frac{\pdLdx\Big|_{x=x_i} + \pdLdx\Big|_{x=x_{i-1}}}{2} \) -
      \(\frac{\pdLdxd\Big|_{x=x_i} - \pdLdxd\Big|_{x=x_{i-1}}}{\Delta t} \).
  \end{align*}
  In the limit $\Delta t \to 0$, (we claim that) the condition $\frac{\partial A}{\partial x_i} = 0$ for $i=1,\ldots, n$ becomes
  \begin{align*}
    0 = \pdLdx - \ddt \pdLdxd,
  \end{align*}
  for all $t$.
\end{proof}

\subsection*{Equivalence of Lagrange equations and Newton's second law}
\begin{exercise}
  Let $\Lag(x, \xdot) = \frac{1}{2}m\xdot^2 - V(x)$. Show that the Euler-Lagrange
  equation $\pdLdx = \ddt \pdLdxd$ is equivalent to Newton's law of motion $F = ma$.
\end{exercise}
\begin{proof}
  On the LHS we have $\pdLdx = -V'(x) = F$.

  We have $\pdLdxd = m\xdot$, therefore on the RHS we have $\ddt \pdLdxd = m\xddot = ma$.
\end{proof}


\subsection*{Generalized coordinates}
With ``generalized coordinates'' $q$ in place of $x$, this captures ``all of classical physics in a nutshell!
If you know what the $q_i$s are, and if you know the Lagrangian, then you have it all.''

\subsection*{Changing coordinate system}

In observer A's frame of reference, a particle is at $x$.

Observer B is moving relative to observer A according to a function $f(t)$.

In observer B's frame of reference, the particle is at $X = x - f(t)$.

According to observer $A$, the Lagrangian is $\frac{1}{2}m\xdot^2 - V(x)$, which under E-L leads to the
equation of motion $m\xddot = -V'(x)$ or $F = ma$.

According to observer $B$, the Lagrangian is $\frac{1}{2}m\(\dot{X}(t) + \dot{f}(t)\)^2 - V(X)$, which under
E-L leads to the equation of motion $m\(\ddot{X} + \ddot{f}\) = -V'(x)$, or $F = ma + m\ddot{f}$. So
observer $B$ sees a fictitious force $m\ddot{f}$, associated with their movement.

\subsection*{Change of coordinates Example 2}

Observer A (reference frame A) uses coordinates $x$ and $y$.

Observer B (reference frame B) uses coordinates $X$ and $Y$.

Reference frame B is rotating relative to A:

\begin{mdframed}
  \includegraphics[width=400pt]{img/physics--susskind--the-theoretical-minimum--1.-the-nature-of-classical-mechanics--6.-the-principle-of-least-action--28fe.png}
\end{mdframed}

To translate between the two reference frames:

\begin{align*}
    x &= X \cos \omega t + Y \sin \omega t \\
    y &= -X \sin \omega t + Y \cos \omega t.
  \end{align*}
  i.e.
\begin{align*}
  \vecMM{x}{y} = \matMMxNN{\cos \omega t}{-\sin \omega t}{\sin \omega t}{\cos \omega t} \vecMM{X}{Y}.
\end{align*}

\begin{align*}
  \vecMM{\xdot}{\ydot} =
        \matMMxNN{ \cos \omega t}{-\sin \omega t}{\sin \omega t}{\cos \omega t} \vecMM{\dot{X}}{\dot{Y}} +
  \omega\matMMxNN{-\sin \omega t}{-\cos \omega t}{\cos \omega t}{-\sin \omega t} \vecMM{X}{Y}
\end{align*}
\TODO


Observer A sees a particle moving with no forces acting on it, so the Lagrangian from their point of view
is $\frac{1}{2}m(\xdot^2 + \ydot^2)$.

From observer B's point of view, we have
\begin{align*}
  \xdot &= -\omega X \sin \omega t + \dot{X}\cos \omega t  +  \omega Y \cos \omega t \\
  \ydot &= -\omega X \cos \omega t - \omega Y \sin \omega t,
\end{align*}
therefore
\begin{align*}
    \xdot^2 &= \omega^2\(X^2\sin^2\omega t - 2XY\sin \omega t \cos\omega t + Y^2\cos^2\omega t\) \\
    \ydot^2 &= \omega^2\(X^2\cos^2\omega t + 2XY\sin \omega t \cos\omega t + Y^2\sin^2\omega t\),
\end{align*}
and so the Lagrangian for observer B is
\begin{align*}
  \frac{1}{2}m(\xdot^2 + \ydot^2) =
  \omega^2\frac{1}{2}m(X^2 + Y^2).
\end{align*}

\subsection*{Conjugate momentum}
Note that if $\Lag = \frac{1}{2}\sum_im\xdot_i^2 - V(x_1, \ldots, x_N)$ as usual,
then $\pdLdxdi = m\xdot_i$. Accordingly, $p_i := \pdLdqdi$ is referred to as the ``conjugate momentum''
to $q_i$.

So a streamlined statement of the Euler-Lagrange equations for classical mechanics is
\begin{align*}
  \dot{p_i} = \pdLdqi.
\end{align*}

In words, the time derivative of the conjugate momentum equals the rate of change of the Lagrangian with respect to the
generalized spatial coordinate.


\subsection*{Conserved quantities}
Momentum is conserved when there is no potential energy function (which would be dependent on the spatial
coordinates).

In general, if there is a generalized coordinate that appears in the Lagrangian only via its velocity (if there
exists a change of coordinates such that this is true), then the corresponding generalized momentum is
conserved.

\section{7.  Symmetries and Conservation Laws}

\begin{enumerate}
\item $\Lag(\qdot, q) = \frac{1}{2}(\dot{q_1}^2 + \dot{q_2}^2) - V(q_1 - q_2)$
\end{enumerate}
Derive the equations of motion:
\begin{align*}
  \ddt \pdLdqdOne &= \pdLdqOne \\
  \ddt \pdLdqdTwo &= \pdLdqTwo \\
\end{align*}
\begin{align*}
  \dot{p}_1        &= -V'(q_1 - q_2) \\
  \dot{p}_2        &= +V'(q_1 - q_2)
\end{align*}
The sign difference is related to the fact that the potential depends on the distance between the two
particles: they are both approaching each other. This has something to do with Newton's third law.

So $\ddt(p_1 + p_2) = 0$: generalized momentum is conserved.

\item $\Lag(\qdot, q) = \frac{1}{2}(\dot{q_1}^2 + \dot{q_2}^2) - V(aq_1 - bq_2)$
\begin{align*}
  \dot{p}_1        &= -aV'(aq_1 - bq_2) \\
  \dot{p}_2        &= +bV'(aq_1 - bq_2)
\end{align*}
\begin{align*}
  \dot{p}_1 + \dot{p}_2 &= (b-a)V'(aq_1 - bq_2) \\
\end{align*}
Now generalized momentum appears not to be conserved in general.

But we see that $bp_1 + ap_2$ is conserved.

\section*{8. Hamiltonian Mechanics and Time-Translation Invariance}

For a Lagrangian with no explicit time dependence we have
\begin{align*}
  L = L({q_i}, {\dot{q_i}}),
\end{align*}
and from the chain rule
\begin{align*}
  \dLdt = \sum_i \(\pdLdqi \dot{q}_i + \pdLdqdi \ddot{q}_i\).
\end{align*}
So, even without explicit time dependence, the Lagrangian does of course vary over time, because the
generalized coordinates and velocities vary with time.

Now consider a Lagrangian with an explicit time dependence: $L = L(q_i, \dot{q}_i, t)$; we now have
\begin{align*}
  \dLdt = \sum_i \(\pdLdqi \dot{q}_i + \pdLdqdi \ddot{q}_i\) + \pdLdt.
\end{align*}
Recall that the Euler-Lagrange equations for classical mechanics $\ddt \pdLdqdi = \pdLdqi$ can be
written $\dot{p_i} = \pdLdqi$, which implies that $p_i = \pdLdqdi$. So we can write the time-derivative of the
Lagrangian as
\begin{align*}
  \dLdt = \sum_i \(\dot{p}_i \dot{q}_i + p_i \ddot{q}_i\) + \pdLdt,
\end{align*}
which via the product rule is
\begin{align*}
  \dLdt = \ddt \sum_i p_i\dot{q}_i + \pdLdt,
\end{align*}
or equivalently
\begin{align*}
  \ddt\(\sum_i p_i\dot{q}_i - \Lag\) = -\pdLdt.
\end{align*}
Accordingly we define the {\it Hamiltonian}
\begin{align*}
  H := \sum_i p_i\dot{q}_i - \Lag,
\end{align*}
so that we have
\begin{align*}
  \dHdt = -\pdLdt.
\end{align*}
Thus the Hamiltonian is a quantity that is constant in time if and only if the Lagrangian has no explicit time
dependence.

In other words, if a system is {\it time-translation invariant}, then the Hamiltonian is conserved.

\subsection*{Example}

Consider a particle in a potential. The Lagrangian is
\begin{align*}
  \Lag = \frac{1}{2}m\dot{x}^2 - V(x).
\end{align*}
The Hamiltonian is then
\begin{align*}
  H
  &= p\dot{q} - \Lag \\
  &= m\dot{x} \cdot \dot{x} - \frac{1}{2}m\dot{x}^2 + V(x) \\
  &= \frac{1}{2}m\dot{x}^2 + V(x),
\end{align*}
i.e. it is the total energy of the system.

This holds for a system comprising any number of particles. $p_i\dot{q}_i$ is like $m\dot{x} \cdot \dot{x}$,
which is twice the $i$-th component of the kinetic energy. So if the Lagrangian is $T - V$, then we have
\begin{align*}
  H
  &= \sum_i p_i\dot{q}_i - \Lag \\
  &= 2T - (T - V) \\
  &= T + V.
\end{align*}
Even if the Lagrangian has a more complex form than $T - V$, the Hamiltonian is defined in the same way, and it
is conserved if and only if the Lagrangian lacks explicit time dependence, and in fact one defines for these
systems:
\begin{align*}
  \text{Energy} ~=~ \text{Hamiltonian}.
\end{align*}
One way (the only way?) in which the Lagrangian may have explicit time dependence is if we are only considering
part of a system: in this case, energy is not in general conserved.

\subsection*{Phase space and Hamilton's equations}

The Lagrangian formulation focuses on motion $q(t)$ through configuration space, and involves second order
differential equations: so we must specify the initial positions, and the initial velocities.

The Hamiltonian formulation focuses on motion $\(q(t), p(t)\)$ through {\it phase space}, and involves first order
differential equations.

For a particle moving on a line, we have
\begin{align*}
  H
  &= \frac{1}{2}m\dot{x}^2 + V(x).
\end{align*}
To derive Hamilton's equations, we rewrite this in terms of momentum $p$ and consider the derivative with
respect to $p$:
\begin{align*}
  H
  &= \frac{1}{2}m\dot{x}^2 + V(x) \\
  &= \frac{1}{2}m\(\frac{p}{m}\)^2 + V(x) \\
  &= \frac{p^2}{2m} + V(x).
\end{align*}
So we have
\begin{align*}
  \pdHdx = \frac{\d V}{\dx} = -m\ddot{x} = -\dot{p},
\end{align*}
and
\begin{align*}
    \pdHdp = \frac{p}{m} = \dot{x}.
\end{align*}
Hamilton's equations for a system of any number of generalized coordinates are
\begin{align*}
  \dot{p}_i &= -\pdHdqi \\
  \dot{q}_i &= ~~\pdHdpi.
\end{align*}
So if you know the form of the Hamiltonian, and you know the values of the coordinates at some point in time,
you can simulate the trajectory of the system through phase space.

\subsection*{The harmonic oscillator Hamiltonian}

Let $q$ be a degree of freedom with potential energy $V(q)$, and that $V(q)$ has a minum representing a stable
equilibrium value of $q$. WLOG we suppose the minimum is at $q=0$. If $q$ stays close to zero then it will be
accurate to approximate $V(q)$ as a quadratic. The constant term may be taken to be zero (since potential
energy is defined relative to some position), and the linear term must be zero since we want $V'(0) = 0$.
Therefore our model is
\begin{align*}
  V(q) = cq^2.
\end{align*}
The Lagrangian may be written
\begin{align*}
  \Lag
  &= T - V \\
  &= \frac{1}{2\omega}\dot{q}^2 - \frac{\omega}{2}q^2.
\end{align*}
\begin{proof}
  Let $x$ be a coordinate (degree of freedom) with Lagrangian $\Lag = \frac{m}{2}\dot{x}^2 - \frac{k}{2}x^2$.
  Now define another coordinate $q = (km)^{1/4}x$. Then
  \begin{align*}
    x^2 &= \frac{1}{(km)^{1/2}}q^2 \\
    \dot{x}^2 &= \frac{1}{(km)^{1/2}}\dot{q}^2 \\
    \Lag &= \frac{m^{1/2}}{2k^{1/2}}\dot{q}^2 - \frac{k^{1/2}}{2m^{1/2}}q^2 \\
    &= \frac{1}{2\sqrt{k/m}}\dot{q}^2 - \frac{\sqrt{k/m}}{2}q^2 \\
    &= \frac{1}{2\omega}\dot{q}^2 - \frac{\omega}{2}q^2,
  \end{align*}
where $\omega = \sqrt{k/m}$.
 \end{proof}

Now, $\dot{q}^2 = \frac{p^2}{m^2}$, hence $T = \frac{1}{2m^2\sqrt{k/m}}p^2$ and the Hamiltonian is
\begin{align*}
  H
  &= T + V \\
  &= \frac{\omega}{2}(p^2 + q^2). ~~~~~~?
\end{align*}
Therefore Hamilton's equations for the Harmonic Oscillator are
\begin{align*}
  \dot{p}_i &= -\pdHdqi = -\omega q \\
  \dot{q}_i &= ~~\pdHdpi = \omega p.
\end{align*}
For comparison, the Lagrangian equation is $\ddot{q} = \omega \dot{p}$.

So, in phase space -- i.e. $(q, p)$-space -- the oscillator moves around a circle centered on the origin. I.e.
both position and momentum oscillate sinusoidally. In general, the phase point always stays on a contour of
constant energy.

\subsection*{General derivation of Hamilton's equations}
\red{TODO}

\section{9. The Phase Space Fluid and the Gibbs-Liouiville Theorem}

Hamilton's equations define a vector field in phase space and we can think of this vector field as representing the
flow of a fluid.

In general, phase space is $2N$-dimensional. Energy conservation means that the fluid flows along surfaces
of $(2N-1)$-dimensions. For the harmonic oscillator with one degree of freedom, we have 2 dimensions and the
phase space fluid flows in concentric circles around an origin.

\begin{definition*}[divergence]
  The {\it divergence} at some location in a fluid flow (vector field) basically measures the extent to which there is a
  net change in the local density of the fluid, as a result of the flow patterns. In 3-dimensional space it's
  defined by considering an infinitesimal cuboid with volume $\dx \dy \dz$. Let $v_x$ be the flow velocity in
  the $x$ direction, and consider an infinitesimal cuboid at some point $\r$.
  If $\frac{\partial v_x}{\partial x}$ at $\r$ is positive, then there is a net loss of fluid from the cuboid
  in the $x$-direction (decrease in density). The amount of fluid lost is given by the rate of loss multiplied
  by the volume:
  \begin{align*}
    \frac{\partial v_x}{\partial x} \dx \dy \dz
  \end{align*}
  The overall change in density is the sum of the changes in density in the 3 spatial
  dimensions, leading to the definition of divergence
\begin{align*}
  \nabla \cdot \v
  := -\(\frac{\partial v_x}{\partial x} +
        \frac{\partial v_y}{\partial y} +
        \frac{\partial v_z}{\partial z}\).
\end{align*}
If a fluid is incompressable, then the divergence is zero everywhere. This means that if you select some volume
of fluid at some point in time, and watch how it changes under the flow, its volume will never change.
\end{definition}

\begin{theorem*}[Gibbs-Liouiville]
  The phase space fluid is incompressable (divergence is everywhere zero).

  \begin{proof}
    Divergence for the $2N$-dimensional phase space fluid is
    \begin{align*}
      \nabla \cdot \v =
      -\sum_{i=1}^N \(\frac{\partial \dot{q}_i}{\partial q_i} +
                     \frac{\partial \dot{p}_i}{\partial p_i}\).
    \end{align*}
    Recall Hamilton's equations:
\begin{align*}
  \dot{p}_i &= -\pdHdqi \\
  \dot{q}_i &= ~~\pdHdpi.
\end{align*}
Therefore
    \begin{align*}
      \nabla \cdot \v :=
      -\sum_{i=1}^N \(\frac{\partial^2H}{\partial q_i \partial p_i} -
                      \frac{\partial^2H}{\partial p_i\partial q_i}\)
      = 0.
    \end{align*}
  \end{proof}

  \begin{intuition*}
    We already knew that the phase fluid flows along constant-energy contours ($(2N-1)$-dimensional surfaces).
    The Gibbs-Liouiville theorem is related to the idea of reversibility: the fact that a blob of continuous
    phase space fluid always retains the same volume is analogous to the notion that, in a finite state machine
    representing a physical process, a given state must always have exactly one precursor state and one
    successor state.
  \end{intuition*}