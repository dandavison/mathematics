\section*{Math 202a - HW5 - Dan Davison - \texttt{ddavison@berkeley.edu}}

\begin{mdframed}
\includegraphics[width=400pt]{img/analysis--berkeley-202a-hw05-781b.png}
\end{mdframed}


\begin{definition*}
  Let $d_n(\om)$ be the $n$-th digit in the binary expansion of $\om$. If $\om$ has two equivalent binary
  expansions, we use the non-terminating one. Define
  \begin{align*}
    A_n := \{\om ~:~ d_n(\om) = 0, ~ \om \in [0, 1]\}.
  \end{align*}
  Thus, for example, $A_1 = (0, 1/2]$ and $A_2 = (0, 1/4] \cup (1/2, 3/4]$.
\end{definition*}

\begin{claim*}
  $A_n$ is Lebesgue measurable for all $n \in \N$.
\end{claim*}

\begin{proof}~\\~\\
  Since $A_n$ is a finite union of intervals of the form $(a, b]$, and since
  \begin{align*}
    (a, b] = \bigcap_{n=1}^\infty (a, b + n^{-1}),
  \end{align*}
  we see that $A_n$ is a finite union of open intervals in $[0, 1]$, hence in the Borel $\sigma$-algebra
  on $[0, 1]$, and hence in the Lebesgue $\sigma$-algebra on $[0, 1]$.
\end{proof}

\begin{claim*}
  $\mu(A_n) > 0$ for all $n$.
\end{claim*}

\begin{proof}~\\~\\
  Note that $(0, 2^{-n}] \subseteq A_n$, therefore by monotonicity of measure
  \begin{align*}
    \mu(A_n) \geq \mu((0, 2^{-n}]) = 2^{-n} > 0.
  \end{align*}
\end{proof}

\begin{claim*}
  $\mu(A_n \Delta A_m) > 0$ if $n \neq m$.
\end{claim*}

\begin{proof}~\\~\\
  Let $m \neq n$ and suppose without loss of generality that $m < n$.

  Then $(0, 2^{-m}] \subseteq A_m$ and also $(2^{-n}, 2^{-(n+1)}] \subset A_m$.
  But $(2^{-n}, 2^{-(n+1)}] \subset A_n^c$ and has non-zero measure, therefore $\mu(A_n \Delta A_m) > 0$.
\end{proof}

\begin{claim*}
  $\mu(A_n \cap A_m) = \mu(A_n)\mu(A_m)$ if $n \neq m$.
\end{claim*}

\begin{proof}~\\~\\
  Let $m \neq n$ and suppose without loss of generality that $m < n$.

  Recall that half of the rank-$i$ dyadic intervals are contained within $A_i$ (those corresponding to
  the $i$-th digit being zero). Therefore $A_i$ is the union of $2^{i-1}$ intervals each of length $2^{-i}$,
  and we have for all $i$
  \begin{align*}
     \mu(A_i) = 2^{i-1}2^{-i} =\frac{1}{2},
  \end{align*}
  therefore $\mu(A_n)\mu(A_m) = \frac{1}{4}$.

  So we need to show that $\mu(A_n \cap A_m) = \frac{1}{4}$.

  Recall that $A_m$ is the union of $2^{m-1}$ dyadic intervals. Let $I$ be one of these intervals. Recall
  that $I$ is partitioned exactly by $2^{n-m}$ rank-$n$ dyadic intervals, each of length $2^{-n}$, and that one
  half of these consist entirely of real numbers with $m$-th digit zero, while the other half have $m$-th digit
  one. Therefore,
  \begin{align*}
    \mu(A_n \cap A_m) = 2^{m-1}2^{n-m}2^{-n}2^{-1} = \frac{1}{4}.
  \end{align*}
\end{proof}
\newpage
\begin{mdframed}
\includegraphics[width=400pt]{img/analysis--berkeley-202a-hw05-2e34.png}\\
It is specified also that $\mu(E) > 0$.
\end{mdframed}


\begin{proof}~\\~\\
  Let $E \subseteq \R$ be Lebesgue measurable and fix $\alpha \in (0, 1)$.

  Set $\eps = \mu(E)\frac{1 - \alpha}{\alpha}$. Since $E$ is Lebesgue measurable there exists an open set $O$
  such that $E \subseteq O$ and $\mu(O \setminus E) < \eps$. Therefore
  \begin{align*}
    \mu(O)
    &= \mu(O \cap E) + \mu(O \setminus E) \\
    &= \mu(E) + \mu(O \setminus E) \\
    &< \mu(E) + \mu(E)\frac{1 - \alpha}{\alpha} \\
    &= \mu(E)/\alpha,
  \end{align*}
  that is,
  \begin{align*}
    \mu(E) = \mu(E \cap O) > \alpha\mu(O).
  \end{align*}
  Now, write $O$ as a countable union of disjoint open intervals $O = \bigcup_{i=1}^\infty I_i$. Note that
    \begin{align*}
    \mu(E)
    &= \mu\Big(E \cap \big(\bigcup_{i=1}^\infty I_i\big)\Big) \\
    &= \mu\Big(\bigcup_{i=1}^\infty E \cap I_i\Big) \\
    &= \sum_{i=1}^\infty \mu(E \cap I_i).
  \end{align*}
  Also $\mu(O) = \sum_{i=1}^\infty \mu(I_i)$. Therefore we have
  \begin{align*}
    \mu(E) = \sum_{i=1}^\infty \mu(E \cap I_i) > \alpha \sum_{i=1}^\infty \mu(I_i).
  \end{align*}
  To see that this implies the desired result, suppose for a contradiction
  that $\mu(E \cap I_i) \leq \alpha \mu(I_i)$ for all $i$. Then we have
  \begin{align*}
    \sum_{i=1}^\infty \mu(E \cap I_i) \leq \alpha \sum_{i=1}^\infty \mu(I_i),
  \end{align*}
  a contradiction.

  Therefore there exists an open interval $I_i$ such that $\mu(E \cap I_i) > \alpha \mu(I_i)$.
\end{proof}


\newpage
\begin{mdframed}
\includegraphics[width=400pt]{img/analysis--berkeley-202a-hw05-ad8b.png}
\end{mdframed}

\begin{remark*}
  If $A$ includes an open interval $(a, b)$ then the result easily follows on considering the
  map $z:(a, b)^2\to [a-b, b-a]$. But $A$ may not include any interval and still have positive measure: the
  generalized Cantor sets provide examples.
\end{remark*}

\begin{remark*}
  I tried for a few hours to answer this without hints but in the end followed the sequence of hints provided
  at \url{https://math.stackexchange.com/a/1079520/397805}.
\end{remark*}

\begin{proof}~\\~\\
  Let $A \subset \R$ be a Borel set of $\R$ with $\mu(A) > 0$.

  Let $F \subseteq A$ be a closed set such that $\mu(A \setminus F) < \epsilon/2$, and let $G \supseteq A$ be
  an open set such that $\mu(G \setminus A) < \epsilon/2$. Thus we have $\mu(G) - \mu(F) > \eps$.

  Let $|d| < \eps$.

  Let $A + d$ denote the set of translates $\{a + d : a \in A\}$. We want to show that
  \begin{align*}
    (A + d) \cap A \neq \emptyset.
  \end{align*}
  Note that $(F + d) \subseteq G$ and $F \subseteq G$, therefore
  \begin{align*}
    \mu((F + d) \cup F)
    &\leq \mu(A) + \mu(G \setminus A) \\
    &= \mu(A) + \eps/2.
  \end{align*}

  Suppose that $(F + d) \cap F = \emptyset$.

  Then $\mu((F + d) \cup F) = \mu(F + d) + \mu(F) = 2\mu(F)$ by additivity and translation invariance of
  measure, and we have
  \begin{align*}
    2\mu(F) \leq \mu(A) + \eps/2.
  \end{align*}
  But this leads to a contradiction since we have
  \begin{align*}
    2\mu(F)
    &\leq \mu(A) + \eps/2 \\
    &=    \mu(F) + \mu(A \setminus F) + \eps/2 \\
    &\leq \mu(F) + \eps/2 + \eps/2,
  \end{align*}
  which requires that $\mu(F) \leq \eps$. \red{TODO} {\bf clarify what the contradiction is}
\end{proof}


\newpage
\begin{mdframed}
\includegraphics[width=400pt]{img/analysis--berkeley-202a-hw05-40cd.png}
\end{mdframed}

\begin{proof}~\\~\\
  We will use the following facts:
  \begin{enumerate}
  \item Every open interval contains a rational.
  \item A Cantor-like set includes no interval and yet has positive measure (and is uncountable).
  \end{enumerate}

  Basically what we want to do is construct $A$ from Cantor-like sets that contains every rational and yet
  includes no interval. Because $A$ includes no interval we will have $\mu(A \cap I) < \mu(I)$ for every
  interval $I$. And because $A$ is Cantor-like we will have $0 < \mu(A \cap I)$.

  Let $C: \Q \to \ms{P}(\R)$ be a function that associates with every rational a certain Cantor-like set.
  Specifically, for $q \in \Q$ we construct $C(q)$ as follows:

\end{proof}

\newpage
\begin{mdframed}
\includegraphics[width=400pt]{img/analysis--berkeley-202a-hw05-3d2c.png}
\end{mdframed}

\begin{proof}~\\~\\

\end{proof}