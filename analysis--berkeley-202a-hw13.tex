\section*{Math 202A - HW13 - Dan Davison - \texttt{ddavison@berkeley.edu}}

\begin{mdframed}
\includegraphics[width=400pt]{img/analysis--berkeley-202a-hw13-26dd.png}
\end{mdframed}

\begin{enumerate}
\item
  \begin{claim*}
    $A \union B$ is connected.
  \end{claim*}

  \begin{proof}
    We must show that $A \union B$ cannot be written as the union of two disjoint non-empty open sets.

    Suppose for a contradiction that $U$ and $V$ are disjoint open sets such that $U \union V = A \cup B$.
    Let $z \in B$ and without loss of generality suppose that $z \in V$. Then $B \subseteq V$ (since $B$ is
    connected). Note that the open ball at $(0, 0)$ contains a point of $B$. But $U$ and $V$ are disjoint,
    therefore $(0, 0) \in V$. But $(0, 0) \in A$ and $A$ is connected, so $A \subseteq V$.
    Therefore $A \union B = V$. But this is a contradiction since $U$ and $V$ are disjoint and non-empty.
    Therefore $A \union B$ are connected.
  \end{proof}


  \begin{claim*}
    $A \union B$ is not path-connected.
  \end{claim*}

  \begin{proof}
    Suppose for a contradiction that there exists a continuous function $f: [0, 1] \to A \union B$ such
    that $f(0) = (0, 0)$ and $f(1) = (0, 1)$.

    Suppose for a contradiction that there exists a continuous function $f: [0, 1] \to A \union B$ such
    that $f(0) = (0, 1)$ and $f(1) = (0, 0)$.

    Write $f = (f_x, f_y)$ where $f_x: [0, 1] \to [0, 1]$ and $f_y: [0, 1] \to [-1, 1]$ give the coordinates of $f$.

    Let $\eps > 0$ and let $(t_n)_{n=1}^\infty$ be an increasing sequence with $0 < t_n < 1$ and $t_n \to 1$.
    Then $f_y(t_n) \to 0$ since $f$ is continuous. Let $N$ be such that $f_y(t_n) < \eps$ for all $n \geq N$. But
    this is a contradiction since $f_y (t_n))$ is oscillating between $-1$ and $1$ but $\eps$ is arbitrary.
    Therefore no such continuous function $f$ exists.
  \end{proof}
\item

  Informally, we will take vertical intervals positioned at each point of a countably infinite sequence of
  rationals and between each successive pair, place a variant of the ``topologist's sine curve​'' constructed so
  that it ``speeds up​'' in both directions. The resulting topological space will be connected but not
  path-connected, because we have seen above that that is what happens when the topologist's sine curve
  approaches a vertical interval.

  Let $\N$ be the natural numers excluding $0$.

  Let $q_n = \sum_{i=1}^n 2^{-n} \in \Q \cap [\frac{1}{2}, 1)$. Then $\{q_n ~:~ n \in \N\}$ is a countably
  infinite set of rationals in $[\frac{1}{2}, 1)$.

  Let $I_n = \{q_n\} \times [0, 1] \subset U$. This is a vertical interval of unit length.

  Let $m_n = (q_{n+1} - q_n)/2$ and let $S_n = \{\sin\((|x - m_n| - m_n)^{-1}\) ~:~ x \in (q_n, q_{n+1})\}$. We
  will refer to this as a ``bidirectional topologist's sine curve​''.

  Finally, let $U = \{I_n ~:~ n \in \N\} \union \{S_n ~:~ n \in \N\}$.

  Informally, $U$ consists of a countable collection of vertical intervals together with a ``bidirectional
  topologist's sine curve​'' between each successive pair of vertical intervals.


  \begin{claim}
    $U$ is a bounded and connected subset of $\R^2$.
  \end{claim}

  \begin{proof}
    $U \subset [0, 1]^2 \subset \R^2$ therefore $U$ is a bounded subset of $\R^2$.

    Recall that we proved above that $V := \(\{0\} \times [0, 1]\) \union \{\sin(1/x) ~:~ x \in (0, 1]\}$ is a
    connected topological space.

    It follows that $I_n \union S_n$ is connected for all $n$ (formally, I believe we can prove this by
    exhibiting a homeomorphism between $V$ and $I_n \union S_n$ and noting that a topological space is
    connected if it is homeomorphic to a different connected topological space. Or possibly I would have to use
    just the ``first half​'' of $S_n$, i.e. $I_n \cup \(S_n \cap \((q_n, m_n) \times [0, 1]\)\)$)

    Similarly, it follows that $S_n \cup I_{n+1}$ is connected for all $n$ (the geometry is identical but with
    left-right orientation reversed).

    It then follows by induction that $U$ is connected.
  \end{proof}

  \begin{claim}
    $U$ has infinitely many path-connected components and infinitely many of these take the form of a vertical
    interval of unit length.
  \end{claim}

  \begin{proof}
    Let $I_n = \{q_n\} \times [0, 1] \subset U$. This is a vertical interval of unit length. Clearly, $I_n$ is
    path-connected. To see this, let $(q_n, y_1), (q_n, y_2) \in I_n$ and without loss of generality suppose
    that $y_1 < y_2$. - Then $f: [0, 1] \to I_n$ defined by $f(t) = (q_n, y_1 + t(y_2 - y_1))$ has the property
    that $f(0) = (q_n, y_1)$ and $f(1) = (q_n, y_2)$.

    Recall that $S_n$ is a ``bidirectional topologist's sine curve​'' in the current construction, and recall also
    that we proved above that $\(\{0\} \times [0, 1]\) \union \{\sin(1/x) ~:~ x \in (0, 1]\}$ is not path
    connected. It follows that neither $I_n \union S_n$ nor $I_n \union S_{n-1}$ is path-connected, and it
    follows from this that $I_n \union I_{m}$ is not path-connected for all $n \neq m$. We
    have $\bigcup_{n \in \N} I_n \subset U$, therefore $U$ contains infinitely many path-connected components
    which take the form of a vertical interval of unit length.
  \end{proof}
\end{enumerate}



\newpage
\begin{mdframed}
\includegraphics[width=400pt]{img/analysis--berkeley-202a-hw13-e2f7.png}
\end{mdframed}

\newpage
\begin{mdframed}
\includegraphics[width=400pt]{img/analysis--berkeley-202a-hw13-983a.png}
\end{mdframed}

\begin{proof}
  This is theorem 20.30 of Bass; the proof is given there. The theorem states that if $X$ is compact Hausdorff
  then $X$ is normal, and the conditions asked for here are part of the definition of normal.\\
  \includegraphics[width=400pt]{img/analysis--berkeley-202a-hw13-7531.png}
\end{proof}


\newpage
\begin{mdframed}
\includegraphics[width=400pt]{img/analysis--berkeley-202a-hw13-9356.png}
\end{mdframed}

First, let's examine some small finite sets and the possible topologies that meet the specified condition. The
following table excludes topologies that differ only by a relabeling of the elements in the underlying set.

\begin{tabular}{l|l|l}
  set&non-trivial&non-trivial\\
     &open subsets&closed subsets \\
  \hline
  $\{\}$             &                                                                                                 & \\
  \hline
  $\{1\}$            &                                                                                                 & \\
  \hline
  $\{1, 2\}$         & $\{1\}$                                                                                         & $\{2\}$\\
  \hline
  $\{1, 2, 3\}$      & $\{1\}$                                                                                         & $\{2, 3\}$ \\
  $\{1, 2, 3\}$      & $\{1\}, \{2\}, \{1, 2\}$                                                                        & $\{2, 3\}, \{1, 3\}, \{3\}$ \\
  $\{1, 2, 3\}$      & $\{1\}, \{1, 2\}$                                                                               & $\{2, 3\}, \{3\}$ \\
  $\{1, 2, 3\}$      & \sout{$\{1\}, \{2, 3\}$}                                                                        & \sout{$\{2, 3\}, \{1\}$} \\
  $\{1, 2, 3\}$      & $\{1, 2\}$                                                                                      & $\{3\}$ \\
  $\{1, 2, 3\}$      & \sout{$\{1\}, \{2\}, \{1, 2\}, \{3\}, \{1, 3\}, \{2, 3\}$}                                      & \sout{$\{2, 3\}, \{1, 3\}, \{3\}, \{1, 2\}, \{2\}, \{1\}$} \\
  $\{1, 2, 3\}$      & \sout{$\{1\}, \{2\}, \{1, 2\}, \{1, 3\}$}                                                       & \sout{$\{2, 3\}, \{1, 3\}, \{3\}, \{2\}$} \\
  \hline
  $\{1, 2, 3, 4\}$   & $\{1\}$                                                                                         & $\{2, 3, 4\}$ \\
  $\{1, 2, 3, 4\}$   & $\{1\}, \{2\}, \{1, 2\}$                                                                        & $\{2, 3, 4\}, \{1, 3, 4\}, \{3, 4\}$ \\
  $\{1, 2, 3, 4\}$   & $\{1\}, \{1, 2\}$                                                                               & $\{2, 3, 4\}, \{3, 4\}$ \\
  $\{1, 2, 3, 4\}$   & $\{1\}, \{2, 3\}, \{1, 2, 3\}$                                                                  & $\{2, 3, 4\}, \{1, 4\}, \{4\}$ \\
  $\{1, 2, 3, 4\}$   & $\{1\}, \{1, 2, 3\}$                                                                            & $\{2, 3, 4\}, \{4\}$ \\
  $\{1, 2, 3, 4\}$   & \sout{$\{1\}, \{2, 3, 4\}$}                                                                     & \sout{$\{2, 3, 4\}, \{1\}$} \\
  $\{1, 2, 3, 4\}$   & $\{1, 2\}$                                                                                      & $\{3, 4\}$ \\
  $\{1, 2, 3, 4\}$   & \sout{$\{1, 2\}, \{3, 4\}$}                                                                     & \sout{$\{3, 4\}, \{1, 2\}$} \\
  $\{1, 2, 3, 4\}$   & $\{1\}, \{2\}, \{1, 2\}, \{3\}, \{1, 3\}, \{2, 3\}, \{1, 2, 3\}$                                & $\{2, 3, 4\}, \{1, 3, 4\}, \{3, 4\}, \{1, 2, 4\}, \{2, 4\}, \{1, 4\}, \{4\}$ \\
  $\{1, 2, 3, 4\}$   & \sout{$\{1\}, \{2\}, \{1, 2\}, \{3\}, \{1, 3\}, \{2, 3\}, \{1, 2, 3\}$},                        & \sout{$\{2, 3, 4\}, \{1, 3, 4\}, \{3, 4\}, \{1, 2, 4\}, \{2, 4\}, \{1, 4\}, \{4\}$}, \\
                     & ~~~~\sout{$\{4\}, \{1, 4\}, \{2, 4\}, \{1, 2, 4\}, \{3, 4\}, \{1, 3, 4\}, \{2, 3, 4\}$}         & ~~~~\sout{$\{1, 2, 3\}, \{2, 3\}, \{1, 3\}, \{3\}, \{1, 2\}, \{2\}, \{1\}$} \\
\end{tabular}

\newpage
\begin{mdframed}
\includegraphics[width=400pt]{img/analysis--berkeley-202a-hw13-aa83.png}\\
\includegraphics[width=400pt]{img/analysis--berkeley-202a-hw13-f49f.png}
\end{mdframed}
