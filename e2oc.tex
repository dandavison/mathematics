\documentclass[12pt]{article}
\usepackage{notes, mdframed,caption}

\DeclareMathOperator{\df}{\text{df}}
\DeclareMathOperator{\dg}{\text{dg}}
\let\dh\relax\DeclareMathOperator{\dh}{\text{dh}}

\begin{document}

\section{Essence of Calculus}
\section{The paradox of the derivative}
\section{Derivatives formulas through geometry}
\section{Visualizing the chain rule and product rule}

More complex functions can be formed by addition, multiplication and
composition of simpler functions. How do we compute derivatives of such more
complex functions?

\subsection{Sum rule}
Suppose $f(x) = g(x) + h(x)$. Visualize an input parameter $x$ represented by
the x-axis, and the graphs of $g$ and $h$, and a third graph of $f$ whose
height at every point is the sum of the other two.

A horizontal nudge $\dx$ to the input causes vertical changes $\d g(x)$ and
$\d h(x)$. The resulting vertical change to $f(x)$ is
\begin{align*}
  \d f(x) = \d g(x) + \d h(x),
\end{align*}
or equivalently
\begin{align*}
  \frac{\d f(x)}{\dx} = \frac{\d g(x)}{\dx} + \frac{\d h(x)}{dx}.
\end{align*}


\subsection{Product rule}
Suppose $f(x) = g(x) \cdot h(x)$. Consider an input parameter $x$ and visualize a
rectangle with one side length $g(x)$ and the other side length $h(x)$. $f(x)$
is the area of the rectangle.

A nudge $\dx$ to the input causes the sides to grow by $\d g(x)$ and $\d h(x)$
respectively. Therefore the change to the area is approximately
\begin{align*}
  h(x) \d g(x) + g(x) \d h(x),
\end{align*}
or equivalently
\begin{align*}
  h(x) \frac{\d g(x)}{dx} + g(x) \frac{\d h(x)}{dx}.
\end{align*}

\subsection{Chain rule: function composition}
Suppose $f(x) = g(h(x))$. Visualize 3 real number lines: at the top the input
parameter $x$; in the middle $h(x)$ and at the bottom $g(h)$.

A nudge $\dx$ to the input causes a change $\d h = \frac{\d h}{\dx} \dx$, which in turn causes a
change $\dg = \frac{\dg}{\dh} \dh$. So we have
\begin{align*}
  \df = \d g(h(x)) = \frac{\dg}{\dh} \frac{\dh}{\dx} \dx,
\end{align*}
or equivalently
\begin{align*}
  \frac{\df}{\dh} = \frac{\d g(h(x))}{\dx} = \frac{\dg}{\dh} \frac{\dh}{\dx}.
\end{align*}

\subsubsection{Example}
$f(x) = \sin(x^2) = g(h(x))$. So the middle number line shows $h(x) = x^2$ and
the output number line at the bottom shows $g(h) = \sin(h)$.

We know that for the outer function, $\dg = \cos h \dh$, and for the inner
function $\dh = 2x \dx$ and , so
\begin{align*}
  \frac{\d g(h(x))}{\dx} = \cos (h) \cdot 2x = \cos(x^2)\cdot 2x.
\end{align*}
\end{document}