\section{Callahan - Advanced calculus: a geometric view}


\begin{enumerate}
\item
  Recall that $\dds \tan s = \sec^2 s = 1 + \tan^2 s$.

  Let $x = \tan s$. Then $\dx = \sec^2 s \ds$, and
  \begin{align*}
      \int \frac{\dx}{1 + x^2}
    = \int \frac{\sec^2 s \ds}{\sec^2 s}
    = \int \ds
    = s + C
    = \arctan x + C.
  \end{align*}
  This is an example of using a pullback substitution to find an antiderivative.

  We can use the antiderivative to evaluate some definite integrals:
  \begin{align*}
      \int_0^\infty \frac{\dx}{1 + x^2}
    &= \lim_{b\to\infty}\Big[\arctan x\Big]_0^b
    = \frac{\pi}{2},\\
      \int_{-\infty}^1 \frac{\dx}{1 + x^2}
    &= \lim_{a\to-\infty}\Big[\arctan x\Big]_a^1
    = -\frac{\pi}{2} - \frac{\pi}{4} = -\frac{3}{4}.
  \end{align*}
\item
  Let $u = 1 + x^2$. Then $\du = 2x \dx$, and
  \begin{align*}
    \int \frac{x \dx}{1 + x^2}
    &= \int \frac{x \frac{\du}{2x}}{u}
     = \frac{1}{2} \int \frac{\du}{u}
     = \frac{1}{2} \ln(1 + x^2) + C.
  \end{align*}
  This is an example of a pushforward substitution.
\item~\\
  \newpage
  \begin{mdframed}
    \includegraphics[width=400pt]{img/calculus--callahan--advanced-calculus-1-3.png}
  \end{mdframed}
  Note that $(R\cos\theta)^2 + (R\sin\theta)^2 = R^2$.

  Let $x = R\cos\theta$. This is a pullback substitution.

  Then $\dx = -R\sin\theta\d\theta$, and
  \begin{align*}
            \int_{-R}^R\sqrt{R^2 - x^2}\dx
    &= -R   \int_{\theta=\arccos(-1)}^{\theta=\arccos(1)}\sqrt{R^2 - R^2\cos^2\theta}~\sin\theta\d\theta\\
    &= -R^2 \int_\pi^0\sin^2\theta\d\theta\\
    &= -\frac{R^2}{2} \int_\pi^0(1 - \cos 2\theta)\d\theta\\
    &= -\frac{R^2}{2} \Big[\theta - \frac{1}{2}\sin 2\theta\Big]_\pi^0\\
    &= \frac{\pi R^2}{2}.
  \end{align*}

\end{enumerate}
