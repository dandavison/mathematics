\section{Sequences and Series}

Notes from Oxford - M1 - Sequences and Series.

\subsection{Axioms for the real numbers}
\begin{mdframed}
\includegraphics[width=400pt]{img/oxford-prelims-M2-analysis-I-axioms-for-real-numbers.png}
\end{mdframed}

\subsection{Approximation property of supremum}
\begin{theorem*}
  Let $S \subset \R$ be non-empty and bounded above (so $\sup S$ exists). For all $\delta > 0$,
  there exists $s_\delta \in S$ such that
  \begin{align*}
    \sup S - \delta < s_\delta \leq \sup S.
  \end{align*}
  \begin{intuition*}
    The supremum is either a member of $S$ or it is ``touching'' an element of $S$ with ``no gap''.
  \end{intuition*}
  \begin{proof}
    If $\sup S \in S$ then we can take $s_\delta = \sup S$ for all $\delta$ and we are done.

    So assume $\sup S \not\in S$. For a contradiction, suppose the negation of the claim, i.e. that
    there exists $\delta > 0$ such that for all $s \in S$ either $s \leq \sup S - \delta$ or
    $s > \sup S$. Since $s > \sup S$ is impossible by definition of $\sup$, we have that
    $s \leq \sup S - \delta$ for all $s \in S$. But then $\sup S - \delta$ is an upper bound for
    $S$ and $\sup S - \delta < \sup S$, a contradiction.
  \end{proof}
\end{theorem*}

\subsection{Archimedean Property of $\N$}
\begin{theorem*}\hspace{0pt}
  \begin{enumerate}
  \item $\N$ has no upper bound.
  \item For all $\epsilon > 0$ there exists $n \in \N$ such that $\frac{1}{n} < \epsilon$.
  \end{enumerate}
\end{theorem*}
\begin{proof}\hspace{0pt}
  \begin{enumerate}
  \item Suppose $\N$ has an upper bound. Then $\sup \N$ exists. By the Approximation Property
    there exists $n \in \N$ such that $\sup \N - \frac{1}{2} < n \leq \sup \N$. But then
    $n + 1 \in \N$ and $n + 1 > \sup N$, a contradiction. Therefore $\sup \N$ does not exist,
    therefore $\N$ has no upper bound.
  \item Since $\N$ has no upper bound, there exists $n \in \N$ such that $n > 1/\epsilon$,
    i.e. $1/n < \epsilon$.
  \end{enumerate}
\end{proof}

\subsection{Well-ordered property of $\N$}
\begin{theorem*}\hspace{0pt}
  Every nonempty subset of $\N$ has a minimum.
\end{theorem*}

\begin{proof}
  Let $\emptyset \neq S \subseteq \N \subset \R$. Note that $S$ is bounded below by 0, therefore
  $\inf S$ exists. Suppose $\inf S \not\in S$. By the Approximation Property, there exists
  $n_1 \in S$ such that $\inf S \leq n_1 < \inf S + 1$.

  We claim that $\inf S = n_1$. Suppose for a contradiction that $\inf S \neq n_1$. Then
  $n_1 = \inf S + \delta$ for some $0 < \delta < 1$. By the Approximation property again, there
  exists $n_2 \in S$ such that $\inf S \leq n_2 < n_1 < \inf S + 1$.

  But since $n_1 > n_2$ we have $n_1 \geq n_2 + 1$, therefore $n_1 \geq \inf S + 1$ which
  contradicts $n_1 < \inf S + 1$. Therefore $\inf S = n_1 \in S$ and $\min S$ exists.
\end{proof}

\begin{remark*}
  Similarly:
  \begin{enumerate}
    \item Every nonempty subset of $\Z$ that is bounded below has a minimum.
    \item Every nonempty subset of $\Z$ that is bounded above has a maximum.
  \end{enumerate}
\end{remark*}

\begin{intuition*}
  Because of the ``gappiness'' of $\N$ and $\Z$, bounded subsets must contain their suprema/infima.
\end{intuition*}

\subsection{Existence of ceil and floor}
\begin{definition*}[floor and ceil]
  Let $x \in \R$. Then floor of $x$ is $\floor{x} = \max\{n \in \Z ~|~ n \leq x\}$ and ceil of $x$
  is $\ceil{x} = \min\{n \in \Z ~|~ n \geq x\}$.
\end{definition*}

\begin{theorem*}[$\floor{x}$ and $\ceil{x}$ exist]~\\
  Let $x \in \R$. Define $S = \{n \in \Z ~|~ n \geq x\} \subset \R$. Note that $S$ is bounded below
  by $x$. Also $S$ is non-empty by the Archimedean Property of $\N$, since otherwise $x$ would be
  an upper bound for $\N$. Therefore $\ceil{x} = \min S$ exists by Well-Ordering.

  Similarly, $\floor{x}$ exists.
\end{theorem*}

\subsection{Existence of $\sqrt 2$}
\begin{theorem*}\label{existence-of-root-2}
  There exists a unique $a \in \R$ such that $a^2 = 2$.
\end{theorem*}

\begin{remark*}
  The only thing that ties the proof to the reals is that it relies on completeness ($\sup$
  exists). We know that $\sqrt 2 \not\in \Q$, therefore $\Q$ is not complete.
\end{remark*}

\begin{proof}
  Let $S = \{s \in \R ~|~ s^2 < 2\}$. Since $S$ is bounded above, $a := \sup S$ exists. We show
  that $a^2 = 2$ by showing that $a^2 < 2$ and $a^2 > 2$ lead to contradictions.

  Note that $1 \in S$, therefore $a \geq 1$.

  \begin{enumerate}
  \item {\bf Suppose $a^2 < 2$}. We seek an $h > 0$ such that $(a + h)^2 < 2$ since this would
    contradict the definition $a := \sup S$. Note that
    \begin{align*}
      (a + h)^2 - 2 &= a^2 + 2ah + h^2 - 2\\
                    &< a^2 - 2 + 3ah ~~~~~~~\text{if $h < a$}\\
                    &< 0             ~~~~~~~~~~~~~~~~~~~~~~\text{if $h < (2 - a^2)/3a$}.
    \end{align*}
    Therefore if we take $h < \min\(a, \frac{2 - a^2}{3a}\)$ then $a + h \in S$ which contradicts
    the definition $a := \sup S$.
  \item {\bf Suppose $a^2 > 2$}. By the Approximation Property for all $0 < h < 1$ we can find
    $s \in S$ such that $a - h < s$.  Therefore $(a - h)^2 < s^2 < 2$. We seek a value of $h$ such
    that $(a - h)^2 \geq 2$, which would be a contradiction. Note that $a^2 - 2ah < (a - h)^2$. If
    we take $h = (a^2 - 2)/2a$ then we have $a^2 - 2ah = 2 < (a - h)^2 < 2$, the desired
    contradiction.
  \end{enumerate}

  Finally to show that $a$ is unique, suppose that there exists $b \in \R$ with $b^2 = 2$. Then
  $0 = a^2 - b^2 = (a + b)(a - b)$ therefore $a = b$.
\end{proof}

\subsection{Connection between sequences and functions}
\begin{theorem*}
  The following two statements are equivalent:
  \begin{enumerate}
  \item $f(x) \to L$ as $x \to a$.
  \item For every sequence $(x_n)$ such that $x_n \neq a$
    \begin{align*}
      \(\lim_{n \to \infty} x_n = a\) \implies \(\lim_{n \to \infty} f(x_n) = f(a)\)
    \end{align*}
  \end{enumerate}
\end{theorem*}

\subsection{Limit of product is product of limits}
\begin{theorem*}\label{limit-of-product}~\\
  Let $\limxa f(x) = L_f$ and $\limxa g(x) = L_g$. Then
  $\limxa f(x)g(x) = L_fL_g$.
\end{theorem*}

\begin{proof}
  Note that
  \begin{align*}
    \limxa f(x)g(x) &= \limxa \Big((f(x) - L_f)(g(x) - L_g) + L_fg(x) + L_gf(x) - L_fL_g\Big)\\
                    &= L_fL_g + \limxa (f(x) - L_f)(g(x) - L_g),
  \end{align*}
  so we need to show that $\limxa (f(x) - L_f)(g(x) - L_g) = 0$. Fix $\epsilon > 0$. Since
  $\limxa (f(x) - L_f) = \limxa (g(x) - L_g) = 0$, there exists $\delta$ (pick the minimum of the
  two $\delta$s) such that whenever $|x - a| < \delta$
  \begin{align*}
    |(f(x) - L_f)| < \sqrt \epsilon ~~~\text{and}~~~|(g(x) - L_g)| < \sqrt \epsilon,
  \end{align*}
  therefore $|(f(x) - L_f)(g(x) - L_g) - 0| < \epsilon$ as required.
\end{proof}

\subsection{Limit of quotient is quotient of limits}
\begin{theorem*}~\\
  Let $\limxa f(x) = L_f$ and $\limxa g(x) = L_g \neq 0$. Then
  \begin{align*}
    \limxa \frac{f(x)}{g(x)} = \frac{L_f}{L_g}.
  \end{align*}
\end{theorem*}

\begin{proof}
  \red{TODO}
  \begin{align*}
    \limxa \frac{f(x)}{g(x)} - \frac{L}{M}
    = \limxa \frac{f(x)}{g(x)} - \frac{1}{g(x)} + \frac{1}{g(x)} - \frac{L}{M}
  \end{align*}

  Let $L_f = \limxa f(x)$ and $L_g = \limxa g(x) \neq 0$.

  Fix $\epsilon > 0$ and let $\delta_f$ and $\delta_g$ be such that
  \begin{align*}
    |x - a| < \delta_f \implies |f(x) - L_f| < \epsilon\\
    |x - a| < \delta_g \implies |g(x) - L_g| < \epsilon.
  \end{align*}
  Let $\delta = \min(\delta_f, \delta_g)$. Then
  \begin{align*}
    \frac{|f(x) - L_f|}{|g(x) - L_f|}
  \end{align*}
\end{proof}



\section{Continuity and Differentiability}
Note from Oxford - M2 - Continuity and Differentiability.

\subsection{Limit point}
\begin{definition*}
Let $E \subset \R$. A point $p \in \R$ is a limit point of $E$ iff for all $\delta > 0$ there
exists $x \in E$ such that $0 < |x - p| < \delta$.
\end{definition*}
\begin{intuition*}
  A deleted ball, of arbitrarily small radius, placed over $p$, will capture at least one point of
  $E$.
\end{intuition*}

\subsection{Limit, Convergence}

\begin{definition*}[Limit of a sequence $(x_n)$]~\\
  $\lim_{n \to \infty} x_n = L$ iff for all $\epsilon > 0$ there exists $N \in \N$ such that
  $n > N \implies |x_n - L| < \epsilon$. The sequence is then said to \textit{converge} to $L$.
\end{definition*}

\begin{definition*}[Limit of a function $f:\R\to\R$]~\\
  $\lim_{x \to a} f(x) = L$ means: for all $\epsilon > 0$ there exists $\delta > 0$ such that
  $0 < |x - a| < \delta \implies |f(x) - L| < \epsilon$.
\end{definition*}

Equivalent notation: $f(x) \to L$ as $x \to a$

\begin{remark*}\hspace{0pt}
  \begin{enumerate}
  \item The value of $f$ at $a$ is irrelevant ($f$ need not be defined at $a$).
  \item $f$ must tend to $L$ from both sides.
  \end{enumerate}
\end{remark*}

\subsection{Limits of functions - Examples}

\begin{example}
  Let $E = \R\setminus \{0\}$ and define $f:E \to \R$ by $f(x) = L$. Then 0 is a limit point of
  $E$ and $f(x) \to L$ as $x \to 0$.
\end{example}

\begin{proof}
  Fix $\delta > 0$. Then $\exists x ~ 0 < |x - 0| < \delta$ is true since we can choose
  $x = \frac{\delta}{2}$. Therefore 0 is a limit point of $E$.

  Fix $\epsilon > 0$. Let $\delta = 1$. Then
  $0 < |x - 0| < \delta \implies |f(x) - L| = 0 < \epsilon$.
\end{proof}

\subsection{Continuity of a function $f$}

\begin{definition*}
$f$ is continuous at $a$ if $\lim_{x \to a} f(x) = f(a)$.
\end{definition*}

Therefore, using the definition of limit, $f$ is continuous at $a$ iff for all $\epsilon > 0$
there exists $\delta > 0$ such that $|x - a| < \delta \implies |f(x) - f(a)| < \epsilon$.

\subsection{Uniform convergence and uniform continuity}

\begin{definition*}[Uniform convergence]
A sequence of functions $\{f_n\}_{n\geq 0}$ has a limit $f$ iff for every point
$x$ in the input set the sequence $\{f_n(x)\}_{n\geq 0}$ has limit $f(x)$.

They \textit{converge uniformly} to $f$ iff the same $m$ works for all input
values.
\end{definition*}

\begin{definition*}[Uniform continuity]
A function $f$ is uniformly continuous iff the same $\delta$ works for all $x_0$.

A function $f$ is uniformly continuous iff for all $\epsilon$, no matter how
small, a $\delta$ exists such that for all $x_0 \in U$, if $x$ is within
$\delta$ of $x_0$ then $f(x)$ is within $\epsilon$ of $f(x_0)$.
\end{definition*}

\subsection{Intermediate value theorem}
\begin{theorem*}
  Let $a, b \in \R$ with $b > a$, and $f:[a,b] \to \R$ be continuous. Let $u$ lie strictly between
  $f(a)$ and $f(b)$. Then there exists $c \in (a, b)$ such that $f(c) = u$.
\end{theorem*}

\begin{proof}
  Define $S := \{x \in [a, b] ~|~ f(x) < u\}$. Since $a \in S$, $S$ is non-empty. By completeness
  of reals $c := \sup S$ exists. The theorem now follows from continuity of $f$ at $c$. (Fix
  $\epsilon > 0$ and consider points $a^* \in (c - \delta, c)$ and $a^{**} \in (c, c + \delta)$,
  noting whether they are in $S$ and the $\epsilon-\delta$ continuity criterion.)
\end{proof}


\subsection{Mean-value theorem}
\begin{theorem*}
  Let $a, b \in \R$ with $b > a$, and $f:[a,b] \to \R$ be continuous on $[a, b]$ and differentiable
  on $(a, b)$. Then there exists $x \in (a, b)$ such that $f'(x) = \frac{f(b) - f(a)}{(b - a)}$.
\end{theorem*}


\subsection{Differentiability implies continuity}
\begin{theorem*}
  Let $f:\R\to\R$ be differentiable. Then $f$ is continuous.
\end{theorem*}

\begin{proof}~\\
  Let $a \in \R$. The claim is that $\limxa f(x) - f(a) = 0$. Since $f$ is differentiable,
  \begin{align*}
    f'(a) &= \lim_{x \to a} \frac{f(x) - f(a)}{x - a}
  \end{align*}
  exists. Therefore by \eqref{limit-of-product}
  \begin{align*}
    \lim_{x \to a} f(x) - f(a) = \lim_{x \to a} (x - a)\frac{f(x) - f(a)}{x - a} = 0\cdot f'(a) = 0.
  \end{align*}
\end{proof}

\begin{remark*}
  Intuitively it seems that differentiability implies continuity because, for the derivative to
  exist, the numerator $f(x) - f(a)$ must get small as $x\to a$, as the denominator $x - a$ does.
\end{remark*}


\newpage
\section{Metric Spaces}

\subsection{Metric space}
\begin{definition}
  Let $X$ be a set. Suppose $d:X \times X \to \R$ satisfies positivity, symmetry and the triangle
  equality. Then $d$ is a metric and $(X, d)$ is a metric space.
\end{definition}

\subsection{Open ball}
\begin{definition}
  Let $(X, d)$ be a metric space, $x \in X$ and $\delta > 0$. Then
  $B(x, \delta) := \{x \in X ~|~ d(x, x) < \delta\}$ iss an open ball of radius $\delta$ centred at
  $x$.
\end{definition}

\begin{remark*}
  Also closed ball, $\leq$. E.g. singleton set.
\end{remark*}

\subsection{Ball-based continuity criterion}
\begin{lemma}
  $f$ is continuous at $x$ if for all $\epsilon > 0$ there exists $\delta > 0$ such that
  $f\(B(x, \delta)\) \subseteq B(f(x), \epsilon))$.

  Equivalently, $B(x, \delta) \subseteq f^\1\(B(f(x), \epsilon)\)$.
\end{lemma}

\subsection{Neighbourhood}
\begin{definition}
  Let $(X, d)$ be a metric space. $N \subseteq X$ is a neighbourhood of $x \in X$ iff there exists
  $\delta > 0$ such that $B(x, \delta) \subseteq N$.
\end{definition}

\begin{remark*}
  $N$ is a neighbourhood of $x$ if a ball can be placed at $x$ without poking outside $N$.
\end{remark*}

\subsection{Open and closed subsets of a metric space}
\begin{definition}
  Let $(X, d)$ be a metric space. Then $U \subseteq X$ is open iff it is a neighbourhood of all of
  its elements.

  $V \seq X$ is closed iff its complement in $X$ is open.
\end{definition}

\subsection{Topology on a metric space}
\begin{definition}
  Let $(X, d)$ be a metric space. The collection $\mc T$ of all open sets in the metric space is
  called the topology of $X$.
\end{definition}

\begin{remark*}
  Note that the definitions so far have the following dependency:

  (open set) $\larrow$ (neighbourhood) $\larrow$ (ball) $\larrow$ (metric),

  so they apply to metric spaces only.
\end{remark*}

\subsection{Open set-based continuity criterion}
\begin{theorem}
  Let $X$ and $Y$ be metric spaces and let $f:X \to Y$. Then

  $f$ is continuous at $x$ iff for every neighbourhood $N \subseteq Y$ of $f(x)$, the preimage
  $f^\1(N)$ is a neighbourhood of $x \in X$.

  $f$ is continuous iff for every open set $U$ of $Y$, $f^\1(U)$ is an open set of $X$.
\end{theorem}

\begin{remark*}
  So we have defined continuity in terms of open sets (the topology). This means that the metric is
  only relevant insofar as it induces the topology; two metric spaces with the same topology have
  the same notion of continuity.
\end{remark*}

\begin{proof}~\\
  Let $f$ be continuous at $x \in X$, and let $N \seq Y$ be a neighbourhood of $f(x)$.

  Then by definition of neighbourhood there exists a ball at $f(x)$ that stays within $N$.

  By continuity of $f$ the preimage of that ball is a superset of a ball at $x$.

  So the preimage of the ball is a neighbourhood of $x$. Therefore the preimage of $N$ is also.

  Conversely, ... similar.

  Let $f$ be continuous on $X$. Now every open set $U$ of $Y$ contains a ball around some point $y$...
\end{proof}

\subsection{Topology on a set, topological space}
\begin{definition}
  A topology on a set $X$ is a collection $\mc T$ of subsets of $X$, which are called the open
  sets. They must satisfy
  \begin{enumerate}
  \item closed under arbitrary unions. In particular, $\emptyset$ is an open set of $X$.
  \item closed under finite intersections. In particular, $X$ is an open set of $X$.
  \end{enumerate}
  A topological space is a pair $(X, \mc T)$.
\end{definition}

\begin{remark*}
  Criteria for closed sets follow by applying de Morgan's laws (closure under finite unions and
  arbitrary intersections).

  $f:X\to Y$ closed iff $f^\1(V)$ is closed for all closed sets $V \seq Y$.
\end{remark*}

\subsection{Limit point}
\begin{definition}
  Let $(X, d)$ be a metric space and $Z \seq X$ be any subset.

  $x \in X$ is a limit point of $Z$ if for all $\delta > 0$ the deleted open ball
  $B(x, \delta)\setminus\{x\}$ has non-empty intersection with $Z$.

  If $z$ is not a limit point of $Z$, then it is an isolated point.

  The set of limit points of $Z$ is denoted $Z'$, and it is clear that
  $Z_1 \seq Z_2 \implies Z_1' \seq Z_2'$.
\end{definition}

\begin{intuition*}
  $x \notin Z$ is a limit point of $Z$ iff it ``touches'' $Z$.

  $z \in Z$ is a limit point of $Z$ if it ``lies in a contiguous region of $Z$''

  An isolated point of $Z$ is what it sounds like.
\end{intuition*}

\begin{example*}~\\
  Let $Z = (0, 1] \cup \{2\}$.

  Intuitively, 0 is a limit point of $Z$ because it ``touches'' $Z$.

  Formally, 0 is a limit point of $Z$ because for all $\delta > 0$ the deleted open ball
  $B(0, \delta)$ contains a point $z > 0 \in Z$.

  Intuitively, 2 is an isolated point.

  Formally, 2 is not a limit point because $\(B(2, 0.5)\setminus\{2\}\) \cap Z = \emptyset$. And
  yet $2 \in Z$, therefore 2 is an isolated point.
\end{example*}

\subsection{Open sets theorems}
\begin{enumerate}
\item An open ball is open
\end{enumerate}

\subsection{Closed sets theorems}
\begin{enumerate}
\item A closed ball is closed
\end{enumerate}


\subsection{Continuity theorems}
\begin{enumerate}
\item $f:X \to Y$ is continuous if for every open ball in $Y$ there is an open ball in $X$ that
  maps inside it.
\item $f:X \to Y$ is continuous if the preimage of $B(f(x), \epsilon)$ in $Y$ is a ball
  $B(x, \delta)$ in $X$.
\item $f:X \to Y$ is continuous if the preimage of the neighbourhood of $f(x)$ is a neighbourhood
  of $x$.
\item $f:X \to Y$ is continuous if the preimage of every open set in $Y$ is an open set in $X$.
\end{enumerate}


\subsection{Continuity of a linear map}
\begin{theorem}
  Let $f:V \to W$ be a linear map between normed vector spaces. Then $f$ is continuous if and only
  if $\{\norm{f(x)} : \norm{x} \leq 1\}$ is bounded.
\end{theorem}

\begin{proof}~\\
  Let $v \in V$.

  Note that $f(v) = f(v) - f(0)$ since $f$ is linear.

  Suppose $f$ is continuous. Then it is continuous at 0.

  Therefore for every $\epsilon > 0$ there exists $\delta > 0$ such that
  $\norm{v} < \delta \implies \norm{f(v)} < \epsilon.$

  $\vdots$

  For the converse, suppose that $\norm{v} \leq 1 \implies \norm{f(v)} < M$.

  Let $\epsilon > 0$ be given.

  Pick $\delta > 0$ such that $\delta M < \epsilon$.

  Now consider two points $u, v \in V$ where $\norm{u - v} < \delta$. We have
  \begin{align*}
    \norm{f(u) - f(v)} = \norm{f(u - v)} = \delta\norm{f\(\frac{u - v}{\delta}\)}.
  \end{align*}

  Note that $\norm{\frac{u - v}{\delta}} < 1$, therefore $\norm{f\(\frac{u - v}{\delta}\)} <
  M$. Therefore we have
  \begin{align*}
    \norm{f(u) - f(v)} < \delta M < \epsilon
  \end{align*}
  as required.
\end{proof}

\subsection{Norm of linear map is bounded}
\begin{theorem}
  $\{\norm{f(x)} : \norm{x} \leq 1\}$ is bounded for linear map $f$, under the Euclidean norm
  $\norm{}_2$.
\end{theorem}

\begin{proof}
  See Oxford A2 Sheet 1 exercises.
\end{proof}
