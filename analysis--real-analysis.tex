\section{Definitions}

\subsection{Limit point}
Let $E \subset \R$. A point $p \in \R$ is a limit point of $E$ iff for all $\delta > 0$ there
exists $x \in E$ such that $0 < |x - p| < \delta$.

\begin{intuition*}
A deleted ball, of arbitrarily small radius, can be placed over $p$ and will capture at least one
point of $E$.
\end{intuition*}

\subsection{Limit, Convergence}

\subsubsection{Limit of a sequence $(x_n)$}
$\limn x_n = L$ iff for all $\epsilon > 0$ there exists an $m \in \N$ such that
$n > m \implies |x_n - L| < \epsilon$. When this is true the sequence is said to \textit{converge}
to $L$.

\subsubsection{Limit of a function $f:\R\to\R$}

$f(x) \to L$ as $x \to a$ means: for all $\epsilon > 0$ there exists $\delta > 0$ such that
$0 < |x - a| < \delta \implies |f(x) - L| < \epsilon$.

Equivalent notation: $\limxa f(x) = L$.

\begin{remark*}
  The value of $f$ at $a$ is irrelevant ($f$ need not be defined at $a$).
\end{remark*}

\subsection{Continuity of a function $f$}
$f$ is continuous at $a$ means: $f(x) \to f(a)$ as $x \to a$.

Therefore, using the definition of limit, $f$ is continuous at $a$ iff for all $\epsilon > 0$
there exists $\delta > 0$ such that $|x - a| < \delta \implies |f(x) - f(a)| < \epsilon$.

\subsection{Uniform convergence and uniform continuity}

\subsubsection{Uniform convergence}
A sequence of functions $\{f_n\}_{n\geq 0}$ has a limit $f$ iff for every point
$x$ in the input set the sequence $\{f_n(x)\}_{n\geq 0}$ has limit $f(x)$.

They \textit{converge uniformly} to $f$ iff the same $m$ works for all input
values.

\subsubsection{Uniform continuity}
A function $f$ is uniformly continuous iff the same $\delta$ works for all $x_0$.

A function $f$ is uniformly continuous iff for all $\epsilon$, no matter how
small, a $\delta$ exists such that for all $x_0 \in U$, if $x$ is within
$\delta$ of $x_0$ then $f(x)$ is within $\epsilon$ of $f(x_0)$.

\section{Theorems}

\begin{theorem*}[Connection between sequences and functions]
  The following two statements are equivalent:
  \begin{enumerate}
  \item $f(x) \to L$ as $x \to a$.
  \item For every sequence $(x_n)$ such that $x_n \neq a$
    \begin{align*}
      \(\limn x_n = a\) \implies \(\limn f(x_n) = f(a)\).
    \end{align*}
  \end{enumerate}
\end{theorem*}

\begin{theorem*}[Limit of product is product of limits]\label{limit-of-product}~\\
  Let $\limxa f(x) = L_f$ and $\limxa g(x) = L_g$. Then
  $\limxa f(x)g(x) = L_fL_g$.
\end{theorem*}

\begin{proof}
  Note that
  \begin{align*}
    \limxa f(x)g(x) &= \limxa \Big((f(x) - L_f)(g(x) - L_g) + L_fg(x) + L_gf(x) - L_fL_g\Big)\\
                          &= L_fL_g + \limxa (f(x) - L_f)(g(x) - L_g),
  \end{align*}
  so we need to show that $\limxa (f(x) - L_f)(g(x) - L_g) = 0$. Fix $\epsilon > 0$. Since
  $\limxa (f(x) - L_f) = \limxa (g(x) - L_g) = 0$, there exists $\delta$ such that whenever
  $|x - a| < \delta$
  \begin{align*}
    |(f(x) - L_f)| < \sqrt \epsilon ~~~\text{and}~~~|(g(x) - L_g)| < \sqrt \epsilon,
  \end{align*}
  therefore $|(f(x) - L_f)(g(x) - L_g) - 0| < \epsilon$ as required.
\end{proof}

\begin{theorem*}[Limit of quotient is quotient of limits]~\\
  Let $\limxa f(x) = L_f$ and $\limxa g(x) = L_g \neq 0$. Then
  \begin{align*}
    \limxa \frac{f(x)}{g(x)} = \frac{L_f}{L_g}.
  \end{align*}
\end{theorem*}

\begin{proof}
  \red{TODO}
  \begin{align*}
    \limxa \frac{f(x)}{g(x)} - \frac{L}{M}
    = \limxa \frac{f(x)}{g(x)} - \frac{1}{g(x)} + \frac{1}{g(x)} - \frac{L}{M}
  \end{align*}

  Let $L_f = \limxa f(x)$ and $L_g = \limxa g(x) \neq 0$.

  Fix $\epsilon > 0$ and let $\delta_f$ and $\delta_g$ be such that
  \begin{align*}
    |x - a| < \delta_f \implies |f(x) - L_f| < \epsilon\\
    |x - a| < \delta_g \implies |g(x) - L_g| < \epsilon.
  \end{align*}
  Let $\delta = \min(\delta_f, \delta_g)$. Then
  \begin{align*}
    \frac{|f(x) - L_f|}{|g(x) - L_f|}
  \end{align*}
\end{proof}


\begin{theorem*}[Intermediate value theorem]
  Let $a, b \in \R$ with $b > a$, and $f:[a,b] \to \R$ be continuous. Let $u$ lie strictly between
  $f(a)$ and $f(b)$. Then there exists $c \in (a, b)$ such that $f(c) = u$.
\end{theorem*}

\begin{proof}
  Define $S := \{x \in [a, b] ~|~ f(x) < u\}$. Since $a \in S$, $S$ is non-empty. By completeness
  of reals $c := \sup S$ exists. The theorem now follows from continuity of $f$ at $c$. (Fix
  $\epsilon > 0$ and consider points $a^* \in (c - \delta, c)$ and $a^{**} \in (c, c + \delta)$,
  noting whether they are in $S$ and the $\epsilon-\delta$ continuity criterion.)
\end{proof}

\begin{theorem*}[Mean-value theorem]
  Let $a, b \in \R$ with $b > a$, and $f:[a,b] \to \R$ be continuous on $[a, b]$ and differentiable
  on $(a, b)$. Then there exists $x \in (a, b)$ such that $f'(x) = \frac{f(b) - f(a)}{(b - a)}$.
\end{theorem*}


\begin{theorem*}[Differentiability implies continuity]
  Let $f:\R\to\R$ be differentiable. Then $f$ is continuous.
\end{theorem*}

\begin{proof}~\\
  Let $a \in \R$. The claim is that $\limxa f(x) - f(a) = 0$. Since $f$ is differentiable,
  \begin{align*}
    f'(a) &= \lim_{x \to a} \frac{f(x) - f(a)}{x - a}
  \end{align*}
  exists. Therefore by \eqref{limit-of-product}
  \begin{align*}
    \lim_{x \to a} f(x) - f(a) = \lim_{x \to a} (x - a)\frac{f(x) - f(a)}{x - a} = 0\cdot f'(a) = 0.
  \end{align*}
\end{proof}

\begin{remark*}
  Intuitively it seems that differentiability implies continuity because, for the derivative to
  exist, the numerator $f(x) - f(a)$ must get small as $x\to a$, as the denominator $x - a$ does.
\end{remark*}





\newpage
\section{Oxford - M2 - Continuity and Differentiability}

\subsection{Limits of functions - Examples}

\begin{example}
  Let $E = \R\setminus \{0\}$ and define $f:E \to \R$ by $f(x) = L$. Then 0 is a limit point of
  $E$ and $f(x) \to L$ as $x \to 0$.
\end{example}

\begin{proof}
  Fix $\delta > 0$. Then $\exists x ~ 0 < |x - 0| < \delta$ is true since we can choose
  $x = \frac{\delta}{2}$. Therefore 0 is a limit point of $E$.

  Fix $\epsilon > 0$. Let $\delta = 1$. Then
  $0 < |x - 0| < \delta \implies |f(x) - L| = 0 < \epsilon$.
\end{proof}

\subsection{Sheet 2}

\begin{mdframed}
\includegraphics[width=400pt]{img/oxford-prelims-M2-analysis-II-sheet-2-1a.png}
\end{mdframed}

\begin{definition*}[continuity]
  $f$ is continuous at $a$ if $f(x) \to f(a)$ as $x \to a$.
\end{definition*}

\begin{definition*}[limit]
  $f(x) \to f(a)$ as $x \to a$ if for all $\epsilon > 0$ there exists $\delta > 0$ such that
  $0 < |x - a| < \delta \implies |f(x) - f(a)| < \epsilon$.
\end{definition*}

\begin{theorem*}
  $f$ is continuous nowhere. I.e. for all $a \in (0, \infty)$ there exists $\epsilon > 0$ such that
  for all $\delta > 0$ there exists $x \in (0, \infty)$ such that $|x - a| < \delta$ and yet
  $|f(x) - f(a)| \geq \epsilon$.
\end{theorem*}

\begin{intuition*}
  However small we make $\delta$, an interval of radius $\delta$ centered at a rational point will
  contain irrational points, and vice versa.
\end{intuition*}

\begin{proof}~\\
  Let $x \in (0, \infty)$.

  Let $q \in \Q \cap (0, \infty)$.




  Fix $\delta > 0$. Note that there are both rational $x$ and
  irrational $x$ satisfying $0 < |x - q| < \delta$. Therefore the maximum value attained by
  $|f(x) - f(q)|$ is $||$
\end{proof}



Let $p \in (0, \infty) \setminus \Q$ be an

Let $p, q \in (0, \infty)$ with $p \notin \Q$ and $q \in \Q$.


\newpage
\section{Oxford - A2 - Metric Spaces}

\subsection{Open sets theorems}
\begin{enumerate}
\item An open ball is open
\end{enumerate}

\subsection{Closed sets theorems}
\begin{enumerate}
\item A closed ball is closed
\end{enumerate}


\subsection{Continuity theorems}
\begin{enumerate}
\item $f:X \to Y$ is continuous if for every open ball in $Y$ there is an open ball in $X$ that
  maps inside it.
\item $f:X \to Y$ is continuous if the preimage of $B(f(x), \epsilon)$ in $Y$ is a ball
  $B(x, \delta)$ in $X$.
\item $f:X \to Y$ is continuous if the preimage of the neighbourhood of $f(x)$ is a neighbourhood
  of $x$.
\item $f:X \to Y$ is continuous if the preimage of every open set in $Y$ is an open set in $X$.
\end{enumerate}



\begin{theorem}
  Let $f:V \to W$ be a linear map between normed vector spaces. Then $f$ is continuous if and only
  if $\{\norm{f(x)} : \norm{x} \leq 1\}$ is bounded.
\end{theorem}

\begin{proof}~\\
  Let $v \in V$.

  Note that $f(v) = f(v) - f(0)$ since $f$ is linear.

  Suppose $f$ is continuous. Then it is continuous at 0.

  Therefore for every $\epsilon > 0$ there exists $\delta > 0$ such that
  $\norm{v} < \delta \implies \norm{f(v)} < \epsilon.$

  $\vdots$

  For the converse, suppose that $\norm{v} \leq 1 \implies \norm{f(v)} < M$.

  Let $\epsilon > 0$ be given.

  Pick $\delta > 0$ such that $\delta M < \epsilon$.

  Now consider two points $u, v \in V$ where $\norm{u - v} < \delta$. We have
  \begin{align*}
    \norm{f(u) - f(v)} = \norm{f(u - v)} = \delta\norm{f\(\frac{u - v}{\delta}\)}.
  \end{align*}

  Note that $\norm{\frac{u - v}{\delta}} < 1$, therefore $\norm{f\(\frac{u - v}{\delta}\)} <
  M$. Therefore we have
  \begin{align*}
    \norm{f(u) - f(v)} < \delta M < \epsilon
  \end{align*}
  as required.
\end{proof}

\begin{theorem}
  $\{\norm{f(x)} : \norm{x} \leq 1\}$ is bounded for linear map $f$, under the Euclidean norm
  $\norm{}_2$.
\end{theorem}

\begin{proof}
  See Oxford A2 Sheet 1 exercises.
\end{proof}


\begin{definition}[Metric space]~\\
  Let $X$ be a set. Suppose $d:X \times X \to \R$ satisfies positivity, symmetry and the triangle
  equality. Then $d$ is a metric and $(X, d)$ is a metric space.
\end{definition}

\begin{definition}[Open ball]~\\
  Let $(X, d)$ be a metric space, $x \in X$ and $\delta > 0$. Then
  $B(x, \delta) := \{x \in X ~|~ d(x, x) < \delta\}$ iss an open ball of radius $\delta$ centred at
  $x$.
\end{definition}

\begin{remark*}
  Also closed ball, $\leq$. E.g. singleton set.
\end{remark*}

\begin{lemma}[Ball-based continuity criterion]~\\
  $f$ is continuous at $x$ if for all $\epsilon > 0$ there exists $\delta > 0$ such that
  $f\(B(x, \delta)\) \subseteq B(f(x), \epsilon))$.

  Equivalently, $B(x, \delta) \subseteq f^\1\(B(f(x), \epsilon)\)$.
\end{lemma}

\begin{definition}[Neighbourhood]~\\
  Let $(X, d)$ be a metric space. $N \subseteq X$ is a neighbourhood of $x \in X$ iff there exists
  $\delta > 0$ such that $B(x, \delta) \subseteq N$.
\end{definition}

\begin{remark*}
  $N$ is a neighbourhood of $x$ if a ball can be placed at $x$ without poking outside $N$.
\end{remark*}

\begin{definition}[Open and closed subsets of a metric space]~\\
  Let $(X, d)$ be a metric space. Then $U \subseteq X$ is open iff it is a neighbourhood of all of
  its elements.

  $V \seq X$ is closed iff its complement in $X$ is open.
\end{definition}

\begin{definition}[Topology on a metric space]~\\
  Let $(X, d)$ be a metric space. The collection $\mc T$ of all open sets in the metric space is
  called the topology of $X$.
\end{definition}

\begin{remark*}
  Note that the definitions so far have the following dependency:

  (open set) $\larrow$ (neighbourhood) $\larrow$ (ball) $\larrow$ (metric),

  so they apply to metric spaces only.
\end{remark*}

\begin{theorem}[Open set-based continuity criterion]~\\
  Let $X$ and $Y$ be metric spaces and let $f:X \to Y$. Then

  $f$ is continuous at $x$ iff for every neighbourhood $N \subseteq Y$ of $f(x)$, the preimage
  $f^\1(N)$ is a neighbourhood of $x \in X$.

  $f$ is continuous iff for every open set $U$ of $Y$, $f^\1(U)$ is an open set of $X$.
\end{theorem}

\begin{remark*}
  So we have defined continuity in terms of open sets (the topology). This means that the metric is
  only relevant insofar as it induces the topology; two metric spaces with the same topology have
  the same notion of continuity.
\end{remark*}

\begin{proof}~\\
  Let $f$ be continuous at $x \in X$, and let $N \seq Y$ be a neighbourhood of $f(x)$.

  Then by definition of neighbourhood there exists a ball at $f(x)$ that stays within $N$.

  By continuity of $f$ the preimage of that ball is a superset of a ball at $x$.

  So the preimage of the ball is a neighbourhood of $x$. Therefore the preimage of $N$ is also.

  Conversely, ... similar.

  Let $f$ be continuous on $X$. Now every open set $U$ of $Y$ contains a ball around some point $y$...
\end{proof}

\begin{definition}[Topology on a set, topological space]
  A topology on a set $X$ is a collection $\mc T$ of subsets of $X$, which are called the open
  sets. They must satisfy
  \begin{enumerate}
  \item closed under arbitrary unions. In particular, $\emptyset$ is an open set of $X$.
  \item closed under finite intersections. In particular, $X$ is an open set of $X$.
  \end{enumerate}
  A topological space is a pair $(X, \mc T)$.
\end{definition}

\begin{remark*}
  Criteria for closed sets follow by applying de Morgan's laws (closure under finite unions and
  arbitrary intersections).

  $f:X\to Y$ closed iff $f^\1(V)$ is closed for all closed sets $V \seq Y$.
\end{remark*}

\begin{definition}[Limit point]~\\
  Let $(X, d)$ be a metric space and $Z \seq X$ be any subset.

  $x \in X$ is a limit point of $Z$ if for all $\delta > 0$ the deleted open ball
  $B(x, \delta)\setminus\{x\}$ has non-empty intersection with $Z$.

  If $z$ is not a limit point of $Z$, then it is an isolated point.

  The set of limit points of $Z$ is denoted $Z'$, and it is clear that
  $Z_1 \seq Z_2 \implies Z_1' \seq Z_2'$.
\end{definition}

\begin{intuition*}[Limit point]~\\
  $x \notin Z$ is a limit point of $Z$ iff it ``touches'' $Z$.

  $z \in Z$ is a limit point of $Z$ if it ``lies in a contiguous region of $Z$''

  An isolated point of $Z$ is what it sounds like.
\end{intuition*}

\begin{example*}~\\
  Let $Z = (0, 1] \cup \{2\}$.

  Intuitively, 0 is a limit point of $Z$ because it ``touches'' $Z$.

  Formally, 0 is a limit point of $Z$ because for all $\delta > 0$ the deleted open ball
  $B(0, \delta)$ contains a point $z > 0 \in Z$.

  Intuitively, 2 is an isolated point.

  Formally, 2 is not a limit point because $\(B(2, 0.5)\setminus\{2\}\) \cap Z = \emptyset$. And
  yet $2 \in Z$, therefore 2 is an isolated point.
\end{example*}
