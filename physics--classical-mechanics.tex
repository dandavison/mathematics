\section{Gravity}
\begin{tabular*}{1.0\linewidth}{l|l}
  $M, m$ & masses of two bodies \\
  $r$    & distance between bodies
\end{tabular*}

Newton's law of gravity: the force between two bodies is $F = G\frac{Mm}{r^2}$, where $G$ is the
gravitational constant.

Units: since $[F] = \frac{LM}{T^2}$, we have $[G] = \frac{LM}{T^2}\frac{L^2}{M^2} = \frac{L^3}{MT^2}$.

\begin{tabular*}{1.0\linewidth}{l|l}
  $M_E = 5.972 \times 10^{24} kg$ & mass of the Earth \\
  $R =  6.371 \times 10^6 m$     & radius of the Earth \\
  $G = 6.67408 \times 10^{-11} m^3 kg^{-1} s^{-2}$        & gravitational constant\\
  $m$ & mass of an object
\end{tabular*}

The acceleration of the object due to gravity at the Earth's surface is
\begin{align*}
  g
  = \frac{F}{m}
  = G\frac{M_E}{R^2}
  = \frac{6.67408 \times 10^{-11} \times 5.972 \times 10^{24}}{(6.371 \times 10^6)^2}
  = 9.82 ms^{-2}.
\end{align*}

\section{Projectile motion}

A projectile of mass $m$ is released with velocity $v_0$ at an angle $\theta$ to the ground. There
is no air resistance.

Note that vertical and horizontal motion are independent of each other. The equations of motion are:

\begin{itemize}
\item {\bf Vertical}\\
  $\ddot{y}(t) = -g$, with initial condition $\dot{y}(0) = v_0\sin\theta$.
\end{itemize}

\subsection{Using integration / FTC to solve the equation of motion}
Informally, since the acceleration is constant, the solution for the velocity function must be
$\dot{y}(t) = v_0\sin\theta - gt$, and therefore by integration the solution for vertical
position is $y(t) = y_0 + v_0\sin\theta t -\frac{1}{2}gt^2$.

More formally, we want to identify the set of functions $y$ that are consistent with the facts:
\begin{align*}
  \ddot{y}(t) &= -g \\
  \dot{y}(0)  &= v_0 \\
  y(0)        &= y_0.
\end{align*}

The first step is to identify the set of first-derivative functions $\dot{y}$ that fit the
facts. Using only the fact that the second derivative is a constant $-g$, we conclude that the
first derivative can be any linear function: $\dot{y}(t) = C - gt$. Then using the initial
velocity, we narrow this further to $\dot{y}(t) = v_0 - gt$.

More formally... the ``antiderivative'' operation maps a single function to a (infinite) set of
functions.
\begin{align*}
  \int \ddot{y}(t) \dt = \int -g \dt = C - gt.
\end{align*}

But the operation of integration maps an $\R \to \R$ function to $\R$.

We have an $\R \to \R$ function $\ddot{y}(t) = -g$. Antidifferentiating tells us that a family of
linear velocity functions are consistent with the acceleration function. What does integration
tell us? It tells us the ``net amount'' of acceleration that has accumulated between time $0$ and
time $t$. (And the FTC tells us that antidifferentiating gives us a trick to find this net amount
easily.)

It's easier to think about velocity and distance. ``Net amount of accumulated velocity'',
i.e. area under the velocity graph, corresponds to net displacement. E.g. $70$ mph $\times$ $2$
hrs equals $140$ miles displacement. What does that familiar calculation correspond to formally?
$140$ miles displacement is saying $x(2) - x(0) = 140$. So the statement is that
\begin{align}
    x(t) - x(0) &= \int_{t'=0}^{t'=t} \dxdt(t') \dt' \label{pcm-ftc} \\
                &= \int_{t'=0}^{t'=t} 70 \dt'.       \nonumber
\end{align}
In this case, \eqref{pcm-ftc} is common sense, because the velocity is constant. But for an
arbitrary velocity function, it would be invoking the FTC.

To compute that integral we can either
\begin{enumerate}
\item Use common sense again: visualize it as the area of a rectangle with height $70$ and width
  $t$.
\item Invoke the FTC a second time: notice that area is increasing linearly with $t$ with a slope
  of $70$, so the answer to the integral is given by the difference in value at $0$ and at $t$ of
  some function which has a slope equal to the integrand. Obvious for a constant integrand, but
  the FTC says that that line of thought still holds when the integrand is any well-behaved
  function. In other words, we need to antidifferentiate: find a function that has derivative
  $70$. In other words, we need to solve a differential equation: $f' = 70$.
\end{enumerate}

So in the velocity/distance problem, the facts were
\begin{align*}
    \dot{x} &= 70 \\
    x(0)    &= 0,
\end{align*}
we wanted to know the function $x$, and we found the solution $x(t) = 70t$ by solving the equation
\begin{align*}
    x(t) - x(0) = \int_0^t 70 \dt.
\end{align*}

Returning to the vertical acceleration of the projectile, the area under the acceleration graph
corresponds to net change in velocity. Recall that the facts are
\begin{align*}
    \ddot{y}(t) &= -g \\
    \dot{y}(0)  &= v_0 \\
    y(0)        &= y_0.
  \end{align*}
  We want to know the function $y$. So, invoking the FTC, we write down the equation
  \begin{align*}
    \int_{t'=0}^{t'=t}\ddot{y}(t') \dt' &= \dot{y}(t) - \dot{y}(0) \\
    \int_{t'=0}^{t'=t}-g \dt' &= \dot{y}(t) - v_0.
\end{align*}
Then, to solve this equation, we invoke FTC again, identifying $-gt$ as an antiderivative, and
conclude that
\begin{align*}
    -gt'\Big|_{t'=0}^{t'=t} &= \dot{y}(t) - v_0 \\
    \dot{y}(t)                   &= v_0 - gt.
\end{align*}
Then, we repeat the procedure, writing down
\begin{align*}
    \int_{t'=0}^{t'=t}\dot{y}(t') \dt' &= y(t) - y(0) \\
    \int_{t'=0}^{t'=t}v_0 - gt \dt'    &= y(t) - y_0 \\
    v_0t' - \frac{1}{2}gt'^2\Big|_{t'=0}^{t'=t} &= y(t) - y_0 \\
    y(t) &= y_0 + v_0t - \frac{1}{2}gt^2.
\end{align*}





\section{Potential energy, conservative force, and work}

The \defn{work} done when a particle moves from $x_0$ to $x_1$ is defined to be
\begin{align*}
  W(x_0 \to x_1) := \int_{x_0}^{x_1} F(x) \dx.
\end{align*}

If this is independent of the path taken between $x_0$ and $x_1$, then we say the force is
\defn{conservative} and define the \defn{potential energy} at $x$, relative to $x_0$ to be
\begin{align*}
  V(x) &:= -\int_{x_0}^x F(x') \dx'.
\end{align*}
If the integral depends on the path taken, the potential energy is undefined.

\blue{A force moves something in space (causes an acceleration). Equivalently, something moves in
  space because a small displacement in some direction is associated with a lower potential
  energy. In fact, the force at a location \emph{is} the gradient in potential energy at that
  location.}

\begin{align*}
  F(x) &= -\frac{\d V(x)}{\dx} \\
  V(x) &= -\int_{x_0}^x F(x') \dx'
\end{align*}

\todo{Understand this}:

A force that depends only on position in one dimension is always conservative, because the integral
depends only on the endpoints. But a force that depends on e.g. time or velocity is
non-conservative. Friction is such a force because although it looks like a constant force
($\mu mg$), its direction (sign) is the opposite of the direction of velocity, so it is in fact
velocity-dependent.

Also, how is this so:
\begin{quote}
  \emph{Since friction always opposes the motion, the contributions to the $W = \int F \dx$ integral are
    always negative, so there is never any cancellation. The result is therefore a large negative
    number.}
\end{quote}
\subsection{Example: gravitational potential energy}

Consider two point masses $M$ and $m$, separated by a distance $r$. Newton's law of gravitation
states that there is a force between them of magnitude $-GMm/r^2$ (the force is attractive, hence
the negative sign).

The potential energy of the system at separation $r$, relative to separation $r_0$, is

\begin{align*}
  V(r) &= -\int_{r_0}^r F(r) \dr \\
       &= -\int_{r_0}^r \frac{-GMm}{r^2} \dr \\
       &= -\(\frac{GMm}{r} - \frac{GMm}{r_0}\).
\end{align*}

Typically in this situation we would choose $r_0 = \infty$ as the reference separation, so that
\begin{align*}
  V(r) = -\frac{GMm}{r}.
\end{align*}

\blue{Potential energy, relative to $r = \infty$, decreases as the two masses get closer: so the
  masses will approach each other. It's always negative because we have measured it relative to
  $r = \infty$, and any separation is more favorable than infinitely large separation.}

\begin{question*}
  What is the gravitational potential energy of a mass $m$ at a height $y$, relative to the Earth's
  surface?
\end{question*}

\begin{proof}
  Let the mass and radius of Earth be $M$ and $R$. Then
  \begin{align*}
    V(y) &= -\(\frac{GMm}{R + y} - \frac{GMm}{R}\) \\
         &= GMm(\frac{R + y - R}{R^2 + Ry}) \\
   &\approx \frac{GMmy}{R^2},
  \end{align*}
  for $y << R$. Recall that the gravitational acceleration of a particle of mass $m$ is $g = \frac{GM}{R^2}$. So
  we can write this in terms of $g$ as
  \begin{align*}
  V(y) &\approx mgy.
  \end{align*}
\end{proof}
\blue{This expression for potential energy decreases as the height $y$ decreases. It does still
  depend on the inverse of the spatial separation, but this factor is approximately constant since
  $y << R$. In any case, the mass falls to Earth, since smaller $y$ has lower potential energy. The
  derivative of the potential energy is the familiar gravitational force
  \begin{align*}
    mg = \frac{GMm}{R^2}.
  \end{align*}
}
