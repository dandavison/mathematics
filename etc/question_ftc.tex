I'd appreciate pointers on the flaws in the following proof sketch of the
Fundamental Theorem of Calculus. I see that textbooks tend to use the Extreme
Value Theorem, whereas in my proof I thought I could use Riemann sums.

$$
\DeclareMathOperator{\d}{\text{d}}
\DeclareMathOperator{\du}{\text{du}}
\DeclareMathOperator{\d}{\frac{\d}{\dx}}
\DeclareMathOperator{\ddx}{\frac{\d}{\dx}}
\renewcommand{\(}{\left(}
\renewcommand{\)}{\right)}
$$


**Definitions**

Define the definite integral of a single variable function $f$ to be a limit of
Riemann sums (to be interpreted as the signed area under the graph of $f$)
\begin{align}
  \int_a^b f(u) \du := \lim_{N \to \infty} \sum_{i=1}^N \frac{b-a}{N} f\(a + \frac{i(b-a)}{N}\),
\end{align}
and define $F(x)$ to be the signed area under the graph of $f$ to the left of $x$:
\begin{align}
  F(x) &:= \int_a^x f(u) \du \\
  &= \lim_{N \to \infty} \sum_{i=1}^N \frac{x-a}{N} f\(a + \frac{i(x-a)}{N}\).
\end{align}

**Claim**

The derivative of $F$ is $f$:
\begin{align}
  \ddx F(x) = f(x).
\end{align}

**Proof**

From the definition of derivative,
\begin{align}
  \ddx F(x) := \lim_{h \to 0} \frac{F(x+h) - F(x)}{h}.
\end{align}
In the numerator is the signed area above a horizontal section of width
$h$. Since our definition of signed area is the limit of Riemann sums, we have
\begin{align}
  \ddx F(x) &= \lim_{h \to 0} \frac{\lim_{N \to \infty} \sum_i^N \frac{h}{N} f\(x + \frac{ih}{N}\)}{h}\\
            &= \lim_{h \to 0} \lim_{N \to \infty} \frac{1}{N} \sum_i^N f\(x + \frac{ih}{N}\)\\
            &= \lim_{N \to \infty} \frac{1}{N} \sum_i^N \lim_{h \to 0} f\(x + \frac{ih}{N}\)\\
            &= \lim_{N \to \infty} \frac{1}{N} \sum_i^N f(x)\\
            &= f(x).
\end{align}

