I know that I don't understand something fundamental about the importance of
conjugation in complex number theory. I will try to give some illustration in
the hope that someone can tell me what it is I am missing.

In outline:

- What does it mean for $(z, \bar z)$ to be ``coordinates''?
- What are the strange partial derivatives used in holomorphic function theory?
- What is it about conjugates that is so central to the notion of holomorphicity?
- If a function is dependent on $\bar z$ then this means it is not holomorphic

- I tend to think of $\mathbb{C}$ as two-dimensional (real and imaginary). But
there's some important sense, I think, in which one can think of $\mathbb{C}$ as
``one dimensional'', and this has to do with conjugacy and holomorphicity. What
is this sense?

As one example take this phrase from p.349 in Penrose's ``The Road To Reality''

> ...by straightforward differentiation with respect to the standard
coordinates $(z, \bar z)$ for the complex plane...

$z$ is not otherwise defined in the paragraph in question. What does it mean
for $(z, \bar z)$ to be ``coordinates''?
