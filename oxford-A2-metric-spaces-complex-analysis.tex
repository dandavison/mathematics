\documentclass[12pt]{article}
\usepackage{enumerate}
\usepackage{notes}
\usepackage{oxford}

\DeclareMathOperator{\diam}{\mathrm{diam}}


\begin{document}

\title{Oxford A2 - Metric Spaces and Complex Analysis
  \footnotetext{\url{https://courses.maths.ox.ac.uk/node/5378}}} \author{Dan Davison}
\author{}
\date{}
\maketitle

\section{Sheet 1}

\subsection{}
\begin{mdframed}
\includegraphics[width=400pt]{img/oxford-a2-1-1.png}
\end{mdframed}

\begin{remark*}
  $d$ is the Hamming distance.
\end{remark*}

For $d$ to be a metric on $X$ we require
\begin{enumerate}
\item \textbf{$d$ is a function $d:X \times X \to $ (a set of non-negative numbers).}\\
  Yes, this is true.
\item \textbf{Positivity}:\\
  Yes, cardinality is never negative and it is clear that $d(a,b) = 0 \iff a = b$.
\item \textbf{Symmetry}:\\
  Yes, follows from the fact that $a_i \neq b_i \iff b_i \neq a_i$.
\item \textbf{Triangle inequality}:\\
  % Want: $d(a, b) \leq d(a, c) + d(c, b)$ for all $a, b, c \in X$.
  Note that $d(a, b) = \sum_{i=1}^n \epsilon_{a_i,b_i}$, where $\epsilon_{ij} :=
  \begin{cases}
    0, ~~~ i = j\\
    1, ~~~ i \neq j
  \end{cases}
$ (the ``negation'' of the Kronecker delta).

  Fix $i \in \{1, 2, \ldots, n\}$. Suppose that $\epsilon_{a_i,b_i} = 0$. Then
  $\epsilon_{a_i,b_i} \leq \epsilon_{a_i,c_i} + \epsilon_{c_i,b_i}$. Alternatively suppose that
  $\epsilon_{a_i,b_i} = 1$. Then we have either $\epsilon_{a_i,c_i} = 0$, in which case
  $\epsilon_{c_i,b_i} = \epsilon_{a_i,b_i} = 1$, or we have $\epsilon_{a_i,c_i} = 1$.

  Therefore $\epsilon_{a_i,b_i} \leq \epsilon_{a_i,c_i} + \epsilon_{c_i,b_i}$, and therefore
  $\sumin \epsilon_{a_i,b_i} \leq \sumin \epsilon_{a_i,c_i} + \sumin \epsilon_{c_i,b_i}$, as required.
\end{enumerate}

Recall that $d_1:\R^n\times\R^n\to\R_{\geq 0}$ is given by $d_1(a, b) := \sum_{i=1}^n|a_i - b_i|$.

Let $\Omega = \{0, 1\} \subseteq \R$ and let $a, b \in \Omega^n \subseteq \R^n$.

Note that $|a_i - b_i| = \epsilon_{a_ib_i}$.

Therefore $d_1(a, b) = \sum_{i=1}^n\epsilon_{a_ib_i} = d(a, b)$, as required.

\newpage
\subsection{}
\begin{mdframed}
\includegraphics[width=400pt]{img/oxford-a2-1-2.png}
\end{mdframed}

\begin{claim*}
  $A \cup B$ is bounded and $\diam(A \cup B) \leq \diam(A) + \diam(B)$.
\end{claim*}

\begin{proof}
  Let $c_1, c_2 \in A \cup B$.

  Note that $c_1, c_2 \in A \implies d(c_1, c_2) \leq \diam(A)$ and
  $c_1, c_2 \in B \implies d(c_1, c_2) \leq \diam(B)$.

  Suppose, without loss of generality, that $c_1 \in A$ and $c_2 \in B$. Let $c_3 \in A \cap
  B$. Then by the triangle inequality we have
  $d(c_1, c_2) \leq d(c_1, c_3) + d(c_3, c_2) \leq \diam(A) + \diam(B)$.

  Therefore $A \cup B$ is bounded and $\diam(A \cup B) \leq \diam(A) + \diam(B)$.
\end{proof}


\begin{claim*}
  If $A \subseteq B$ then $\diam(A) \leq \diam(B)$.
\end{claim*}

\begin{proof}
  Let $A \subseteq B$, and suppose for a contradiction that $\diam(A) > \diam(B)$. Then there exist
  $a_1, a_2 \in A$ such that $d(a_1, a_2) > \diam(B)$. But since $A \subseteq B$ we have
  $a_1, a_2 \in B$. This contradicts the definition of $\diam(B)$ as the supremum over distances
  between pairs of elements of $B$. Therefore $\diam(A) \leq \diam(B)$.
\end{proof}


\newpage
\subsection{}
\begin{mdframed}
\includegraphics[width=400pt]{img/oxford-a2-1-3.png}
\end{mdframed}

\begin{lemma*}
  Let $f:V \to W$ be a linear map between normed vector spaces. Then $f$ is continuous if and only
  if $\{\norm{f(x)} : \norm{x} \leq 1\}$ is bounded.
\end{lemma*}

\begin{proof}
  See $\epsilon-\delta$ argument in lecture notes.
\end{proof}

% \begin{claim*}
%   The Euclidean norm is independent of basis.
% \end{claim*}

% \begin{proof}~\\
%   Let $B$ and $B'$ be bases of $\R^n$ and let $P$ be a matrix with the elements of $B'$ as its
%   columns.

%   I.e. if $v$ contains the coordinates of a vector with respect to basis $B$, Then $Pv$...

%   With respect to $B$, we have $\norm{v} = \sqrt{\sumin v_i^2}$.

%   With respect to $B'$, we have $\norm{v} = \sqrt{\sumin (Pv_i)^2}$.
% \end{proof}

\begin{claim*}
  $\{\norm{\alpha(x)} : \norm{x} \leq 1\}$ is bounded. \Intuition{the image of the unit sphere is
    bounded in norm}
\end{claim*}

\begin{proof}
  Note that $\norm{\alpha(x)}^2 := \sum_{i=1}^n \alpha(x)_i^2$, where $\alpha(x)_i$ is the $i$-th
  coordinate of $\alpha(x)$ with respect to the standard basis.

Therefore it is sufficient to show that $\{|\alpha(x)_i| : \norm{x} \leq 1\}$ is bounded for all
$1 \leq i \leq n$.

Let $a_i$ be the $i$-th row of the $m \times n$ matrix of $\alpha$.

Then $|\alpha(x)_i| = |\langle a_i, x \rangle| \leq \norm{a_i}\norm{x}$.
\end{proof}

% \begin{intuition*}
%   The proof of the (linear, continuous) $\iff$ (linear, image bounded in norm) lemma uses
%   $\epsilon-\delta$ arguments. Then we use Cauchy-Schwarz to show that the image of the unit
%   sphere is bounded in norm as prescribed.
% \end{intuition*}

\newpage
\subsection{}
\begin{mdframed}
\includegraphics[width=400pt]{img/oxford-a2-1-4.png}
\end{mdframed}

\begin{proof}~\\
  Let $B(x, r)$ be an open ball of radius $r$ centered at $x$.

  Note that for every $\delta > 0$ there exists $N \in \N$ such that $d(x_n, \ell) < \delta$ for all
  $n > N$.

  Let $y \neq \ell$ and let $N = \min\{n ~|~ d(x_n, \ell) < d(y, \ell)\}$.

  Let $M = \argmin_{n \in \{1, \ldots, N\}} D(n)$, where $D:\N \to \R$ is defined by
  $D(n) :=
  \begin{cases}
    d(x_n, y), &x_n \neq y\\
    \infty, &x_n = y
  \end{cases}$.

  Informally, $x_M$ is the element of $S$ that is closest but not equal to $y$.

  Note that $\Big(B(\ell, \frac{x_M - y}{2})\setminus\{\ell\}\Big) \cap S = \emptyset$.

  Therefore $y \neq \ell \implies y \notin S'$, or equivalently, $S' \seq \{\ell\}$.

  Now suppose that $S$ is such that there exists $N \in \N$ such that $x_n = \ell$ for all $n > N$.

  Let $M = \max\{n ~|~ x_n \neq \ell\}$.

  Note that $\Big(B(\ell, \frac{x_M - \ell}{2})\setminus\{\ell\}\Big) \cap S = \emptyset$.

  Therefore it is possible that $\ell \notin S'$, or equivalently, $S' = \emptyset$.
\end{proof}

\newpage
\begin{mdframed}
\includegraphics[width=400pt]{img/oxford-a2-1-5-1.png}
\end{mdframed}

\begin{lemma*}
  Let $X$ and $Y$ be metric spaces. $h:X \to Y$ is continuous iff for every open set $U \in Y$, the
  preimage $h^\1(U)$ is an open set in $X$.
\end{lemma*}

\begin{claim*}
  $(g \circ f):R \to T$ is continuous.
\end{claim*}

\begin{proof}
Let $U_T$ be an open set of $T$.

Note that $(g \circ f)^\1(t) = f^\1(g^\1(t))$.

By the lemma, since $f$ and $g$ are continuous, $f^\1(g^\1(U_T))$ is an open set in $R$.
\end{proof}

\begin{mdframed}
\includegraphics[width=400pt]{img/oxford-a2-1-5-2.png}
\end{mdframed}

Where does it fail?

\begin{enumerate}
\item Positivity?
\item Symmetry?
\item Triangle inequality?
\end{enumerate}

\begin{mdframed}
\includegraphics[width=400pt]{img/oxford-a2-1-5-3.png}
\end{mdframed}

\begin{definition*}
  A subset $X \seq M$ is open if at every point $x \in X$ there exists a $\delta$ such that
  $B(x, \delta) \seq X$.
\end{definition*}

\begin{proof}~\\
  If $M$ is empty it is vacuously true, so suppose $M$ is not empty.

  $M$ itself is open since there are no points outside $M$, so any ball centred on a point of $M$
  must be a subset of $M$.

  Suppose that $X \subset (M, d)$ is not open.

  Then there exists $x \in X$ with the following property: there does not exist $\delta > 0$ such
  that $B(x, \delta) \seq X$.

  However, $M$ is finite. Therefore we may pick the element of $M$ that is closest to $x$, and set
  $\delta$ to be half this distance.

  This contradiction shows that no such non-open $X$ exists.
\end{proof}

\begin{mdframed}
\includegraphics[width=400pt]{img/oxford-a2-1-6-1.png}
\end{mdframed}

\begin{enumerate}
\item $(-5, 1) \cup (0, \infty) = (-5, \infty)$ open, not closed
\item $(-\infty, 2]$ not open, closed
\item $\{0\}$ not open, closed
\item $(0, 2]$ not open, not closed
\item $\R$ open, closed
\item $\Q$ not open, not closed (any interval $I \seq \R$ contains both rationals and irrationals)
\item $\Z$ not open, closed
\item $\emptyset$ open, closed
\end{enumerate}

\begin{mdframed}
\includegraphics[width=400pt]{img/oxford-a2-1-6-2.png}
\end{mdframed}

\begin{enumerate}
\item $[0, 1] \times \{0\}$ not open, closed\\
  \begin{tiny}
    A ball at any point will contain points outside, so not open.\\
    Complement is open, so closed.
    \par
  \end{tiny}
\item $(0, 1) \times \{0\}$ not open, not closed\\
  \begin{tiny}
    A ball at any point will contain points outside (by poking out in the y direction), so not open.\\
    Complement is not open (e.g. contains origin)
    \par
  \end{tiny}
\item $\{(x,y) ~|~ 1 < 4x^2 + y^2 < 4\}$ open, not closed\\
  {\tiny
    Region sandwiched between inner and outer ellipse.\\
    Clearly open, complement not open.
    \par}
\item $\{(x,y) ~|~ xy = 1\}$ not open, closed\\
  {\tiny
    clearly not open, e.g. ball at $(1,1)$ leaves the set\\
    complement is open
    \par}
\item $\Z \times \R$ not open, closed\\
  {\tiny
    Collection of horizontal lines.\\
    Not open, any ball will poke up/down in the y-direction, leaving the set.\\
    Complement is $\R^2$ with a collection of horizontal lines deleted.\\
    So complement is open.
    \par}
\item $\{(x,y) ~|~ x \in \Z \text{~and~} y > 0\}$ not open, not closed\\
  {\tiny
    Vertical lines, starting just above the y=0 line.\\
    Clearly not open.\\
    Complement not open, since it includes the y=0 line.
    \par}
\item $\{(x,y) ~|~ \exp(x^2 + y^2) = 1 + (y^3 - x^3)(x^7 + y^7)\}$ not open, closed\\
  \begin{tiny}
    One equation in 2D ambient space $\implies$ no solutions or a line of solutions.\\
    0 is a solution $\implies$ line of solutions $\implies$ not open, closed.\\
    \par
  \end{tiny}
\end{enumerate}

\newpage
\begin{mdframed}
\includegraphics[width=400pt]{img/oxford-a2-1-7.png}
\end{mdframed}

% \begin{remark*}
%   Since on a closed interval every continuous function is bounded and attains its bounds, the
%   supremum metric here is just a max metric.
% \end{remark*}

\begin{mdframed}
\begin{lemma*}
  Let $f:V \to W$ be a linear map between normed vector spaces. Then $f$ is continuous if and only
  if $\{\norm{f(v)} : \norm{v} \leq 1 \}$ is bounded.
\end{lemma*}

\begin{proof}
  See question 3.
\end{proof}
\end{mdframed}

% \url{https://math.stackexchange.com/questions/975759/are-differentiation-and-integration-continuous-functions}

\begin{enumerate}[label=\roman*)]
\item \textbf{Is $C^1[a,b]$ a closed subset of $C[a,b]$?}
  \red{TODO}

  $C^1[a,b] \seq C[a, b]$ is closed iff its complement is open.

  To answer this question we can either show that its complement is not open (could be done by
  counter-example), or open (requires positive proof).

  The complement of $C^1[a,b] \seq C[a, b]$ is the set of continuous functions that are not
  continuously differentiable. An example is $x \mapsto |x|$.

  Counter-example approach: if we could exhibit a pair of functions $f \in (C^1[a,b])^C$ and
  $g \in C^1[a,b]$, such that the distance between $f$ and $g$ can be made arbitrarily small, then
  this would prove that $(C^1[a,b])^C$ is not open and hence that $C^1[a,b]$ is not a closed subset
  of $C[a,b]$.

% \item Let $f,g \in C^1[a,b]$, let $h = f - g$ and fix $\epsilon > 0$.

%   For continuity at $f$ we seek $\delta > 0$ such that
%   \begin{align*}
%     & \sup_{x \in [a,b]} |f(x) - g(x)| < \delta \implies \sup_{x \in [a,b]} |f'(x) - g'(x)| < \epsilon\\
%     \iff & \sup_{x \in [a,b]} |(f - g)(x)| < \delta \implies \sup_{x \in [a,b]} |(f - g)'(x)| < \epsilon\\
%     \iff & \sup_{x \in [a,b]} |h(x)| < \delta \implies \sup_{x \in [a,b]} |h'(x)| < \epsilon.
%   \end{align*}
%   Suppose that such a $\delta$ exists.

\item \textbf{Is differentiation continuous?}\\
  Note that the differentiation operator $D:C^1[a,b] \to C[a,b]$ is a linear map between vector
  spaces.

  Define a norm on $C[a,b]$ as $\norm{f} := \sup_{x\in [a,b]}|f(x)|$.

  Then, according to the lemma, $D$ is continuous if and only if
  $\{\norm{D(f)} : \norm{f} \leq 1 \}$ is bounded.

  Suppose for a contradiction that $M$ is such a bound. But consider the logistic function
  $f:[a,b] \to (0, 1)$ given by $f(x) := (1 + e^{-4(M+1)(x-a)})^{-1}$. Its derivative is
  $$f'(x) = 4(M+1)f(x-a)(1-f(x-a)).$$ So we have $f \in C^1[a,b]$ and $\norm{f} \leq 1$ and yet
  $f'(a) = M+1$.

  Therefore no such bound $M$ exists. Therefore the differentiation operator is not continuous.

\item \textbf{Is integration continuous?}\\
  Note that the integration operator defined by $F(f) := \int_a^x f$ is a linear map between vector
  spaces.

  Define a norm on $C[a,b]$ as $\norm{f} := \sup_{x\in [a,b]}|f(x)|$.

  Then, according to the lemma, $F$ is continuous if and only if
  $\{\norm{F(f)} : \norm{f} \leq 1 \}$ is bounded.

  Note that on a closed interval every continuous function is bounded.

  Let $f \in C[a,b]$ and let $L$ and $M$ be the lower and upper bounds of $f$ respectively.

  Then $(F(f))(x) \leq (M-L)x$, for all $x \in [a,b]$.

  Therefore $F$ is continuous.

\end{enumerate}

\newpage
\begin{mdframed}
\includegraphics[width=400pt]{img/oxford-a2-1-8-1.png}
\end{mdframed}

\begin{table}[h!]
  \begin{tabular}{c|c|l}
    $S$         &$S'$         &                \\
    \hline
    $(0, 1)$    & $[0, 1]$    &                \\
    $\{0\}$     & $\emptyset$ & 0 is an isolated point \\
    $\R$        & $\R$        &                \\
    $\Q$        & $\Q$        & Any ball placed on a rational contains both other rationals and irrationals.\\
    $\Z$        & $\emptyset$ & All integers are isolated points.               \\
    $\emptyset$ & $\emptyset$ &
  \end{tabular}
\end{table}

\begin{mdframed}
\includegraphics[width=400pt]{img/oxford-a2-1-8-2.png}
\end{mdframed}

\begin{proof}~\\
  \begin{comment}
  Let $S \subset \R$.

  Let $u \in (S')'$.

  Then $\Big(B(u, \epsilon) \setminus \{u\}\Big) \cap S' \neq \emptyset$ for all $\epsilon > 0$.

  Therefore there exists $t \in (S')'$ such that $t \in S'$.

  If $t \in S$

  Want: $\Big(B(u, \epsilon) \setminus \{u\}\Big) \cap S \neq \emptyset$ for all $\epsilon > 0$.
\end{comment}
\end{proof}

\begin{mdframed}
\includegraphics[width=400pt]{img/oxford-a2-1-8-3.png}
\end{mdframed}

\begin{table}[h!]
  \begin{tabular}{l|c|l|l}
    $S$                                 &  open          &   closed      &    \\
    $\{(x, y) \in S ~|~ x \geq 1\}$     &  open          &   closed      &    \\
    $\{(x, y) \in S ~|~ x > 0\}$        &  not open      &   open        &    \\
    $\{(1 + 1/n, 1) ~|~ n \in \N\}$     &  not open      &   not closed  & complement contains $(1,1)$   \\
    $\{(1/n, 1) ~|~ n \in \N\}$         &  not open      &   closed      & unlike previous, complement excludes $(0, 0)$\\
  \end{tabular}
\end{table}


\newpage
\begin{mdframed}
\includegraphics[width=400pt]{img/oxford-a2-1-9.png}
\end{mdframed}

\begin{proof}~\\
  \textbf{Well-defined:}

  \textbf{Positivity:}: by Cauchy-Schwartz we have $1 - \frac{\langle v,w\rangle^2}{\norm{v}^2\norm{w}^2} > 0$
  therefore $d(L_1, L_2) > 0$ since we take the positive square root.

  \textbf{Symmetry:}

\end{proof}

\end{document}
