\newcommand{\pzpx}{\frac{\partial z}{\partial x}}
\newcommand{\pzpy}{\frac{\partial z}{\partial y}}


\section{3blue1brown - Essence of Calculus}
\subsection{The paradox of the derivative}
\subsection{Derivatives formulas through geometry}
\subsection{Visualizing the chain rule and product rule}

More complex functions can be formed by addition, multiplication and
composition of simpler functions. How do we compute derivatives of such more
complex functions?

\subsection{Sum rule}
Suppose $f(x) = g(x) + h(x)$. Visualize an input parameter $x$ represented by
the x-axis, and the graphs of $g$ and $h$, and a third graph of $f$ whose
height at every point is the sum of the other two.

A horizontal nudge $\dx$ to the input causes vertical changes $\d g(x)$ and
$\d h(x)$. The resulting vertical change to $f(x)$ is
\begin{align*}
  \d f(x) = \d g(x) + \d h(x),
\end{align*}
or equivalently
\begin{align*}
  \frac{\d f(x)}{\dx} = \frac{\d g(x)}{\dx} + \frac{\d h(x)}{dx}.
\end{align*}


\newpage
\subsection{Product rule}
Suppose $f(x) = g(x) \cdot h(x)$. Consider an input parameter $x$ and visualize a
rectangle with one side length $g(x)$ and the other side length $h(x)$. $f(x)$
is the area of the rectangle.

A nudge $\dx$ to the input causes the sides to grow by $\d g(x)$ and $\d h(x)$
respectively. Therefore the change to the area is approximately
\begin{align*}
  \d f(x) = h(x) \d g(x) + g(x) \d h(x),
\end{align*}
or equivalently
\begin{align*}
  \frac{\d f(x)}{\dx} = h(x) \frac{\d g(x)}{dx} + g(x) \frac{\d h(x)}{dx}.
\end{align*}

\subsection{Integration by Parts}

TODO: graphical intuition (see wikipedia page)

\newpage
\subsection{Chain rule: function composition}
Suppose $f(x) = g(h(x))$. Visualize 3 real number lines: at the top the input
parameter $x$; in the middle $h(x)$ and at the bottom $g(h)$.

A nudge $\dx$ to the input causes a change $\d h = \frac{\d h}{\dx} \dx$, which in turn causes a
change $\dg = \frac{\dg}{\dh} \dh$. So we have
\begin{align*}
  \df = \d g(h(x)) = \frac{\dg}{\dh} \frac{\dh}{\dx} \dx,
\end{align*}
or equivalently
\begin{align*}
  \frac{\df}{\dx} = \frac{\d g(h(x))}{\dx} = \frac{\dg}{\dh} \frac{\dh}{\dx}.
\end{align*}

\subsubsection{Example}
$f(x) = \sin(x^2) = g(h(x))$. So the middle number line shows $h(x) = x^2$ and
the output number line at the bottom shows $g(h) = \sin(h)$.

We know that for the outer function, $\dg = \cos h \dh$, and for the inner
function $\dh = 2x \dx$, so
\begin{align*}
  \d g(h(x)) = \cos (h) \cdot 2x \dx = \cos(x^2)\cdot 2x \dx.
\end{align*}

\newpage
\subsection{Implicit differentiation}

Consider the circle defined by $x^2 + y^2 = 5$. Here, on the face of it, we
don't have a function with input and an output; we just have a set of points in
2D defined by some condition which they satisfy (an implicit curve).

How do we find the tangent to the circle at the point $(3,4)$? We want $\dydx$.

Consider a related problem. A ladder of length 5 is leaned against a wall, with
initial height 4, and is slipping down at 1 m/s. Define $y(t)$ to be its height
at time $t$, so $\dydt = -1$. What is $\dxdt$?

Clearly the starting point is that $x(t)^2 + y(t)^2 = 5^2$. One solution is\\

\begin{mdframed}
  \begin{align*}
    x(t) &= \(5^2 - y(t)^2\)^{1/2}\\\\
    \dxdt &= \frac{-2y\dydt}{2 \(5^2 - y(t)^2\)^{1/2}} = \frac{y}{x}.
  \end{align*}
\end{mdframed}

Another solution is to note that the sum of the squares is constant:\\

\begin{mdframed}
  \begin{align*}
    \frac{\d \(x(t)^2 + y(t)^2\)}{\dt} &= 0\\\\
    2x\dx + 2y\dy &= 0\\
    \frac{\dx}{\dt} &= -\frac{y \dy}{x\dt} = \frac{y}{x}.
  \end{align*}
\end{mdframed}
~\\
In the case of the ladder problem, it was clear what was going on since we
could differentiate $x(t)^2 + y(t)^2$ with respect to $t$.

Going back to the implicit curve $x^2 + y^2 = 5$, there is in fact a function
there: a function of two variables:
\begin{align*}
  z(x, y) = x^2 + y^2
\end{align*}

We want $\dydx$. What is $\dydx$? It's a ratio of two nudges to the two input
variables. OK, but those nudges could be anything; the ratio is not
determined. But we have a condition: the two nudges must stay on a tangent line
to the circle. So,
\begin{align*}
  (x + \dx)^2 + (y + \dy)^2 &= 5\\
  x^2 + 2x\dx + y^2 + 2y\dy &= 5\\
  2x\dx + 2y\dy &= 0\\
  \dydx  &= -\frac{x}{y}
\end{align*}



The derivative of $z$ in the direction of the vector
$\vec u = \cvecc{\dx}{\dy}$ is
\begin{align*}
  \D_u z &= \pzpx \dx + \pzpy \dy\\
         &= 2x \dx + 2y \dy.
\end{align*}
And the condition for staying on the tangent to the circle is that $z$ stays
constant:
\begin{align*}
  \D_u z &= 2x \dx + 2y \dy = 0\\
   \frac{\dy}{\dx} &= \frac{-x}{y}.
\end{align*}


\begin{align*}
  y^2\sin x = x
\end{align*}
