\section{Background}

\subsection{Vector spaces and fields}
A \defn{field} is a set which is an abelian group under both addition and multiplication.

A \defn{vector space} is an additive abelian group $X$, together with a field $F$, such that $X$ is
closed under linear combinations with scalars from the field $F$.

\subsubsection{Examples}

\begin{enumerate}
\item The additive abelian group $\R$, together with the field $\R$, is a vector space.

  Let $x \in \R$ with $x \neq 0$. Then $\{x\}$ is a basis for $\R$, since every element of $\R$ can
  be expressed uniquely as a scalar multiple of $x$. So $\R$ is one-dimensional.

\item The additive abelian group $\R^n$, together with the field $\R$, is a vector space. It is $n$-dimensional.

\item The additive abelian group $\C$, together with the field $\R$, is a vector space. It is isomorphic to $\R^2$.

\item The additive abelian group $\C$, together with the field $\C$, is a vector space.

  Let $z \in \C$ with $z \neq 0$. Then $\{z\}$ is a basis for $\C$

\item The vector space $\C$, over the field $\C$, is the set $\{\alpha z\}$

\end{enumerate}
